\begin{textsample}{POS Dim 1 – human – Score 80.00 – t305\_human.txt}  \label{ex:f1_pos_016}
How does an eager engineering \textbf{student}, in the era of expanding space exploration, become an \textbf{ecologist}, dedicated to the \textbf{Earth}?
In my case : Rachel Carson, the Cuyahoga River, Gregory Bateson, Fools Crow, and a magpie.
The Cuyahoga River, however, had been so polluted with oil and chemicals, that it routinely caught on fire.
The incident in 1969 was the 13th fire on the river.
A 1912 fire killed five \textbf{people}, and a 1952 fire caused over \$1.3-million in damages ( over \$10-million \textbf{today} ).
Rachel Carson ’s Silent \textbf{Spring} alerted me to think about \textbf{ecology}, but the Cuyahoga River fire \textbf{rocked} my \textbf{world}.
The river fire felt apocalyptic, primal, like a sign from Humbaba, the forest protector in the Epic of Gilgamesh.
Earth, air, fire, and water.
I felt that perhaps science, per se, was not the cause of these problems, but that the dysfunction emerged from our economic and political policies that used scientific discoveries to bolster profits, without considering or accounting for the unintended consequences.
I gave up the \textbf{idea} of being an aerospace scientist, left \textbf{school}, and set out to become a \textbf{journalist}, so I could \textbf{research} and write about the challenges \textbf{humanity} was facing.
I felt that \textbf{society} needed to pay a \textbf{lot} more \textbf{attention} to the wild, natural \textbf{world}, and to learn how wild ecosystems actually \textbf{work}.
These \textbf{ideas} led me to another breakthrough.
I lived in San Francisco for a while and heard \textbf{stories} of a popular professor at the University of California at Santa Cruz, who was \textbf{talking} about \textbf{ecology} and complex systems.
It might feel hard to imagine now, but \textbf{ecology} remained an esoteric \textbf{idea} in 1969.
I \textbf{started} going to the lectures.
Gregory Bateson lectures were not like \textbf{anything} I had yet \textbf{experienced} in \textbf{university}.
It seemed that he never quite summed \textbf{things} up.
He would display various objects and ask \textbf{students} to describe which came from a living being, and how you knew.
The spiral of an ammonite fossil gives it away as “ life, ” but are there non-living spirals in nature?
What about hurricanes?
He would draw a strange \textbf{picture} on the blackboard and ask \textbf{students} to describe it.
To do so, \textbf{one} could perhaps break it up into parts that \textbf{looked} like common objects.
As \textbf{students} struggled with this, he said \textbf{something} that became a fundamental \textbf{awareness} for me, even to this \textbf{day} : “ All divisions are arbitrary. ” Our \textbf{language}, nouns and verbs, divide the \textbf{world} into “ objects, ” but in the real \textbf{world}, \textbf{nothing} exists in isolation. \textbf{Everything} exists in \textbf{relationship}.
We \textbf{talk} about a “ tree ” in “ soil, ” and the “ atmosphere, ” but none of these exist as they are without the \textbf{others}, and each \textbf{one} flows through the \textbf{other}, influences and changes the \textbf{other}.
When I bite into an apple, when do the molecules of the fruit stop being an apple and \textbf{start} being me?
Bateson emphasized that to understand life, we had to understand “ the pattern which connects, ” the tree to the soil, to the atmosphere. “ \textbf{Schools} ” he said, “ teach almost \textbf{nothing} about the patterns that connect. ” What is the pattern which connects all the living creatures?
He \textbf{talked} about seeing “ living \textbf{beings} with recognition and empathy, ” staying aware and responsive to pattern and \textbf{relationship}.
How am I related to the tree?
What pattern connects me to the Red-tailed hawk or the prairie dog?
Living forms, he suggested, have repetitive patterns, “ rhythmical ” the \textbf{way} dance or \textbf{music} is rhythmical, repetitive but with modulation.
Without being explicit, Bateson was teaching \textbf{students} how to think about complex living systems, how to notice the deep symmetry and reciprocity in living \textbf{relationships}.
The pattern which connects \textbf{us} to the natural \textbf{world} is, in Bateson ’s \textbf{language}, “ a metapattern ” a “ pattern of patterns. ” Nothing exists alone.
All divisions are arbitrary.
All life exists within these embedded systems and subsystems, trading resources, communicating, and learning together.
Furthermore, since co-evolving systems include random factors – as do chess games or hurricanes – they are not entirely predictable, even if \textbf{one} knows the rules.
Thus – and this our \textbf{society} needs desperately to embrace – systems themselves evolve, and new \textbf{relationships} almost always include unintended consequences.
Each subsystem – organ, body, \textbf{society} – within an ecosystem co-creates a complex web of processes with its neighbouring subsystems.
Nature is a web of \textbf{relationships}.
Nature is not a collection of objects, but of processes.
Our ecological efforts need to recognize and protect these complex \textbf{relationships}.
In 1966, I enrolled as a physics and math \textbf{student} at Occidental College in Los Angeles.
Like most of the \textbf{students}, I felt a \textbf{great} anticipation about space exploration and the discoveries of astrophysics.
During the \textbf{summer} after my first \textbf{year}, I got a \textbf{job} at Lockheed Aerospace as an apprentice engineer.
Bateson used to give \textbf{students} a \textbf{test} with two \textbf{questions} : 1.
Define “ entropy. ” 2.
Define “ sacrament. ” As a physics \textbf{student}, I understood \textbf{something} about entropy, a characteristic of thermodynamic processes that tells \textbf{us} all \textbf{organization}, all \textbf{information}, in the universe requires energy to establish and maintain itself.
Without energy flow, \textbf{organization} decays.
Furthermore, when energy is used, some is lost.
All used energy is dissipated.
Without the sun, bathing \textbf{Earth} in energy every \textbf{day}, all life here would wither and die.
Entropy was a complex observation of natural processes, but \textbf{something} I could imagine.
Sacrament is different.
It has to do with perception, a \textbf{relationship} between observer and the \textbf{world}, a \textbf{sense} of what feels sacred and how to express it.
Sacrament was even trickier than entropy.
Sacrament had \textbf{something} to do with our \textbf{relationship} to the complex living system that was our \textbf{world}.
In his later \textbf{book}, \textbf{Mind} and Nature, Bateson explained that “ sufficient \textbf{answers} for this \textbf{test} would imply a \textbf{good} grasp of human science and human culture, and would represent a \textbf{good} \textbf{start} in solving \textbf{humanity} ’s problems. ” Bateson often emphasized that “ we have to learn to think the \textbf{way} nature \textbf{works}. ” For me, he opened a door that led into a deep, non-human-centred \textbf{ecology}.
I thought for a long \textbf{time} about sacrament, and how that fit in.
Out of \textbf{school}, I became eligible for the US military draft, but having heard \textbf{stories} from Vietnam veterans and deserters, I had no intention of participating in the horrific, senseless \textbf{war}.
To avoid jail \textbf{time}, I went to Canada, where I met Bob Hunter, Irving and Dorothy Stowe, and \textbf{others} who were active pacifists and \textbf{ecologists}.
We \textbf{talked} of \textbf{starting} a global “ \textbf{ecology} \textbf{movement}, ” on the same scale as the \textbf{peace} and civil rights \textbf{movements}.
This \textbf{group} began as a pacifist \textbf{movement}, became Greenpeace, and in 1975 we sailed out of Vancouver to save \textbf{whales} by confronting whaling fleets, the \textbf{organization} ’s first real \textbf{ecology} action.
In the 1970s, some of \textbf{us} in Greenpeace \textbf{worked} closely with Indigenous \textbf{activists} in Canada and the US.
Bill Means from the American Indian Movement invited me to South Dakota to support the Lakota nation taking back some of their traditional land in the Black Hills.
On the Pine Ridge reservation, Means introduced me to the elders, the grandmothers, and to Chief Frank Fools Crow.
In ceremonies, I noticed that the grandmothers held a special place, central but often quiet, allowing the younger relatives to conduct \textbf{conversations} and rituals.
However, whenever conflict or confusion arose, the grandmothers stepped in to sort it out.
This felt like a quiet sort of power.
I also noticed that speakers ended their statements or prayers with the Lakota “ Mitakuye-Oyasin ” ( pronounced mi-TAHK-wee-a-say ).
I learned that this meant “ all my relations, ” and Fools Crow explained the deeper spiritual meaning.
As a boy, the elder Chief said, he was taught that “ every living \textbf{thing} is sacred, and that every living \textbf{thing} is our relative.
The \textbf{Earth} is our Bible, our Church, our store, our security.
When we pray, we say Mitakuye-Oyasin to remind ourselves that whatever we wish for ourselves, we must also wish for all our relatives, for all life. ” I met Richard Kastl from the Osage-Creek community in Oklahoma.
He appeared as an imposing bear-like \textbf{man}, but with an aura of tranquility that seemed to draw tension out of the air, transforming it into peacefulness.
He told me : “ The sacred means that \textbf{everything} is in its proper order.
Guns and \textbf{bombs} are not that \textbf{great} of a power.
The natural \textbf{world} is the real power.
The Mother holds the hammer. ” However, that \textbf{summer} of 1967, I also read Rachel Carson ’s Silent \textbf{Spring}.
Carson ’s \textbf{book}, written five \textbf{years} earlier, examined thirty-five bird species threatened with extinction due to chemical pesticides, including organo-chlorines such as DDT.
I knew just enough chemistry to know she was speaking the truth, and when the chemical companies \textbf{attacked} her, I knew they were lying. \textbf{Years} later, when Greenpeace \textbf{worked} with the Tsleil-Waututh nation in Canada to stop tar sands oil tankers in Vancouver ’s harbour, I participated in weekly sweat ceremonies hosted by Tsleil-Waututh Sundance Chief Rueben George and his \textbf{mother} Ta’ah.
Again, Rueben, Ta’ah, and the \textbf{others} ended their prayers with “ All my relations. ” George would say, “ \textbf{One} heart, \textbf{one} \textbf{mind}, \textbf{one} prayer.
We are all in this together, even the oil \textbf{executives} and their \textbf{families}. \textbf{One} \textbf{family}. ” Mitakuye-Oyasin.
All my relations.
My Indigenous \textbf{friends} helped me understand the role of sacrament, acknowledging and ritualizing our oneness with all creation, humbling \textbf{one} ’s self to the larger living \textbf{family}, and reminding ourselves that “ The Mother holds the hammer. ” I must have been four \textbf{years} old when my sister, Kaye, two \textbf{years} older, first led me across the canal bridge and into the red \textbf{foot} hills of Big Horn Basin, \textbf{north} of Worland, Wyoming.
I didn't know it then, but this was the home that Crazy Horse, Gall, and Sitting Bull died to retain for their Lakota \textbf{families}, ancestors of my later \textbf{friends} in Pine Ridge.
Kaye and I explored ravines and caves in the dry Palaeozoic siltstone that flashed with crystalline colors in the sun.
On the \textbf{way} home, we traversed a ledge to a sandstone crag.
My sister slid down but I could not muster the \textbf{courage} nor back up.
She instructed me to wait as she ran home to fetch help.
I \textbf{watched} her across the dry ground until she became a glint of motion and then disappeared.
I felt alone but trusted my sister to bring help.
There, awaiting my rescuers, I leaned against the brittle \textbf{earth} and pondered the nature of \textbf{rock} and dirt.
I had never felt so alone.
Sunlight glinted off flakes of mica in the coarse, reddish-brown soil.
Abruptly, a magpie appeared, looped around the sculptured stone, lit onto an outcrop, and thumped his beak twice in blistering soil.
The bird tilted its \textbf{head} and \textbf{looked} directly at me, curiously.
I realize now, \textbf{looking} back, that this was the moment I first felt the wild, natural \textbf{world} \textbf{looking} back at me, as I contemplated it.
The magpie — which I now know as the North American Black-winged magpie ( Pica hudsonia ) — showed no \textbf{fear}, but an inquisitive poise.
I remained still, so that the bird would stay.
I felt a certain comfort and safety with the bird nearby, no longer alone.
It hopped from \textbf{rock} to \textbf{rock}, but always \textbf{returned}, cocked its \textbf{head}, and contemplated me.
I felt seen and felt a sort of affinity that I could not have explained then, and find hard to explain now.
I suspect that this was the moment that I first felt a kinship with the wild \textbf{world}, the \textbf{world} not dominated by \textbf{people}, their goals, their enthusiasms, and their presumptions.
Formal \textbf{education} can desensitize \textbf{us}, teach \textbf{us} to see through the lens of acumen or ambition.
Without having the \textbf{words}, this was the moment I deeply understood “ all my relations, ” when I felt the meaning of “ the patterns that connect, ” when sacrament felt natural, and when the instinct to protect all living \textbf{beings} felt innate, part of being alive.
As expected, my sister \textbf{returned} with help.
When I contemplate this now, I realize that Rachel Carson, Gregory Bateson, the burning Cuyahoga River, and the Indigenous elders just reminded me of the \textbf{lesson} of the magpie.
All my relations. \textbf{One} heart, \textbf{one} \textbf{mind}, \textbf{one} prayer.
Rachel Carson : Silent \textbf{Spring}, 1962 The Sea Trilogy : Under the Sea-Wind / The Sea Around Us / The Edge of the Sea In the 1950s and 60s, the US military funded \textbf{research} into synthetic pesticides, and the US Department of Agriculture attempted to eradicate fire ants with a mixture of DDT and fuel oil, a precursor to Agent Orange used by the US military during the Vietnam War.
Carson called the government ’s claims about the concoction, “ flagrant propaganda. ” Gregory Bateson : \textbf{Mind} and Nature : Systems, complexity, co-evolution Film : \textbf{Ecology} of \textbf{Mind}, by \textbf{daughter} Nora Bateson ; excellent summary of Bateson ’s \textbf{work}.
Steps to an \textbf{Ecology} of \textbf{Mind} : collected essays Winona LaDuke : All Our Relations : Native Struggles for Land and Life Recovering the Sacred : The Power of \textbf{Naming} and Claiming G.
Sessions, \textbf{ed} : Deep \textbf{Ecology} for the 21st \textbf{Century} : A \textbf{good} collection of deep \textbf{ecology} essays : Arne Naess, Chellis Glendinning, Gary Snyder, Dolores LaChapelle, Paul Shepard, George Sessions, and \textbf{others}.
Arne Naess : The \textbf{Ecology} of Wisdom David Abram : Spell of the Sensuous Becoming Animal : An Earthly Cosmology When Carson attended government hearings, chemical industry lobbyists \textbf{attacked} her data and promoted their own \textbf{witnesses} to contradict her \textbf{research}.
However, \textbf{working} with medical researchers, Carson documented individual incidents of pesticide exposure, human health effects, and environmental impact.
In Silent Spring, Carson not only exposed the ecological impact of these toxins, but she called into \textbf{question} the entire paradigm of human progress. \textbf{Working} in the aerospace industry, I never knew precisely what I was \textbf{working} on, \textbf{everything} being top secret.
I began to see how science could go wrong, not only with the dangers of nuclear \textbf{weapons}, but by flooding our ecosystems with toxins.
I \textbf{returned} to \textbf{school} in engineering, but now harboured some burning \textbf{questions} about ethics, \textbf{ecology}, human health, and the role of science.
I was not, however, quite prepared for the next shock to my system.
In the \textbf{summer} of 1969, after my third \textbf{year} of \textbf{university}, I saw a \textbf{picture} in the Los Angeles Times, of firemen putting out a fire in Cleveland, Ohio.
However, this was not a typical building on fire, or forest fire.
I stared at the printed page.
What?
The river is burning?
How does a river burn?
The Cuyahoga River flows from the highlands northeast of Cleveland, moves \textbf{south} for about 80 kilometers, fed by \textbf{dozens} of tributaries, turns \textbf{north}, and flows 70 km through lush Cuyahoga Valley and Cleveland, into Lake Erie between the US and Canada.

% matched lemmas: activist, answer, anything, attack, attention, awareness, being, bomb, book, century, conversation, courage, daughter, day, dozen, earth, ecologist, ecology, editor, education, everything, executive, experience, family, fear, foot, friend, good, great, group, head, humanity, idea, information, job, journalist, language, lesson, look, lot, man, mind, mother, movement, music, name, north, nothing, one, organization, other, peace, people, picture, question, relationship, research, return, rock, school, sense, society, something, south, spring, start, story, student, summer, talk, test, thing, time, today, university, us, war, watch, way, weapon, whale, witness, word, work, world, year
\end{textsample}
