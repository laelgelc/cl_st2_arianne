\begin{textsample}{POS Dim 1 – human – Score 99.00 – t280\_human.txt}  \label{ex:f1_pos_003}
Two of the Greenpeace founders, Irving and Dorothy Stowe, grew up during \textbf{world} \textbf{wars}, the dawn of a global \textbf{peace} \textbf{movement}, and a non-violent direct action \textbf{movement} inspired by Mahatma Gandhi.
In the \textbf{spring} of 1960, protesters from A.J.
Muste ’s Campaign for Nonviolent Action \textbf{marched} from Boston to Groton, Connecticut, to protest US submarines designed to carry nuclear missiles.
When the marchers passed through Providence, Irving and Dorothy hosted them and continued with the \textbf{march} to the \textbf{boat} yard.
A small \textbf{group} paddled \textbf{boats} in \textbf{front} of launching submarines, \textbf{boarded} them, were arrested, and received 19-month jail sentences.
In 1961, when President Kennedy began sending US Marines to Vietnam, Irving and Dorothy decided to leave the US in protest against paying taxes for \textbf{war} and \textbf{weapons} \textbf{testing}.
They moved to New Zealand, where Irving joined the University of Auckland law faculty and Dorothy got a \textbf{position} as a social worker.
In New Zealand, they adopted the \textbf{name} “ Stowe, ” after nineteenth \textbf{century} American abolitionist, feminist, and \textbf{author}, Harriet Beecher Stowe, who praised Quaker pacifists in her \textbf{books}.
They led \textbf{peace} \textbf{marches} to the US consulate and attended Quaker \textbf{meetings}.
However, in 1965, under pressure from the US, New Zealand began sending troops to Vietnam, the Stowes felt outraged, and once again \textbf{looked} for a new home.
In April 1966, Irving stopped in Vancouver, Canada on his \textbf{way} back from Rhode Island.
The city was bathed in sun, fruit trees were in bloom, and the blue water and snow-capped mountains soothed his heart.
Canada was not sending troops to Vietnam and was quietly accepting American pacifists evading the military draft.
Irving and Dorothy decided to move their \textbf{family} — Bobby now 11, and Barbara 10 — to Canada.
They purchased a two-story, wooden home in the quiet neighborhood of Point Grey, near the University of British Columbia.
The Stowes were not naive ; they knew that Canadian companies mined uranium for American \textbf{weapons} and that some 40 Canadian \textbf{universities} provided military \textbf{research} to the \textbf{war} effort.
Irving discovered that Canadian contractors in Suffield, Alberta devised chemical and biological \textbf{weapons} for the US, including napalm and Agent Orange used in Vietnam.
There was nowhere to run.
They decided to remain in Canada and \textbf{work} for \textbf{peace}.
Irving, now 51 \textbf{years} old, told Dorothy that the \textbf{time} had come for him to be a full-time \textbf{activist}.
Dorothy found a \textbf{job} in \textbf{family} \textbf{services}, and Irving registered as a private estate planner, but he would dedicate the \textbf{rest} of his life to stopping the \textbf{bomb} and bringing about \textbf{peace}.
Their quiet home in Vancouver would soon become a hub of global significance.
Two \textbf{journalists} in Vancouver caught Irving ’s \textbf{attention}.
Bob Hunter wrote a column in the Vancouver Sun that addressed progressive issues including \textbf{peace}, civil rights, \textbf{women} ’s rights, and \textbf{ecology}.
Ben Metcalfe hosted a CBC naturalist show, “ Klahanie, ” featuring local \textbf{ecology} \textbf{stories}.
Irving began writing to both \textbf{journalists}, promoting \textbf{peace} \textbf{movement} events.
They met Quaker pacifists Jim and Marie Bohlen at an anti-war \textbf{demonstration} and became close \textbf{friends}.
The Stowes supported virtually every disarmament or ecological \textbf{campaign} in Vancouver.
They belonged to the Quaker Friends Service Committee, the World Peace Council, and the BC Environmental Coalition ( BCEC ).
With the Society to Advance Vancouver Environment ( SAVE ), they promoted a pedestrian mall for the city core, and with the Stop Pollution from Oil Spills ( SPOILS ) they protested oil tanker traffic in the Georgia Straight.
Dorothy kept track of the correspondence and replied to every offer of support.
She not only earned their \textbf{family} income, she also acted as secretary for several \textbf{groups}.
Her file cabinet was a sea of acronyms.
Then, in August 1969 the US announced a 1-megaton nuclear \textbf{bomb} \textbf{test}, “ Milrow, ” scheduled for October on Amchitka Island, 4,000 kilometers northwest of Vancouver in the Aleutian archipelago.
Irving and Dorothy led \textbf{peace} \textbf{marches} to the US embassy, where they met Deeno Birmingham from the BC Voice of Women and Lille d’Easum, who wrote about nuclear \textbf{radiation}.
Bob Hunter wrote a column about the threat of a tsunami caused by the detonation.
For a \textbf{demonstration} at the US border, Hunter made a sign that read : “ Don't Make a Wave : Stop the \textbf{Bomb}. ” The Stowes met Hunter at the \textbf{demonstration}, felt inspired to \textbf{start} an \textbf{organization} dedicated to stopping the US \textbf{bomb} \textbf{tests}.
However, the next \textbf{morning}, the US detonated the Milrow \textbf{bomb} \textbf{test} some 1,300 \textbf{meters} underground.
The \textbf{island} surface heaved five \textbf{meters} into the air.
Fish were blown from \textbf{lakes}, house-size chunks of granite fell into the sea, and nesting birds and sea otter carcasses floated in the shoreline foam.
The University of Victoria recorded a 6.9 Richter scale shockwave.
When the US announced a follow-up \textbf{test}, five-times as powerful, for the fall of 1971, the Stowes, Bob and Zoe Hunter, Ben and Dorothy Metcalfe, Birmingham, d’Easum, the Bohlens, and \textbf{other} \textbf{peace} \textbf{activists} met at the Stowe home to plan \textbf{strategy} to stop the blast.
Borrowing Hunter ’s slogan, they called the new \textbf{organization} “ The Don't Make A Wave Committee. ” In 1915, Irving Strasmich was \textbf{born} in Providence, Rhode Island, in a Talmudic Jewish \textbf{family} that emphasized worldly \textbf{education}.
In 1936, he graduated from Brown University in economics, and entered Yale law \textbf{school}.
He believed in Franklin Roosevelt ’s New Deal, \textbf{hope} for \textbf{working} \textbf{class} \textbf{people} within the structure of American capitalism.
Out of law \textbf{school}, he helped draft antitrust legislation, earned his pilot ’s license, and joined the US World War II effort, where he was sent to the Office of Strategic Services ( OSS ), President Roosevelt ’s precursor to the CIA.
Irving believed the US could help establish democratic \textbf{freedom} around the \textbf{world}.
However, he was so upset by the Hiroshima \textbf{bomb} that he broke down in tears in \textbf{front} of the White House, \textbf{returned} home, and served as a \textbf{lawyer} for the Rhode Island Tax Commission.
As attendance grew, they moved \textbf{meetings} to the Unitarian Church.
Marie Bohlen, borrowing a Quaker tactic, suggested, “ We should just sail a \textbf{boat} into the \textbf{test} zone. ” The \textbf{idea} appealed to \textbf{everyone}, and they decided to find a \textbf{boat}.
At the end of the \textbf{meeting}, Irving in his typical manner, flashed the “ V ” sign and said “ \textbf{peace}. ” Twenty-two year-old Bill Darnell who had organized an “ \textbf{Ecology} Caravan ” in the region, responded : “ Make it a Green \textbf{Peace}! ” Darnell ’s spontaneous slogan perfectly articulated the merging \textbf{peace} and \textbf{ecology} \textbf{movements}. \textbf{Group} \textbf{members} began to call the \textbf{campaign} “ Green Peace, ” and Marie Bohlen ’s \textbf{son}, Paul, designed a button with green lettering on a yellow \textbf{background}, with the \textbf{ecology} symbol, the \textbf{peace} symbol, and in the \textbf{middle}, the single \textbf{word} : “ Greenpeace. ” Stowe reckoned they would need about \$45,000 for a \textbf{boat} charter, fuel, and provisions for the voyage.
Twenty-five-cent buttons weren't going to do it, so Stowe decided to organize a benefit \textbf{music} concert.
He wrote to Joan Baez, whom he had met eight \textbf{years} earlier through the Committee for Nonviolent Action.
Baez could not perform, but she donated \$1,000, recommended Joni Mitchell and Phil Ochs and gave Stowe their \textbf{phone} numbers.
When both agreed to perform, Irving set a \textbf{date} for October 16, 1970.
Joni Mitchell invited rising star James Taylor, Canadian band Chilliwack joined the show, and the concert was a sellout.
Phil Ochs, the senior pacifist \textbf{artist} at 31, opened the show and spoke directly to the raison d’etre of the evening with his “ I Ain't Marchin ’ Anymore. ” Chilliwack got the crowd in a \textbf{good} \textbf{rock} and roll frenzy.
James Taylor stunned the crowd with the cryptic “ Carolina In My \textbf{Mind}, ” and “ Fire And Rain. ” Joni Mitchell ’s “ Chelsea Morning ” and “ Big Yellow Taxi, ” brought shrieks of joy from the crowd.
Irving Stowe, grinning and raising the \textbf{peace} sign, delivered flowers to Mitchell.
After all expenses were paid, the event netted over \$17,000.
By the end of October, the Don't Make A Wave Committee had \$23,467.02 in the bank, all accounts kept impeccably by Irving and Dorothy Stowe.
Crossing the Gulf of Alaska during the autumn storm season was dangerous.
Some mariners told them they were crazy.
Nevertheless, on the waterfront \textbf{south} of Vancouver, Jim Bohlen met fisherman John Cormack, who agreed to take them to Amchitka Island on his 63-foot fishing \textbf{boat}.
Irving Stowe was vulnerable to severe sea sickness, so he could not go.
Marie Bohlen had planned to join the \textbf{crew}, but a few \textbf{days} before departure, she decided it was too risky for both her and her husband Jim to leave their \textbf{children}, so she stepped down.
At dusk on September 15, 1971, the Phyllis Cormack, rechristened “ Greenpeace ” for the voyage departed Vancouver.
Irving, Dorothy, Marie \textbf{worked} with \textbf{other} \textbf{volunteers} from Vancouver.
Dorothy Metcalfe, a seasoned \textbf{journalist}, operated a marine radio link to the \textbf{boat}, released \textbf{stories} to the \textbf{media}, and pressured Canadian Prime Minister Pierre Trudeau to put pressure on the US government.
The voyage created massive public \textbf{attention}.
Robert and Barbara Stowe helped organize Vancouver \textbf{students} to walk out of \textbf{schools} in protest.
Irving and Dorothy achieved their \textbf{vision} of walking with business leaders and \textbf{union} leaders in a protest \textbf{march}. \textbf{Time} magazine reported : “ Seldom, if ever, had so many Canadians felt so deep a \textbf{sense} of resentment over a single US action. ” Thirty U.S. senators urged the Nixon government to halt the \textbf{test}.
On the \textbf{way} up the \textbf{coast}, the Kwakwa̱ka̱ʼwakw Indigenous community at Alert Bay, BC, invited the \textbf{crew} to a ceremony and blessed the voyage.
When they reached the Aleutian Islands, the US Coast Guard seized the vessel and sent it back to Sand Spit, Alaska to go through customs, delaying the \textbf{trip}.
On November 6, 1971, before the \textbf{boat} could reach Amchitka, the US detonated the 5.2-megaton \textbf{bomb}.
The blast created a molten cavern inside the \textbf{rock}, fissured the volcanic substrate, and blew a mile-wide crater in the \textbf{island} surface.
At first, the \textbf{crew} felt that they had failed, but when they \textbf{returned} to Vancouver, they were treated as heroes.
The voyage succeeded in capturing the imagination of \textbf{activists} around the \textbf{world}.
Opposition to the Amchitka \textbf{tests} became so intense that President Nixon cancelled the program on the \textbf{island} a \textbf{year} later.
Eventually, Amchitka became a bird sanctuary, and \textbf{one} of the most influential environmental \textbf{groups} of the twentieth \textbf{century} had established its original and highly visible protest tactic.
Dorothy Rabinowitz, was \textbf{born} in Providence in 1920.
Her \textbf{mother}, a Hebrew \textbf{teacher}, and her \textbf{father}, a jeweler, had immigrated to the US from Galicia, a once-independent region of central Europe torn by fighting during World War I.
As a young \textbf{woman}, Dorothy met distinguished Jewish political \textbf{activists} in her home, including the future first \textbf{president} of Israel, Chaim Weizmann.
She graduated from Pembroke College in English literature, became a purchasing \textbf{officer} for the US Navy during World War II, joined the Women ’s International League for Peace and Freedom, and became \textbf{president} of Rhode Island ’s first state employees \textbf{union}.
When she won a 33 % wage increase for social workers, Democratic governor John Pastore called her “ a communist. ” In May 1972, the \textbf{group} officially changed its \textbf{name} to The Greenpeace Foundation.
Hunter, Darnell, and \textbf{others} envisioned future \textbf{ecology} actions, which led to the 1975 \textbf{whale} \textbf{campaign} and eventually to a global Greenpeace \textbf{organization}.
On October 28, 1974, Irving Stowe passed away from pancreatic \textbf{cancer} at the \textbf{age} of fifty-nine.
His passing gave the \textbf{group} pause and reminded \textbf{people} : this \textbf{mission} is serious. “ No \textbf{one} could say that Irving wasted his \textbf{time} here, ” Bob Hunter wrote in his Vancouver Sun column. “ When \textbf{others} were waiting to see what nightmare would materialize next, Irving was moving like a human whirlwind toward the goal of \textbf{heading} the nightmare off. ” Dorothy Stowe continued to support \textbf{peace} and social justice for decades and got to \textbf{witness} the evolution of Greenpeace around the \textbf{world}.
In 2005, when Irish \textbf{rock} band U2 played Vancouver, singer Bono invited Dorothy to the show and dedicated the song “ Original of the Species ” to her.
On July 23, 2010, Dorothy, at the \textbf{age} of 89, passed away peacefully at UBC Hospital in Vancouver.
Irving and Dorothy Stowe were natural leaders, who, by their own example, inspired commitment in \textbf{others} and thus launched \textbf{one} of the \textbf{world} ’s most effective environmental and \textbf{peace} \textbf{organizations}.
Irving and Dorothy met in 1951, on a blind \textbf{date} at Brown University.
Irving played the violin, and frequented Providence ’s Celebrity Club, where he made recordings of Louis Armstrong and Duke Ellington.
The Black jazz musicians had invited the Jewish \textbf{lawyer} into the National Association for the Advancement of Colored \textbf{People}, the NAACP.
The \textbf{couple} married in 1953, with a Rabbi presiding and with renowned British jazz pianist George Shearing as \textbf{best} \textbf{man}.
They spent their wedding \textbf{night} at a benefit banquet for the NAACP.
Dorothy and Irving joined the Quaker Society of Friends, became committed pacifists, and protested nuclear \textbf{weapons}.
They adopted the Quaker \textbf{idea} of “ \textbf{bearing} \textbf{witness}, ” to injustice and speaking out to \textbf{others}.
Irving offered pro bono legal \textbf{services} to \textbf{citizens} called before Senator Joseph McCarthy ’s US House Un-American Activities Committee.
He argued convincingly that pacifism was not unpatriotic.
An avid letter \textbf{writer} and early riser, he could send off letters to President Eisenhower, to Secretary of Defense Charles Wilson, and to the \textbf{editors} of The New York Times all before breakfast.
In 1955, Irving and Dorothy Strasmich had a \textbf{son}, Robert, and the following \textbf{year} a \textbf{daughter}, Barbara.
From the example of Gandhi, they believed that \textbf{citizens} acting with integrity and \textbf{courage} could defeat powerful forces.
From the Quaker \textbf{tradition}, Irving adopted the \textbf{idea} of “ Replacing the Death culture with a Life culture, ” a rally cry he would champion for the \textbf{rest} of his life.
After the horrors of World War II, the global \textbf{peace} \textbf{movement} gained momentum.
In London, Bertrand Russell formed the Union of International Scientists opposed to nuclear \textbf{bomb} \textbf{tests}.
On his deathbed, Albert Einstein drafted a disarmament declaration signed by fifty-two Nobel laureates.
Meanwhile, Harvard biologist Dr.
Barry Commoner collected deciduous teeth from \textbf{children} in St.
Louis and discovered the presence of strontium-90, a radioactive by-product of nuclear explosions.
Strontium-90 in \textbf{children} ’s teeth was the sort of horrifying \textbf{image} that \textbf{people} could understand, and a threat to which they would respond.
The \textbf{world} ’s militaries had become the \textbf{world} ’s worst polluters.
The \textbf{peace} \textbf{movement}, the civil rights \textbf{movement}, and the \textbf{ecology} \textbf{movement} began to merge.
When US federal law made civil defense drills compulsory, Irving and Dorothy joined the War Resisters League by refusing to take shelter.
Irving declared that hiding from nuclear \textbf{attack} was a fraud, the arms \textbf{race} was a waste of resources, and \textbf{radiation} from \textbf{weapons} \textbf{testing} was killing \textbf{people}.
Irving and Dorothy emphasized that the disarmament \textbf{movement} integrated \textbf{class}, religions, and cultures, and that business leaders and \textbf{union} leaders stood side by side at \textbf{peace} \textbf{marches}.

% matched lemmas: activist, age, artist, attack, attention, author, background, bear, board, boat, bomb, book, campaign, cancer, century, child, citizen, class, coast, couple, courage, crew, date, daughter, day, demonstration, ecology, editor, education, everyone, family, father, freedom, friend, front, good, group, head, hope, idea, image, island, job, journalist, lake, lawyer, look, man, march, medium, meeting, member, meter, middle, mind, mission, morning, mother, movement, music, name, night, officer, one, organization, other, peace, people, phone, position, president, race, radiation, research, rest, return, rock, school, sense, service, son, south, spring, start, story, strategy, student, teacher, test, testing, time, tradition, trip, union, university, vision, volunteer, war, way, weapon, whale, witness, woman, word, work, world, writer, year
\end{textsample}
