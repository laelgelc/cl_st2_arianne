\begin{textsample}{POS Dim 1 – human – Score 99.00 – t998\_human.txt}  \label{ex:f1_pos_002}
In Vancouver, on Canada ’s Pacific \textbf{coast}, Greenpeace set off on a voyage in 1972 which is still continuing.
At “ English Bay ” the most successful environmental \textbf{organization} in the \textbf{world} was launched by just a \textbf{dozen} \textbf{women} and \textbf{men}.
Their principles – non-violence and direct action – have been followed by Greenpeace the \textbf{world} over right up to the present \textbf{day}.
And those pioneers from the early \textbf{days} : Dorothy Stowe, widow of Irving Stowe, Dorothy Metcalfe, Jim Bohlen, and Bob Hunter were proud of what they achieved when they gathered, in 1996, to mark the 25-year anniversary of that first voyage.
Here ’s how these first-generation Rainbow \textbf{Warriors} described those early \textbf{days} : Dorothy Stowe : Of \textbf{course} you ’re \textbf{talking} about how it all began.
Actually, it all \textbf{started} to happen in this \textbf{house}.
I can still see Irving sitting on the bed with the telephone in his \textbf{hand}, and \textbf{someone} is telling him about the atomic \textbf{tests} about to be held on Amchitka.
They ’re telling him that the Aleutian Islands are an important habitat for sea-otters, and that they are jeopardized by the \textbf{tests} because their eardrums are in danger of bursting as a result of the explosion.
The very \textbf{idea} of this outraged Irving just as much as the atomic \textbf{tests} themselves.
So he called Jim.
Bob : “ Jim, do \textbf{something}! ” Jim : The \textbf{rest} is \textbf{history}.
Dorothy Stowe : Not yet.
Jim : Irving called me because I was \textbf{head} of the “ Sierra Club ” of British Columbia.
But our \textbf{head} \textbf{office} in San Francisco didn't want to run a \textbf{campaign} against the nuclear \textbf{tests}.
So we – Paul Cote, Irving Stowe and myself, together with our wives – set up a splinter \textbf{group}, the “ Don't Make a Wave Committee ”, the germ of Greenpeace.
Dorothy Stowe : The Bohlens and the Stowes had already been active in the \textbf{peace} \textbf{movement} for a long \textbf{time}.
Jim : And Paul Cote had been present at the first atomic \textbf{test} protests in 1969 on the border between the USA and Canada as well.
Bob had reported on it and had been in contact with \textbf{us} since then.
In spite of that we had to \textbf{work} hard on him to persuade him to come along to Amchitka.
Bob : Well, after all I had to write a column for the “ Vancouver Sun ” every \textbf{day}.
Jim : A hippie column.
But Bob was very important for \textbf{us} as a \textbf{media} \textbf{man}.
Out of a \textbf{crew} of twelve, half were \textbf{journalists}.
We did a \textbf{lot} for the \textbf{media} from the very \textbf{beginning}, for example for the Canadian Broadcasting Company, CBC.
We were almost out of \textbf{port} when their camera \textbf{team} showed up late.
What did we do?
We turned around and cast off again.
Dorothy Stowe : Irving wisely decided to stay at home.
He already had \textbf{cancer} at that \textbf{time}, which nobody knew, and he died just three \textbf{years} later.
Bob : My goodness, Jim.
I last saw you five \textbf{years} ago, at the 20-year celebrations.
Is it really you?
Jim : On shore he \textbf{worked} like crazy, collecting \textbf{money} and hanging on to every last cent.
Dorothy Stowe : He wrote \textbf{everything} down in detail. “ Collection box : 6 dollars 41 cents ”.
Later we even found receipts from the \textbf{post} \textbf{office} for 37 cents.
Jim : We were really short of \textbf{money} at that \textbf{time}.
We financed ourselves from donations, sold Greenpeace buttons on busy \textbf{street} corners for 25 cents each and Greenpeace T-shirts for three dollars.
Bob : Then Irving had the \textbf{idea} of a solidarity concert to finance the \textbf{trip} to Amchitka.
Three dollars a ticket for a concert with Joni Mitchell.
Dorothy Stowe : That had its funny side : a few \textbf{days} before the concert the \textbf{phone} rang.
It ’s Joni Mitchell on the line from Los Angeles.
Suddenly Irving puts his \textbf{hand} over the mouthpiece and hisses across to \textbf{us}, “ \textbf{Anyone} of you know who James Taylor is? ”.
Nobody knew him.
My \textbf{daughter} shouted to him, “ God, dad.
That ’s that \textbf{black} blues singer. ” She ’d mixed him up with James Brown.
Irving was still at a loss. “ What am I going to do?
She wants him to be at the concert with her.
Is he \textbf{good}? ”, he asked Barbara.
And James Taylor ’s new album was just at the top of the charts.
We hadn't noticed any of this because we ’d been so busy getting \textbf{things} ready for the \textbf{trip}.
Jim : We even haggled for every item on the list of provisions until a dealer gave \textbf{us} \textbf{everything} for free.
We had ice-boxes full of steaks.
Bob, do you remember how we later toyed with the \textbf{idea} of going on hunger strike?
Bob : You bet!
As I recall, Captain John Cormack was very enthusiastic about the \textbf{idea} ; if we all starved to death he would have more to eat.
Jim : That was a \textbf{good} \textbf{reason} not to go ahead with it.
Dorothy Metcalfe : I was a \textbf{kind} of thirteenth \textbf{crew} \textbf{member} on shore.
I supplied the \textbf{media} with \textbf{news} from on \textbf{board} the Greenpeace.
At the height of the \textbf{campaign} I didn't leave the \textbf{house} for 15 \textbf{days} on end to make sure I didn't miss any radio messages.
Jim : If it hadn't been for you wewould have been in serious difficulties.
In those \textbf{days} you couldn't transmit from the ship direct.
You were our relay \textbf{station} and did some fantastic \textbf{work} to make sure that our \textbf{stories} were sorted out and edited before they reached the \textbf{media}.
Jim : You \textbf{’d} better \textbf{watch} out! – \textbf{may} have got a \textbf{lot} older but I ’m still the same old Jim.
Dorothy Metcalfe : Well, we had to be absolutely reliable to make sure the press believed our reports.
Jim : In spite of that some \textbf{people} claimed that the first \textbf{trip} was a failure.
Dorothy Metcalfe : If it had been, the Greenpeace that we know \textbf{today} would not exist.
Dorothy Stowe : Don't forget that seven \textbf{tests} had been planned for Amchitka and only three were actually carried out.
The US Government had to admit that the \textbf{others} were canceled as a result of public pressure.
Later they decared Amchitka a nature reserve.
Jim : For \textbf{us} it was a sign of \textbf{hope} that \textbf{people} can change \textbf{things}.
And our action gave the entire \textbf{ecology} \textbf{movement} a new \textbf{name} : Green.
That was \textbf{better} than ecology- a \textbf{word} hardly \textbf{anyone} understood.
Dorothy Stowe : I remember how we tried to think of a \textbf{name} for the ship.
And then Bill Darnell came up with the combination of “ Green ” and “ \textbf{Peace} ”.
Jim : We were aware that no-one would be interested in the fate of a Phyllis Cormack. \textbf{Someone} said that the \textbf{word} “ Green ” would have to appear in it somewhere.
Irving said the \textbf{word} “ Peace ” was more important.
In response to this, Bill, later our ship ’s cook on the Phyllis Cormack, threw in his famous suggestion.
Then, when my \textbf{son} Paul designed the first button he had real problems trying to get the \textbf{words} Green and Peace on it as two \textbf{words}.
So I said that he should write it as \textbf{one} \textbf{word} : GREENPEACE.
Dorothy Metcalfe : The secret of our \textbf{success} was being cheeky. \textbf{Everyone} was amazed : how dare they? \textbf{Attacking} governments and demanding the end of atomic \textbf{tests}.
That was sensational – David versus Goliath.
Only not \textbf{everyone} was on David ’s side : while you were on your \textbf{trip} I was a \textbf{guest} on various \textbf{talk} shows.
And there \textbf{people} would call up to say they \textbf{hoped} that this bunch of hippies would all drown.
That was hard to handle.
Exactly at that \textbf{time} the ship was \textbf{battling} through a storm with waves ten \textbf{meters} high.
Jim : Apart from Cormack we all threw up.
It wasn't an adventure – it was a serious undertaking in every respect. \textbf{Everyone} had had to take six \textbf{weeks} off \textbf{work} for the \textbf{trip}.
They even wanted to fire me as I was \textbf{working} for the government.
Dorothy Stowe : \textbf{Others} put their own \textbf{money} into the project.
Irving, for instance, completely gave up his \textbf{job} as a highly qualified \textbf{lawyer} specializing in marine law.
I supported the \textbf{family} from my salary as a therapist.
Bob : Well, you didn't recognize me straight off either.
Jim : Not enough \textbf{people} know just what an important part the \textbf{women} played in all this.
Greenpeace would probably never have been so successful if Dorothy hadn't made it possible for Irving to devote all his energy to the cause.
And without Irving ’s commitment a \textbf{lot} would have been left undone.
Dorothy Metcalfe : We were just a handful of \textbf{people} from different \textbf{backgrounds}, but on \textbf{one} \textbf{thing} we agreed – this planet is in danger.
Jim : I ’m still surprised to \textbf{day} that we found a \textbf{job} for every talent – and a talent for every \textbf{job}.
Bob : Take Paul Spong, the well-known marine biologist.
He approached \textbf{us} inorder to use our \textbf{good} \textbf{name} for protecting \textbf{whales}.
That ’s how we came to take up the subject of \textbf{whales}.
And David McTaggart was another \textbf{man} in the right place at the right \textbf{time}.
Jim : Or Dorothy ’s ex-husband Ben Metcalfee, a television \textbf{journalist} who joined in the first voyage of the Greenpeace as a \textbf{media} observer.
Like you, Bob, he “ mutated ” into an \textbf{activist} and became \textbf{head} of our press \textbf{office}.
In 1972, when he was chairman of the association, he wanted to do \textbf{something} against French nuclear \textbf{tests} and was \textbf{looking} for \textbf{people} to join him in New Zealand.
That ’s how we fell in with David McTaggart with his yacht, Vega.
We were worried because nobody knew \textbf{anything} about the guy.
Ben just said, “ we ’ll give the \textbf{man} a radio transmitting set and a few hundred dollars and we ’ve already got a \textbf{campaign}. ” We thought, “ OK, what have we got to lose? ” Dorothy Metcalfe : In those \textbf{days} McTaggart was less bothered about the French \textbf{testing} atomic \textbf{bombs} on Moruroa than the \textbf{fact} that they were blocking off a huge area of sea although they were only entitled to the twelve-mile limit.
He wasn't interested in environmental matters.
But he was out for adventure and realized that Greenpeace offered a \textbf{platform} for getting \textbf{something} meaningful done.
Jim : It wasn't until later that he became a convinced \textbf{environmentalist}, after the French had given him such a bad beating.
That was their mistake.
Bob : Yes, on the Vega ’s second voyage to Moruroa, the \textbf{crew} was really given a roughing up by the French.
But David ’s girlfriend, Anne-Marie Horne, managed to take some photos of it.
She smuggled the \textbf{film} off the ship in her vagina and took it to Vancouver, where we developed it and immediately realized what we had got hold of.
At the \textbf{time} David was still in hospital.
Jim : We \textbf{attacked} the French for their orgy of violence.
The government in Paris claimed that David had slipped up and got his bruises and \textbf{eye} injury from that.
Bob : Only then did we publish the photos.
It was a complete knock-out.
Jim : Right.
I thought to myself, “ Who is that guy!? ” Dorothy Metcalfe : After 1974 the direction Greenpeace took changed so much that many of the old campaigners no longer wanted to follow.
Instead of fighting against the atomic threat they took up the cause of protecting seals and \textbf{whales}.
Dorothy Stowe : At that \textbf{time} Irving no longer had the energy to stand up against this development.
Jim : And I moved out into the country.
Sold my \textbf{house} here in the city and built up a farm to do \textbf{research} into new \textbf{ways} of living self-sufficiently as far as energy was concerned.
We called it the”Greenpeace Experimental Farm ”. \textbf{Watching} from the outside, I thought that the whole outfit would fold.
Bob : And there were a \textbf{lot} of fights : Ben and David hated each \textbf{other}.
David felt that he had been left in the lurch by Greenpeace in his legal \textbf{battle} against France.
Bob : From 1975 onwards we had a few awful \textbf{years} – \textbf{nothing} but in-fighting.
Jim : In those \textbf{days} though there were some pretty strong characters rubbing each \textbf{other} up the wrong \textbf{way}.
What was to be the future of the \textbf{organization}? \textbf{Opinions} on this differed widely.
The direction Greenpeace should take has always been worth arguing about.
Bob : Did you know that for David McTaggart the \textbf{history} of Greenpeace doesn't \textbf{start} until Greenpeace International was founded in 1979?
Jim : The founders of Greenpeace are three \textbf{people}.
Or the twelve who risked their asses on the first voyage in 1971.
When David got a prize as the “ Greenpeace Founder ” in Mexico City I was absolutely fuming.
Bob : Sometimes I think it ’s a miracle that Greenpeace has survived all the fights.
Jim : So it ’s true what they say : “ You can't sink a \textbf{rainbow}! ” Whenever we were in a bad \textbf{way} and had no \textbf{money} left, some government would make a mistake and that would put \textbf{us} on our \textbf{feet} again.
Ultimately the \textbf{history} of Greenpeace is based on a \textbf{lot} of coincidences.
Bob : Appearances change, but then your character gradually \textbf{starts} to form.
Before we completely disappear we grow again to our \textbf{greatest} size … like a supernova.
Bob : Like in 1975. \textbf{Everything} was very much in the balance at that \textbf{time}.
We were broke and urgently had to pay a whole load of invoices.
We were completely desperate because \textbf{someone} had run off with 8000 dollars from a concert.
Then I come into the \textbf{office} in the evening and there ’s this brown paper bag on the desk.
A \textbf{man} with terminal \textbf{cancer} had given \textbf{us} a donation.
A whole bag full of ten and five dollar bills.
Only 50 dollars short of the exact amount we needed to pay off our debts.
And this \textbf{kind} of \textbf{thing} didn't happen just once.
We used to call it “ cosmic accounting ”.
Jim : A \textbf{lot} of \textbf{people} tended to depend too much on this \textbf{kind} of \textbf{thing}.
Correct accounting or precise planning were alien to them.
Bob : I have always fought for \textbf{us} to have a certain amount of bureaucracy.
Decision making structures and such like.
The hippie faction thought I was completely gaga.
My \textbf{answer} was : if we carry on in this chaotic fashion \textbf{one} \textbf{day} we \textbf{will} be completely burnt out.
But the government and multi-nationals ’ bureaucracies \textbf{will} last forever.
That ’s why we have to create a bureaucratic machinery with enough \textbf{strength} and staying power to fight against the \textbf{other} big bureaucratic machines.
Jim : You have to fight fire with fire.
Every step is \textbf{history} : in 1969 the small \textbf{park} on English Bay in Vancouver was to be paved over for a shore-side road.
Jim, Bob and Irving Stowe got in the \textbf{way} – in \textbf{front} of the bulldozers.
Bob : For a \textbf{time} though, my biggest \textbf{fear} was that I would die in my boots at a Board \textbf{meeting}.
But then you just can't get 30 countries co-ordinated just like that.
Jim : You can only combat the big multi-nationals internationally.
And that ’s why in 25 \textbf{years} Greenpeace should have its own \textbf{office} in every country in the \textbf{world}.
Bob : Just think of China in 50 \textbf{years}.
Greenpeace could play a major role in discussions on the environment there.
Or in India.
Jim : It always seems to me like \textbf{watching} your own kid grow up.
Greenpeace was and is our baby.
And we have \textbf{worked} hard to bring it up.
Bob : But it really has got pretty big, hasn't it Jim?
Jim : But for its old folks a kid \textbf{will} always be a kid.
When \textbf{someone} from Greenpeace calls me up \textbf{today} I react in the same \textbf{way} as with my real kids.
The first \textbf{thing} I ask is, “ Is \textbf{everything} OK? ” Jim : Charming.
After all it ’s \textbf{better} to \textbf{look} forward to the apocalypse than to slow decay.
By the \textbf{way}, I ’m just reading the \textbf{book} you ’re carrying in your pocket there, “ Rogue Primate ”.
From a vast number of possible \textbf{books} we have both chosen the same \textbf{one}.
Once aligned and we ’re still transmitting on the same wave length.
Bob : The emotional tie to this outfit is really strong.
I \textbf{experienced} \textbf{one} of the finest moments in my life in 1976 in James Bay.
We were standing on the bridge of our ship \textbf{watching} the Russian whaling fleet running away from \textbf{us}, and I though, “ Wow.
We ’ve got you. ” A wonderful moment.
Jim : The \textbf{best} \textbf{thing} that has happened to me was meeting my second wife.
Bob : \textbf{Good} Lord!
I forgot to mention my wife.
Jim : If she finds that out … Bob : The \textbf{thought} of being a co-founder of Greenpeace just goes beyond what my \textbf{mind} can handle.
I was in the right place at the right \textbf{time}.
When my last \textbf{hour} comes, I ’ll be able to say to myself, “ You didn't waste your life away meaninglessly. ” Jim : For me at any rate, Greenpeace was the crowning achievement of my \textbf{working} life.
Determining \textbf{everything} yourself, doing \textbf{everything} yourself – for yourself and for \textbf{others}.
I only wonder why so few \textbf{people} listen to Greenpeace \textbf{today}.
We \textbf{talk} about the dying planet, about the Greenhouse effect, the ozone hole – and nobody really listens.
Only when it ’s too late do they say that we were right.
In my \textbf{opinion}, Greenpeace has to become more militant.
Not in the \textbf{sense} of sinking ships.
We must be less willing to compromise on our demands.
Bob : While we ’re on the subject of criticism : I think that Greenpeace has always been too ashamed of its spiritual side. \textbf{Anyone} who has \textbf{looked} a \textbf{whale} in the \textbf{eye} knows just how much more \textbf{Man} should feel at \textbf{one} with nature.
But anyway we did give the outfit its main tools – non-violent action and \textbf{media} \textbf{work}.
It surprises even me just how important a feature of Greenpeace this still is \textbf{today}.
And there is \textbf{one} benefit for me in all this.
Dean, the barkeeper in my local bar, asks for \textbf{one} dollar from me instead of two and a half, for the \textbf{rest} of my life.
Because he doesn't know \textbf{anyone} else who has founded a world-wide \textbf{organization}.
That ’s \textbf{something} isn't it?
Michael Friedrich is a former \textbf{editor} of Greenpeace Magazine, Germany.
Bob : Not just aligned.
Welded together, on the Greenpeace alias the Phyllis Cormack.
Jim : You mean you ’ve got your armpit in my nose just like I ’ve got yours?
Even \textbf{today} I could recognize \textbf{everyone} on \textbf{board} by their smell – we lived that close together on \textbf{board}.

% matched lemmas: activist, answer, anyone, anything, attack, background, battle, beginning, black, board, bomb, book, campaign, cancer, coast, course, crew, daughter, day, dozen, ecology, editor, environmentalist, everyone, everything, experience, eye, fact, family, fear, film, foot, front, good, great, group, guest, hand, head, history, hope, hour, house, idea, job, journalist, kind, lawyer, look, lot, man, may, medium, meeting, member, meter, mind, money, movement, name, news, nothing, office, one, opinion, organization, other, park, peace, people, phone, platform, port, post, rainbow, reason, research, rest, sense, someone, something, son, start, station, story, street, strength, success, talk, team, test, testing, thing, thought, time, today, trip, us, warrior, watch, way, week, whale, will, woman, word, work, world, year
\end{textsample}
