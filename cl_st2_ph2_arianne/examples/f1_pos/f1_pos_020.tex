\begin{textsample}{POS Dim 1 – human – Score 75.00 – t495\_human.txt}  \label{ex:f1_pos_020}
Greenpeace cofounder Dorothy Metcalfe passed away on December 10.
Dorothy operated the radio link, connecting the \textbf{boats} to international \textbf{media}, during the first two Greenpeace \textbf{campaigns} to stop nuclear \textbf{testing} in 1971 and 1972.
She waged a \textbf{media} \textbf{battle} with Canadian Prime Minister Pierre Trudeau and organized an audience with Pope Paul VI.
She was a seasoned \textbf{journalist} and a creative campaigner, who knew how to capture and hold the interest of the \textbf{media} and public.
The sophisticated \textbf{media} operation became a fundamental \textbf{difference} between Greenpeace and earlier protest \textbf{boats} such as the Quaker Golden Rule, and Dorothy Metcalfe provided the hub of that \textbf{media} \textbf{campaign}.
The US detonated the \textbf{bomb} \textbf{test} in 1971, but then cancelled all future \textbf{tests} due to the opposition.
After the \textbf{success} of the Amchitka \textbf{campaign} a Canadian reporter accused the Metcalfes of being “ anti-American, ” and claimed that they would never mount a similar \textbf{campaign} against French nuclear \textbf{testing}. “ \textbf{Good} \textbf{idea}, ” Dorothy told Ben, and within \textbf{days}, they formulated a \textbf{campaign} against the French \textbf{tests} on Moruroa atoll in the South Pacific.
A popular French \textbf{film} at that \textbf{time}, Alain Resnais ’s 1959 Hiroshima Mon Amour, told a \textbf{love} \textbf{story} set against the tragedy of the Hiroshima nuclear explosion.
Dorothy called their \textbf{campaign} “ Mururoa Mon Amour, ” which immediately struck a nerve in France. ( At the \textbf{time}, Greenpeace used the French colonial misspelling “ Mururoa, ” which persisted on most maps. ) Ben and Dorothy ran ads in New Zealand and Australia to find a \textbf{boat} and skipper who would saild to the French \textbf{test} \textbf{site}, which led David McTaggart to Greenpeace.
They began a letter-writing \textbf{campaign} directed at French President Pompidou and arranged to attend the 1972 UN Conference on the Human Environment, in Stockholm, to get atmospheric nuclear \textbf{testing} on the agenda.
To put more pressure on predominantly Catholic France, Dorothy arranged an audience with Pope Paul VI in Rome.
She received a reply from the Canadian Archbishop, saying the Pope would see them at the Vatican in June.
Ben and Dorothy Metcalfe and their assistant Madeleine Reid arrived in Paris at the end of May.
They made their \textbf{way} along the Seine to meet Greenpeace \textbf{activist} Rod Marining for a planned protest at Notre Dame Cathedral.
At Quai St.
Michel, French security agents disguised as hippies surrounded them, demanded their passports, and placed the three Canadians in a decrepit-looking Citroen.
Exotic \textbf{communications} gear had been built into the car and the dashboard \textbf{looked} like the cockpit of a jet airplane.
Inside a windowless \textbf{room} at the Securité National headquarters, an agent told them they would be deported back to Canada. “ Non! ” said Dorothy Metcalfe defiantly. “ You can't. ” “ Pourquoi? ” Dorothy reached into her handbag and produced the Vatican cable. “ We have an audience with the Pope, ” she insisted.
The agent grabbed the cable.
There followed a \textbf{great} deal of stomping back and forth from the adjoining \textbf{room}, voices on the telephone, and finally the \textbf{officer} in \textbf{charge} said, “ Fine, we ’re deporting you to Italy. ” “ We have to get our luggage, ” Dorothy insisted.
The \textbf{officer} said he would arrange to collect their \textbf{things} from the hotel, but Dorothy refused the offer. “ It ’s personal. ” More stomping and \textbf{phone} calls.
Reluctantly, the agents allowed Dorothy and Madeleine to \textbf{return} to their hotel to retrieve their own luggage, but they held Ben Metcalfe in custody.
As Dorothy left, her \textbf{eyes} met her husband ’s and a faint smile crossed her face.
On their \textbf{way} to the Left Bank hotel, Dorothy and Madeleine stopped at the Reuters \textbf{office}, reported their arrest and deportation, and left \textbf{information} about the Moruroa \textbf{campaign}.
At the hotel, Dorothy called their Canadian \textbf{friend} Lyle Thurston in Rome and told him they would meet him on the Spanish Steps the following \textbf{day}.
The two \textbf{women} \textbf{returned} to the Securité National \textbf{office}, where a young agent, assigned to escort them to Rome, ushered them into a cab.
He carried a thin briefcase and appeared nervous in his new trench coat.
Dorothy was \textbf{born} April 16, 1931, in Winnipeg, Canada.
Shortly after her \textbf{birth}, her Ukrainian-Polish \textbf{parents} changed the \textbf{family} \textbf{name} from Hrushka to “ Harris, ” to fit into Canadian \textbf{society}.
She grew up through the depression, alcohol prohibition, and World War II.
Dorothy \textbf{loved} \textbf{history} and literature, and became a \textbf{journalist} for the Winnipeg Tribune.
At a gathering of Winnipeg reporters, drinking bootleg whiskey, she met her future husband Ben Metcalfe.
In the 1950s, Dorothy and Ben travelled Europe, filing \textbf{stories} for the North America Newspaper Alliance.
Dorothy gave \textbf{birth} to their first \textbf{child}, \textbf{daughter} Michelle, in London.
In the cab, the Metcalfes spoke in French with the agent, which seemed to relax him.
When they arrived at Gare \textbf{de} Lyon, a Reuters \textbf{photographer} stood waiting at the \textbf{train} \textbf{platform}.
The agent threw up his \textbf{hands}. “ Non!
Non! ” he protested, but it was too late.
The \textbf{photographer} weaved and crouched, clicking his shutter.
A crowd gathered to see the celebrities.
The Canadians waved and smiled.
The agent attempted to hustle them onto the \textbf{train}, but Dorothy and Madeleine took their \textbf{time} and chatted with the crowd. “ Madame.
Madame, ” the agent pleaded. “ S’il vous plaît.
Please. ” Once aboard, Ben and Dorothy opened the windows and waved as the \textbf{train} pulled out.
The UPI \textbf{photographer} took more \textbf{pictures}.
Photographs of Madeleine, Ben, and Dorothy circulated on the wire \textbf{services} with the \textbf{story}.
The three Canadians \textbf{looked} like international jewel thieves, well dressed, suave, but in custody.
The flustered agent appeared to be on his first big assignment, discreetly whispering to the conductors.
A youthful waiter in the dining car heard they were from Greenpeace, shook their \textbf{hands}, and \textbf{returned} with a free round of cognac for the four travellers.
The agent enjoyed his cognac, so Ben bought another round.
Then the agent felt compelled to buy a round.
His confidence bolstered, he now appeared pleased to be escorting such famous villains.
The drinking continued through Dijon, the Alps, and into Torino.
Ben and Dorothy kept the agent entertained with \textbf{stories} of the Canadian prairies and Europe after World War II.
The young \textbf{man} \textbf{shared} \textbf{stories} of his boyhood in the French countryside.
Ben and Dorothy kept ordering cognac.
Madeleine pasted a Greenpeace “ Mururoa Mon Amour ” sticker onto the agent ’s briefcase.
Ben and Dorothy explained to him the horror of nuclear \textbf{weapons} and \textbf{radiation}.
The young \textbf{man} defended France ’s right to protect itself, but soon agreed that nuclear \textbf{bombs} might not be the \textbf{best} solution to \textbf{world} problems.
The agent fought off sleep as they roared \textbf{south} toward Rome.
When they arrived at the Rome terminal, sixteen \textbf{hours} from Paris, the \textbf{story} of their deportation had appeared in the international newspapers.
The French agent stepped gingerly along the \textbf{platform}, pale and appearing queasy.
Madeleine and the Metcalfes \textbf{headed} off to see the Pope and left him in the \textbf{train} \textbf{station} with the “ Mururoa Mon Amour ” sticker still on his briefcase.
Dorothy Metcalf was a skilled campaigner, who knew how to create public interest, and who was fearless in the face of governments and corporations.
There \textbf{will} be a Celebration of Life for Dorothy on Tuesday, February 18 at the Ferry Building Gallery in West Vancouver, 5 to 7 pm.
In 1954, when the US detonated a massive thermonuclear \textbf{bomb} \textbf{test} in the Pacific ocean, spreading radioactive fallout around the globe, Dorothy and Ben became dedicated \textbf{peace} \textbf{activists}.
Back in Canada, Ben found \textbf{work} at The Province newspaper in Vancouver.
Their \textbf{son} Michael was \textbf{born} in West Vancouver in 1956 and Christopher two \textbf{years} later.
Dorothy continued her \textbf{peace} advocacy from home, often doing \textbf{research} for Ben ’s newspaper \textbf{stories} and for his CBC nature show, Klahanie.
In 1969, to promote the \textbf{idea} of \textbf{ecology}, Ben and Dorothy spent \$4,000 ( about \$20,000 in 2020 Canadian dollars ), to place twelve billboards around Vancouver, proclaiming : “ \textbf{Ecology}? \textbf{Look} it up!
You ’re involved. ” In August 1969, when the US announced a nuclear \textbf{bomb} \textbf{test} for Amchitka Island in Alaska, Dorothy and Ben joined forces with Irving and Dorothy Stowe, Bob and Zoe Hunter, Jim and Marie Bohlen, Bill Darnell, and \textbf{others} to launch a \textbf{campaign} to stop the \textbf{test}.
This small \textbf{group} of Canadians and expatriate Americans became Greenpeace.
Borrowing a Quaker tactic, the \textbf{group} decided to send a small fish \textbf{boat} into the US nuclear \textbf{test} zone.
Ben Metcalfe joined the \textbf{crew}, and relayed \textbf{news} \textbf{stories} back to Dorothy, who had set up a radio and wire \textbf{service} in her home.
Dorothy recorded radio calls from the \textbf{boat} and passed them along to Canadian and international \textbf{journalists}.
Dorothy ’s political savvy and \textbf{journalist} network made the \textbf{campaign} a huge \textbf{story} across Canada and the US.
Over the radio, Dorothy told Ben. “ Nixon ’s under pressure from his own \textbf{party}.
Gravel and Inouye in the Senate are opposed.
Hirohito is going to meet with Nixon in Anchorage this \textbf{month} and he ’s not in favor of the \textbf{bomb}.
No \textbf{one} wants to come out in favour of the \textbf{bomb}.
It ’s going to come down to the Supreme Court. ” Since the US Air Force was attempting to keep track of the Greenpeace vessel, Dorothy remained in daily contact with the US Coast Guard and learned that a directive had gone out to pilots : \textbf{SEARCH} CAN VES GREEN \textbf{PEACE}.
Dorothy Metcalfe gave the \textbf{coast} guard the ship ’s \textbf{position} and told them : “ We have no secrets.
Our ship is \textbf{heading} for Amchitka Island to stop this nuclear \textbf{test}. ” Dorothy wrote personally to Canadian Prime Minister Pierre Trudeau, insisting that he urge the Americans halt the \textbf{test}.
When he took no action, Dorothy chastised him in the press for being “ cowardly. ” She sent Trudeau a personal message through the \textbf{media} : “ From the wives and \textbf{families} of the \textbf{men} on \textbf{board} the Greenpeace.
Our \textbf{men} are risking their lives for the benefit of all mankind. ” When some Canadian supporters rebuked her for calling the Prime Minister a coward, she told them, “ This is a democracy. \textbf{People} have a responsibility to speak their \textbf{minds}. ”

% matched lemmas: activist, battle, bear, birth, board, boat, bomb, campaign, charge, child, coast, communication, crew, daughter, day, de, difference, ecology, eye, family, film, friend, good, great, group, hand, head, history, hour, idea, information, journalist, look, love, man, medium, mind, month, name, news, office, officer, one, other, parent, party, peace, people, phone, photographer, picture, platform, position, radiation, research, return, room, search, service, share, site, society, son, south, station, story, success, test, testing, thing, time, train, way, weapon, will, woman, work, world, year
\end{textsample}
