\begin{textsample}{POS Dim 1 – human – Score 82.00 – t995\_human.txt}  \label{ex:f1_pos_012}
In July 2010, two Greenpeace founders – Jim Bohlen, 84, and Dorothy Stowe, 89 – passed away.
Both lived full lives as agents of social change, that leave \textbf{us} much to ponder and to emulate.
Jim Bohlen, who passed away on July 5, 2010, served in the U.S.
Navy at the end of World War II, and \textbf{witnessed} the destruction of Hiroshima, which influenced his life-long commitment to pacifism.
He met Marie at a Quaker \textbf{peace} \textbf{march} in Pennsylvania in 1958 ; they married and immigrated to Canada in 1967 to keep Marie ’s \textbf{son} Paul out of the Vietnam War.
In Vancouver, Jim helped form the Committee to Aid War Objectors.
When I arrived in Vancouver in 1972, as a draft resister from the U.S., my wife and I slept in a shelter provided by this \textbf{group}.
Jim and Marie Bohlen attended an End the Arms Race rally in Vancouver, where they first met Irving and Dorothy Stowe.
After the U.S. announced a series of nuclear \textbf{tests} on Amchitka Island in Alaska, the Bohlen ’s and Stowes wanted to do \textbf{something} to stop these \textbf{tests}.
Meanwhile, Bob Hunter wrote in his newspaper column that the underground nuclear \textbf{tests} could cause a tsunami.
For a \textbf{demonstration} at the U.S. embassy, Hunter made a placard that read : “ Don't Make a Wave, ” in reference to the tsunami threat.
The Bohlen ’s and Stowes used this \textbf{image} to \textbf{name} their new \textbf{organization} “ The Don't Make a Wave Committee. ” Jim Bohlen told Marie that he felt frustrated because the \textbf{group} had no plan to disrupt the \textbf{bomb} \textbf{tests}, and Marie, borrowing an \textbf{idea} from the Quakers, said, “ Why not sail a \textbf{boat} up there? ” By happenstance, a reporter \textbf{phoned} that \textbf{day} and asked what the \textbf{group} might be planning, and Jim Bohlen said, “ We \textbf{hope} to sail a \textbf{boat} to Amchitka Island. ” The \textbf{story} ran the next \textbf{day}, although the \textbf{group} had no \textbf{boat} and no \textbf{money} to charter \textbf{one}.
The momentum of the \textbf{idea}, however, would not be stopped, and a \textbf{year} later, the \textbf{group} launched the first “ Greenpeace \textbf{campaign} ”, a seagoing voyage that set the tone for later Greenpeace campagins.
Dorothy Stowe passed away on July 23, 2010 at the \textbf{age} of 89.
She was \textbf{born} Dorothy Anne Rabinowitz in Providence, Rhode Island on December 22, 1920, from Jewish immigrant \textbf{parents} from Russia and Galicia.
She fondly recalled that her \textbf{father} Jacob “ \textbf{cared} about justice not only for Jewish \textbf{people}, but for \textbf{everyone}. ” Dorothy ’s \textbf{mother}, Rebecca Miller, taught Hebrew and inspired Dorothy to pursue an \textbf{education}.
Dorothy became a psychiatric social worker, and served as the first \textbf{president} of her local civic employees \textbf{union}.
During the repressive McCarthy era in the U.S., when she threated a strike, the state governor called her a “ communist, ” but Dorothy was not intimidated.
She stood up to the bullies and won a pay raise for her \textbf{union}.
In 1953 Dorothy married civil rights \textbf{lawyer} Irving Strasmich.
They celebrated their wedding dinner at the National Association for the Advancement of Colored \textbf{People} ( NAACP ), the \textbf{organization} that launched the U.S. civil rights \textbf{movement}.
They changed their \textbf{family} \textbf{name} to Stowe in honour of Harriet Beecher Stowe – pioneering feminist and abolitionist, who helped end slavery in the U.S.
The Stowes had two \textbf{children}, Robert, \textbf{born} in 1955 and Barbara in 1956, both now living in Vancouver.
In the 1950s, Dorothy and Irving Stowe began campaigning against nuclear \textbf{weapons}, adopting the Quaker \textbf{ideas} of “ bearing \textbf{witness} ” to wrong-doing and “ speaking truth to power. ” In 1961, to avoid supporting the Vietnam War with their taxes, Dorothy and Irving immigrated to New Zealand.
However, when New Zealand sent troops to Vietnam in 1965, the Stowes moved their \textbf{family} to Canada.
In Vancouver, Dorothy \textbf{worked} as a \textbf{family} therapist, supporting Irving ’s full \textbf{time} \textbf{peace} activism.
When the Stowes formed the “ Don't Make a Wave Committee, ” Dorothy Stowe recruited social workers and \textbf{women} ’s \textbf{groups} to boycott US products until the nuclear \textbf{tests} were cancelled.
Social change often ignites on the fringe, in hybrid cultures, where the status quo loses its grip on perception.
This is the case with Greenpeace, which evolved from a grassroots \textbf{peace} and \textbf{ecology} \textbf{movement} in Vancouver, Canada between 1968 and 1972.
The \textbf{movement} began innocently enough on the \textbf{streets}, in private kitchens, coffee shops, and pubs, where two distinct cultures and generations mixed.
While the \textbf{men} – Ben Metcalfe, Bob Hunter, Jim Bohlen, and Irving Stowe – made headlines, Dorothy Stowe quietly \textbf{worked} behind the \textbf{scenes}, supporting her \textbf{family}, hosting \textbf{meetings} in her home, \textbf{answering} correspondence, keeping the files, and creating a broad coalition with churches, \textbf{unions}, feminist \textbf{groups}, and \textbf{other} \textbf{activists}.
Over the \textbf{years} since, Dorothy has hosted hundreds of of young \textbf{activists}, who made the pilgrimage to her home for \textbf{inspiration}.
When the band U2 visited Vancouver in 2005, singer Bono made a special effort to meet Dorothy Stowe.
Dorothy never \textbf{rested} on past \textbf{success} or stopped \textbf{working} for social change.
A \textbf{month} before she passed away, Dorothy hosted a brunch for then newly installed Greenpeace International Executive Director Kumi Naidoo.
This was her last public act, and she summoned every ounce of energy to host the Greenpeace contingent in her home where many of the first Greenpeace \textbf{meetings} were held more than forty \textbf{years} ago.
She was thrilled that Kumi, with a \textbf{background} in civil rights, had taken this leadership role in Greenpeace.
The most fitting memorial for Dorothy Stowe, Jim Bohlen, and the \textbf{others} who have passed away, is that we simply get up each \textbf{morning} and go back to \textbf{work} in the \textbf{service} of \textbf{peace}, justice, and the living \textbf{Earth}.
This is all they would have asked of \textbf{us}.
War played a role in creating this hybrid culture.
In the 1960s, over \textbf{one} million \textbf{people} fled the United States in protest against nuclear \textbf{weapons}, militarism, and the Vietnam \textbf{war}.
Some 150,000 – including the Stowes, the Bohlens, and \textbf{others} in the Greenpeace \textbf{movement} – immigrated into Canada, the largest single political exodus in US \textbf{history}.
The U.S. immigrants brought \textbf{ideas} and progressive enthusiasm from civil rights and \textbf{peace} \textbf{movements}.
Meanwhile, Canada possessed deeply seeded, broad-based \textbf{peace} and \textbf{ecology} \textbf{movements}, which influenced the new immigrants.
Canada was naturally multicultural, with a grounded French community, and Asian and European communities that had retained their cultural roots.
Canadians were more introspective than the Americans and tended to possess a more self-effacing \textbf{sense} of humour.
Canadians could laugh at themselves.
The mixture of these cultures gave Greenpeace an international quality from the very \textbf{beginning}.
Secondly, two distinct generations of \textbf{activists} merged to create Greenpeace.
On the \textbf{one} \textbf{hand}, in the 1960s and 1970s, a creative youth culture emerged globally, linked by television and radio, with \textbf{shared} values ( \textbf{peace}, \textbf{ecology}, natural living ) and shared culture ( \textbf{rock} and jazz \textbf{music}, \textbf{film}, \textbf{activist} \textbf{art}, and personal liberation ).
This radicalized youth culture influenced Greenpeace but was balanced by an older generation of committed \textbf{peace} \textbf{activists} represented by the Stowes and Bohlens.
The young \textbf{activists} were willing to take risks, perform wild stunts, and get arrested, but the older \textbf{activists} – influenced by the Quakers and Gandhi – contributed \textbf{experience}, political \textbf{awareness}, thoughtful \textbf{strategy}, and a \textbf{sense} of calm. \textbf{Looking} back now, I believe this mix of generations and cultures helped make Greenpeace stronger in the \textbf{beginning}.
Another defining quality of Greenpeace at that \textbf{time}, was an \textbf{awareness} of social \textbf{media}, changing \textbf{society} by telling \textbf{stories} that would move on existing \textbf{media} networks.
Canadian \textbf{media} theorist Marshall McLuhan – who introduced the \textbf{idea} of the “ global \textbf{village} ” and who warned that “ Television does more educating than all the \textbf{schools} ” — influenced our perception of \textbf{media} ’s power to change \textbf{society}.
Three experienced Canadian \textbf{journalists} – Ben and Dorothy Metcalfe and Bob Hunter – inculcated these \textbf{media} \textbf{ideas} within Greenpeace.
All three had grown up in Winnipeg, on the Canadian prairies, \textbf{worked} at the Winnipeg Tribune, and had traveled in Europe as \textbf{writers} and correspondents.
On the first Greenpeace \textbf{campaign}, Hunter and Ben Metcalf filed \textbf{stories} to the \textbf{media} from the \textbf{boat} and Dorothy Metcalfe converted her home into a radio \textbf{room}, relaying audio reports from the \textbf{boat} to the \textbf{world} \textbf{media}.
It might seem like normal practice now, but in 1971, these \textbf{media} practices for \textbf{activists} were pioneering.
Ben Metcalfe passed away in 2003, Hunter in 2005, and we ’ve lost Jim Bohlen and Dorothy Stowe in 2010.
Many \textbf{people} contributed to the creation of Greenpeace, but four \textbf{couples} were consistently at the core of early Greenpeace actions : Dorothy and Irving Stowe, Marie and Jim Bohlen, Dorothy and Ben Metcalfe, and Bob and Zoe Hunter.
Zoe was active in the \textbf{Campaign} for Nuclear Disarmament in London, where she met Bob, and remains active \textbf{today} in Amnesty International. \textbf{Others} can be mentioned among the founders – notably Paul Cote, who served on the \textbf{board} of the original Don't Make a Wave \textbf{committee} that became Greenpeace ; Bill Darnell, who coined the \textbf{word} “ green \textbf{peace} ” to add \textbf{ecology} to the original \textbf{peace} initiative ; Rod Maringing, a creative street-theatre \textbf{activist} ; Bobbi Hunter, who organized the first public \textbf{office} in Vancouver ; Paul Spong, who introduced the \textbf{whale} \textbf{campaign} ; and many \textbf{others}.
Nevertheless, the four founding \textbf{couples} – the Stowes, Bohlens, \textbf{Hunters}, and Metcalfes – hold a special place in the \textbf{history} of Greenpeace.

% matched lemmas: activist, age, answer, art, awareness, background, bear, beginning, board, boat, bomb, campaign, care, child, committee, couple, day, demonstration, earth, ecology, education, everyone, experience, family, father, film, group, hand, history, hope, hunter, idea, image, inspiration, journalist, lawyer, look, man, march, medium, meeting, money, month, morning, mother, movement, music, name, office, one, organization, other, parent, peace, people, phone, president, rest, rock, room, scene, school, sense, service, share, society, something, son, story, strategy, street, success, test, time, today, union, us, village, war, weapon, whale, witness, woman, word, work, world, writer, year
\end{textsample}
