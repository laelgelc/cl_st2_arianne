\begin{textsample}{POS Dim 1 – human – Score 101.00 – t451\_human.txt}  \label{ex:f1_pos_001}
May 2, 2020 marks 15 \textbf{years} since the passing of Greenpeace co-founder Bob Hunter.
Although Hunter remains legendary within Greenpeace lore, readers \textbf{may} not know about his life outside of Greenpeace or the social-change philosophy that inspired his ecological \textbf{strategy}.
Bob had sailed on the first Greenpeace \textbf{campaign} ( the \textbf{group} was still called “ The Don't Make a Wave Committee then ) to protest the US nuclear \textbf{test} in Alaska, a tactic borrowed from the Quakers.
Greenpeace was launching a second \textbf{campaign} against French nuclear \textbf{testing}.
We had already learned that strontium-90, a radioactive by-product of the nuclear \textbf{bomb} \textbf{tests}, was appearing in the deciduous teeth of \textbf{children} around the \textbf{world}.
This realization provided a key link between the \textbf{peace} \textbf{movement} and \textbf{ecology}.
Bob had written an analysis of cultural change, and \textbf{ecology}, “ The Enemies of Anarchy ”.
The “ real anarchists, ” he believed, were militarized elites, who ran selfishly about the planet, ignoring the laws of nature, devastating environments.
The “ enemies ” of this institutionalized anarchy were the advocates for a new consciousness, for \textbf{peace}, unity, and \textbf{ecology}.
The necessary changes would not come from the political process, which was too slow and corrupted.
Bob proposed that violent \textbf{revolution} “ accomplishes \textbf{nothing}, except a changing of the guard. ” Violence “ diverts \textbf{us} from the real struggle, which is to attain a higher level of consciousness. ” From Betty Friedan and the feminist \textbf{writers}, Bob had realized that the new consciousness would be more sensual than intellectual. “ \textbf{Ecology} is the \textbf{thing}, ” Bob said.
This new consciousness, he believed, arose from understanding natural \textbf{relationships}. “ In nature, ” Bob quoted from Rachel Carson ’s Silent \textbf{Spring}, “ \textbf{nothing} exists alone. ” The \textbf{peace} and civil rights \textbf{movements} recognized the whole human \textbf{family}, but this “ whole ” did not stop with the human community.
We are all part of a much more fundamental ecological community.
Furthermore, the necessary \textbf{revolution} is a spiritual journey because the \textbf{Earth} is sacred, and our \textbf{relationships} with all \textbf{Earth} ’s creatures are sacred \textbf{relationships}.
In November 1969, the Cuyahoga River that runs through Cleveland, Ohio caught on fire due to massive chemical and oil pollution. “ The rivers are burning, ” I recall Bob shaking his \textbf{head}. “ It ’s Biblical. \textbf{Humanity} won't endure a slow evolution of \textbf{awareness}. ” From the first \textbf{time} we met, Bob and I agreed that global \textbf{society} needed an \textbf{ecology} \textbf{movement}, and we spent the next decade together trying to make that happen.
Paul Shepard had described \textbf{ecology} as the “ subversive science. ” An honest \textbf{understanding} of \textbf{ecology} would change \textbf{everything} about human \textbf{society}.
It was going to change our \textbf{art}, psychology, \textbf{politics}, science, and public discourse. \textbf{Writers} such as Carson, Shepard, Paul Ehrlich, Donella Meadows, Arne Naess, Gregory Bateson, and \textbf{others} were our mentors, calling for change. \textbf{Someone} had to make it happen, quickly, on a grand scale.
When I met Hunter, he had just written his second non-fiction \textbf{book}, Storming of the \textbf{Mind}.
Bob had been reading Canadian \textbf{media} analyst Marshall McLuhan, who noted that electronic \textbf{media} ( television and radio in those \textbf{days} ) had converted the \textbf{world} into a “ global \textbf{village}, ” in which \textbf{people} could communicate from previously isolated regions, creating a “ unified field of \textbf{experience}. ” From McLuhan, Hunter had learned to think of the electronic \textbf{media} as a global nervous system, a “ delivery system ” for \textbf{ideas}.
Thus, Bob believed, it was not necessary or effective to think of \textbf{revolution} as an armed struggle. “ Revolution these \textbf{days}, ” he said, “ is a \textbf{communications} struggle, a \textbf{war} of \textbf{images}.
Instead of storming the Bastille, ” he said, ” we ’re storming the \textbf{minds} of millions of \textbf{people}.
Instead of lobbing bullets and \textbf{bombs}, were lobbing \textbf{mind} \textbf{bombs}, revolutionary \textbf{images} that \textbf{will} explode in \textbf{people} ’s \textbf{heads}. ” To create an \textbf{ecology} \textbf{movement}, then, we had to come up with \textbf{images} that would circulate the globe, \textbf{images} that would inspire \textbf{people} to recognize their fundamental ecological nature, their kinship with every living creature on Earth.
Hunter was \textbf{born} on October 13, 1941 in St.
Boniface, Manitoba, the French \textbf{district} of Winnipeg in Canada.
Young Bob rarely saw his war-veteran \textbf{father}, and then \textbf{one} \textbf{day} he was gone.
Curiosity about his absent \textbf{father} ’s \textbf{war} \textbf{experience} led Hunter to World War II \textbf{books}, and these \textbf{books} inspired the young \textbf{man} to write his own fantasy adventures.
We \textbf{staged} actions around Vancouver, blocking toxic drain pipes into the Fraser River, and so forth, but the big \textbf{idea} that captured our imaginations came from cetacean researcher, Dr.
Paul Spong in 1972 : Save the \textbf{whales}!
From that \textbf{day} forward, we formulated plans to take a \textbf{boat} into the Pacific, find the Russian and Japanese \textbf{whaling} fleets, blockade them, and record the confrontation on \textbf{film} for the \textbf{world} \textbf{media}.
Two-and-a-half \textbf{years} later, on April 27, 1975, we launched the 80-foot halibut \textbf{boat}, the Phyllis Cormack, Greenpeace V, from Vancouver, and in June we clashed with Russian whalers off the \textbf{coast} of California.
The \textbf{rest} is Greenpeace \textbf{history}.
Bob often mocked himself, and the “ hokus pokus ” \textbf{stuff}, but he was deeply spiritual about \textbf{ecology}.
Bob and I \textbf{shared} a \textbf{passion} for Buddhism ( compassion for all sentient \textbf{beings} ) and Taoism ( nature as a model for human action ).
Bob believed that \textbf{awareness} itself was curative.
With a deeper \textbf{awareness} of \textbf{one} ’s own organic nature, a \textbf{person} could \textbf{let} the organism take over, without interfering, like the Taoists, trusting our instinctive wisdom to make decisions.
By 1976, Greenpeace had become famous. \textbf{Money} and support flooded in, and that changed \textbf{everything}. \textbf{One} \textbf{day}, during the second \textbf{whale} \textbf{campaign}, aboard the minesweeper the James Bay, christened Greenpeace VII, I \textbf{watched} Bob on the forward deck, alone, gripping a halyard, staring into the gray void of the sea.
He wore the same white wool sweater and sandals he ’d worn for \textbf{weeks}, but now a long scrub-brush dangled from his belt.
The pressure of holding the centre of this expanding \textbf{movement} and competing egos had driven him to the edge of sanity.
The crush for a \textbf{piece} of the action, a spot on the \textbf{crew}, access to the budget, or for a \textbf{share} of the \textbf{media} notoriety had transformed a small cadre of \textbf{ecology} \textbf{activists} into \textbf{something} akin to a \textbf{touring} \textbf{rock} band.
Twenty Greenpeace \textbf{offices} now operated around the \textbf{world}.
Even with supply lines and \textbf{communication} lines intact, which ours were not, the forward motion of a social \textbf{movement} can flounder on the delicacies of administering \textbf{relationships}.
Bob ’s natural style was to include \textbf{everyone}.
Yet, Bob, at the hub of an expanding \textbf{movement}, and not prone to authoritarian rule, suffered exhaustion and private anguish.
We had been called “ crazy ” by more than \textbf{one} observer, but most of our antics were all in fun.
The mystical side of Greenpeace did not imply a cult or collection of psychotics.
We understood the importance of having a political \textbf{strategy}, a clear message, and a skilled \textbf{team}.
But we also appreciated the value of a \textbf{good} myth and a \textbf{good} laugh.
The unique blend that had become Greenpeace – \textbf{ecology}, \textbf{media} \textbf{awareness}, spirituality, humour, and sea-going direct action – grew from a balance of myth-making and realism.
However, by 1976, Bob appeared to be burning out like a meteor racing through the thick political atmosphere within the burgeoning \textbf{organization}.
His apparent madness on \textbf{board} the ship, however, concealed a method.
Bob took seriously the advice from his mentor, the poet Allen Ginsberg, regarding power : “ \textbf{Let} it go. ” As a theatrical retort to the crush for power on the ship, and within Greenpeace in \textbf{general}, Bob had assigned himself the role of ship ’s “ latrine \textbf{officer} ” and took to wearing the latrine brush, which now dangled from his belt as he shuffled about the ship.
He could be found each \textbf{morning} in the latrine, cheerfully scrubbing away at the sinks and toilet bowls.
To some, this was pure delirium, but his latrine \textbf{officer} act was a little internal mindbomb, not madness. “ Don't take yourself too seriously, ” was the message.
Bob inspired \textbf{people} around him to contribute, made \textbf{others} feel essential.
He was a natural leader, but not really cut out for internal \textbf{politics}.
In 1977 he retired from Greenpeace to a rural life with his wife Bobbi and their \textbf{son} Will.
Canadian journalism gladly took him back.
Bob knew how to light up a public audience.
By the \textbf{age} of 14, Bob had filled ten thick notebooks with handwritten science-fiction \textbf{stories}.
His \textbf{mother} bought him an Underwood typewriter for his fifteenth birthday, his most treasured possession.
He wrote a 75-page \textbf{story} called “ The Long Twilight, ” in which a boy is kidnapped by a flying saucer and ends up alone in the universe.
Twenty \textbf{years} later, on Greenpeace ships, Bob still pecked at the typewriter with two fingers ; the fastest two-finger typist I ’ve ever met. \textbf{Time} magazine later \textbf{named} Hunter as \textbf{one} of the “ Eco-Heroes ” of the 20th \textbf{century}.
In 1991, he won the Canadian Governor General ’s Award for his \textbf{book} “ Occupied Canada : A Young White Man Discovers His Unsuspected Past “.
In 1998, doctors diagnosed Bob with prostate \textbf{cancer} and on May 2, 2005 he passed on.
His ashes were scattered in northern Canada near the Arctic, at Tortuga Bay in the Galapagos Islands, and in Antarctica by his \textbf{daughter} Emily during the 2006 Sea Shepherd \textbf{campaign} against whaling.
He is survived by his wife Bobbi ; his four \textbf{children} Emily, Will, Conan, and Justine ; and by a grateful Greenpeace \textbf{organization}, which owes to Hunter many of its fundamental principles and \textbf{vision}.
Hunter, Robert ; “ Erebus, ” 1968, McClelland and Stewart.
Hunter, Robert, “ The Enemies of Anarchy : A Gestalt Approach to Change, ” 1970, McClelland-Stewart ; Viking Hunter, Robert ; “ The Storming of the \textbf{Mind} : Inside the Consciousness \textbf{Revolution}, ” 1971, McClelland and Stewart ; Doubleday.
Hunter, Robert ; “ \textbf{Warriors} of the Rainbow : A Chronicle of the Greenpeace Movement, ” 1979, McClelland and Stewart.
Hunter, Robert ; “ Occupied Canada, ” 1991, McClelland and Stewart.
At 17, Bob wrote “ After the \textbf{Bomb}, ” a futurist novel about a post-nuclear-holocaust civilization.
He quit high \textbf{school} and left Winnipeg with his beloved typewriter to see the \textbf{world}.
He made it as far as Vancouver, lived on hot dogs and water in a skid row hotel, began a coming-of-age novel, ran out of \textbf{money}, and hitchhiked back to Winnipeg. “ The \textbf{world} ’s shortest \textbf{world} \textbf{tour}, ” he would later recall.
While \textbf{working} as a copy boy at the Winnipeg Tribune, an \textbf{editor} asked, “ Who wants a byline? ” Hunter \textbf{raced} across the newsroom and stood at \textbf{attention}. “ Hunter, ” said the \textbf{editor}. “ Of \textbf{course}. ” The assignment, a skydiving \textbf{story}, required that he leap from an airplane, which he did, but on landing he seriously injured his back.
Decades later, Hunter would describe his chronic back pain as, “ my reminder of the karmic consequences of ego. ” At 20, he travelled to Paris and London, where he began what would be his first published novel, Erebus, a searing, dark farce about finding \textbf{love} while \textbf{working} at an abattoir.
In London, he met Zoe Rahim, a \textbf{member} of the \textbf{Campaign} for Nuclear Disarmament.
Hunter ’s Bohemian \textbf{writer} life path turned sharply towards \textbf{politics}.
Zoe took him to Speaker ’s Corner, where he heard Bertrand Russell expound on pacifism, and to the historic 1963 \textbf{peace} \textbf{march} to Aldermaston.
They married, flew to Winnipeg, where their \textbf{children}, Conan and Justine, were \textbf{born}, and then to Vancouver, where Bob got a \textbf{job} at The Province, as a copy boy.
When Erebus was published and earned a Governor General ’s Award nomination, Hunter moved up to \textbf{news} reporter at the Vancouver Sun.
When the \textbf{editor} ran a contest for a vacant columnist \textbf{job}, Bob won and began writing his own column.
I arrived in Vancouver from the US as a \textbf{war} resister and got a \textbf{job} as \textbf{journalist} and \textbf{photographer} at The North Shore News.
Bob Hunter ’s columns seemed to me like the most interesting writing in Vancouver, so I \textbf{phoned} him, and we met at the Press Club, a small, unadorned pub on Granville Street.
We arrived around 4pm and closed the pub down after midnight.
Bob was a fascinating conversationalist, who listened as well as he spoke.
He appeared brilliant.
We \textbf{shared} a \textbf{passion} for advocacy journalism, \textbf{peace} activism, and the emerging science of \textbf{ecology}.
It is hard to imagine now, but in the late 1960s and early 70s, \textbf{ecology} was still a radical \textbf{idea}.
There was no \textbf{ecology} \textbf{movement}, as there is \textbf{today}, no environment ministers, no \textbf{ecology} field \textbf{trips} in high \textbf{schools}, or \textbf{courses} in \textbf{universities}.

% matched lemmas: activist, age, art, attention, awareness, bear, being, board, boat, bomb, book, campaign, cancer, century, child, coast, communication, communications, course, crew, daughter, day, district, earth, ecology, editor, everyone, everything, experience, family, father, film, general, good, group, head, history, humanity, idea, image, job, journalist, let, love, man, march, may, medium, member, mind, money, morning, mother, movement, name, news, nothing, office, officer, one, organization, other, passion, peace, people, person, phone, photographer, piece, politics, race, relationship, rest, revolution, rock, school, share, society, someone, something, son, spring, stage, story, strategy, stuff, team, test, testing, thing, time, today, tour, trip, understanding, university, us, village, vision, war, warrior, watch, week, whale, will, work, world, writer, year
\end{textsample}
