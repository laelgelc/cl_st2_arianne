\begin{textsample}{POS Dim 1 – human – Score 88.00 – t297\_human.txt}  \label{ex:f1_pos_007}
With Greenpeace ’s 50th anniversary on the horizon, I was asked to host a series of virtual “ mailbag ” calls connecting \textbf{activists} across generations and regions.
The outpouring of interest — and \textbf{questions} — from those at Greenpeace \textbf{today} was moving.
This made \textbf{us} laugh, and we thought : This is perfect!
We promptly announced to the \textbf{media} that we were going to do a “ \textbf{test} blockade ” of the oil tanker \textbf{test} run.
We knew we were onto \textbf{something} when the reporters laughed.
We \textbf{phoned} our \textbf{friend} Dennis Feroce, who agreed to pilot his \textbf{boat}, The Meander, to take \textbf{us} to the entrance of Juan \textbf{de} Fuca Strait, where the tanker would have to enter.
Three \textbf{days} later, we sat ready in the water with our Zodiacs and a small flotilla of sailboats, as television camera \textbf{crews} flew overhead in helicopters.
We stopped the tanker dead in the water, \textbf{film} and photographs went all over the US and Canada, and we got arrested by the US \textbf{coast} guard and taken to jail in Everett, \textbf{north} of Seattle.
The \textbf{media} followed \textbf{us}, as we told the \textbf{police} that it was “ just a \textbf{test}. ” We made jokes about \textbf{testing} the handcuffs, \textbf{testing} the jail, and the \textbf{police} also laughed. \textbf{Everyone} was on our side.
When an \textbf{officer} brought \textbf{us} our meal ( fast food burgers ), she dropped the bag on the \textbf{table} in the cell and said “ \textbf{test} this. ” The entire \textbf{campaign} was hilarious, and the oil \textbf{port} never got built.
This is \textbf{one} of my favorite \textbf{campaigns} because we pulled it off in three \textbf{days}, \textbf{everyone} had fun, and the \textbf{idea} came not from a seasoned \textbf{activist} or a \textbf{campaign} \textbf{committee}, but from our unassuming \textbf{office} \textbf{manager}, Julie.
To shift the needle in social action, \textbf{one} has to be creative and do \textbf{something} that is not expected.
When environmental \textbf{groups} do what \textbf{everyone} expects them to do, \textbf{nothing} changes.
For example, regarding global warming, I would suggest don't do the predictable \textbf{thing} and go to the next climate conference.
Flip the script.
Boycott the climate conference.
Explain why : Because the next conference in Glasgow in October \textbf{will} be the 34th international climate conference since the first \textbf{one} in 1979, and these conferences have achieved \textbf{nothing} significant.
Thirty-four climate conferences in 42 \textbf{years}, and during that \textbf{time} human carbon emissions have doubled.
Atmospheric CO 2 levels have gone from 337 parts per million ( ppm ) to 420 ppm ; the oceans are choked with acidification, the coral beds are dying, forests are burning, and the \textbf{world} ’s \textbf{politicians} are burning jet fuel to twiddle their thumbs at these climate \textbf{meetings}.
I would suggest : Stop validating this nonsense.
Boycott.
Organize the \textbf{ecology} \textbf{groups} to boycott together, and state the \textbf{reasons}.
Hold your own separate \textbf{meetings} regionally and on the internet.
Denounce the phoney and hollow promises by governments.
Go instead to every major seaside city on Earth and paint the future waterline on buildings to depict sea rise after the Antarctic and Greenland ice melts.
Give \textbf{people} a new \textbf{picture}, not the failing routine.
The \textbf{great} social \textbf{movements} of \textbf{history} that have actually changed \textbf{things} have been able to find a \textbf{way} to do the unexpected, to blow up the prevailing paradigm, to make \textbf{people} think in new \textbf{ways}.
The pandemic is linked to human overshoot, our occupation of wild habitats, our destruction of biodiversity, our growth and consumption beyond the capacity and limits of the global ecosystems.
So, yes, the conditions for the swift transmission of this pandemic are created by human activity.
However, nature does not really “ fight back. ” Evolution does not appear to have goals or preferences, and does not hold grudges.
Nevertheless, pandemics \textbf{will} likely continue to be \textbf{one} consequence of neglecting our ecological crisis.
I don't assume that Greenpeace can or should do \textbf{everything} that needs to be done.
I know from \textbf{experience} that \textbf{people} project onto Greenpeace the responsibility for every environmental urgency, an impossible expectation for a single \textbf{organization}.
Thus, I think of this \textbf{question} more as “ What do I wish the environmental \textbf{movement} would do that it isn't doing? ” I wish the ecology/environmental \textbf{movement} was more realistic about our real, fundamental crisis : Ecological overshoot and the conditions that create it, namely : unfettered growth.
Overshoot is natural.
Most successful species overshoot their habitats, a pack of wolves \textbf{will} overshoot the capacity of a watershed, algae \textbf{will} overshoot the capacity of a \textbf{lake}, and we can see in our own gardens how \textbf{everything} grows into its neighbours, tangling and pushing the limits of space and resources.
Natural evolution teaches all species how to grow, reproduce, and consume, but evolution does not teach species when to stop, when to restrain itself.
I wish the \textbf{ecology} \textbf{movement} would more directly address the \textbf{fact} that \textbf{humanity} has overshot Earth ’s capacity.
Calculated estimates range from about 50 % overshoot ( Footprint Network ) to 100 % or more.
The significant point is that all pathways out of overshoot — for any species anywhere, no exceptions — involve contraction : The wolves die back until their game recovers ; the algae die back to the limit of available nutrients ; plants push back on each \textbf{other} until they reach a new dynamic homeostasis.
We make a mistake if we behave as if \textbf{humanity} does not have to also get smaller in numbers and consumption, the two issues we tend to avoid.
Government, industry, and even some environmental \textbf{groups} focus on new technologies and a presumed “ green growth, ” avoiding the inevitable contraction of human economic activity, numbers, and consumption.
Addressing the frivolous, wasteful consumption of the rich is a \textbf{good} place to \textbf{start}, but not the full \textbf{story}.
To actually reverse overshoot, we also need to address the global economic system of capitalism, which requires unrealistic, endless growth ; we need to address equitable \textbf{ways} to reverse human population growth ( \textbf{women} ’s rights and accessible contraception ) ; and we need to be realistic about what is required to clean up our technologies.
I would like to see the environmental \textbf{movement} be more active and serious about all three of these necessary steps to reversing overshoot.
There is a huge \textbf{difference} between a \textbf{movement} and an \textbf{organization}.
The \textbf{movements} for \textbf{peace}, civil rights, \textbf{women} ’s rights, and \textbf{others} have survived for \textbf{centuries} because they have a robust constituency and clear goals that have not yet been fully achieved.
Similarly, the \textbf{ecology} \textbf{movement} \textbf{will} likely endure long into the future.
While I could not address all the \textbf{great} \textbf{questions} and comments in \textbf{one} \textbf{article} or \textbf{one} call, here is the first batch.
These are my \textbf{thoughts} and \textbf{ideas}. \textbf{Someone} else might have different \textbf{ideas}.
I make no claim to arriving at complete and definitive \textbf{answers} to these \textbf{questions}. \textbf{Organizations}, on the \textbf{other} \textbf{hand}, can come and go.
Social \textbf{organizations} gain support and prominence because they address an issue that \textbf{people} \textbf{care} about and they appear effective.
I say “ appear ” because an \textbf{organization} \textbf{may} endure, through reputation and self-promotion, even if it becomes less effective than its supporters believe.
For an effective social \textbf{organization} to endure, however, requires a constituency that believes the \textbf{organization} plays an essential role, that the leadership has integrity, and that it can achieve genuine change.
Typically, successful \textbf{organizations} arise because some \textbf{group} of \textbf{people} have a creative \textbf{idea} about how to address a problem about which the public is generally aware. \textbf{Creativity} is essential in the \textbf{birth} and growth of an \textbf{organization}.
However, as an \textbf{organization} grows and becomes more structured, it is possible that \textbf{creativity} is stifled rather than encouraged.
Maintaining \textbf{creativity} is a key quality of successful \textbf{organizations}.
By definition, there is no formula for \textbf{creativity}.
Rather than attempt to formalize \textbf{creativity}, successful \textbf{organizations} learn to create the conditions for \textbf{creativity}, to overcome bureaucratic roles, and allow new \textbf{ideas} to flourish at every level of the \textbf{organization}.
This is a popular \textbf{question} these \textbf{days}, I believe because so many of \textbf{us} feel the concern about \textbf{humanity} ’s future.
We meet discouraging obstacles, resistance, subversion, and indifference, so we naturally seek hopefulness. \textbf{Hope} is a \textbf{good} frame of \textbf{mind}, because it opens paths to action, but \textbf{hope} is not a \textbf{strategy}.
To solve problems, \textbf{one} must deeply understand the problem, the conditions, and appreciate the larger systems and forces at \textbf{work}.
Delusional \textbf{hope} is definitely not helpful. \textbf{Humanity} exists now in a tremendous bind.
The powerful and wealthy have plundered the \textbf{Earth} to enrich a few, while billions live on the edge of starvation.
Meanwhile, species diversity plummets, the atmosphere fills with carbon-dioxide, the oceans turn acidic and are choked with plastic, and we face myriad ecological tragedies.
I do not find \textbf{hope} in political conferences, governments, corporations, nor in appeals to the \textbf{general} \textbf{good} of \textbf{humanity}.
In my \textbf{experience}, most \textbf{people} are decent and fair, but greed, \textbf{fear}, and ignorance can create chaos and harm.
I find \textbf{hope} in nature itself, in the wild \textbf{world}, in the power of life to create new conditions, in the \textbf{shared} learning and co-evolution of all nature.
This is where, I believe, \textbf{humanity} has to turn.
We are not solving our problems with conferences, technologies, space travel, or economic growth.
These delusions create more problems.
I believe we have to rejoin the ecological community.
We have to become apprentices to nature and learn how the natural \textbf{world} actually endures and survives as a living system.
I believe we have to shed our human pride and re-align ourselves with all our relatives, with the systems and processes of the entire natural \textbf{Earth}.
I ’m with the Taoists and certain Indigenous \textbf{teachers} on this : We have to learn to respect our Mother Earth, the source of all life. \textbf{Nothing} survives alone.
We only survive in \textbf{relationship} with the living matrix.
That is where I \textbf{look} for \textbf{hope}.
I was fortunate as a young \textbf{child} to live in natural settings, with rivers, forests, hills, and ocean to explore.
However, as a \textbf{child}, I didn't know how vulnerable all this was.
Later, I \textbf{witnessed} pristine natural settings obliterated for shopping malls, highways, and parking \textbf{lots}.
Rachel Carson ’s Silent \textbf{Spring} taught me more about the crisis, and in 1969 when the Cuyahoga River in Ohio caught on fire from pollution, I realized the urgency of the crisis.
From Gregory Bateson, Arne Naess, the Taoist \textbf{writers}, and my Indigenous \textbf{friends}, I began to learn a deeper respect for how to think and live the \textbf{way} nature \textbf{works}.
I recently wrote about this for Greenpeace.
Sort of.
In the 1970s, we set out to create a global \textbf{ecology} \textbf{movement}, which did not exist at the \textbf{time}, and we expected that the \textbf{movement} would expand around the \textbf{world}.
In the \textbf{beginning}, I think most of \textbf{us} were more interested in a global \textbf{movement} than in a global \textbf{organization}.
We wanted \textbf{people} to rise up everywhere and defend biodiversity and vulnerable ecosystems. \textbf{Friends} of the \textbf{Earth} and \textbf{other} \textbf{organizations} arose at the same \textbf{time}.
As Greenpeace \textbf{offices} \textbf{sprang} up all over the \textbf{world}, to maintain clear \textbf{communications} about our \textbf{work}, we had to create a global structure, and thus we created Greenpeace International in Amsterdam in 1979.
The \textbf{movement} is strong enough now, that it \textbf{will} continue with or without any single \textbf{organization}.
Fridays for Future and Extinction Rebellion are examples of how the \textbf{movement} evolves.
This is more or less what I \textbf{hoped} would happen.
There are many fond \textbf{memories}, and most are linked to the camaraderie of \textbf{working} with \textbf{people} to achieve \textbf{something} larger, more important than ourselves : Sailing on the fish \textbf{boat} in 1975, at \textbf{night} under the stars as if adrift in the universe, playing \textbf{music}, learning maritime \textbf{skills}, the \textbf{day} we found the whalers, the shared euphoria, the shared heartbreak at \textbf{witnessing} the slaughter, the shared satisfaction when our \textbf{pictures} and \textbf{story} circled the globe, and the \textbf{feeling} that we had achieved \textbf{something} significant.
I felt most at risk the first \textbf{time} we actually maneuvered our Zodiacs between the \textbf{whales} and the whalers in 1975.
During the previous two \textbf{years} of organizing this \textbf{campaign}, and two \textbf{months} at sea \textbf{looking} for the whaling fleets, I had not thought about the consequences of what we planned.
However, once I stood in the bow of the Zodiac, with the \textbf{whales} in \textbf{front} of \textbf{us} and a harpoon \textbf{boat} behind, it occurred to me that if we were hit by \textbf{one} of these 200-pound exploding harpoons, we would be instantly killed.
I suddenly felt a chilling tremor of \textbf{fear}, but with no alternative but to stay in place.
That remains the most frightening moment of my Greenpeace \textbf{experience}.
I would advise my younger self to pay more \textbf{attention} to the \textbf{relationships}, internal and external, to be more aware of \textbf{others} ’ motivations and intentions.
I believe I was sometimes naive and perhaps too tolerant of big egos.
I would also advise more modesty in the face of the challenge we set for ourselves.
Confidence is helpful, but overconfidence is not.
I was unaware of how easily our stated values and \textbf{visions} for an ecological \textbf{society} could be misunderstood and even subverted.
I would also advise more boldness and less compromise in certain cases.
We often made compromises to appease \textbf{other} factions in \textbf{society} and in the environmental \textbf{movement}.
Sometimes those compromises might have been helpful, but in some cases, we \textbf{may} have allowed our message to be blunted.
The 1970s \textbf{whale} \textbf{campaign} was probably the most successful because it achieved two significant goals : It led to the 1982 whaling moratorium and many populations of \textbf{whales} began to recover.
We also had the intention that this \textbf{campaign} would help launch an \textbf{ecology} \textbf{movement}, which it did. \textbf{One} of my favorite \textbf{campaign} actions, however, was our “ \textbf{test} blockade ” of a supertanker.
In 1981, as we were \textbf{working} in our Vancouver \textbf{office}, Rod Marining read in the newspaper that in three \textbf{days}, an oil tanker was going to enter the Salish Sea between Seattle and Vancouver, loaded with water, as a “ \textbf{test} ” to demonstrate how easily an oil tanker could maneuver in these inside waters, promoting an oil \textbf{port}.
We were discussing what we might do when our \textbf{office} \textbf{manager} Julie McMaster said casually, “ You should do a \textbf{test} blockade. ”

% matched lemmas: activist, answer, article, attention, beginning, birth, boat, campaign, care, century, child, coast, committee, communication, creativity, crew, day, de, difference, earth, ecology, everyone, everything, experience, fact, fear, feeling, film, friend, front, general, good, great, group, hand, history, hope, humanity, idea, lake, look, lot, manager, may, medium, meeting, memory, mind, month, movement, music, night, north, nothing, office, officer, one, organization, other, peace, people, phone, picture, police, politician, port, question, reason, relationship, share, skill, society, someone, something, spring, start, story, strategy, table, teacher, test, thing, thought, time, today, us, vision, way, whale, will, witness, woman, work, world, writer, year
\end{textsample}
