\begin{textsample}{POS Dim 1 – human – Score 81.00 – t724\_human.txt}  \label{ex:f1_pos_013}
At the University of Minnesota Dr.
Nate Hagens teaches an honours \textbf{course} called “ Reality 101 : A Survey of the Human Predicament. ” Hagens operated his own hedge fund on Wall Street until he glimpsed, “ a serious disconnect between capitalism, growth, and the natural \textbf{world}. \textbf{Money} did not appear to bring wealthy clients more well \textbf{being}. ” Hagens became \textbf{editor} of The Oil Drum, and now sits on the Board of the Post Carbon Institute and the Institute for Integrated Economic Research.
Build community cohesion with \textbf{communication}, events, joy, sharing, etc.
The Oil Drum Post Carbon Institute Nora Bateson, “ Small Arcs of Larger Circles ” : Deep Green review and \textbf{book} at ( Triarchy Press, 2016.
William Catton, Overshoot, University of Illinois, 1980.
William Rees, “ The Way Forward : Survival 2100, ” Solutions Journal v.3, \#3, June 2012 Donella Meadows, et. al., Limits to Growth ( D.
H.
Meadows, D.
L.
Meadows, J.
Randers, W.
Behrens, 1972 ; New American Library, 1977 ) ; and Limits to Growth : The 30-Year Update ( Chelsea Green, 2004 ).
Preserve and restore local ecosystems ; protect wild places Teach, educate, learn, \textbf{share} \textbf{information} Promote local energy systems Plant gardens, grow food Learn localized community health \textbf{care} Accept complexity The \textbf{question}, “ What can I do? ” typically seeks a linear \textbf{answer} to a complex, whole-system challenge. “ What can I do? ” often wants a “ solution ” for a “ problem. ” This sort of linear thinking helped create the predicament we ’re in.
Changing a complex living system is not a linear, mechanistic “ solution. ” We have to remain humble in this struggle.
We are small.
Life is short.
Nature is expansive, complex, and long. \textbf{Love} and trust nature Spend \textbf{time} in the natural \textbf{world} without trying to “ fix ” it.
Sit with wildness and absorb it, \textbf{love} it, and respect it.
Apprentice yourself to nature, and what you learn \textbf{will} help when you engage in the human realm to defend that wildness.
Trust nature.
She \textbf{will} be fine.
Humans \textbf{will} not “ destroy the \textbf{Earth}. ” We cause harm to the biosphere, drive species to extinction, and alter Earth ’s climate, but we cannot touch the regenerative power of wild nature.
Earth \textbf{will} be fine.
Reality 101 addresses \textbf{humanity} ’s toughest challenges : economic decline, inequality, pollution, biodiversity loss, and \textbf{war}. \textbf{Students} learn about systems \textbf{ecology}, neuroscience, and economics. “ We ask hard \textbf{questions}, ” says Hagens. “ What is wealth?
What are the limits to growth?
We attempt to face our crises \textbf{head} on. ” “ Sharpen the sword ” This is a Buddhist precept.
You are the sword.
You are the tool that you take into \textbf{battle}.
Keep that tool sharp.
Be prepared.
In Buddhism, the sharpening comes from meditation and acts of compassion.
There are \textbf{other} methods, such as yoga, \textbf{art}, and the worship of mystery.
We sharpen the sword by \textbf{working} on ourselves, making ourselves better human \textbf{beings} and \textbf{better} agents of change.
In my \textbf{experience}, the weakest link in social \textbf{movements} is the ego : pride, wanting credit, wanting fame, wanting to be admired, wanting power, and so forth.
When we sharpen the sword, we quiet our own ego so that we become a calming influence rather than a source of anxiety for \textbf{others}.
These five principles are the bedrock for me.
And still, this is just the \textbf{beginning}, because once we unlock the confidence to act, and as we turn out to the \textbf{world}, the more challenging \textbf{work} begins.
We \textbf{may} benefit if we simultaneously hold two extremes of action ; both the huge, universal \textbf{movements} for \textbf{ecology} and justice and the daily, personal actions that help slightly and make \textbf{us} \textbf{better} examples to \textbf{others}.
Our priorities of action are unlikely to be the same as the priorities of status quo \textbf{society}. \textbf{Humanity} is in a state of ecological overshoot, and all pathways out of overshoot require contraction.
Few institutions like the \textbf{idea} of getting smaller, simplifying, or reversing the scale of human activity.
Technology can provide benefits, but there are no technologies that eliminate the ecological requirement of contraction to heal the biological foundation of our civilization.
Here are the areas that need the most \textbf{attention} : Consumption \textbf{Humanity} has been hugely successful at consuming Earth ’s bounty, but we have already overshot many of her limits.
Reducing consumption is imperative, and of \textbf{course}, this has to \textbf{start} with the frivolous, wasteful consumption of the rich \textbf{world}.
Some \textbf{ideas} : \textbf{Start} a \textbf{campaign} to reduce extravagant travel.
Some \textbf{students} feel inspired to action, and some report finding the material “ depressing. ” \textbf{One} \textbf{student} \textbf{shared} the \textbf{course} material with a \textbf{family} \textbf{member}, who asked, “ So what can I do? ” The \textbf{student} struggled to \textbf{answer} this \textbf{question}, and the listener chastised her : “ why did you explain all this to me, if you can't tell me what to do?! ” Lobby for heavy tax incentives to slow indulgent, leisure consumption.
Transform the \textbf{idea} of “ fashion. ” Make modesty the new fashion statement.
Organize your community to recycle and repair \textbf{everything}.
Help popularize modest consumption and a simpler \textbf{lifestyle}. \textbf{Start} a \textbf{campaign} for shoppers to leave all packaging at the stores.
Population Find \textbf{ways} to help stabilize and reduce human population.
Some human rights \textbf{activists} \textbf{fear} that population efforts might violate human rights, but crowding already erodes human rights.
Humans and our livestock now comprise 96 % of all mammal biomass on Earth.
There are limits.
All we need to do is reduce the human growth rate from +1 % per \textbf{year} to -1 % per \textbf{year}.
Reversing human sprawl makes life better for \textbf{everyone} and shows respect for all life.
The most graceful and effective \textbf{strategies} to stabilize and reduce the growth rate are simple and have \textbf{other} social benefits : Help establish universal \textbf{women} ’s rights, the right to plan pregnancy and childbirth. \textbf{Campaign} for universally available free contraception.
A fair \textbf{question}. \textbf{One} that, as \textbf{environmentalists}, we often get asked.
At the request of Dr Hagens, here is my list : Overcome the \textbf{fear} and taboo about discussing the human population growth rate.
Help popularize smaller \textbf{families} and \textbf{family} planning.
Energy Find \textbf{ways} to help reduce energy demand, reduce fossil fuel use, and support renewable energy.
Militarism \textbf{Campaign} to end militarism and \textbf{weapons} industries in all forms at every level.
Consumption, population, petroleum fuels, and militarism remain the four major drivers of our ecological crisis.
The underlying psychological drivers \textbf{may} be greed, \textbf{fear} and ignorance.
Meanwhile, there are hundreds, thousands of interconnected issues that need \textbf{attention} too.
Here are just 19 : Reduce meat consumption, reduce livestock herds, through taxes and \textbf{lifestyle} changes.
Support and preserve the cultures and \textbf{lifestyles} among Indigenous and modest farmer communities.
I have been asking this \textbf{question} all of my adult life.
As I ’ve \textbf{witnessed} the crisis intensify, I ’ve \textbf{experienced} \textbf{feelings} of panic, anger, and helplessness.
Nevertheless, I also feel at \textbf{peace}.
I \textbf{love} my \textbf{family} and \textbf{friends}, I enjoy life in my community, and \textbf{love} my \textbf{time} in the natural \textbf{world}.
Here are some of the \textbf{ways} I believe we can deal with anxiety about the \textbf{world} and take action : \textbf{Campaign} to limit corporate power in \textbf{politics}. \textbf{Campaign} to publicly fund \textbf{universities}, all \textbf{education}, to limit corporate corruption of \textbf{education}. \textbf{Start} an economic de-growth \textbf{group}. \textbf{Start} a \textbf{campaign} to create a new micro-economic system in your community, your state, your county, your nation, your company, your \textbf{family}. \textbf{Start} a \textbf{school} for the homeless and disenfranchised ; teach localized, useful \textbf{skills}, gardening, tool repairs.
Lobby your local government to create community gardens.
Study and create renewable energy systems that can be built, operated, and maintained locally. \textbf{Campaign} to consume only locally produced products ; reduce the energy cost of transported \textbf{goods}. \textbf{Start} or join \textbf{campaigns} to preserve ecosystems, rivers, \textbf{lakes}, the oceans, forests, biodiversity, and all non-human habitats.
Open or join a clinic and begin to \textbf{research} localized, small-scale healthcare.
Lobby governments to create walking neighbourhoods ; ban cars from city centres, create public transit projects, and make cities serve community.
Stay active Start a company that uses local resources and local \textbf{skills} to create useful locally consumed tools and resources. \textbf{Start} a “ free store ” in your community, where \textbf{people} can drop off used \textbf{goods}, and pick up useful used items they \textbf{may} need. \textbf{Start} a local support \textbf{group} or psychology practice and begin to learn and support community therapy ; build community trust ; help \textbf{others} deal with depression and anxiety.
The \textbf{best} therapy is a \textbf{friend}.
Legal support : are you a \textbf{lawyer}, or do you want to be?
Could you \textbf{work} as a paralegal? \textbf{Start} a practice to defend \textbf{ecology} \textbf{activists}, and \textbf{start} \textbf{class} action lawsuits against corporations that pollute. \textbf{Start} or join a \textbf{campaign} to impose carbon taxation and \textbf{other} pollution \textbf{charges} on contaminating products ; lobby for resource depletion fees, true cost pricing, and import tariffs on ecologically dangerous \textbf{goods}.
Help restore damaged ecosystems ; lobby governments and corporations to make funds available to restore damaged ecosystems ; plant trees, build soils, re-establish natural water flows. \textbf{Start} or join a \textbf{campaign} to achieve whatever is close and dear to your heart.
Even if small, personal actions might not shift the whole \textbf{world}, those actions count.
Your personal actions become a model for \textbf{others}, and the personal \textbf{lifestyle} changes of individuals add up.
These 12 actions \textbf{will} bring you closer to nature, closer to yourself, and closer to \textbf{friends} and allies, who \textbf{share} your \textbf{beliefs} and concerns : Grow food, plant gardens, learn horticulture, plant fruit trees.
Spend as much \textbf{time} in wild nature as possible, pay \textbf{attention}, observe, contemplate.
It can feel \textbf{good} to simply resist the destructive acts of governments and corporations, to stand up for the dispossessed, abused, and for the natural \textbf{world}. \textbf{Caring} about \textbf{others} can be the \textbf{greatest} gift to \textbf{one} ’s own soul and \textbf{peace} of \textbf{mind}.
Fix \textbf{everything}.
Have a fix-it shop with tools and supplies.
Fix \textbf{things} for your \textbf{family}, \textbf{friends} and neighbours.
Teach \textbf{others} how to fix \textbf{things}.
Repair clothes.
Stand up to bullies in every possible \textbf{way} ; don't \textbf{let} individuals, corporations, or governments bully you, your \textbf{family}, or your neighbours.
You can do this with kindness and grace, and with inner \textbf{strength}.
And don't be bullied by popular, conventional perceptions. \textbf{Share} \textbf{everything} you can.
Help \textbf{others} trust in sharing.
Create community cohesion by organizing \textbf{ways} to \textbf{share} resources, tools, or public land.
Take in a homeless foster \textbf{child} ; give them some \textbf{love} and security ; help create \textbf{one} less wounded soul, floundering and struggling in the \textbf{world}.
Find \textbf{ways} to use your training, career, or \textbf{job} to further ecological and social justice goals. \textbf{Talk} to coworkers.
Create recycling, \textbf{sharing}, and promote modest consumption in your workplace.
Create \textbf{art}, \textbf{music}, theatre, dance.
Artistic \textbf{work} can express human \textbf{creativity} without frivolous consumption ; \textbf{art} builds self-confidence and leads to creative interaction with \textbf{others}.
Create \textbf{art} events, \textbf{start} a gallery or performance space.
Help young \textbf{people} find their creative \textbf{spirit}.
Help your community learn to entertain itself with its own \textbf{creativity} rather than rely on globalized, electronic, high-consumption entertainment.
Accept that there is no miracle technology that is going to allow \textbf{us} to continue living this endless growth, high consumption, self-indulgent, expanding population, fossil-fueled, presumptuous, human-centred life.
Change is inevitable.
Simplicity is the new “ progress. ” Accept it and be at \textbf{peace} with that.
Create discussion \textbf{groups}, in \textbf{person} and online, about all of these actions.
Help \textbf{others} feel comfortable living simpler lives, taking action, and building a genuinely sustainable future \textbf{world}.
Find a spiritual practice that helps you calm down and see the \textbf{world} with more compassion and patience, and that helps you appreciate the more-than-human \textbf{world}.
Educate yourself, forever.
The issues are complex, non-linear, and linked.
Learn how complex living systems actually \textbf{work}.
Educate yourself about wild nature, evolution, and scientific complexity.
Accept that the universe is beyond comprehension, but continue the effort to comprehend : Localize Read “ Small Arcs of Larger Circles ” by Nora Bateson.
Read “ The Collapse of Complex \textbf{Societies} ” by Joseph Tainter.
Read Arne Naess, Chellis Glendinning, David Abram, and Paul Shepard.
Read Gregory Bateson, Janine Benyus, William Catton, and \textbf{other} \textbf{ecology} \textbf{writers}.
Read Rachel Carson, Basho, Li Po, William Blake, Mary Oliver, Denise Levertov, Gary Snyder, Susan Griffin, Nanao Sakaki, and \textbf{other} poets who honour nature.
Go to \textbf{art} galleries.
Contemplate the connection between creative artistic expression and change in a complex system.
See the \textbf{art} in nature and the nature in \textbf{art}.
Learn about the errors of modern, neoliberal economics, and learn about \textbf{other} \textbf{ways} to approach economics.
Read : N.
Georgescu-Roegen, Herman Daly, Donella Meadows, Mark Anielski.
Learn about how energy really \textbf{works}.
Read Vaclav Smil, Bill Rees, and Howard Odum Read Wendell Berry : “ Solving for Pattern ” and “ Gift of Good Land. ” See if you can fall in \textbf{love} with \textbf{something} that ’s not human.
See if you can fall in \textbf{love} with wild nature.
Even as I engage in global \textbf{battles}, my life revolves around \textbf{family}, neighbours, \textbf{friends}, and finding \textbf{ways} to help strengthen my community.
Protect your local habitat ; preserve a local river, a \textbf{lake}, or forest.
I believe that most genuine “ solutions ” that matter \textbf{will} appear at a community-in-habitat level.
The priorities : Practice equanimity, calmness even in the face of uncertainty or tragedy ; the first rule of all First Aid training : the responder should remain calm.
Help re-establish \textbf{terms} such as the common \textbf{good}, public interest, and collective benefit back into political and social discourse.
Accept that “ the \textbf{world} ” is a complex living system, made from living subsystems out of your control. \textbf{Let} go of “ changing the \textbf{world} ” with human cleverness, and be \textbf{content} to influence your community and ecosystems where you can.
Get creative about helping. \textbf{Talk} with \textbf{friends} and \textbf{colleagues}.
Invent new \textbf{ways} to contribute to the principles of slower consumption, smaller populations, cooperative communities, \textbf{peace}, and restored ecosystems.
There are many actions we can take to help.
Take your pick.
They all count.
Teach them.
Discuss them.
Add to the list.
Collaborate with \textbf{others} who \textbf{share} your values ; divide the complexity into manageable parts.
Simon Grant Take \textbf{care} of \textbf{one} ’s own physical and mental health.
Create daily windows free from anxiety-inducing \textbf{information} ( most electronic \textbf{media} ).
It \textbf{works} best if such a window precedes bed-time so that you can close your \textbf{eyes} without thinking about the \textbf{news}.
Lucas Durand Resources and Links : “ Planetary Boundaries : Exploring the Safe Operating Space for \textbf{Humanity}, ” J.
Rockström, et. al., \textbf{Ecology} and Society, 2009.
The nine planetary boundaries, Stockholm Resilience Centre

% matched lemmas: activist, answer, art, attention, battle, beginning, being, belief, book, campaign, care, charge, child, class, colleague, communication, content, course, creativity, earth, ecology, editor, education, environmentalist, everyone, everything, experience, eye, family, fear, feeling, friend, good, great, group, head, humanity, idea, information, job, lake, lawyer, let, lifestyle, love, may, medium, member, mind, money, movement, music, news, one, other, peace, people, person, politics, question, research, school, share, skill, society, something, spirit, start, strategy, strength, student, talk, term, thing, time, university, us, war, way, weapon, will, witness, woman, work, world, writer, year
\end{textsample}
