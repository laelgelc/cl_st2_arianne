\begin{textsample}{POS Dim 1 – human – Score 92.00 – t023\_human.txt}  \label{ex:f1_pos_004}
Mr.
Mindbomb is a biography of Greenpeace co-founder Robert Lorne Hunter ( 1941-2005 ), told by his \textbf{friends} and \textbf{family}.
The \textbf{book} was edited by Bob ’s wife, 1970s Greenpeace organizer Bobbi Hunter.
Scores of \textbf{campaign} and \textbf{media} \textbf{colleagues} contribute accounts of his life.
A \textbf{picture} emerges of a \textbf{man} who deeply touched \textbf{people} ’s hearts, inspired commitment, helped give \textbf{birth} to the modern \textbf{ecology} \textbf{movement}, and did it all with extraordinary humility and humour.
I met Hunter at the press club and began \textbf{working} with the \textbf{peace} and \textbf{ecology} \textbf{group}, which in 1972 changed its \textbf{name} to the “ Greenpeace Foundation. ” After successful \textbf{campaigns} against US and French nuclear \textbf{testing}, Hunter wanted to \textbf{stage} a global scale environmental action, and we had begun to think about how to publicize forests, rivers, the toxins exposed by Rachel Carson, and the plight of \textbf{other} species.
The big \textbf{idea} came to \textbf{us} from a cetologist from New Zealand.
Dr.
Paul Spong picks up the \textbf{story} here, telling how he approached Hunter in the \textbf{spring} of 1973 with the \textbf{idea} to “ save the \textbf{whales} ” using Greenpeace direct action and mindbomb tactics. “ We met around the pool \textbf{table} at the back of the Cecil Hotel pub, ” Spong recalls, “ drank 25-cent beers, and plotted. ” He tells of taking Hunter to the Vancouver Aquarium and introducing him to the Orca \textbf{whale} Skana, who playfully “ seized Bob ’s \textbf{head} in her mouth. ” This \textbf{experience} — \textbf{feeling} both the power and gentleness of the \textbf{whale} — left Bob inspired and thoroughly committed.
In 1975, we launched the first Greenpeace global \textbf{ecology} action, confronted Russian whalers in the North Pacific, and circulated the photographs and \textbf{film} around the \textbf{world}.
Greenpeace had become a global \textbf{ecology} \textbf{movement}, and four \textbf{years} later, in 1979, we formed Greenpeace International, which now includes national and regional \textbf{organisations} in over 55 countries.
In this \textbf{book}, Mr.
Mindbomb, \textbf{campaign} \textbf{colleagues} from around the \textbf{world} — Carlie Trueman in Canada, Czech sailor George Korotva, Australian \textbf{ecologist} Chris Pash, and \textbf{dozens} more — tell of their adventures with Bob Hunter during Greenpeace \textbf{campaigns}.
Bob never left Greenpeace in his heart, but he stepped away after the formation of a global \textbf{organization} and \textbf{returned} to journalism. \textbf{Other} \textbf{friends} and \textbf{colleagues} pick up the \textbf{story} after Bob ’s Greenpeace career.
Special \textbf{education} \textbf{teacher} Linda Weinberg tells of how she recruited Bob to help stop an LNG plant in Canada, and how he not only devised a protest plan and wrote \textbf{media} \textbf{stories} in support, but how he made the \textbf{work} fun and inspiring.
The LNG plant was never built. \textbf{Media} pioneer Moses Znaimer, a “ Jewish refugee … \textbf{born} in Tajikistan, ” and founder of Canada ’s CityTV and scores of popular television shows, channels, and \textbf{stations}, hired Hunter to be “ the \textbf{world} ’s first \textbf{ecology} reporter. ” He describes Hunter as “ a philosopher of depth and foresight … demonstrating a crazy combination of recklessness and \textbf{courage}. … Bob Hunter didn't have to hector or browbeat or command, because he could persuade, and what he achieved is far \textbf{greater} than being the founder of a company. ” Znaimer writes, ” Hunter galvanized millions in defence of the planet. ” Janine Ferretti, a Professor of Global Development Policy at Boston University, tells how Bob helped her husband lobby \textbf{politicians} for sound ecological policies.
She tells how, after Bob ’s passing, that she began using his 2002 global warming \textbf{book}, 2030 : Thermageddon, with her \textbf{students}. “ Bob cut through the complexity of climate science and the noise of climate \textbf{politics}, ” she writes. “ With his trademark of hard \textbf{facts}, flawless logic and effortless clarity, Bob Hunter asked a new generation to take a stand and to act … I was moved to hear my \textbf{students} reflect on his writing, and I \textbf{witnessed} that, \textbf{years} after his death, Bob is still communicating and inspiring. ” Elizabeth May, former \textbf{executive} \textbf{director} of Sierra Club Canada and Canada ’s first Member of Parliament from the Green Party, writes an Afterword, and environmental \textbf{activist} and Sea Shepherd founder Paul Watson writes an Introduction to the \textbf{book}.
Editor Bobbi Hunter describes her husband ’s message as : “ Follow your heart and \textbf{start} walking in any direction, change \textbf{will} happen … any single \textbf{person} can make a \textbf{difference}.
Just do \textbf{something}. ” Bob ’s \textbf{daughter}, Emily, embraces the sad but noble \textbf{task} of describing her \textbf{father} ’s passing on May 2, 2005, “ with his \textbf{family} surrounding him just as he ’d wished … We each ceremonially took turns speaking to him and telling him what our \textbf{love} for him meant and gave him permission to \textbf{let} go of this plane of existence.
My \textbf{father} taught me a \textbf{lot} about the macro \textbf{stuff} that matters – the planet ’s well-being, our collective well-being, independent thinking, and interdependent \textbf{relationships} – but most of all he taught me to \textbf{love}. ” Fourteen \textbf{years} later, her first \textbf{son} was \textbf{born} on May 2, the anniversary of her \textbf{father} ’s death.
She and her husband Ryan Dyment \textbf{named} the boy Phoenix. “ The \textbf{world} we know \textbf{today} is dying, ” writes Emily. “ But in that same breath, a new \textbf{world} is being \textbf{born} ( or reborn ).
I don't believe my \textbf{father} is still here.
Instead, I believe my \textbf{father} ’s \textbf{spirit} had to pass from this realm, perhaps into another.
Yet I can't deny that there is \textbf{something} here – \textbf{something} bigger that my \textbf{father} ’s \textbf{spirit} was a part of, that is instilled in me, and now is instilled in my \textbf{son}. ” I last spoke to Hunter by \textbf{phone}, from his hospital bed in Mexico, where he was fighting the \textbf{cancer} that finally took him.
He made a comment that still reminds me of his devotion to realism, his upbeat \textbf{spirit}, and his uncanny ability to discover the humour in \textbf{everything} : “ Hey, Rex, ” he exclaimed cheerfully, “ I found out \textbf{today} that my blood type sums up my life ’s philosophy! ” Yeah, what ’s that, Bob?, I asked. “ B-positive! ” Mr.
Mindbomb : Eco-Hero and Greenpeace Co-founder Bob Hunter Greenpeace \textbf{activist} Dr.
Myron MacDonald traces Hunter ’s \textbf{family} heritage to the Hunterston forests of Scotland, \textbf{west} of Glasgow on the Firth of Clyde.
The epigram from the \textbf{family} crest, “ Cursum perficio, ” or “ journey complete, ” implies “ I accomplish my purpose. ” Bob ’s brother Don tells of their \textbf{father} who \textbf{one} \textbf{day} stopped coming home, their modest childhood on the Manitoba prairies raised by their French-Canadian \textbf{mother}, Augustine Bernadette, and of Bob reading the newspaper to his \textbf{mother} after she lost her sight.
He recounts their \textbf{mother} telling long, rambling \textbf{stories} and how Bob acquired the \textbf{art} of the raconteur. “ The big \textbf{difference} was that Bob ’s \textbf{stories} were funny, ” Don recalls, “ interesting, and more to the point.
He learned the \textbf{art} of storytelling by improving on Mom ’s style. ” Edited by Bobbi Hunter Rocky Mountain Books April 2023 Bob Hunter became a \textbf{writer} as a \textbf{child}, creating his own illustrated \textbf{books}.
He set off at \textbf{age} 19 to see the \textbf{world}, packing the typewriter that his \textbf{mother} had given him.
Canadian-British physicist Walt Patterson describes \textbf{meeting} Hunter in London in 1962, hanging out in pubs with aspiring \textbf{writers} and living the bohemian life. “ Bob was a virtuoso storyteller, ” Patterson recalls, “ with a vast repertoire, a rich and resonant speaking voice, and impeccable timing. ” In London, Hunter met Zoe Rahin, two \textbf{years} older and active in the \textbf{peace} \textbf{movement}.
Zoe introduced Hunter to philosopher Bertrand Russell, a life-changing encounter, and Hunter followed Zoe and the renowned philosopher into the pacifist \textbf{movement}.
In turn, Hunter inspired Walt Patterson to use his physics \textbf{background} to support the \textbf{peace} and \textbf{ecology} \textbf{movements}.
In 1970, Patterson edited the UK ’s first environmental magazine and in 1972 he represented \textbf{Friends} of the \textbf{Earth} at the UN Conference on the Human Environment in Stockholm. “ Before long I was a full-time and high-profile environmental campaigner, ” writes Patterson, “ all because Bob had triggered the spark … not only the \textbf{information} and insight he provided but also the burning ardour and headlong exhilaration he brought with it. ” Bob and Zoe married and moved to Canada, where Hunter \textbf{worked} at the Winnipeg Tribune, and their \textbf{son} Conan and \textbf{daughter} Justine were \textbf{born}.
Justine Hunter, now a seasoned reporter at Canada ’s Globe and Mail, picks up the \textbf{story} there, adding observations from her older brother Conan.
She recounts a 1967 \textbf{trip} to Mexico in a Volkswagen van.
On a remote desert road, they were pulled over by Mexican \textbf{police}, who concluded that the “ penniless gringo in bare \textbf{feet}, ” was a poor prospect for a fine or bribe.
On the \textbf{way} home, \textbf{everyone} sick with dysentery, US border guards pulled apart the van \textbf{looking} for drugs.
Conan ’s last \textbf{memory} of Mexico was “ the chrome trim of the headlights sitting on the dusty ground. ” Zoe grew weary of Winnipeg winters and fled to Vancouver, and Bob soon followed.
In 1968, Hunter published his first \textbf{book}, the coming-of-age novel Erebus, nominated for the Canadian Governor General ’s Award, boosting his reputation in Canadian journalism circles.
Justine explains that on the \textbf{west} \textbf{coast} Hunter “ reinvented himself.
The former clean-cut newspaper reporter from Winnipeg now sported long hair and a beard, promoted on billboards as the Vancouver Sun ’ s counterculture columnist. ” Justine recalls how their modest East Vancouver home became a \textbf{meeting} place for Vancouver pacifists opposing US nuclear \textbf{testing} in Alaska.
Conan recalls that, “ Not only were they going to change the \textbf{world}, but they were forming the club that would do it. ” Rod Marining, \textbf{working} as a copy boy at the Vancouver Sun met Hunter in the newspaper lunchroom, and they became \textbf{activist} allies.
Marining ’s \textbf{group} Rent-a-Demo \textbf{staged} protests for worthy causes. “ Bob was rising in respect and popularity, ” recalls Marining. “ I \textbf{started} to notice young \textbf{people} buying the newspaper and going right to his column.
They had found an advocate in the Vancouver Sun. ” In 1971, Hunter published The Storming of the \textbf{Mind}, in which he introduced his \textbf{idea} of the “ \textbf{mind} \textbf{bomb}. ” Activism, he proposed, was not an “ armed struggle ” such as the storming \textbf{images} of the Bastille that \textbf{will} “ explode in \textbf{people} ’s \textbf{heads} … and change perception ”, but rather a “ \textbf{communications} struggle, ” with the \textbf{media} as a “ delivery system [ for ] mindbombs. ” This \textbf{idea} would become a cornerstone of Greenpeace \textbf{activist} philosophy.
In the Vancouver Sun, Hunter wrote a column suggesting that the underground nuclear \textbf{tests} in Alaska might cause a tsunami that could hit the \textbf{coast} of Canada, and he made a placard that read “ Don't Make a Wave. ” Irving and Dorothy Stowe — Quaker pacifists and \textbf{US} expatriates — proposed the \textbf{group} call itself the “ Don't Make a Wave Committee, ” and they registered the \textbf{organization} as a Canadian charity.
When Irving Stowe ended a \textbf{meeting} with the salutation “ \textbf{Peace}, ” \textbf{ecology} \textbf{activist} Bill Darnell added quietly, “ Make it a green \textbf{peace}, ” including the ecological crisis and creating a new \textbf{image} that would soon reverberate around the \textbf{world}.
The \textbf{group} borrowed a Quaker \textbf{strategy}, proposed by Marie Bohlen, to sail a \textbf{boat} into the nuclear \textbf{test} \textbf{site}, and in 1971, Hunter joined this voyage, reporting for the Vancouver Sun.

% matched lemmas: activist, age, art, background, bear, birth, boat, bomb, book, campaign, cancer, child, coast, colleague, communications, courage, daughter, day, difference, director, dozen, earth, ecologist, ecology, education, everyone, everything, executive, experience, fact, family, father, feeling, film, foot, friend, great, group, head, idea, image, information, let, look, lot, love, man, media, medium, meeting, memory, mind, mother, movement, name, one, organisation, organization, other, peace, people, person, phone, picture, police, politician, politics, relationship, return, site, something, son, spirit, spring, stage, start, station, story, strategy, student, stuff, table, task, teacher, test, testing, today, trip, us, way, west, whale, will, witness, work, world, writer, year
\end{textsample}
