\begin{textsample}{POS Dim 1 – human – Score 75.00 – t933\_human.txt}  \label{ex:f1_pos_019}
How the Sisiutl symbol came to be part of Greenpeace ’s identity is both an important part of our origins and a troubling \textbf{story} of cultural appropriation.
Now, we ’re taking a closer \textbf{look} at our \textbf{history}.
The \textbf{organisation} ’s first \textbf{campaign} was protesting \textbf{US} nuclear \textbf{testing} at Amchitka, Alaska.
To do this, a small fishing vessel, the Phyllis Cormack, was chartered to take a \textbf{crew} of a \textbf{dozen} to the \textbf{front} lines of the nuclear \textbf{test}.
Along the \textbf{way} it stopped at the indigenous community of ‘ Yalis ( or ‘ Yalis, its actual Kwakwaka’wakw \textbf{name} ) on Cormorant Island, located on Canada ’s \textbf{west} \textbf{coast}. ‘ Yalis is located in the traditional territory of the ‘ Namgis First Nation, and here \textbf{members} of \textbf{other} several Kwakwaka’wakw nations gifted the \textbf{crew} ( which included Greenpeace founder Bob Hunter ) with wild coho salmon and a blessing for their journey.
This also happened to be the young \textbf{organisation} ’s first interaction with an indigenous community.
As they continued on to Amchitka, the US Coastguard prevented the Greenpeace \textbf{crew} from going any further without a proper entry visa, and eventually with deflated \textbf{spirits} they \textbf{headed} back to Vancouver, British Columbia.
En route, the Phyllis Cormack stopped once again in ‘ Yalis, where the \textbf{crew} was invited to a celebration in their honor at the ceremonial Big House.
They were acknowledged by \textbf{members} of the ‘ Namgis and \textbf{other} Kwakwaka’wakw nations for their \textbf{courage} in standing up to the Unites States military industrial complex with a special ceremony that included song, dance and food.
The \textbf{crew} had Kwakwaka’wakw traditional regalia placed on them, with eagle down on their \textbf{heads}, and they even participated in \textbf{one} of the ceremonial dances – the Peace Dance ( or Tla’sala, in the Kwak’wala \textbf{language} ).
This was all a \textbf{great} honor.
Bob writes in \textbf{Warriors} of the Rainbow that the \textbf{crew}, “ were made into brothers of the [ Kwakwaka’wakw ] \textbf{people}. ” – or, in \textbf{other} \textbf{words}, they were ceremonially adopted by the community.
All of this uplifted their \textbf{spirits}. [ I should clarify that ’ Kwakwaka’wakw ’ ( mistakenly referred to in the recent past as ‘ Kwakiutl ’ ) is not \textbf{one} First Nation but rather are the \textbf{people} who speak Kwak’wala, and so there are many nations in the region – seventeen by \textbf{today} ’s count.
The \textbf{people} that Bob and the \textbf{crew} encountered to and from ‘ Yalis, in addition to the ‘ Namgis, would have belonged to any of these seventeen nations ].
Shortly after that important ceremony Bob was given a blue cloth with a Sisiutl crest by a leading \textbf{member} of \textbf{one} of the Kwakwaka’wakw communities ( according to some of the original \textbf{crew} \textbf{members}, this \textbf{person} was James Sewid, who came from \textbf{one} of the surrounding \textbf{villages} and who was also the first elected Chief Councilor of the ‘ Namgis First Nation at ‘ Yalis ).
The Sisiutl symbol is a powerful spiritual crest with many origin \textbf{stories} amongst various coastal First Nations of British Columbia, including the Kwakwaka’wakw.
The Sisiutl itself is a supernatural creature with two serpent \textbf{heads}, \textbf{one} at each end and a human \textbf{head} in the \textbf{middle}.
For the Kwakwaka’wakw, the Sisiutl symbolizes the balance of life between \textbf{good} and evil.
Wearing it and telling its \textbf{stories} helps \textbf{warriors} and healers with their \textbf{work}.
It is also used to protect canoes and Big \textbf{Houses}.
Only certain Kwakwaka’wakw individuals and \textbf{families} ( as well as \textbf{other} coastal First Nations ) have the cultural and spiritual rights to display the Sisiutl in ceremonies, so to have the Sisiutl \textbf{shared} with Bob and Greenpeace was a huge honor.
We don't actually know at this point what the original Sisiutl on that \textbf{piece} of blue cloth \textbf{looked} like or why it was \textbf{shared} with Bob.
We can only surmise that given its meanings of protection and \textbf{general} association with the sea and \textbf{warriors} that it was a blessing or gift intended to help the Greenpeace \textbf{team} succeed in its important \textbf{work} after the Amchitka \textbf{campaign}.
This \textbf{story} was originally \textbf{posted} by Greenpeace Canada.
What we do know is that its form was dramatically altered sometime between 1974 and 1975, when Bob and \textbf{other} Greenpeace \textbf{members} decided to use it as the basis for a new symbol for the anti-commercial whaling \textbf{campaign} that was to launch in 1975.
Bob wrote in \textbf{Warriors} of the Rainbow that he believed Greenpeace had permission to adapt the crest because the original \textbf{crew} had been adopted into the Kwakwaka’wakw during the 1971 ceremony.
In How to Change the \textbf{World} you can see rare footage of the altered Sisiutl/Greenpeace symbol being painted on the Phyllis Cormack ’s sail for the first \textbf{time}, and then subsequently worn as t-shirts by Greenpeace \textbf{volunteers} and staff, on \textbf{office} \textbf{walls}, etc.
And, interestingly, there are also photos of Bob showing a tattered flag with the hybrid Sisiutl/Greenpeace symbol to Kwakwaka’wakw elders when he and \textbf{other} \textbf{crew} \textbf{members} of the anti-whaling \textbf{campaign} \textbf{returned} to ‘ Yalis later in 1975 for another ceremony at the Big House.
Bob recounts in \textbf{Warriors} of the Rainbow that they \textbf{returned} to ‘ Yalis as a \textbf{way} to honor the Kwakwaka’wakw, “ to give them the flag we had flown since embarking on the voyage to save the \textbf{whales} ”.
But in making that fateful decision to adopt and adapt the Sisiutl as Greenpeace ’s own, the sacred crest was in \textbf{fact} inappropriately altered and hybridized, its meaning forever changed.
With the benefit of hindsight, and with \textbf{greater} \textbf{understanding} \textbf{today} of Kwakwaka’wakw customs and \textbf{traditions}, we believe that while the Sisiutl symbol itself was \textbf{shared} with Greenpeace, we didn't have the right to alter it and make it our own – especially to the degree that the hybrid continues to be displayed in Greenpeace \textbf{offices} around the \textbf{world}.
In addition to the continued widespread use of the altered symbol there are also inaccurate \textbf{stories} of how we came by it, and misunderstandings of what the Sisiutl actually is.
For example, over the \textbf{years} Greenpeace has believed, and has communicated out to the \textbf{world}, that the Sisiutl symbol is of two \textbf{whales} forming the ‘ infinite cycle of nature ’, and this is incorrect.
With all this in \textbf{mind}, Greenpeace Canada with support from Greenpeace International approached \textbf{representatives} of Kwakwaka’wakw communities to inquire about redesigning or restoring the Sisiutl to a more culturally appropriate form, and to explore the possibility of having it re-dedicated by \textbf{representatives} of these First Nations in the most meaningful and culturally appropriate \textbf{way}. \textbf{One} \textbf{representative} suggested we commission well-known Kwakwaka’wakw \textbf{artist} and cultural leader Beau Dick to help \textbf{us} with this important \textbf{task}.
We approached Beau, and he agreed to redesign the Sisiutl Crest for our use in perpetuity, but with our written promise to never alter it again.
Following protocol, we committed to a re-dedication ceremony of the Sisiutl crest at a potlatch that would also renew ties with the Kwakwaka’wakw hereditary chiefs.
We again \textbf{returned} to ‘ Yalis in March 2015, and there we had the honor and privilege of having the renewal ceremony take place at the Big House, at a potlatch hosted by the Willie Family, an important hereditary \textbf{family} in region ( see \textbf{family} \textbf{representative} Mike Willie ’s blog ).
Greenpeace International Executive Director Kumi Naidoo and Greenpeace Canada Executive Director Joanna Kerr along with staff, joined surviving \textbf{members} of the original 1971 \textbf{crew} and Mike Willie, on the potlatch floor for the special ceremony.
Kumi and Joanna addressed the many hereditary chiefs and \textbf{members} of ‘ Yalis and surrounding communities, and apologized for the misuse and misappropriation of the original Sisiutl crest.
Beau ’s redesign of the Sisiutl was then revealed, with this historic occasion \textbf{witnessed} by all in attendance, including a few of the community \textbf{members} who were present at the 1971 ceremony as youngsters.
Later that \textbf{night}, Mark Worthing ( who originally brought the altered Sisiutl/Greenpeace symbol to our \textbf{attention} and has been crucial to the renewal project ) and myself were taken to the back area of the Big House.
A few \textbf{months} ago I had the privilege to \textbf{watch} an advance screening of the extraordinary documentary on Greenpeace ’s early \textbf{history}, How to Change the World.
Jerry Rothwell ’s compelling \textbf{film} brings to life an important and formative chapter of how Greenpeace began and where its roots lie, as seen through the \textbf{eyes} of late cofounder Bob Hunter.
I felt deeply moved and nerve-wracked by that moment, disbelieving of what was happening to me as they placed traditional regalia on me, with the headdress sitting firmly on my glasses and eagle down atop it beginning to fall gently all around me.
We were given a quick \textbf{lesson} on how to dance the \textbf{Peace} Dance before being ushered, with several community \textbf{members}, onto the ceremonial floor.
And on the floor, I just lost track of \textbf{time} and place and, well, self.
Barbara Stowe, \textbf{daughter} of Greenpeace co-founders Irving and Dorothy Stowe, later said to me : “ That was \textbf{one} of the most resonant and moving \textbf{experiences} of my life.
The honor accorded to you and Mark was obvious, and through you this community had just accorded enormous respect and honor to Greenpeace. ” That moment, in echoing that historic \textbf{day} in the fall of 1971 when the \textbf{crew} of the Phyllis Cormack were also asked to dance the \textbf{Peace} Dance in full regalia, I believe also helped reinvigorate the direct ties between our Greenpeace community and the Kwakwaka’wakw.
We once again have become interconnected deeply and ceremonially with the Kwakwaka’wakw as a whole.
As meaningful and moving as that ceremony and potlatch was, equally as important was re-introducing the new Sisiutl crest to our fleet of Greenpeace ships ( especially since that is how our \textbf{relationship} with the Sisiutl began- with our first ship, the Phyllis Cormack ).
It was very serendipitous then that \textbf{one} of our vessels, the Esperanza, happened to be in my region a few \textbf{months} after the Willie Family Potlatch.
We took advantage of this and organized a ceremonial event in the Squamish First Nation traditional territory of North Vancouver, British Columbia.
Squamish \textbf{representatives} officially welcomed the Esperanza.
Beau Dick and his delegation from Kwakwaka’wakw communities dressed in full regalia formally blessed the ship and presented the renewed Sisiutl Crest, in the form of a flag, to ship ’s Captain Pep Barpal and \textbf{others} from the Greenpeace community.
The flag with the new Sisiutl Crest and the Greenpeace logo placed beside it was then hoisted to fly proudly alongside the international Greenpeace flag.
As a result, the renewed Sisiutl has now been brought back in a culturally appropriate form to the Greenpeace fleet.
How to Change the World is an important \textbf{film} that materially demonstrates how a small \textbf{group} of impassioned individuals can make a \textbf{difference} in the \textbf{world}, but in a \textbf{way} that also authentically reveals all-too human frailties.
These frailties in some \textbf{way} also helps underscore what happened to the Sisiutl – in the \textbf{desire} to do \textbf{good} in the \textbf{world}, ego-driven decisions sometimes lead to mistakes that leave lasting legacies that need fixing.
We acknowledge the heroics of those incredible \textbf{people} involved in the heady \textbf{days} of the \textbf{organization} ’s origins, and in the case of the Sisiutl we must now also take responsibility for the legacy of its cultural appropriation.
This is \textbf{one} small but important step at reconciliation with this particular \textbf{group} of indigenous \textbf{people}.
Aside from following proper protocols and undertaking ceremonies that now authorize \textbf{us} to carry the new Sisiutl, we are rewriting the \textbf{story} of how we came to the symbol and what it actually means ( this blog \textbf{post} is a means towards that end ).
We plan to circulate not only prints of Beau Dick ’s Sisiutl crest as we retire the old hybrid \textbf{one}, but also the newly corrected narrative of the Sisiutl-Greenpeace \textbf{story}, to Greenpeace \textbf{offices} around the \textbf{word}.
As I mentioned previously, there is significant misunderstanding of what the Sisiutl means and how we came by this amongst the Greenpeace community so we \textbf{will} continue in our efforts to reconcile the legacy of these past \textbf{beliefs} and mistakes.
The \textbf{film} is rich with passionate \textbf{people} who founded this intrepid \textbf{organization}, and overall I felt a \textbf{sense} of connectedness through my \textbf{work} as a forest campaigner to the inspiring legacy left by such a small but powerful and dedicated \textbf{group} of \textbf{women} and \textbf{men}.
Managing the Sisutl Renewal Project has a been deeply meaningful journey not only towards helping the \textbf{organization} decolonize a symbol sacred to Pacific Northwest \textbf{peoples} that is intertwined with Greenpeace ’s identity, but also towards decolonizing my own \textbf{mind} and thinking processes.
It ’s an ongoing process.
It has also been a \textbf{great} metaphor for a \textbf{mind} \textbf{bomb}.
Bob would be proud.
Eduardo Sousa is senior forest campaigner for Greenpeace Canada However there was \textbf{one} \textbf{image} seen throughout the \textbf{film} that brought both a knowing smile and a slight wince.
This \textbf{image} was that of the old Greenpeace \textbf{ecology} and \textbf{peace} logo encircled by a double-headed serpent ( at the \textbf{time} thought to be two \textbf{whales} ).
Many \textbf{times} seen emblazoned on Greenpeace ships, \textbf{office} windows, etc, the \textbf{image} is in \textbf{fact} an altered, hybridized \textbf{version} of an ancient crest called the Sisiutl and it is sacred to ( amongst \textbf{others} ) the indigenous Kwakwaka’wakw, Heiltsuk, Nuxalk and Nuu-cha-nulth \textbf{peoples} of the Pacific Northwest.
How this symbol came to appear most prominently on the sail of the first Greenpeace ship, Phyllis Cormack, and continues to be part of the Greenpeace identity is an intimate and intrinsic part of Greenpeace ’s origin \textbf{story}.
It is also a \textbf{story} of cultural appropriation.
And my \textbf{job} for the past \textbf{year} has been managing the Sisiutl Renewal Project, with the goal of both restoring the dignity of the crest at Greenpeace and the \textbf{spirit} with which it was \textbf{shared} with \textbf{us}.
The Sisiutl Renewal Project is \textbf{one} small but important step towards decolonizing our \textbf{organization}. \textbf{Other} steps include the recently developed policies Greenpeace USA and Greenpeace Canada each now have in reconciling and \textbf{working} with indigenous \textbf{peoples}.
But \textbf{let} ’s go back to 1971, to Greenpeace ’s very origins.

% matched lemmas: artist, attention, belief, bomb, campaign, coast, courage, crew, daughter, day, desire, difference, dozen, ecology, experience, eye, fact, family, film, front, general, good, great, group, head, history, house, image, job, language, lesson, let, look, man, member, middle, mind, month, name, night, office, one, organisation, organization, other, peace, people, person, piece, post, relationship, representative, return, sense, share, spirit, story, task, team, test, testing, time, today, tradition, understanding, us, version, village, volunteer, wall, warrior, watch, way, west, whale, will, witness, woman, word, work, world, year
\end{textsample}
