\begin{textsample}{POS Dim 1 – human – Score 80.00 – t228\_human.txt}  \label{ex:f1_pos_015}
From 1974 to 1982, I served as \textbf{photographer} on Greenpeace \textbf{campaigns}.
Here are a \textbf{dozen} more photographs from those \textbf{years} along with some \textbf{memories} that they evoke : In 1976, we chartered a Canadian minesweeper, the James Bay, that would be able to go farther into the Pacific and could keep pace with the whaling fleets.
This photograph was taken on the last \textbf{day} in Canada, before departing to find the whalers, who were operating near Hawaii.
Prior to 1975, Greenpeace had no public \textbf{office}, and we often met in our own kitchens, in coffee shops, and in pubs.
The skipper, John Cormack, and \textbf{others} on the \textbf{crew}, did not drink, but pub culture was part of Greenpeace \textbf{history}.
Many \textbf{campaigns} were conceived and fleshed out in Canadian pubs.
David “ Walrus ” Garrick ( left ) was the ship ’s cook and Greenpeace librarian, with a vast collection of \textbf{ecology} journals, legislation, and \textbf{books}, essentially the \textbf{search} \textbf{engine} of our \textbf{movement}.
Bob Hunter, the \textbf{author} of The Storming of the \textbf{Mind}, \textbf{Warriors} of the Rainbow, and \textbf{other} \textbf{books} was already a Canadian \textbf{media} legend, a brilliant \textbf{campaign} strategist, and a life-long \textbf{ecologist}.
In 2003, Hunter wrote Thermageddon : Countdown to 2030, about global heating from human carbon emissions, and our failure to act on the science.
In 1975, we first disrupted the Russian Pacific whaling fleet.
In 1976, we actually stopped this harpoon ship dead in the water.
The previous \textbf{year}, they had fired a 250-pound exploding harpoon over our \textbf{heads}, killing a female Sperm \textbf{whale}, and our \textbf{films} and photographs of that encounter circled the globe via public \textbf{media}.
Obviously, \textbf{word} had come down from the Soviet bosses to avoid confrontation, so when we disrupted the hunt this \textbf{year}, the ships throttled down and stopped.
We had been \textbf{working} on this \textbf{campaign} for three \textbf{years}, we were all exhausted, we were broke, and operating on borrowed \textbf{money}.
No \textbf{one} was getting paid.
Bob was in tears as we approached this idle ship.
He reached out and laid his \textbf{hand} on the rusting hull.
This was a culmination of the \textbf{dream} we had nourished for three long \textbf{years}, and the moment felt historic.
Probably the most heartbreaking \textbf{campaign} I ever \textbf{worked} on was to stop the slaughter of weeks-old infant Harp seals off Labrador \textbf{coast} in eastern Canada.
Industrial-scale ships from Norway had been appearing on the ice floes for \textbf{years}, clubbing infant seals only \textbf{weeks} old, because their white coats were highly prized by the fashion industry.
This photograph was taken from our helicopter, \textbf{looking} down on the ravaged seal nursery.
On the ice, the cries of the infant seals tore me up, as they sounded so much like a human infant in pain.
Most of \textbf{us} were in tears before we left.
This \textbf{campaign} brought \textbf{us} into conflict with some Newfoundlanders, who got a few \textbf{weeks} of \textbf{work} from the Norwegian fur companies.
In the end, we compromised with the “ landsmen, ” who hunted seals near the shoreline, avoiding a clash with the local \textbf{hunters}, and focusing on the Norwegian industry and the European fashion trade.
The eventual ban on seal furs in Europe also hurt some of our Indigenous \textbf{friends} in Canada, who hunted seals.
The Inuit \textbf{people}, for example, objected to our \textbf{campaign} because even though they were not the target, the ban in Europe affected their lives.
The Inuit hunted seals for subsistence, and the lean meat is rich in iron, zinc, and vitamins.
Greenpeace apologized for the disruption and agreed to avoid any conflict with the Indigenous seal hunt.
Our action in Labrador, however, \textbf{may} have saved the Harp seal from extinction, and certainly gave thousands of seal pups the opportunity to mature and live a normal seal ’s life.
Bob Hunter and I had met Allen Ginsberg in Vancouver, and \textbf{shared} \textbf{ideas} about \textbf{ecology}, Buddhism, and direct action.
Meanwhile, I was \textbf{working} with a Colorado \textbf{group}, The Rock Flats Truth Force, in an \textbf{alliance} that included Greenpeace, to close the Rocky Flats nuclear facility that made the triggers for the entire US nuclear \textbf{weapons} arsenal.
The soil and air around the facility were contaminated with tritium and \textbf{other} radioactive by-products, and Colorado doctors had linked the \textbf{cancer} spike in the region to this contamination.
Ginsberg taught at the Buddhist Naropa Institute in Boulder, near the nuclear factory.
He had just written a now-famous poem, “ Plutonian Ode ” : “ My oratory advances on your vaunted Mystery … Behind your concrete \& iron \textbf{walls}, inside your fortress.. ” We attended his poetry reading in Boulder and invited him to the occupation of the Rocky Flats \textbf{site}.
Ginsberg sat with \textbf{others} on the rail tracks, blocking shipments of uranium to the \textbf{site}, and was arrested.
Ginsberg was a committed pacifist and \textbf{ecologist}, \textbf{one} of my mentors and heroes.
He passed away in 1997, leaving behind a vast record of poetry, documenting decades of social upheaval, spiritual insight, and creative consciousness.
The Clamshell Alliance had been formed in 1976 to oppose the Seabrook, New Hampshire power \textbf{station}.
I had become \textbf{friends} with \textbf{photographer} Lionel Delevingne, who was documenting the anti-nuclear \textbf{movement}, and we went to Seabrook together in the \textbf{summer} of 1980.
Some 3,000 \textbf{activists} arrived at the \textbf{site}, some began to cut through wire fencing, and \textbf{others} blocked delivery of the plant ’s first nuclear reactor containment vessel.
Police attempted to clear the road and fence line with tear gas, shortly before this \textbf{picture} was taken.
The Clamshell Alliance, a decade before the Chornobyl meltdown, inspired clean energy \textbf{activists} around the \textbf{world} by following Albert Einstein ’s advice to take “ the \textbf{facts} of atomic energy.. to the \textbf{village} square ”.
Since then, Chornobyl, Fukushima, a still-intact nuclear \textbf{weapons} arsenal, a persistently unsolved nuclear waste boondoggle, the impact of radioactivity, and the carbon-cost of mining, confirm that nuclear energy is still not remotely green or peaceful.
Greenpeace, originally an anti-war \textbf{group}, had been active in local ecological issues in Canada, but wanted to \textbf{stage} an \textbf{ecology} action that would have global impact.
In 1972, Canadian \textbf{whale} scientist Paul Spong met \textbf{author} Farley Mowat — who had just published A \textbf{Whale} for the Killing — and learned about the near-collapse of \textbf{whale} populations in all oceans of the \textbf{world}.
Spong came to Greenpeace, wanting to use the sea-going confrontation tactics to stop the \textbf{whale} slaughter.
This was it!
We all felt that this \textbf{campaign} would help launch a global \textbf{ecology} action \textbf{movement}, and we set to \textbf{work}.
In January of 1981, as we \textbf{worked} in our Vancouver Greenpeace \textbf{office}, Rod Marining read in the newspaper that an oil consortium planned on bringing an oil tanker in the Salish Sea, the inside strait between the Canadian mainland and Vancouver Island.
They announced that the tanker would only be loaded with water and they intended to demonstrate how safe it would be to build an oil \textbf{port} in these waters.
The tanker was expected in a few \textbf{days}, they announced, and was “ just a \textbf{test}. ” As we discussed what to do, our \textbf{office} \textbf{manager} Julie McMaster said off-handedly, “ Why don't you \textbf{stage} a \textbf{test} blockade? ” We all laughed.
Yes, this would be perfect.
We announced that \textbf{day} that we were launching a “ \textbf{test} blockade ” of the supertanker, the S/S B.T.
San Diego, a 188-thousand tonne oil tanker, the largest ever built on the \textbf{west} \textbf{coast} of North America.
We organized a flotilla of \textbf{boats} from Vancouver, Victoria, and Seattle, and \textbf{headed} out to Juan \textbf{de} Fuca Strait, the entrance from the Pacific.
This photograph shows our \textbf{friends} from Seattle, who had opened a Greenpeace \textbf{office} there.
We stopped the tanker dead in the water and were arrested by the US Coast Guard.
As they hauled \textbf{us} up the wharf in Washington, we told the \textbf{media} that this was “ just a \textbf{test} ” and we were going to now \textbf{test} the local jail.
The \textbf{police} appeared to be on our side, and treated \textbf{us} kindly.
They went out to buy \textbf{us} dinner, \textbf{returned} with a large bag of sandwiches, dropped it on the \textbf{table} in the \textbf{middle} of our cell and said, “ Test this. ” The supertanker \textbf{campaign} was always \textbf{one} of my favorites because the \textbf{idea} came spontaneously from our \textbf{office} \textbf{manager}, and we organized the entire project in a few \textbf{days}, we spent almost no \textbf{money}, the “ just a \textbf{test} ” humour endured through the whole exercise, and \textbf{best} of all, we won over public and political sentiment.
Captain Cormack, who skippered the first Greenpeace action to stop nuclear \textbf{bomb} \textbf{testing}, agreed to contribute his \textbf{services} and his fishing \textbf{boat}.
In February 1975, we announced the plan to the public in an abandoned Canadian Navy hanger at Vancouver ’s Jericho Beach.
Event promoter Alan Clapp ( left, sitting next to Bob Hunter ) negotiated with the Navy, and organized the \textbf{media} conference. \textbf{Crew} \textbf{member}, musician Mel Gregory sits in the centre ( with a headband ). \textbf{Engine} and inflatable-boat mechanic, Carlie Trueman, stands behind ( in pigtails ).
Vancouver \textbf{activist} and \textbf{campaign} \textbf{media} \textbf{director} Rod Marining sits on the far right.
At the \textbf{time}, this felt like just another \textbf{day} in the \textbf{campaign} ; in retrospect, this was the moment that Greenpeace announced to the \textbf{world} that we would put human lives in harm ’s \textbf{way} to save \textbf{other} creatures, that there was going to be a global \textbf{ecology} \textbf{movement} to save the \textbf{Earth}.
In 1975, we \textbf{headed} out in the fishing \textbf{boat}, Phyllis Cormack, to find the whalers, somewhat naive about the \textbf{task} before \textbf{us}.
As we \textbf{headed} \textbf{north} along the \textbf{west} \textbf{coast} of Canada, we tucked into Rose Harbour on Kunghit Island, the southern tip of Haida Gwaii, and discovered this abandoned whaling \textbf{station}.
The Haida, Inuit, and \textbf{other} Aboriginal nations practiced sustainable sustenance whaling for \textbf{centuries} before colonization by Europeans.
On the \textbf{other} \textbf{hand}, industrial, commercial whaling on Canada ’s \textbf{west} \textbf{coast} lasted only 61 \textbf{years}, from 1905 to 1967.
Rose Harbour was \textbf{one} of five shore \textbf{stations} that processed \textbf{whales}, including Blue, Fin, Humpback, Sei, Sperm, Right, Gray, Minke, and Baird ’s beaked \textbf{whales}.
Some 25,000 \textbf{whales} were caught before the populations collapsed and the industry closed.
The beach at Rose Harbour was still littered with \textbf{whale} bones, rusting winches, and harpoon \textbf{heads}.
It felt as if we had stumbled upon the remains of a horrendous \textbf{battle}, a lasting reminder of the limits to \textbf{humanity} ’s industrial plunder of our fragile \textbf{Earth}.
During the first \textbf{whale} \textbf{campaign} in May 1975, we passed by remote Triangle Island on our \textbf{way} to the Dellwood Seamounts where we thought we might find \textbf{whaling} fleets.
We spent a \textbf{morning} out of the wind, preparing to venture into the open ocean.
This photograph reminds me of what a \textbf{relationship} to the \textbf{ecology} of \textbf{one} ’s habitat meant for me then, and still now : Protecting what is wild and free in the \textbf{world}, unencumbered by human cleverness and enterprise.
I sometimes \textbf{dream} of a \textbf{world} in which all creatures, including humans, could conduct their \textbf{relationships} with their habitat in \textbf{peace}.
Flocks of petrels fly each \textbf{night} into these remote \textbf{rocks} and burrow under the wind-pruned salmonberry and salal.
Gulls, cormorants, auklets, oyster-catchers, and \textbf{other} sea birds nest among the \textbf{rocks} here and \textbf{north} along the \textbf{coast} of Haida Gwaii.
We didn't find the whalers at the Dellwood Seamounts, but we began to learn how to be a half-decent ship ’s \textbf{crew}.
After the first \textbf{whale} confrontation with Russian whalers in 1975, we had to travel two \textbf{days} into San Francisco to process \textbf{film} and send \textbf{images} out on the wire \textbf{services}.
Of \textbf{course}, there was no Internet, and we had no capacity to process \textbf{film} on \textbf{board} our vessel.
When we arrived at the dock in San Francisco, all the major \textbf{media} \textbf{services} — UPI, AP, Reuters, and so forth — met \textbf{us} and clamored for \textbf{pictures}.
We made an agreement with The Associated Press to process the \textbf{film} at their studio and make \textbf{images} available to all the \textbf{services}.
The 16mm \textbf{film} aired that \textbf{night} on CBS News in the United States.
I rose early the next \textbf{morning}, and \textbf{headed} straight out to find the newspapers, to see if my \textbf{images} appeared. “ Above the fold ” on the \textbf{front} page was comparable to being on top of the Google \textbf{search} \textbf{today}.
When I saw the harpoon ship \textbf{image} on the cover of the San Francisco Chronicle, I captured the moment.
Later at the newsstand, I saw our photographs in most of the \textbf{world} ’s newspapers. \textbf{Today}, we move so many \textbf{images} around the \textbf{world}, so quickly, it might be difficult to imagine how this felt in 1975, but, this felt to me like the achievement of a goal two \textbf{years} in the making.
Greenpeace was never the same after this \textbf{day}.
Our little Canadian \textbf{peace} and \textbf{ecology} project was about to go global.

% matched lemmas: activist, alliance, author, battle, board, boat, bomb, book, campaign, cancer, century, coast, course, crew, day, de, director, dozen, dream, earth, ecologist, ecology, engine, fact, film, friend, front, good, group, hand, head, history, humanity, hunter, idea, image, look, manager, may, media, medium, member, memory, middle, mind, money, morning, movement, night, north, office, one, other, peace, people, photographer, picture, police, port, relationship, return, rock, search, service, share, site, stage, station, summer, table, task, test, testing, time, today, us, village, wall, warrior, way, weapon, week, west, whale, word, work, world, year
\end{textsample}
