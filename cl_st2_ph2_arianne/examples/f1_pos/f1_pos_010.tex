\begin{textsample}{POS Dim 1 – human – Score 85.00 – t320\_human.txt}  \label{ex:f1_pos_010}
On March 8, International Women ’s Day, Dr.
Vandana Shiva published a short address, in which she examined the connection between the colonization of Earth and of \textbf{women}.
She discussed food security and Indigenous seed protection as foundations for \textbf{women} ’s emancipation globally.
Stephanie Mills gave a famous college commencement address in 1969, “ The Future Is a Cruel Hoax, ” in which she anticipated vast ecological impact from runaway human population and consumption, and announced her decision to not bring \textbf{children} into this \textbf{world}, a commitment she called “ the ecofeminist \textbf{version} of burning a draft card. ” I met Mills at \textbf{ecology} events in San Francisco in the 1970s, as she became influential as a \textbf{writer} and \textbf{editor} for the \textbf{Friends} of the \textbf{Earth} ’s Not \textbf{Man} Apart magazine.
Mills published seven \textbf{books}, including Epicurean Simplicity ( 2002 ), Tough Little Beauties ( 2007 ), and the 1989 Whatever Happened to \textbf{Ecology}?, a personal narrative of her ecological journey and \textbf{work} in the bioregional \textbf{movement}, ( New Catalyst \textbf{Books}, edition, 2008 ). “ The \textbf{Earth} is an organism of organisms, an interrelated whole, thriving in balance, with no preference for \textbf{one} species over another, ” she wrote. “ The \textbf{task} of our species is to find our \textbf{way} back into the web. ” Donella Meadows, earned a Ph.D. in biophysics from Harvard in 1968, became a \textbf{research} fellow in system dynamics at the Massachusetts Institute of Technology, and in 1972, she was the lead \textbf{author} of the ground-breaking study Limits to Growth ( with her husband Dennis Meadows, Jørgen Randers, and William Behrens, for the Club of Rome ).
I met Meadows in 1980 at MIT.
We passed a \textbf{summer} \textbf{day} on the lawn, where she \textbf{talked} about systems theory and how her \textbf{team} had arrived at their projections of Earth ’s natural limits to economic growth.
Meadows ’ insights now appear prophetic.
Limits to Growth : The 30-Year Update shows how well the original predictions have held up as industrial \textbf{society} has plundered and degraded wild ecosystems to fuel global economic growth, while creating a waste stream that toxifies our ecosystems and heats up the atmosphere.
In her 1999 \textbf{book}, Leverage Points, Meadows examines the most effective \textbf{ways} to intervene in complex systems.
Typically, she writes, \textbf{world} leaders exhibit a “ backward intuition … pushing with all their might in the wrong direction. ” She suggests that typical efforts to mitigate environmental decline — market subsidies, taxes, and regulations — are the least effective \textbf{ways} to alter a complex system.
Far more effective are efforts to \textbf{work} with the system ’s self-organized structures and goals.
The most effective of all are efforts that \textbf{work} with the paradigm out of which the system ’s structures and goals arise, and to transcend the social paradigms that blind \textbf{us} to the system as a whole.
In her 1995 \textbf{book}, Paradigms in Progress, economist Hazel Henderson, made pioneering contributions to the \textbf{idea} of an ecological economy, tracking ecosystem health and human well-being rather than mere financial activity.
In My \textbf{Name} is Chellis and I ’m in Recovery from Western Civilization, ( Shambhala, 1994 ), Chellis Glendinning, compares the ecological crisis to personal trauma, and recovery.
She points to deep \textbf{ecology} and Indigenous wisdom as paths back to our “ innate wholeness. ” Janine Benyus in Biomimicry, 2002, and Nancy Jack Todd in A Safe and Sustainable World, 2006, introduced innovative \textbf{ideas} about how to use nature ’s own patterns to fashion \textbf{society} through ecological design.
Nora Bateson founded the “ Warm Data Labs ” as a means to help \textbf{groups} shift discussions away from typical biases or binary squabbles to what she calls ” trans-contextual mutual learning. ” Within living ecosystems, all learning is mutual, and a reflection of various contexts, each \textbf{one} influenced by \textbf{other} contexts.
In 2017, the Harvard Innovation Lab selected her \textbf{book}, Small Arcs of Larger Circles : Framing through \textbf{Other} Patterns, ( Triarchy Press, 2016 ), as a core text for incoming \textbf{students}. “ There is no \textbf{language} to define the spiraling processes of the vast context we are participants in, ” Bateson writes. “ We do not have \textbf{names} for the patterns of interdependency. ” Alice Friedemann \textbf{shares} a vast knowledge of energy systems in her \textbf{books} and essays.
Friedemann operated logistic networks for some of the \textbf{world} ’s largest delivery systems.
Her \textbf{book}, When Trucks Stop Running : Energy and the Future of Transportation, ” ( Springer, 2015 ) surveys our dependence on heavy freight, and the challenges ahead to replace diesel \textbf{engines} with electric motors for ships and large trucks.
Friedemann publishes regularly at Energyskeptic and Resilience, on energy, agriculture, and transportation.
Her most recent \textbf{book}, Life after Fossil Fuels : A Reality Check on Alternative Energy, ( Springer, 2021, with free chapter previews ), examines the range of energy alternatives – wind, solar, hydrogen, geothermal, nuclear, biomass and more.
Shiva inspires me as \textbf{one} of the \textbf{world} ’s leading environmental \textbf{activists}, because she so seamlessly connects human justice with a deep \textbf{understanding} of \textbf{ecology}.
Ten \textbf{years} ago, I wrote a brief \textbf{history} about some of the \textbf{women} who were essential to the founding of Greenpeace.
This \textbf{year}, Shiva ’s Women ’s Day statement inspired me to think about some of \textbf{women} who have been essential to ecological \textbf{awareness} in human \textbf{society}.
A lyrical approach to nature also matters.
Some of my favorite poets, who reflect deep insights of nature, are Mary Oliver, Denise Levertov, Susan Griffin, and Diane di Prima ; and novelists/essayists such as Annie Dillard, Camille Dungy, Andrea Wulf, and Helen MacDonald.
Dorothy Stowe was the first \textbf{president} of her local civic employees \textbf{union} in Rhode Island, spent her wedding \textbf{night} at a civil rights dinner, and immigrated to Canada with her husband Irving in protest against the US \textbf{war} in Vietnam.
She hosted early Greenpeace \textbf{meetings} in her home and infused the radical \textbf{politics} with a \textbf{sense} of openness and community.
Marie Bohlen was a nature illustrator, Sierra Club \textbf{member}, a Quaker, and pacifist.
When her \textbf{son} Paul became eligible for the US military in 1967, she and her husband Jim Bohlen immigrated to Vancouver, Canada, where they met the Stowes.
Borrowing a Quaker tactic, Marie proposed the \textbf{idea} to sail a \textbf{boat} into the US nuclear \textbf{test} zone in Alaska, the first Greenpeace action.
Deeno Birmingham led the B.C.
Voice of Women in Vancouver and raised funds and public \textbf{awareness} for the first \textbf{campaign}.
Her \textbf{colleague} Lille d’Easum wrote the first Greenpeace technical report, a study of \textbf{radiation} effects.
Dorothy Metcalfe, a seasoned \textbf{journalist}, converted her home into a radio \textbf{room} during the first \textbf{campaign} and relayed \textbf{news} reports to the \textbf{media}.
She was later arrested in Paris protesting nuclear \textbf{testing} and arranged an audience with the Pope at the Vatican to bless the Greenpeace \textbf{mission}.
Dorothy Metcalfe was a brilliant campaigner.
Some of her feats are recounted here, in this short \textbf{history}.
In 1963, Zoe Hunter ( Rahim ), was a \textbf{member} of the \textbf{Campaign} for Nuclear Disarmament, when she met Canadian Bob Hunter in London and took him to his first \textbf{peace} \textbf{march}.
Later, in Canada, they both helped launch the first Greenpeace \textbf{campaign}.
In 1970, at the \textbf{age} of 27, Canadian musician Joni Mitchell headlined the benefit concert that raised the \textbf{money} for the first Greenpeace \textbf{campaign}.
The first two \textbf{women} to sail on a Greenpeace \textbf{campaign} were Ann-Marie Horne and Mary Lornie from New Zealand, on \textbf{board} the Vega, which sailed into the French nuclear \textbf{test} \textbf{site} at Moruroa atoll in 1973.
Taeko Miwa, and Carlie Trueman sailed on the first Greenpeace \textbf{whale} \textbf{campaign} in 1975.
Trueman \textbf{trained} the \textbf{crews} in the operation of the inflatable \textbf{boats} that became a Greenpeace icon.
Miwa had led \textbf{campaigns} against mercury poisoning and air pollution in Japan.
Bobbi Hunter ( Innes ) managed the first public Greenpeace \textbf{office} in Vancouver.
In 1976, Bobbi and Marilyn Kaga were the first \textbf{women} to blockade a whaling ship, the Russian Vlasny harpoon \textbf{boat}.
In 1977, in London, Susi Newborn and Denise Bell acquired and outfitted the first ship that Greenpeace ever owned, the 134-foot trawler Sir William Hardy, which became the Rainbow Warrior.
In 1978, they helped lead an international \textbf{crew}, which confronted Icelandic and Spanish whalers and exposed the UK ship Gem illegally dumping nuclear waste into the ocean.
Newborn ’s A Bonfire in my Mouth is her personal account of the Rainbow Warrior \textbf{story}.
Perhaps the first \textbf{ecology} \textbf{activists} were the Bishnois Hindus of Khejarli, India, who in 1720 attempted to protect their local forests from loggers supplying lumber to the Maharaja of Jodhpur.
Bishnois spiritual \textbf{beliefs} acknowledged the sacredness of all living \textbf{beings}, including the trees.
According to historian Jyotsna Kamat ’s account of the Bishnoi, a young \textbf{woman}, Amrita Devi, confronted the loggers and embraced a tree to protect it. \textbf{Others} joined her, and soldiers beheaded 363 Bishnoi villagers.
When the Maharaja learned of this, he appeared horrified, apologized, and designated the Bishnoi state as a protected area, legislation that still exists \textbf{today}.
In 2013, to call \textbf{attention} to Shell ’s Arctic oil and gas drilling, six \textbf{women} climbers — Sabine Huyghe ( Belgium ), Sandra Lamborn ( Sweden ), Victoria Henry ( Canada ), Ali Garrigan ( UK ), Wiola Smul ( Poland ) and Liesbeth Debbens ( Netherlands ) — climbed London ’s tallest building, the Shard, a 310-metre glass tower.
Each \textbf{stage} of the climb required a lead climber to free-climb a section of the building. “ Deep rooted social injustices, from worker rights to \textbf{gender} inequality, go \textbf{hand} in \textbf{hand} with the climate emergency, ” wrote Greenpeace Executive Director Jennifer Morgan on Women ’s \textbf{day} this \textbf{year}. “ Amplifying the voices of the most marginalised and vulnerable, while boosting their access to opportunities and \textbf{platforms}, is central to the \textbf{mission} of Greenpeace. ” In an address a \textbf{year} earlier, Shiva proclaimed : “ I want to see the decolonization of Earth, \textbf{women}, workers, indigenous \textbf{people}, and of the future. ” During the 2018 heat wave and wildfires in Sweden, Greta Thunberg changed ecological \textbf{history} when she \textbf{staged} a \textbf{school} strike outside the Swedish Riksdag, demanding that the Swedish government reduce carbon emissions.
Her actions — alongside Vanessa Nakate, Jamie Margolin, Xiye Bastida, and so many \textbf{other} \textbf{women} climate \textbf{activists} — have inspired \textbf{student} strikes in over 300 cities around the \textbf{world}.
In December 2018, Thunberg spoke at the UN climate conference in Poland and chastised the politicized delegates for their \textbf{history} of failure. “ Until you \textbf{start} focusing on what needs to be done rather than what is politically possible, there is no \textbf{hope}.
We can't solve a crisis without treating it as a crisis, ” she said. “ You only \textbf{talk} about moving forward with the same bad \textbf{ideas} that got \textbf{us} into this mess, even when the only sensible \textbf{thing} to do is pull the emergency brake. “ “ The \textbf{world} is waking up, ” Thunberg said, “ And change is coming whether you like it or not. ” “ A message from Vandana Shiva, ” International Women ’s \textbf{day}, Navdanya, March 8, 2021 Ecofeminism, Maria Mies and Vandana Shiva, Zed Books, 1993, second edition 2014.
Vandana Shiva, Staying Alive : \textbf{Women}, \textbf{Ecology} and Development in India ( Zed Press/Penquin, 1988 ).
Vandana Shiva, Leipzig Appeal for Women ’s Food Security, 1996, IATP.org Fast forward to 1973 in northern India, when \textbf{village} \textbf{women} in the Alaknanda valley, inspired by the Bishnois and by Gandhi, defended their forest from commercial logging by embracing the trees, the original tree-huggers.
Their action launched the Chipko \textbf{movement} ( Chipko : “ to embrace ” ) that spread across northern India.
Mahila Anna Swaraj.
Earth Rising, Women Rising : Regenerating the \textbf{Earth}, Seeding the Future.
Navdanya, March, 2021.
Karen Warren, \textbf{ed}., Ecological Feminist Philosophies, University of Indiana Press, 1996.
Jyotsna Kamat, “ The Bishnoi Community. ” Geographica India. historian from Bangalore Melissa Petruzzello, Chipko \textbf{movement}, Britannica, 2015.
The \textbf{women} who founded Greenpeace, Rex Weyler, Greenpeace International, 2010.
Alice Hamilton, MD : Exploring The Dangerous Trades, ( Little, Brown,1943, NWU Press 1985 ) Rachel Carson : Silent \textbf{Spring}, Houghton Mifflin, 1962 Chellis Glendinning, “ Stephanie Mills : A Life of the \textbf{Mind}, ” Wild Culture, 2019.
Stephanie Mills, In Praise of Nature, ( Island Press, 1990 ) Donella Meadows : Leverage Points : Places to Intervene in a System, The Sustainability Institute, 1999.
In the late 19th \textbf{century}, Alice Hamilton grew up in an Irish/German \textbf{family} in the eastern United States, served as a professor of pathology at the Woman ’s Medical School of Northwestern University in Illinois, and became the first \textbf{woman} appointed to the faculty of Harvard University.
In the 1920s, she began a \textbf{research} career, examining industrial toxicology.
She uncovered carbon monoxide poisoning among steelworkers, mercury poisoning among hatters, spastic anemia among limestone cutters, and an “ unusually high incidence of pulmonary tuberculosis ” among granite carvers.
Her discoveries laid the groundwork for widespread industrial reform.
When she uncovered lead poisoning from leaded gasoline, and met aggressive resistance from the oil and automobile industries, she accused General Motors of “ willful murder. ” Governments failed to ban lead in gasoline for decades, until the 1950s, when industrial smog choked major cities worldwide, and 4,000 \textbf{people} perished in London ’s infamous 1952 “ killer fog. ” The British Parliament finally passed the first Clean Air Act, as Hamilton had recommended decades earlier.
Donella Meadows, et. al., Limits to Growth ( D.
H.
Meadows, D.
L.
Meadows, J.
Randers, W.
Behrens, 1972 ; New American Library, 1977 ) ; and Limits to Growth : The 30-Year Update ( Chelsea Green, 2004 ).
Chellis Glendinning, My \textbf{Name} is Chellis and I ’m in Recovery from Western Civilization, Shambhala, 1994.
Janine Benyus, Biomimicry : Innovation Inspired by Nature, Harper, 2002.
Nancy Jack Todd, A Safe and Sustainable \textbf{World} : The Promise Of Ecological Design, Island Press, 2006 Nora Bateson, Small Arcs of Larger Circles : Framing through \textbf{Other} Patterns, ( Triarchy Press, 2016 ).
Hazel Henderson, Paradigms in Progress, ( Berrett-Koehler, 1995 ).
Alice J.
Friedemann, When Trucks Stop Running : Energy and the Future of Transportation, ” Springer, 2015 ; and essays at Energyskeptic.
In 1962, Rachel Carson ‘ s publication of Silent \textbf{Spring}, might have been the most influential event to inspire modern ecological \textbf{awareness}.
Carson ’s \textbf{book} examined thirty-five bird species threatened with extinction due to organo-chlorine chemicals such as DDT, the nastiest toxic garbage in the industrial waste stream. “ For the first \textbf{time} in the \textbf{history} of the \textbf{world}, ” Carson wrote, “ every human \textbf{being} is now subjected to contact with dangerous chemicals. ” Carson ’s \textbf{work} set her on a crash \textbf{course} with the chemical industry.
Since chemical companies found it difficult to safely dispose of these toxins, they created a market for them and spread them around the \textbf{world}.
Chemical corporations attempted to vilify Carson. \textbf{Today} ’s chemical polluters are still at it, spinning out new Rachel Carson smear \textbf{campaigns}, blaming her for malaria deaths because, allegedly, we haven't sprayed enough organochlorines around the \textbf{world}.
In \textbf{fact}, Carson supported responsible disease control, and malaria mosquitoes become immune to most pesticides. “ The ‘ control of nature ’ is a phrase conceived in arrogance, ” Carson wrote. “ It is a wholesome and necessary \textbf{thing} for \textbf{us} to turn again to the \textbf{earth} and in the contemplation of her beauties to know the \textbf{sense} of wonder and humility. ” Her writing launched a re-evaluation of our \textbf{relationship} with nature and ignited the modern environmental \textbf{movement}.
In addition to some of the Greenpeace founders ( see below ), two \textbf{women} who inspired me in the 1970s were Stephanie Mills and Donella Meadows.

% matched lemmas: activist, age, attention, author, awareness, being, belief, board, boat, book, campaign, century, child, colleague, course, crew, day, earth, ecology, editor, engine, fact, family, friend, gender, group, hand, history, hope, idea, journalist, language, man, march, medium, meeting, member, mind, mission, money, movement, name, news, night, office, one, other, peace, people, platform, politics, president, radiation, relationship, research, room, school, sense, share, site, society, son, spring, stage, start, story, student, summer, talk, task, team, test, testing, thing, time, today, train, understanding, union, us, version, village, war, way, whale, woman, work, world, writer, year
\end{textsample}
