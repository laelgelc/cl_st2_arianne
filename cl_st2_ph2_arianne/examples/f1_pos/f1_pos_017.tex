\begin{textsample}{POS Dim 1 – human – Score 77.00 – t785\_human.txt}  \label{ex:f1_pos_017}
James ( Jon ) Castle – 7 December 1950 to 12 January 2018 In 1980 Castle and the Rainbow Warrior \textbf{crew} confronted Norwegian and Spanish \textbf{whaling} ships, were again arrested by Spanish authorities, and brought into custody in the El Ferrol naval base.
The Rainbow Warrior remained in custody for five \textbf{months}, as the Spanish government demanded 10 million pesetas to compensate the whaling company.
On the \textbf{night} of November 8, 1980, the Rainbow Warrior, with Castle at the helm, quietly escaped the naval base, through the North Atlantic, and into \textbf{port} in Jersey.
In 1995, Castle skippered the MV Greenpeace during the \textbf{campaign} against French nuclear \textbf{testing} in the Pacific and led a flotilla into New Zealand to replace the original Rainbow Warrior that French agents \textbf{bombed} in Auckland in 1985.
Over the \textbf{years}, Castle became legendary for his maritime \textbf{skills}, \textbf{courage}, compassion, commitment, and for his incorruptible integrity. “ Environmentalism : That does not mean a \textbf{lot} to me, ” he once said, “ I am here because of what is right and wrong.
Those \textbf{words} are \textbf{good} enough for me. ” \textbf{One} of the most successful Greenpeace \textbf{campaigns} of all \textbf{time} began in the \textbf{summer} of 1995 when Shell Oil announced a plan to dump a floating oil storage tank, containing toxic petroleum residue, into the North Atlantic.
Castle signed on as skipper of the Greenpeace vessel Moby Dick, out of Lerwick, Scotland.
A \textbf{month} later, on 30 April 1995, Castle and \textbf{other} \textbf{activists} occupied the Brent Spar and called for a boycott of Shell \textbf{service} \textbf{stations}.
When Shell security and British \textbf{police} sprayed the protesters with water cannons, \textbf{images} flooded across \textbf{world} \textbf{media}, \textbf{demonstrations} broke out across Europe, and on May 15, at the G7 summit, German chancellor Helmut Kohl publicly protested to British Prime Minister John Major.
In June, 11 nations, at the Oslo and Paris Commission \textbf{meetings}, called for a moratorium on sea disposal of offshore installations.
After three \textbf{weeks}, British \textbf{police} managed to evict Castle and the \textbf{other} occupiers and held them briefly in an Aberdeen jail.
When Shell and the British government defied public sentiment and began towing the Spar to the disposal \textbf{site}, consumers boycotted Shell \textbf{stations} across Europe.
Once released, Castle took \textbf{charge} of the chartered Greenpeace vessel Altair and continued to pursue the Brent Spar towards the dumping ground.
Castle called on the master of another Greenpeace ship, fitted with a helideck, to alter \textbf{course} and rendezvous with him.
Using a helicopter, protesters re-occupied the Spar and cut the wires to the detonators of scuppering \textbf{charges}. \textbf{One} of the occupiers, young recruit Eric Heijselaar, recalls : “ \textbf{One} of the first \textbf{people} I met as I climbed on \textbf{board} was a red-haired giant of a \textbf{man} grinning broadly at \textbf{us}.
My first \textbf{thought} was that he was a deckhand, or maybe the bosun.
So I asked if he knew whether a cabin had been assigned to me yet.
He gave me a lovely warm smile, and reassured me that, yes, a cabin had been arranged.
At dinner I found out that he was Jon Castle, not a deckhand, not the bosun, but the \textbf{captain}.
And what a \textbf{captain}! ” With \textbf{activists} occupying the Spar once again, Castle and the \textbf{crew} kept up their pursuit when suddenly the Spar altered \textbf{course}, \textbf{heading} towards Norway.
Shell had given up.
The company announced that Brent Spar would be cleaned out and used as a foundation for a new ferry terminal.
Three \textbf{years} later, in 1998, the Convention for the Protection of the Marine Environment of the North-East Atlantic ( OSPAR ) passed a ban on dumping oil installations into the North Sea. “ There was no \textbf{question} among the \textbf{crew} who had made this possible, who had caused this to happen, ” Heijselaar recalls. “ It was Jon Castle.
His quiet enthusiasm and the trust he put into \textbf{people} made this \textbf{crew} \textbf{one} of the \textbf{best} I ever saw.
He always knew exactly what he wanted out of a \textbf{campaign}, how to gain momentum, and he always found the right \textbf{words} to explain his philosophies.
He was that rare combination, both a mechanic and a mystic.
And above all he was a very loving, \textbf{kind} human being. ” Over four decades Captain Jon Castle navigated Greenpeace ships by the twin stars of ‘ right and wrong ’, defending the environment and promoting \textbf{peace}.
Greenpeace chronicler, Rex Weyler, recounts a few of the \textbf{stories} that made up an extraordinary life.
After the Brent Spar \textbf{campaign}, Castle \textbf{returned} to the South Pacific on the Rainbow Warrior II, to obstruct a proposed French nuclear \textbf{test} in the Moruroa atoll.
Expecting the French to occupy their ship, Castle and engineer, Luis Manuel Pinto da Costa, rigged the steering mechanism to be controlled from the crow’s-nest.
When French commandos \textbf{boarded} the ship, Castle \textbf{stationed} himself in the crow’s-nest, cut away the access ladder and greased the mast so that the raiders would have difficulty arresting him.
Eventually, the commandos cut a hole into the engine-room and severed cables controlling the \textbf{engine}, radio, and steering mechanism, making Castle ’s remote control system worthless.
They towed the Rainbow Warrior II to the \textbf{island} of Hao, as three \textbf{other} protest vessels arrived.
Three thousand demonstrators gathered in the French \textbf{port} of Papeete, demanding that France abandon the \textbf{tests}.
Oscar Temaru – leader of Tavini Huiraatira, an anti-nuclear, pro-independence \textbf{party} – who had been aboard the Rainbow Warrior II when it was raided, welcomed anti-testing supporters from Britain, Ireland, New Zealand, Australia, Japan, Sweden, Canada, Germany, Brazil, the Netherlands, Luxembourg, the Philippines, and American Samoa.
Eventually, France ended their \textbf{tests}, and atmospheric nuclear \textbf{testing} in the \textbf{world} ’s oceans stopped once and for all.
Through these extraordinary \textbf{missions}, Jon Castle advocated “ self-reflection ” not only for individual \textbf{activists}, but for the \textbf{organisation} that he \textbf{loved}. \textbf{Activists}, Castle maintained, required “ moral \textbf{courage}. ” He cautioned, “ Don't seek approval. \textbf{Someone} has to be \textbf{way} out in \textbf{front} … illuminating territory in advance of the main body of \textbf{thought}. ” He opposed “ corporatism ” in \textbf{activist} \textbf{organisations} and urged Greenpeace to avoid becoming “ over-centralised or compartmentalised. ” He felt that \textbf{activist} decisions should emerge from the actions themselves, not in an \textbf{office}.
We can't fight industrialism with “ \textbf{money}, numbers, and high-tech alone, ” he once wrote in a personal manifesto. \textbf{Organisations} have to avoid traps of “ self-perpetuation ” and focus on the \textbf{job} “ upsetting powerful forces, taking on multinationals and the military-industrial complex. ” He recalled that Greenpeace had become popular “ because a gut message came through to the thirsty hearts of poor suffering \textbf{people} … feeling the destruction around them. ” \textbf{Activists}, Castle felt, required “ \textbf{freedom} of expression, spontaneity [ and ] an integrated \textbf{lifestyle}. ” An \textbf{activist} \textbf{organisation} should foster a “ \textbf{feeling} of community ” and exhibit “ moral \textbf{courage}. ” Castle felt that social change \textbf{activists} had to “ \textbf{question} the materialistic, consumerist \textbf{lifestyle} that drives energy overuse, the increasingly inequitable \textbf{world} economic tyranny that creates poverty and drives environmental degradation, ” and must maintain “ honour, \textbf{courage} and the creative edge. ” Susi Newborn, who was there to welcome Jon aboard the Rainbow Warrior \textbf{way} back in 1977, and who gave the ship its \textbf{name}, wrote about her \textbf{friend} with whom she felt “ welded at the heart : He was a Buddhist and a vegetarian and had an earring in his ear.
He liked poetry and classical \textbf{music} and could be very dark, but also very funny.
Once, I cut his hair as he downed a bottle or two of rum reciting The Second Coming by Yeats. ” Newborn recalls Castle insisting that \textbf{women} steer the ships in and out of \textbf{port} because, “ they got it right, were naturals. ” She recalls a \textbf{night} at sea, Castle “ lashed to the wheel facing \textbf{one} of the biggest storms of last \textbf{century} \textbf{head} on.
I was flung about my cabin like a rag doll until I passed out.
We never \textbf{talked} about the storm, as if too scared to summon up the behemoth we had encountered.
A small handwritten note pinned somewhere in the mess, the sole acknowledgment of a skipper to his six-person \textbf{crew} : ‘ Thank You. ’ ” \textbf{Others} remember Castle as the Greenpeace \textbf{captain} that could regularly be found in the galley doing kitchen duty.
In 2008, with the small yacht Musichana, Castle and Pete Bouquet \textbf{staged} a two-man invasion of Diego Garcia \textbf{island} to protest the American bomber base there and the UK ’s refusal to allow evicted Chagos Islanders to \textbf{return} to their homes.
They anchored in the lagoon and radioed the British Indian Ocean Territories officials on the \textbf{island} to tell them they and the US Air Force were acting in breach of international law and United Nations resolutions.
When arrested, Castle politely lectured his captors on their immoral and illegal conduct.
In \textbf{one} of his final actions, as he \textbf{battled} with his failing health, Castle helped \textbf{friends} in Scotland operate a soup kitchen, quietly prepping food and washing up behind the \textbf{scenes}.
James ( Jon ) Castle first opened his \textbf{eyes} virtually at sea.
He was \textbf{born} 7 December 1950 in Cobo Bay on the Channel Island of Guernsey, UK.
He grew up in a \textbf{house} known locally as Casa del Mare, the closest \textbf{house} on the \textbf{island} to the sea, the second \textbf{son} of Robert Breedlove Castle and Mary Constance Castle.
Upon hearing of his passing, Greenpeace ships around the \textbf{world} – the Arctic Sunrise, the Esperanza, and the Rainbow Warrior – flew their flags at half mast.
Jon is fondly remembered by his brother David, ex-wife Caroline, their \textbf{son}, Morgan Castle, \textbf{born} in 1982, and their \textbf{daughter}, Eowyn Castle, \textbf{born} in 1984.
Morgan has a \textbf{daughter} of eight \textbf{months} Flora, and and Eowyn has a \textbf{daughter}, Rose, who is 2.
Rex Weyler was a \textbf{director} of the original Greenpeace Foundation, the \textbf{editor} of the \textbf{organisation} ’s first newsletter, and a co-founder of Greenpeace International in 1979.
Young Jon Castle \textbf{loved} the sea and \textbf{boats}.
He \textbf{worked} on De Ile \textbf{de} Serk, a cargo \textbf{boat} that supplied nearby Sark \textbf{island}, and he studied at the University of Southampton to become an \textbf{officer} in the Merchant Navy.
Jon became a beloved skipper of Greenpeace ships.
He sailed on many \textbf{campaigns} and famously skippered two ships during Greenpeace ’s action against Shell ’s North Sea oil \textbf{platform}, Brent Spar.
During his \textbf{activist} career, Jon spelt his \textbf{name} as “ Castel ” to avoid unwanted \textbf{attention} on his \textbf{family}.
Jon had two personal obsessions : he \textbf{loved} \textbf{books} and \textbf{world} knowledge and was extremely well-read.
He also \textbf{loved} sacred \textbf{sites} and spent personal holidays walking to stone circles, standing stones, and holy wells.
As a young \textbf{man}, Jon became acquainted with the Quaker \textbf{tradition}, drawn by their dedication to \textbf{peace}, civil rights, and direct social action.
In 1977, when Greenpeace purchased their first ship – the Aberdeen trawler renamed, the Rainbow Warrior – Jon signed on as first mate, \textbf{working} with skipper Peter Bouquet and \textbf{activists} Susi Newborn, Denise Bell and Pete Wilkinson.
In 1978, Wilkinson and Castle learned of the British government dumping radioactive waste at sea in the deep ocean trench off the \textbf{coast} of Spain in the Sea of Biscay.
In July, the Rainbow Warrior followed the British ship, Gem, \textbf{south} from the English \textbf{coast}, carrying a load of toxic, radioactive waste barrels.
The now-famous confrontation during which the Gem \textbf{crew} dropped barrels onto a Greenpeace inflatable \textbf{boat}, ultimately changed maritime law and initiated a ban on toxic dumping at sea.
After being arrested by Spanish authorities, Castle and Bouquet \textbf{staged} a dramatic escape from La Coruńa harbour at \textbf{night}, without running lights, and \textbf{returned} the Greenpeace ship to action. \textbf{Crew} \textbf{member} Simone Hollander recalls, as the ship entered Dublin harbour in 1978, Jon cheerfully insisting that the entire \textbf{crew} help clean the ship ’s bilges before going ashore, an action that not only built camaraderie among the \textbf{crew}, but showed a mariner ’s respect for the ship itself.
In 1979, they brought the ship to Amsterdam and participated in the first Greenpeace International \textbf{meeting}.

% matched lemmas: activist, attention, battle, bear, board, boat, bomb, book, campaign, captain, century, charge, coast, courage, course, crew, daughter, de, demonstration, director, editor, engine, eye, family, feeling, freedom, friend, front, good, head, house, image, island, job, kind, lifestyle, lot, love, man, medium, meeting, member, mission, money, month, music, name, night, office, officer, one, organisation, other, party, peace, people, platform, police, port, question, return, scene, service, site, skill, someone, son, south, stage, station, story, summer, talk, test, testing, thought, time, tradition, us, way, week, whale, woman, word, work, world, year
\end{textsample}
