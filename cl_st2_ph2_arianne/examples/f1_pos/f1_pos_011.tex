\begin{textsample}{POS Dim 1 – human – Score 85.00 – t185\_human.txt}  \label{ex:f1_pos_011}
All around the \textbf{world}, \textbf{women} and girls are making enormous contributions to climate action.
They are vital agents of change for the planet, but their voices are often missing from the decision-making \textbf{table}.
Many disabled folks want to be more environmentally conscious and play their part in taking climate action.
Still, a legacy of non-engagement and eco ableism has created distrust between many disabled communities and environmental and climate \textbf{organisations}.
Building and restoring credibility, trust and \textbf{relationships} \textbf{will} take \textbf{time}.
But if \textbf{anything} \textbf{will} keep me going, it ’s \textbf{hope}, that climate \textbf{movements} *can* make shifts to be intersectionally centring the voices of those most marginalised, including disabled \textbf{people}, and campaigning in solidarity with \textbf{us} for a future of \textbf{care} and flourishing for both planet and \textbf{people}. ” Áine Kelly-Costello ( she/they ) is a proudly multiply disabled climate campaigner advocating in Aotearoa New Zealand \& internationally.
They are passionate about furthering climate justice and creating an accessible, inclusive \textbf{world}, especially by harnessing the perspectives of disabled \textbf{people}.
They are newly exploring identifying as genderqueer \& want to shout out to all the trans \& \textbf{gender} non-conforming folks who \textbf{bear} the brunt of transphobia \& \textbf{gender} inequities.
Explore further : Eco Ableism and the Climate Movement ( Catrina Randall, Young Friends of the Earth Scotland ) Reading list on disability justice and climate justice ( compiled by SustainedAbility – scroll down page for list ) The Missing \textbf{Conversation} about Disabled Leadership in Climate Justice ( Áine Kelly-Costello, \textbf{Stuff} ) “ In my Khadia tribe, my surname ‘ Soreng ’ means ‘ \textbf{rock} ’.
Many indigenous communities have surnames related to nature showing how intertwined they are with the natural \textbf{world}.
My grandfather was a pioneer of a community-led forest protection in our \textbf{village} in Odisha ’s Sundergarh \textbf{district}.
He was a staunch believer that we should have a sustainable \textbf{relationship} with nature, only then can there be sustainable living.
My \textbf{father} was an advocate of indigenous healthcare.
It was my \textbf{parents} who said to me that : “ If you really want to contribute back to \textbf{society}, you need to enter into policy making ”.
Through my studies in \textbf{humanities} and environmental science, I realized that it is important for indigenous communities to write and speak for ourselves, and to \textbf{share} our perspective and worldview through our own lens.
It is important to raise our voice, because our voice matters.
In the \textbf{month} of March, Greenpeace had highlighted the voices of diverse \textbf{women} who are leading on climate.
These are just 10 \textbf{women} from the Asia Pacific region who are raising their voices to help shape the climate \textbf{conversation} and to heal our planet.
My \textbf{father} ’s death in 2017 triggered in me an urgent need to learn traditional practices and their contribution towards climate action and biodiversity conservation, and to document them so that upcoming generations \textbf{will} know about these practices.
Through the \textbf{years} I have interacted with numerous tribal communities in the state.
Every tribe is unique but all their cultures are sustainable and respectfully intertwined with nature.
Climate change affects all of \textbf{us} but not equally.
Tribal communities are the \textbf{ones} who have been in the frontlines of protecting the forest and nature through their \textbf{way} of living, traditional knowledge and practices, even at the cost of their lives, yet they are the \textbf{ones} who are most adversely affected due to the impact of the Climate Crisis.
We need to make tribal communities an integral part of the decision making process of Climate Action. ” Archana Soreng, a \textbf{member} of the Khadia tribe in Odisha, India, and a \textbf{member} of UN Secretary General ’s Youth Advisory Group on Climate Change “ I am Eunbin Kang, living in Seoul, South Korea.
I was interested in the Korean \textbf{War} and division issues, and I studied Political Science and Diplomacy at \textbf{university}.
It ’s been my all-time routine to be vigilant about waste management and meat-eating issues.
In September 2019, I took part in the Climate Crisis Emergency Action Parade in Seoul.
The slogan “ Climate Crisis, Not Global Warming ” touched me.
The \textbf{word} ‘ climate crisis ’ was a wake up call that the reality in \textbf{front} of \textbf{us} is a crisis that ordinary \textbf{people} must \textbf{work} through together, not just experts or \textbf{people} in power In 2020, I \textbf{started} the climate \textbf{movement} in Youth Climate Emergency Action.
In 2021, we took direct action against Doosan Heavy Industries \& Construction, a \textbf{representative} Korean company that is building coal power plants in various parts of Asia.
Currently, we are undergoing trials including a claim for damages of 18.4 million won from Doosan Heavy Industries \& Construction.
Rather than despair in the face of the climate crisis, we have chosen to stand up and resist those in power.
I remember the saying that our very existence itself gives \textbf{someone} \textbf{hope} and stimulation.
I want to continue the climate \textbf{movement}, remembering that \textbf{hope} is within \textbf{us} and that \textbf{love} overcomes \textbf{everything}. ” Eunbin Kang, co-founder of the Youth Climate Emergency Action \textbf{group} in South Korea “ Growing up, I have always fancied doing public \textbf{service}, inspired by the biographies of our national heroes who fought for our country ’s independence.
I became the first Muslim \textbf{student} President in a very conservative and Catholic-denominated all-girls \textbf{school}.
Being \textbf{woman} and relatively young were challenging, given our patriarchal and tribal \textbf{society}.
I am a Maranao — a Muslim tribe from a war-torn province, \textbf{south} of Philippines — carving a space in the “ imperial ” Manila.
In order to be heard and earn “ legitimacy, ” I tread the path of \textbf{lawyering} to earn a voice not only for myself, but for the \textbf{people} I seek to serve and represent—women, \textbf{children}, elderly, indigenous \textbf{peoples}, and \textbf{other} vulnerable sectors.
After almost 5 \textbf{years} of honing my \textbf{skills} as a \textbf{general} litigation \textbf{lawyer} in a firm mostly serving corporate clients, I embarked on a journey to realize my \textbf{dream} to serve.
Destiny led me to Greenpeace to \textbf{work} with 32 audacious petitioners who sued 47 big coal, oil, gas, and cement companies, a landmark national investigation against multinational corporations that dealt with the cross-cutting issues of climate change and human rights.
Five \textbf{years} \textbf{working} with different communities and vulnerable \textbf{groups} heightened my \textbf{passion} to even \textbf{look} for opportunities to mainstream climate justice with the \textbf{hope} that every Filipino \textbf{will} take the matter seriously and with the same level of urgency as the current pandemic.
We cannot just \textbf{look} the \textbf{other} \textbf{way} when it comes to the realities on the ground.
We must demand accountability and resist injustice. “ I ’m a Nakhi, native to the foothills around Lijiang, where I was \textbf{born} and raised.
Lijiang sits among the Himalaya and Hengduan mountain ranges in a region that \textbf{houses} the most biodiversity of anywhere in China.
It ’s a haven of alpine plants, scattered with wild orchids, rhododendron, and innumerable precious plant life.
They are remarkably sensitive to the impact of climate change.
As the snow line moves higher and higher up the mountains and glaciers melt, we \textbf{may} see the extinction of more and more species, including those that are still undiscovered or understudied.
There are many species we still don't understand but nonetheless do know that they are on the brink of extinction.
While awaiting for the impending release of the final report on our legal action, I \textbf{hope} we have opened up the space and built power for communities to reclaim their rights and take action to achieve a safe climate and a healthy environment.
Indeed, there is no such \textbf{thing} as David vs.
Goliath if you are fighting for a \textbf{good} cause.
And there is no \textbf{better} cause than the cause for the environment and human rights. ” Hasminah D.
Paudac-Tawano, Legal Advisor, Climate Justice and Liability Program at Greenpeace Southeast Asia ( Philippines ) “ I joined the Greenpeace Climate and Energy Project in 2017, after receiving my PhD from the Chinese Academy of Social Sciences.
As the Program \textbf{Manager} of Climate \& Energy ( GPEA ), I lead the Beijing Office Climate Risk Project and Research Unit where I am committed to making the public and stakeholders aware of climate change and to take active actions.
In 2018, I visited the glaciers in the western plateau of China together with scientists.
We observed the catastrophic impact of glacial flooding under the influence of climate change.
In 2021, my \textbf{colleagues} and I participated in a post-disaster rescue after the Henan floods due to heavy rains.
From what we have \textbf{witnessed}, the climate change crisis is looming.
At the same \textbf{time}, we are also noticing that climate change is widening the inequality gap, exposing vulnerable populations to more severe risks.
Climate change is such a grand and complex issue, engulfing a series of crises in \textbf{politics}, economy, health, development, and fairness.
Climate crisis is like an abyss.
It is incalculable and \textbf{may} even swallow our \textbf{courage} to act. ” “ The only \textbf{way} to escape the abyss is to \textbf{look} at it, gauge it, sound it out and descend into it. ” — Cesare Pavese Liu Junyan, Programme \textbf{Manager} of Climate \& Energy at Greenpeace East Asia “ Pua Lay Peng is a local \textbf{activist} leading a grassroots environmental \textbf{group} called Persatuan Tindakan Alam Sekitar Kuala Langat ( Kuala Langat Environmental Action Group ) in Malaysia.
The \textbf{group} was active in \textbf{campaigning} against the imported plastic waste problem that was affecting their small town, Jenjarom.
In 2018, the global plastic waste trade was disrupted when China banned most plastic waste imports.
Southeast Asian countries like Malaysia had picked up the slack, accepting the outpouring of plastic waste from high-income countries that led to a spike in the number of illegal dumpsites and burning facilities in the country.
I was a \textbf{journalist} in traditional \textbf{media} for more than ten \textbf{years}.
Now, I use social \textbf{media} to record and exhibit beautiful but fragile plant life to get more \textbf{people} to pay \textbf{attention} to alpine plants, as well as climate change and its impact on biodiversity.
We quickly find joy in the gifts of nature, but can also easily ignore the step-by-step progression of this crisis.
A chemist, Pua had moved back to Jenjarom where she \textbf{witnessed} the devastating impact that the plastic recycling industry was having on her community.
With over 40 illegal plastic factories emitting toxic gases into the air and polluting the local rivers and waterways, they were making \textbf{people} very sick.
Alarmed by the environmental and health impacts of the illegal industries, Pua took action.
She led the \textbf{group} as they organised \textbf{meetings}, conducted fieldwork and published reports exposing the plastic problem.
They also \textbf{worked} with the authorities to act against the illegal plastic waste trade, leading to the closure of several hundred illegal facilities.
Despite receiving death threats, Pua and the \textbf{group} continued with their courageous and relentless efforts, empowering \textbf{other} communities to stand together and fight against environmental pollution. ” “ My \textbf{name} is Sylvia Wu.
As a legal professional, I aspire to use my legal knowledge to bring about positive changes.
In February of 2021, we filed the first climate litigation in Taiwan, Greenpeace East Asia and \textbf{others} v.
Ministry of Economic Affairs..
I \textbf{worked} closely with the individual plaintiffs, local NGOs, and our project \textbf{team} to urge the government to be more ambitious in climate change law amendments and to challenge the rigid judicial system.
Climate justice should not be a privilege.
It ’s a fundamental human right.
That is also our ultimate goal in the case–to create a path for ordinary \textbf{people} to access climate justice. \textbf{One} of the biggest challenges we faced was to overcome the procedural barriers in a highly traditional court system like Taiwan ’s.
Moreover, most Taiwanese have never heard of climate litigation or climate justice.
Therefore, we endeavored to amplify the influence of our case and to ensure that our plaintiffs ’ voices got heard while the case was still pending.
We collaborated with local environmental law \textbf{groups} to initiate a petition among \textbf{lawyers} advocating for adding \textbf{citizen} suit provision, hold forums allowing more \textbf{people} to understand climate justice, and publish \textbf{articles}.
In addition to climate litigation, I devoted myself to supporting our Distant Water Fishery project.
Due to the discriminated-based two-tier employment system, many distant water fishery workers \textbf{experienced} forced labor or human trafficking in the dark ocean.
We assessed and developed legal and political advocacies to safeguard their rights and to help them break out from the chains of an unfair system.
The \textbf{people} I have been fighting for are the highlight of my \textbf{story}.
What makes me most proud of my \textbf{work} is that I get to stand with them, amplify their voices, and fight together against the flaws in the system. ” Sylvia Wu, Legal coordinator at Greenpeace Taiwan “ My activism is sustained by what I ’ve learnt from my \textbf{experiences} while \textbf{working} with communities. \textbf{Movements} are created from the ground up and building a \textbf{movement} is building \textbf{people} power in the process.
It is built on values of shared responsibilities as much as a shared leadership.
As a community organizer, these \textbf{lessons} have become my core values too.
What matters most is supporting and enabling each \textbf{other} to act together so as to achieve our shared \textbf{vision}.
Taking action on climate change means each \textbf{person} gets involved.
So far, we haven't yet done enough. ” I \textbf{hope} that \textbf{others} would aspire for a \textbf{better} life, not just for survival.
That they realize that they have the power to achieve this through persistent collective activism.
We must be at the forefront of the struggle and confront the system that makes the \textbf{people} poor and miserable, thus perpetuating injustices.
Through our activism we have an opportunity to desconstruct this system that prioritizes profit over the right and welfare of the \textbf{people} and planet.
I \textbf{hope} to show that aspiring for a \textbf{better} life and \textbf{society} is \textbf{nothing} short of demanding for a systemic shift.
While we have managed to achieve small victories in the past, the \textbf{better} \textbf{society} that we aspire to is still a long \textbf{way} away.
Challenges have been non-stop but the biggest \textbf{one} at this \textbf{time} is the context of the shrinking democratic space and the growing impunity in many countries.
Places like in the Philippines where activism is being criminalized.
Even so, as I immersed myself in community \textbf{work}, my commitment gets deeper as an \textbf{activist}.
Despite the difficult personal and work-related challenges along the \textbf{way}, my motto is to go back to the \textbf{start} and remember the \textbf{reasons} why I chose this life. ” Veronica Cabe is a Coordinator of the Nuclear and Coal-Free Bataan Movement in the Philippines, a community-based network of \textbf{organizations} and individuals that \textbf{campaigns} for the protection of communities against the perils of nuclear and fossil-fuel energies. “ My \textbf{name} is Yaewon Hwang, and I am an Equity, Diversity and Inclusion Partner for Greenpeace East Asia.
I believe that many transgender \textbf{women} are \textbf{born} \textbf{activists} as we ’ve had to fight so hard to fit into a Cisgender-centric \textbf{society} from a young \textbf{age}.
We are taught from the moment we were \textbf{born} that \textbf{everything} we are about and that \textbf{everything} we want to be is ‘ wrong ’.
We ’ve been made to feel very lonely and isolated from \textbf{society}.
What I have gone through has made me broaden my horizons and taught me what the core of my activism should be, and that is \textbf{love}.
I \textbf{will} overcome hate with \textbf{love}.
I have, and always \textbf{will} be, vocal about what I believe in and what is right, especially when it comes to trans justice.
I believe that \textbf{conversations} about transgender \textbf{movement} have only just \textbf{started} in the last few \textbf{years}.
We are nowhere near where it needs to be. \textbf{People} tend to \textbf{fear} what they do not know and transphobia comes from ignorance and not being educated on this \textbf{topic}.
Trans \textbf{people} are your neighbors, \textbf{colleagues}, \textbf{friends}, and \textbf{family}.
We are no different from you.
I believe that we can overcome \textbf{people} ’s \textbf{fears} with exposure and I am very glad to see so many incredible trans \textbf{women} on many different \textbf{platforms} these \textbf{days}.
It really helps cis \textbf{people} to understand that we live and \textbf{love} just like them.
Achieving trans justice \textbf{starts} from our daily lives.
If you are a cis \textbf{person}, please be our ally because the trans community needs more allies who \textbf{will} stand up for \textbf{us} and who \textbf{will} speak out for what is right.
For example, if \textbf{one} of your \textbf{family} \textbf{members} or \textbf{friends} are making discriminatory comments or jokes about the trans community, please correct them. \textbf{Let} them know that their discriminatory comments are not acceptable, and that we all need to respect each \textbf{other} in order for \textbf{us} to move forward for a \textbf{better} future. ” Yaewon Hwang, Equity, Diversity and Inclusion Partner for Greenpeace East Asia Ai Ji is a Nakhi conservationist using documentation and public \textbf{awareness} activism to highlight the fragile beauty of her home, Lijiang, as well as climate change and its impact on biodiversity. “ \textbf{Politicians}, \textbf{journalists} and the climate \textbf{movement} still barely ever \textbf{talk} about the disproportionate \textbf{ways} in which climate breakdown and responses to it are leaving disabled \textbf{people} behind.
Multiple layers of marginalisation, including indigenous status and \textbf{gender}, compound the harms many disabled communities face.
Often if we do get mentioned, disabled \textbf{people} are relegated to a list of “ vulnerable \textbf{groups} ”, taking away our agency.
I ’m determined to help rewrite that \textbf{story}.
Disabled \textbf{people} are change agents well-adapted to a \textbf{world} not created with \textbf{us} in \textbf{mind}.
We have lived \textbf{experience} and expertise which is integral to shaping an accessible, inclusive, safe and climate-resilient future.
I \textbf{’d} \textbf{love} for environmental and climate campaigners to learn about how eco ableism can show up, and figure out how their \textbf{organisations} can \textbf{start} to dismantle it.
That includes \textbf{looking} internally at organisational systems, processes and practices.
Engaging with disabled \textbf{members} and consultants to develop and embed accessibility and inclusion guidance and training is crucial.

% matched lemmas: activist, age, anything, article, attention, awareness, bear, campaign, care, child, citizen, colleague, conversation, courage, day, district, dream, everything, experience, family, father, fear, friend, front, gender, general, good, group, hope, house, humanity, journalist, lawyer, lesson, let, look, love, manager, may, medium, meeting, member, mind, month, movement, name, nothing, one, organisation, organization, other, parent, passion, people, person, platform, politician, politics, reason, relationship, representative, rock, school, service, share, skill, society, someone, south, start, story, student, stuff, table, talk, team, thing, time, topic, university, us, village, vision, war, way, will, witness, woman, word, work, world, year
\end{textsample}
