\begin{textsample}{POS Dim 1 – human – Score 75.00 – t270\_human.txt}  \label{ex:f1_pos_018}
This \textbf{month}, Greenpeace celebrates 50 \textbf{years} of environmental activism, \textbf{dating} from the first \textbf{campaign} to stop a nuclear \textbf{bomb} \textbf{test} in Alaska, launched from Vancouver, Canada, on Sept. 15, 1971.
I was \textbf{one} of some 50,000 American draft resisters, opposing the Vietnam War, who slipped \textbf{north} into Canada between 1965 and 1973.
In Vancouver, I soon met \textbf{peace} \textbf{activists} such as Hunter, the Metcalfes, and the Stowes.
A single event united these \textbf{activists}.
In November 1969, the United States announced a 5-megaton thermonuclear \textbf{bomb} \textbf{test}, scheduled for October 1971 on remote Amchitka Island, 4,000 kilometers northwest of Vancouver.
The \textbf{island} was a US Wildlife Refuge for 131 species of sea birds.
An earlier, smaller \textbf{test} had registered 6.9 on the Richter scale and killed wildlife around the \textbf{island}.
The 1971 \textbf{test} was going to be five-times more powerful.
Bob Hunter wrote a column about the risks, proposing that the explosion could cause a tsunami on Canada ’s \textbf{west} \textbf{coast}.
For a \textbf{demonstration} at the US/Canada border, he created a sign, declaring : “ DO N’T MAKE A WAVE. ” At the protest, Irving Stowe proposed a \textbf{citizen} ’s \textbf{group} to halt the \textbf{bomb}.
Early \textbf{meetings} included Hunter, Bill Darnell, the Metcalfes, the Bohlens, Deeno Birmingham with the Voice of Women, and radical \textbf{activists} Rod Marining and Paul Watson.
Stowe suggested they \textbf{name} the ad hoc \textbf{group} “ The Don't Make a Wave Committee. ” The \textbf{activists} were familiar with a 1958 Quaker protest \textbf{boat}, the Golden Rule, that sailed from California to the Enewetak Island nuclear \textbf{test} \textbf{site} in the Philippine Sea. \textbf{One} \textbf{morning}, Marie Bohlen told her husband, “ We should just sail a \textbf{boat} to Alaska. ” That same \textbf{day}, a Vancouver Sun reporter called, asking what the new \textbf{group} might do to stop the \textbf{test}.
Jim Bohlen blurted out, “ We \textbf{hope} to sail a \textbf{boat} to Amchitka to confront the \textbf{bomb}. ” \textbf{Everyone} \textbf{loved} the \textbf{idea} and set to \textbf{work}.
The Committee met at the Unitarian Church to discuss how they might find a \textbf{boat} and skipper willing to make the \textbf{trip}.
As the \textbf{meeting} ended, Irving Stowe flashed the “ V ” sign, and said “ \textbf{Peace}. ” Bill Darnell replied quietly, “ make it a green \textbf{peace}. ” This \textbf{term}, “ green \textbf{peace} ” articulated the merging \textbf{peace} and \textbf{ecology} \textbf{movements}, and stuck in \textbf{everyone} ’s \textbf{mind}.
In September 1971, the \textbf{group} launched the 64-foot fish \textbf{boat}, the Phyllis Cormack, skippered by fisherman John Cormack.
For the protest voyage, the \textbf{boat} was re-christened as “ Greenpeace. ” The US detonated the \textbf{bomb} before the Greenpeace ship was able to reach Amchitka Island, but the voyage created such an uproar in Canada, the US, and in Europe, that the US cancelled all future nuclear \textbf{tests} in Alaska.
The following \textbf{year}, the Committee adopted the \textbf{name} that so perfectly articulated the emerging zeitgeist : Greenpeace Foundation.
Ben and Dorothy Metcalfe took over leadership and launched a similar \textbf{campaign} against French nuclear \textbf{bomb} \textbf{tests} in the South Pacific.
After three \textbf{years} of protests, France also relented.
The \textbf{environmentalists} in Vancouver had discovered the \textbf{greatest} \textbf{inspiration} for any social visionary : they could win.
Some of \textbf{us} in Vancouver felt that we had to launch a \textbf{campaign} that would speak directly to \textbf{ecology}, that would give voice to all of nature ’s \textbf{beings}.
In 1972, Canadian \textbf{author} Farley Mowat met New Zealand brain researcher, Dr.
Paul Spong, who was performing studies with a captured Orcinus orca ( killer \textbf{whale} ) at the Vancouver aquarium.
Mowat told Spong that Soviet and Japanese whaling fleets were decimating the \textbf{whales}, and that the Atlantic grey \textbf{whale} was already extinct.
Spong came to Greenpeace with an \textbf{idea} to launch a \textbf{campaign} to save the \textbf{whales}, using tactics similar to the anti-nuclear-bomb tactics.
Spong ’s \textbf{idea} seemed perfect for our global \textbf{ecology} \textbf{campaign}.
The \textbf{whales} needed help, and the \textbf{campaign} would give voice to the notion of \textbf{ecology}, \textbf{looking} after the \textbf{Earth} itself and all the creatures of the \textbf{Earth}.
Several \textbf{books} published at that \textbf{time} inspired ecological \textbf{awareness} : Rachel Carson ’s Silent \textbf{Spring}, in 1962, the ground-breaking Limits to Growth by Donella Meadows, Dennis Meadows, Jørgen Randers, and their \textbf{colleagues} at the Club of Rome in 1972.
We faced two critical obstacles.
The nuclear \textbf{tests} had been held at fixed locations.
The whaling fleets, on the \textbf{other} \textbf{hand}, moved constantly.
We did not know how we might find them.
Secondly, we would have to blockade moving ships and could not likely achieve this with the fish \textbf{boat} or a sail \textbf{boat}.
To solve the first challenge, Spong undertook an international espionage \textbf{trip} that led him to Sandefjord, Norway, where he gained access to International Whaling Commission archives and located previous routes of the whaling fleets.
To solve the second challenge, we borrowed an \textbf{idea} from the French commandos, who had \textbf{boarded} the Greenpeace protest \textbf{boat}, Vega, in 1973.
The French sailors had used inflatable Zodiac \textbf{boats} to chase down and \textbf{board} the Greenpeace \textbf{boat}. “ That ’s it, ” said Hunter.
The inflatable protest vessel has since become an icon of Greenpeace sea-going actions.
In April 1975, we set off on the Phyllis Cormack to find the whaling fleets.
In June, we intercepted the Russian whaling factory ship Dalnyi Vostok, some 50 nautical miles \textbf{west} of California at the Mendocino Ridge sea-mounts, right where Spong ’s maps suggested they would be.
We confronted the whaling ships, and took photographs and 16mm \textbf{film} to document the encounter.
A \textbf{week} later, when the \textbf{film} and photographs went out on wire \textbf{services} around the \textbf{world}, Greenpeace became recognized internationally.
We followed this \textbf{campaign} with efforts to save Harp seals and \textbf{other} marine mammals, to stop toxic dumping in the oceans, and to save forests. \textbf{Friends} of the \textbf{Earth} arose at about the same \textbf{time}, and within a few \textbf{years}, \textbf{ecology} \textbf{groups} were forming all over the \textbf{world}.
At last, the \textbf{world} had an \textbf{ecology} action \textbf{movement}.
By 1977, public donations allowed \textbf{colleagues} in London, Susi Newborn and Denise Bell, to purchase the first ship that Greenpeace actually owned, an Aberdeen trawler, re-christened Rainbow Warrior.
In 1979, we formed Greenpeace International in Amsterdam.
Fifty \textbf{years} after that first \textbf{campaign} ship departed from Vancouver for Alaska, Greenpeace now has national and regional \textbf{offices} throughout the \textbf{world}.
The long \textbf{battle} to save \textbf{Earth} from ecological calamity continues.
To imagine how we might make progress during the next 50 \textbf{years}, perhaps we should take stock of progress over the last 50 \textbf{years}.
We ’ve \textbf{witnessed} 50 \textbf{years} of environmental activism ; we have thousands of environmental \textbf{groups}, environmental ministers, environmental agencies, protected zones, legislation, scientific papers, warnings, data, conferences, government promises, and public outcry.
We have millions of allegedly green products, sustainable processes, and eco-friendly corporate \textbf{mission} statements.
That might feel potentially promising, but we have to ask \textbf{one} simple \textbf{question} : Is \textbf{humanity} on Earth more sustainable \textbf{today} than we were in 1971?
The data reveals that the \textbf{answer} to this \textbf{question} is clearly “ No. ” We are not more sustainable in 2021 than we were in 1971.
There is less biodiversity, more greenhouse gases in the atmosphere, more toxins in our soils, less carbon in our soils, less fertile soil, more dry rivers and dead \textbf{lakes}, more ocean dead zones, less forest, more desert, more humans, a billion \textbf{people} living on the edge of starvation, and ever-greater human demands on all the dwindling resources.
We are clearly less sustainable.
Still, \textbf{ecology} was not a fashionable \textbf{idea} in the early 1970s.
Conservation \textbf{groups} protected \textbf{parks}, but there was no global \textbf{activist} \textbf{movement} giving voice to Earth itself.
In the 1970s in Vancouver, a small \textbf{group} of \textbf{people} set out to create an \textbf{ecology} action \textbf{movement} on the same scale as the \textbf{peace} or civil rights \textbf{movements}.
Since 1971, human population has more than doubled ( from 3.76 to 7.67 billion ) while vertebrate abundance has declined by over 30 % and, collectively, fish, amphibians, reptiles, birds, and mammals have declined by 60 %.
Commercial fish stocks have been depleted by 85 % and oxygen-depleted ocean zones have increased by 75 %.
We continue to lose about 13 million hectares of forests every \textbf{year}.
Meanwhile, since 1971, human carbon emissions have increased by 250 %, from about 4 to 10 Giga-tonnes/year.
So, to be honest, \textbf{everything} we have achieved is not enough.
This implies a second critical \textbf{question} : So, what is enough?
What do we have to do differently?
Three \textbf{years} ago, I wrote a summary of my \textbf{answer} to this \textbf{question}, “ What can we do?. ” My \textbf{greatest} concern for the next 50 \textbf{years} is that we risk misunderstanding the problem we face, and fail to make the necessary deep changes in consciousness and behaviour.
For example, we have a robust climate \textbf{movement} and we ’ve held 34 international climate \textbf{meetings} over the last 42 \textbf{years}, but our global carbon emissions keep going up.
Our next important step \textbf{may} be to re-frame the challenges for the next 50 \textbf{years}.
We are facing much more than a climate crisis.
We face a complex, global-scale ecological crisis.
Climate disruption is \textbf{one} symptom of this crisis. \textbf{Other} symptoms include biodiversity collapse, fresh water depletion, nutrient cycle disruption, and the myriad \textbf{other} challenges, including \textbf{war} and social injustice. \textbf{Ecologists} call our \textbf{general} crisis “ overshoot. ” Most successful species tend to overshoot their habitats.
Evolution teaches species to reproduce and consume.
Evolution does not teach species when to stop.
Wolves overshoot watersheds, algae \textbf{will} overshoot the nutrient capacity of a \textbf{lake}, and locusts \textbf{will} overshoot their available food.
We can \textbf{witness} overshoot in our gardens, or in the forest, where plants grow over each \textbf{other}, entangle with each \textbf{other}, and expand to and beyond the limits of their habitat.
Overshoot is natural, and the solutions to overshoot in the natural \textbf{world} all involve a contraction of the hyper-successful species.
Whatever else we do over the coming decades, we \textbf{will} not likely solve our ecological crisis unless we contract our numbers and our consumption, especially the frivolous consumption in the rich nations.
We need to address changes to our unnatural, unecological, and unjust economic system.
Some observers object, claiming that “ there is no limit to growth, ” or “ no limit to human ingenuity, ” but the \textbf{lessons} of the natural \textbf{world} tell \textbf{us} otherwise.
Some extreme technology-obsessed billionaires imagine that we ’ll colonize \textbf{other} planets and leave the depleted \textbf{Earth} behind.
I ask this \textbf{question} to myself virtually every \textbf{day} : How do we change the human trajectory, away from growth, chaos, and collapse?
What path \textbf{will} lead \textbf{us} toward genuine sustainability?
I suspect that our progeny — and all \textbf{other} species — \textbf{will} be far \textbf{better} off if we embrace our \textbf{relationship} with the living \textbf{Earth}, learn from nature itself, consciously contract, slow our economies, and allow wild, untrammeled natural habitats to expand. “ Irving and Dorothy Stowe : Mentors to a \textbf{movement}, ” Rex Weyler, Greenpeace, August, 2021.
Rachel Carson, “ Silent \textbf{Spring}, ” Houghton Miffli n, 1962.
During the post-World-War-II nuclear arms \textbf{race}, a global \textbf{peace} \textbf{movement} gained momentum.
In the US, biologist, Dr.
Barry Commoner found traces of strontium-90, a carcinogenic byproduct of nuclear explosions, in \textbf{children} ’s teeth.
Similar studies in Russia and Europe revealed that radioactive nuclear byproducts contaminated every region of Earth.
The \textbf{peace} \textbf{movement} and the \textbf{ecology} \textbf{movement} began to merge.
Donella Meadows, et al., Limits to Growth by the Club of Rome in 1972, Rex Weyler, “ Greenpeace : The Inside Story, ” Raincoast Books, 2005.
William R.
Catton, “ Overshoot : The Ecological Basis of Revolutionary Change, ” ( University of Illinois Press, 1982 ). “ \textbf{Earth} Overshoot \textbf{Day} back to July amid post-pandemic emissions rebound, ” inews.co.uk, 2021. “ Personal actions to address climage change, ” Earth Overshoot org Rex Weyler, “ What can we do?, ” Greenpeace, 2018. “ Planetary Boundaries : Exploring the Safe Operating Space for \textbf{Humanity}, ” J.
Rockström, et. al., \textbf{Ecology} and Society, 2009.
William Rees, “ The Way Forward : Survival 2100, ” Our World, June 2012 “ The Mining of Minerals and the Limits to Growth, ” Simon P.
Michaux, Geological Survey of Finland, April 2021.
In the Eastern United States, Irving and Dorothy Strasmich ( later “ Stowe ” ) became \textbf{peace} advocates.
In 1966, in opposition to the US \textbf{war} in Vietnam, they moved to Vancouver, on Canada ’s \textbf{west} \textbf{coast}.
They attended Quaker \textbf{meetings}, led \textbf{peace} \textbf{marches}, and supported Indigenous rights \textbf{groups}, and \textbf{worked} with Canada ’s Voice of Women. ( To learn more \textbf{check} out : Irving and Dorothy Stowe : Mentors to a \textbf{movement}. ) They met Bob Hunter, a columnist for the Vancouver Sun newspaper, writing about \textbf{ecology}, civil rights, and pacifism.
Stopping \textbf{war} wasn't enough, Hunter believed ; we had to stop the destruction of the natural \textbf{world}.
He told his \textbf{friends}, “ \textbf{Ecology} is the \textbf{thing}. ” Jim and Marie Bohlen came to Vancouver to avoid the US military draft for their \textbf{sons}, Lance and Paul.
They met the Stowes at a \textbf{peace} \textbf{march} and became \textbf{friends}.
Meanwhile, 22- year-old Bill Darnell organized an “ \textbf{Ecology} Caravan, ” which \textbf{toured} the province.
Hunter introduced the Stowes to \textbf{journalists} Ben and Dorothy Metcalfe, who had been so disturbed by forest devastation and bellowing smokestacks from pulp mills, that they placed 12 billboards around Vancouver that declared : \textbf{Look} it up!
You ’re involved.

% matched lemmas: activist, answer, author, awareness, battle, being, board, boat, bomb, book, campaign, check, child, citizen, coast, colleague, date, day, demonstration, earth, ecologist, ecology, environmentalist, everyone, everything, film, friend, general, good, great, group, hand, hope, humanity, idea, inspiration, island, journalist, lake, lesson, look, love, march, may, meeting, mind, mission, month, morning, movement, name, north, office, one, other, park, peace, people, question, race, relationship, service, site, son, spring, term, test, thing, time, today, tour, trip, us, war, week, west, whale, will, witness, work, world, year
\end{textsample}
