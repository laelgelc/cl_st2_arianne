\begin{textsample}{POS Dim 1 – human – Score 85.00 – t300\_human.txt}  \label{ex:f1_pos_009}
This September 2021, Greenpeace \textbf{will} celebrate 50 \textbf{years} of environmental activism, \textbf{dating} from the launch of the first Greenpeace \textbf{campaign} to stop a nuclear \textbf{bomb} \textbf{test} in Alaska.
In 1966, Irving and Dorothy Stowe, in opposition to the US \textbf{war} in Vietnam, moved to Vancouver, on Canada ’s \textbf{west} \textbf{coast}, with their two \textbf{children}, Robert and Barbara.
They attended Quaker \textbf{meetings}, led \textbf{peace} \textbf{marches} to the US embassy, and corresponded with Bob Hunter, who was now writing for the Vancouver Sun newspaper, and with Ben and Dorothy Metcalfe, who were reporting for the CBC.
They \textbf{worked} with Indigenous rights \textbf{groups} and with Deeno Birmingham and Lille d’Easum from Canada ’s Voice of Women.
Hunter wrote about \textbf{ecology}, civil rights, and the \textbf{peace} \textbf{movement} in his newspaper column, and \textbf{worked} on his first non-fiction \textbf{book}, The Enemies of Anarchy His \textbf{book} addressed the “ consciousness of interrelationships ” that he had picked up from Rachel Carson, a cultural \textbf{revolution} that Hunter believed would involve social diversity, \textbf{gender} equality, electronic \textbf{media}, and \textbf{ecology}.
He grew convinced that the next big change in \textbf{society} would be an ecological \textbf{revolution}.
He told his \textbf{friends} at the pub, “ \textbf{Ecology} is the \textbf{thing}. ” Ben and Dorothy Metcalfe uncovered a scheme to swindle B.C. ’s Sekani First Nation out of their homeland to construct a hydro-power dam financed by Axel Wennergren, a Swedish industrialist suspected of \textbf{working} with the Nazis.
Ben Metcalfe ’s \textbf{story} in the Vancouver Province newspaper was picked up by Toronto \textbf{media}, inciting Liberal Cabinet Minister Jack Pickersgill to blurt out, “ I ’m not interested in sick Indians. ” The incident blew up across Canada and Metcalfe became a \textbf{media} celebrity.
In 1969, Ben Metcalfe went fishing in Howe Sound, near Vancouver, and \textbf{witnessed} the stench from bellowing smokestacks at the Port Mellon pulp mill.
A few \textbf{weeks} later, he attended a Forestry Commission \textbf{meeting} and asked the \textbf{politicians} what they planned to do about the foul air in Howe Sound. “ We have to accept it, ” an industry \textbf{executive} told Metcalfe. “ No we don't, ” Metcalfe declared.
On their own initiative, at a cost of \$4,000, the Metcalfes placed twelve billboards around the city.
They created a logo to represent the environment, two waves joined together into a spiral maze. “ If you can promote companies and products, ” he told his \textbf{friends}, “ you can promote \textbf{ideas}. ” The billboards declared : \textbf{Look} it up!
You ’re involved.
An \textbf{ecology} \textbf{movement} was being \textbf{born} in Vancouver.
A Green Peace I was \textbf{one} of some 50,000 American draft resisters, opposing the Vietnam War, who slipped \textbf{north} into Canada between 1965 and 1973.
I soon met the \textbf{peace} \textbf{activists} such as Hunter and the Stowes.
Vancouver was an eclectic city.
Chinese and Japanese communities flourished, with Buddhist temples, Tibetan meditation centers, Quakers, beat poetry coffeehouses, and a radical network of back-to-the-land farmers, naturalists, and conservationists.
Jim and Marie Bohlen came to Vancouver to avoid the military draft for their \textbf{sons}, Lance and Paul.
Jim from New York ’s West Bronx had joined the US Navy and, like Metcalfe, had \textbf{witnessed} Japan after the bombings.
He met Marie — a nature illustrator, a \textbf{member} of the Sierra Club — at a Quaker gathering in Pennsylvania.
In Vancouver, they joined the Sierra Club, met the Stowes, and became close \textbf{friends}.
In the working-class neighborhood of East Vancouver, twenty-two year-old Bill Darnell organized an “ \textbf{Ecology} Caravan, ” which \textbf{toured} the province.
When the government proposed a highway through Vancouver ’s beach \textbf{front}, Darnell helped organize protests — with the Stowes, the \textbf{Hunters}, and \textbf{others} — that blockaded bulldozers and halted the project.
With this \textbf{campaign}, the \textbf{environmentalists} in Vancouver discovered the \textbf{greatest} \textbf{inspiration} to any social visionary : they could win.
Leading up to the anniversary, Greenpeace \textbf{will} reflect on — and we \textbf{will} see \textbf{media} coverage about — the early \textbf{campaigns}, and subsequent \textbf{years} of \textbf{lessons}, risks, failures, and \textbf{successes}.
Greenpeace, however, did not arise out of thin air.
It ’s important to consider some of the cultural context, circumstances, and \textbf{movements} that gave rise to Greenpeace in Vancouver, Canada in 1971.
A single event brought all these \textbf{people} together.
In November 1969, the United States announced a 5-megaton thermonuclear \textbf{bomb} \textbf{test}, code \textbf{name} “ Cannikan, ” scheduled for October 1971 on remote Amchitka Island, 4000 kilometers northwest of Vancouver, across the Gulf of Alaska, among the Aleutian Islands.
The \textbf{island} was supposedly a US Federal Wildlife Refuge for 131 species of sea birds.
An earlier, smaller \textbf{test}, had registered 6.9 on the Richter scale and killed wildlife all around the \textbf{island}.
The Cannikin \textbf{test} was going to be five-times more powerful.
Bob Hunter wrote a column about the risks, proposing that the explosion could cause a tsunami that might swamp western Canada.
For a \textbf{demonstration} at the US/Canada border, he created a sign, declaring : “ DO N’T MAKE A WAVE. ” At the protest he met Irving Stowe in \textbf{person}, who proposed forming a \textbf{citizen} ’s \textbf{group} to halt the \textbf{bomb}.
Stowe called Deeno Birmingham with the Voice of Women, Bill Darnell, the Metcalfes, and the Bohlens.
Hunter reached out to radical \textbf{activists} Rod Marining and Paul Watson.
They formed an ad hoc \textbf{group}, technically a \textbf{committee} of the Sierra Club, that they called “ The Don't Make a Wave Committee. ” The \textbf{group}, however, did not yet have a plan.
Jim and Marie were familiar with a 1958 Quaker protest \textbf{boat}, the Golden Rule, that sailed from California to Enewetak Island nuclear \textbf{test} \textbf{site} in the Philippine Sea.
The US Coast Guard intercepted the ship and arrested the \textbf{captain}, Albert Bigalow, but \textbf{pictures} of the ship appeared around the \textbf{world}, stirring the pacifist \textbf{movements}. \textbf{One} \textbf{morning}, over coffee, Marie told her husband, “ We should just sail a \textbf{boat} to Alaska. ” That same \textbf{day}, a Vancouver Sun reporter called, asking what the Sierra Club might be planning to stop the \textbf{test}.
Caught off guard, Bohlen blurted out, “ We \textbf{hope} to sail a \textbf{boat} to Amchitka to confront the \textbf{bomb}. ” The Sun ran the \textbf{story} the next \textbf{day}, and suddenly, the Don't Make a Wave Committee had a plan.
The Committee met at the Unitarian Church to discuss this \textbf{idea}, and ponder how they would find a \textbf{boat} and skipper willing to make the \textbf{trip}.
As the \textbf{meeting} ended, Irving Stowe flashed the “ V ” sign, and said “ \textbf{Peace}. ” Bill Darnell replied quietly, in the same off-handed manner that Marie Bohlen had suggested the \textbf{boat}, “ make it a green \textbf{peace}. ” This \textbf{term}, “ green \textbf{peace} ” articulated the merging \textbf{peace} and \textbf{ecology} \textbf{movements}, and stuck in \textbf{everyone} ’s \textbf{mind}.
When Lille d’Easum, the 71-year-old \textbf{executive} of the BC Voice of Women wrote a \textbf{research} paper in March 1970, “ Nuclear Testing in the Aleutians, ” the Committee published it under the “ Greenpeace ” banner, the \textbf{world} ’s first Greenpeace pamphlet.
Ex-Navy \textbf{officer} Jim Bohlen \textbf{toured} the waterfront, \textbf{looking} for a \textbf{boat}.
At the Fraser River docks, he met Captain John Cormack, 60, who owned an 80-foot halibut \textbf{boat} the Phyllis Cormack, \textbf{named} after his wife.
Cormack had 40 \textbf{years} \textbf{experience} fishing the \textbf{west} \textbf{coast}.
The \textbf{idea} of taking his \textbf{boat} across the treacherous Gulf of Alaska in the fall storm season did not faze him.
He agreed to take the charter.
When the Sierra Club rejected the \textbf{campaign} \textbf{idea}, the Don't Make a Wave Committee proceeded independently, incorporated as a non-profit \textbf{society}, and prepared to launch the 80-foot fish \textbf{boat}, which Captain Cormack agreed could be re-christened for the voyage as “ Greenpeace. ” Irving Stowe called his pacifist \textbf{friend} Joan Baez to \textbf{stage} a benefit concert to fund the \textbf{campaign}.
Baez could not attend, but introduced Stowe to Joni Mitchell, who agreed, and who brought rising star James Taylor with her.
They were joined by pacifist \textbf{music} legend Phil Ochs, and by popular Canadian band Chilliwack.
In October, 1970, the event raised \$17,000, enough for the \textbf{boat} charter and some basic expenses.
The global Zeitgeist after World War II resonated with a \textbf{desire} for \textbf{peace}.
Even so, the Cold War between Russia and the European/American allies led to \textbf{dozens} of surrogate conflicts — Korea, Vietnam, Palestine, Cuba — and a chilling nuclear arms \textbf{race}.
Greenpeace had emerged spontaneously, out of the social stirrings of the 1960s, civil rights, \textbf{women} ’s rights, Indigenous rights, workers ’ rights, pacifism, and the emerging \textbf{awareness} of \textbf{ecology}.
After the first \textbf{campaign}, the Don't Make A Wave Committee adopted the \textbf{name} that so perfectly articulated a new, emerging zeitgeist : Greenpeace Foundation.
During the 1950s, common \textbf{citizens} around the \textbf{world} began to hear new \textbf{words} such as “ fallout ” and “ genetic mutation, ” and the \textbf{fear} of nuclear holocaust gripped the \textbf{world}.
A nuclear disarmament \textbf{movement} \textbf{started} in Japan, in response to the \textbf{experiences} at Hiroshima and Nagasaki, and this \textbf{movement} connected with older pacifist \textbf{traditions} around the \textbf{world}.
In Providence, Rhode Island, in the United States, Irving and Dorothy Strasmich ( later “ Stowe ” ) were among millions influenced by the nuclear \textbf{bomb} threats.
Dorothy had organized the first social workers \textbf{union} in Rhode Island and became \textbf{president} of the state employees \textbf{union}.
Irving was a \textbf{lawyer} and jazz enthusiast, and his Black musician \textbf{friends} invited him to join the National Association for the Advancement of Colored \textbf{People}, the NAACP.
Dorothy and Irving married in 1953, with a reception dinner at NAACP headquarters.
The \textbf{couple} attended Quaker \textbf{meetings} and later took the \textbf{name} “ Stowe ” after Harriet Beecher Stowe, Quaker advocate for \textbf{women} ’s rights and the abolition of slavery.
Two decades later, they would help launch Greenpeace.
The Stowes were fighters. “ Find out just what \textbf{people} \textbf{will} submit to, ” I recall Dorothy quoting abolitionist Frederick Douglass, “ and you have found out the exact amount of injustice and wrong that \textbf{will} be imposed upon them. ” Canadian Ben Metcalfe lied about his \textbf{age} to get into the British Air Force during World War II.
While he served the British in India, Congress Party leader Mohandas Gandhi refused to cooperate with the British \textbf{war} effort.
Metcalfe sympathized with Gandhi ’s pacifist \textbf{movement} that made the British \textbf{look} like hypocrites.
To avoid bombing pro-Gandhi \textbf{villages} as ordered, Metcalfe and his Hawker Demon bomber pilot dropped their \textbf{bombs} in fallow fields while villagers below \textbf{watched} and waved.
The airmen ’s defiance was probably an act of treason under British law, but Metcalfe and his pilot supported Gandhi ’s views.
After the \textbf{war}, Ben became a \textbf{journalist} in Winnipeg, Canada and married \textbf{colleague} \textbf{journalist} Dorothy Harris.
The \textbf{couple} moved to Vancouver in 1956 and they both became instrumental in the founding of Greenpeace.
Bob Hunter learned about \textbf{bombs} and radioactive fallout in grade \textbf{school} in Winnipeg.
As a teenager, he heard about US Army General James Gavin telling the US Senate that a Soviet nuclear \textbf{attack} could leave vast regions of North America uninhabitable, which inspired him to write a short futurist novel, After the \textbf{Bomb}, about a post-nuclear-holocaust civilization.
Hunter quit \textbf{school} in 1958, after grade 11, and set out to be a \textbf{writer}.
In London, he met his future wife, Zoe, who introduced him to Bertrand Russell during a nuclear disarmament \textbf{march} in London.
In 1962, at the \textbf{age} of 21, Hunter read Rachel Carson ’s Silent \textbf{Spring} and began to think about a new \textbf{idea} : \textbf{Ecology}.
He realized that Carson ’s statement “ in nature, \textbf{nothing} exists alone, ” was literally true, and this changed the \textbf{way} he saw the \textbf{world}.
Stopping militarism wasn't enough ; we had to stop another \textbf{war} against the natural \textbf{world}.
Meanwhile, a young biologist, Dr.
Barry Commoner, had been collecting deciduous teeth from \textbf{children} in St.
Louis and documenting the absorption of strontium-90, a carcinogenic byproduct of nuclear explosions.
Militarism was now a source of deadly pollution.
The \textbf{peace} \textbf{movement} and the \textbf{ecology} \textbf{movement} began to merge.

% matched lemmas: activist, age, attack, awareness, bear, boat, bomb, book, campaign, captain, child, citizen, coast, colleague, committee, couple, date, day, demonstration, desire, dozen, ecology, environmentalist, everyone, executive, experience, fear, friend, front, gender, great, group, hope, hunter, idea, inspiration, island, journalist, lawyer, lesson, look, march, medium, meeting, member, mind, morning, movement, music, name, north, nothing, officer, one, other, peace, people, person, picture, politician, president, race, research, revolution, school, site, society, son, spring, stage, start, story, success, term, test, thing, tour, tradition, trip, union, village, war, watch, way, week, west, will, witness, woman, word, work, world, writer, year
\end{textsample}
