\begin{textsample}{POS Dim 1 – human – Score 91.00 – t226\_human.txt}  \label{ex:f1_pos_005}
From organising peaceful protests and fighting forest fires, to painting banners and \textbf{working} on digital \textbf{content} to spread \textbf{awareness}, Greenpeace \textbf{volunteers} come from all \textbf{ages} and \textbf{backgrounds} but are united in their \textbf{passion} for climate justice.
Despite the challenges, the courageous young \textbf{woman} stands ready to do what she can to protect the environment. “ As a firefighting \textbf{volunteer}, I must be ready when called upon in the event of a fire and ready to be placed wherever I am needed. ” What do you \textbf{hope} for with your action?
I want to continue to see the sky as it is now, blue and without the haze, and we can live our lives as usual.
I want the \textbf{children} to be able to play happily outside the \textbf{house} and all of \textbf{us} do not have to worry about the threat of respiratory infections anymore.
It was early November in 2013 when Ronan Renz Napoto and his \textbf{family} in Eastern Visayas, Philippines, heard over the \textbf{news} that there was a typhoon coming. “ Living in the Pacific, we ’re used to having typhoons so we weren't very worried, ” he said.
When Typhoon Haiyan hit, they were unprepared for its ferocity as it ripped through the Philippines. \textbf{One} of the most powerful typhoons in \textbf{history}, it caused widespread devastation and loss of lives.
For \textbf{years} after, Ronan would have nightmares of the \textbf{day}, often waking up with tears running down his face. “ I can still remember the sequence of \textbf{everything}, even right now \textbf{everything} keeps on flashing back to me.
It ’s still painful to remember those events, ” he said.
It was on his journey to process the trauma that led Ronan down the path of climate advocacy.
Already a youth community leader and trash crusader, the natural disaster drove his \textbf{awareness} of the urgency of the climate crisis and his need to act.
After taking part in a \textbf{story} sharing event organised by Greenpeace Philippines, he \textbf{started} to actively \textbf{volunteer} with Greenpeace, participating in brand audits and helping with administrative duties.
When the Rainbow Warrior was anchored in Tacloban as part of the Climate Justice \textbf{tour}, he \textbf{volunteered} to be a guide.
Ronan is also engaged in influencing policy makers in his community about creating effective environmental and plastic policies.
An eloquent orator and natural storyteller, he often speaks at Greenpeace events about climate justice and the science behind climate change. “ I also \textbf{talk} about food and agriculture, the impact of agriculture on climate change, and \textbf{other} \textbf{topics} revolving around climate, ” he said.
His most memorable activity with Greenpeace is also the \textbf{one} closest to his heart, and that is collecting \textbf{stories} from the different communities in the yearly commemoration of Typhoon Haiyan. “ It reminds me that behind the science of climate change, there are real \textbf{people} with real \textbf{stories}, ” he said. “ Statistics are important but we don't want to be just remembered as numbers, we want to be remembered and our \textbf{stories} to be remembered about who we are and how we struggled. ” Ronan, who \textbf{works} with different \textbf{organisations} on \textbf{research} about community engagement and building resilience, is also the founder of Balud, a youth-led \textbf{organisation} that promotes ecological consciousness in the Visayas. “ Coming from the provinces, I wanted to highlight the youth leaders from outside the big cities.
We want to create more opportunities for \textbf{people} who are considered to be minorities or coming from vulnerable communities so that everybody acknowledges that we also have powerful \textbf{stories}, ” he said.
Meet a few of our \textbf{volunteers} in Asia as they \textbf{share} their inspiring \textbf{stories} about \textbf{volunteering} with Greenpeace and how they are fighting to protect our planet.
The process has not been easy for the young \textbf{man} but Ronan ’s determination and \textbf{passion} keeps him focused on his advocacy. “ We have in our local \textbf{language} the \textbf{word} Padayon, which translates as ‘ to keep going ’.
Because advocacy can be very hard and sometimes, you never know if there is going to be a light at the end of the tunnel.
But it ’s worth it to keep on going, to keep on trying and moving forward, especially to bring your agenda forward. \textbf{Nothing} \textbf{will} happen if we just stop. ” Lee Hui Ling was exposed to environmental and social issues at a young \textbf{age}. \textbf{Born} into a \textbf{family} of \textbf{artists}, her \textbf{mother} had a strong environmental conscience, which she had expressed through her \textbf{art} and imparted on her \textbf{daughter}. “ As a \textbf{child}, I would worry about the ozone layer expanding, acid rain, rubbish pollution, flora and fauna going extinct.
Very serious \textbf{topics} for a little girl! ” After graduating from the Sarah Lawrence college in New York and moving back to Malaysia, Hui Ling ’s concerns for the environment grew, particularly after the Fukushima nuclear crisis in 2011, her \textbf{mother} ’s hometown. “ Greenpeace was very active in investigating the extent of the pollution at that \textbf{time} and I was very impressed by how transparent they were with the \textbf{information}, as opposed to a \textbf{lot} of cover ups, ” she said.
Hui Ling responded by setting up a Greenpeace Malaysia online community on various social \textbf{media} \textbf{platforms}.
This was instrumental in the eventual setting up of the Malaysian \textbf{office} in 2017 A committed \textbf{volunteer} and a natural leader, Hui Ling was involved from the very \textbf{beginning}.
She helped to organise and lead at meet-ups in cafes and community halls, as well as run workshops, training and retreats.
Taking direct action, she has participated in various \textbf{campaigns} such as Radioactive Ruse, Stop The Haze, and Break Free From Plastic. “ Environmental activism has taught me that doing \textbf{good} is not a sprint, but a marathon, and we need to develop the endurance and resilience to make it through the difficult \textbf{times}, ” she said. “ I think activism has been normalised, and with that \textbf{kind} of normalisation, it brings a level of safety.
It becomes a very effective \textbf{way} to speak about social issues and affect change.
An \textbf{artist} and educator, Hui Ling organised participatory \textbf{art} projects in line with Greenpeace Malaysia \textbf{campaigns} on deforestation, plastic pollution and consumerism. \textbf{One} of them was the Wings of Paradise project, where she led a \textbf{team} of 30 youth \textbf{volunteers} in creating a 64-meter long mural as part of a global \textbf{street} \textbf{art} \textbf{campaign} against deforestation in Papua. “ We see the inequalities between the Global North and Global South exacerbated through climate change.
This is a climate emergency, ” said Hui Ling. “ However, there is a beacon of \textbf{hope} in the youth activism of the last few \textbf{years}.
The youths of \textbf{today} are well organised, articulate and passionate in expressing their \textbf{desire} for positive change and a green and sustainable future for all. ” What change would you most like to see in the \textbf{world}?
As a whole, I would like \textbf{society} to be more \textbf{kind}, generous and inclusive.
I would also like to see a more proactive sharing economy and sustainable models of entrepreneurship, production and consumption.
Dwi Agustya Ningrum, also known as Tya to her \textbf{friends}, \textbf{started} \textbf{volunteering} as a Greenpeace firefighter after the \textbf{great} forest fire in Riau in 2015. “ The sky was filled with smoke.
Visibility was only about 1 to 2 \textbf{meters}, ” remembered Tya.
What does it mean to be a Greenpeace \textbf{volunteer}?
I think, first and foremost, \textbf{one} comes with the \textbf{idea} of wanting to effect change.
It comes from a place of empathy and being passionate about environmental issues in a very deep and involved \textbf{way}.
Greenpeace \textbf{volunteers} are different in a \textbf{sense} that they ’re more involved in environmental issues, it ’s not just greenwashing or a PR \textbf{thing}.
I think that ’s what sets Greenpeace \textbf{volunteers} apart, that there ’s more direct action.
Amika Jamjansri was a first \textbf{year} \textbf{student} at Mahidol University, Thailand, when enticed by the promise of diving \textbf{lessons}, she signed up for a beach clean up activity where the \textbf{volunteers} had to collect, separate and \textbf{group} all the plastic on the beach.
As Amika set to \textbf{work}, she was shocked by the magnitude of the plastic problem. “ This trash that we collected is only just a small amount of the plastic, what about the \textbf{rest} of it that ’s out there in the ocean? ” That exposure to the plastic waste problem signalled the \textbf{start} of Mika ’s environmental crusade, and she \textbf{started} adopting plastic-free habits.
In her \textbf{passion} to take action to help the planet, she joined Greenpeace Thailand as a \textbf{volunteer} in 2020.
She was \textbf{working} on the Solar Generation \textbf{campaign} when the pandemic hit and the country went into lockdown.
Despite the lack of mobility and activities being confided online, Mika continued \textbf{volunteering} with Greenpeace, joining the meat and dairy \textbf{campaign} where she helped to conduct online surveys.
When restrictions for in-person activities were lifted, she participated in brand audits for the plastic \textbf{campaign} and was an emcee for a Facebook Live event.
Her favourite \textbf{experience} was \textbf{training} with the \textbf{boat} \textbf{team} and the camaraderie she \textbf{experienced} with \textbf{other} like-minded \textbf{volunteers}. “ It was so fun!
I ’ve never done \textbf{anything} like this before and I learnt a \textbf{lot} about driving \textbf{boats} by the end of it, ” she said.
Besides being passionate about going plastic free, Amika is most concerned about the impact of climate change on \textbf{humanity} and wildlife, and the loss of biodiversity. “ There are so many incidents now of flooding and fires, and it sets off a chain reaction.
To tackle the problem, it comes down to law and implementation.
Those in power are not serious enough despite the big conferences.
There is just not enough real action. ” What \textbf{one} change would you most like to see?
I would like to see international law being changed in favour of solving environmental issues more effectively and that businesses are made to take responsibility for their actions.
Minseop Kim ’s awakening to the climate emergency came early this \textbf{year} when his home in Seoul, the capital city of South Korea and home to 10 million, was flooded after abnormally heavy rains. “ \textbf{Lots} of single-person households in their 20s in South Korea live in places that are very vulnerable to extreme weather like floods, storms, and rainfall.
Climate crisis is happening to \textbf{people} like me, who migrated to Seoul with less economic power to buy a \textbf{house} resilient to floods.
Based on the current situation with global warming, we ’re going to \textbf{experience} extreme weather more frequently and more severely, ” he said. “ I was scared whenever I heard of heavy rainfalls in the \textbf{news}.
But the \textbf{fear} doesn't change \textbf{anything}.
If I move from here to another place, \textbf{someone} \textbf{will} go through a similar situation as mine.
So I decided to find \textbf{something} I can do to make a \textbf{difference}. ” “ My \textbf{family} was my main \textbf{reason} to sign up.
My \textbf{parents} had a respiratory infection caused by the haze.
I had to see them use oxygen cylinders at home.
Small \textbf{children}, who are supposed to play with \textbf{friends} their \textbf{age} outside the \textbf{house}, were also forced to stay at home.
What is certain is that we are directly impacted by the forest and peat fires. ” Minseop joined Greenpeace as a \textbf{volunteer}, using his \textbf{skills} as a web designer to design effective, powerful visuals for Greenpeace Korea ’s Climate Suffrage \textbf{campaign}.
The \textbf{campaign} is aimed at changing climate policy in South Korea, including the Korean government ’s carbon neutral policy, the national law and the presidential candidates ’ commitment to climate action.
He also participated in the ‘ Green New Deal Civic Action ’, where he monitored the legislative action of congress \textbf{members} and made calls to the \textbf{office} of congress \textbf{members} to relate his personal \textbf{experience} regarding climate change and to urge them to take more ambitious climate action. “ I got inspired by \textbf{volunteers} abroad who called the local \textbf{politicians} demanding stronger climate action as a \textbf{citizen} lobbyist.
I feel empowered taking real climate action, more than voting, ” he said. “ \textbf{Volunteering} is important because \textbf{people} who are concerned about similar issues are connected and act together.
We can learn from each \textbf{other}, from their different perspectives on \textbf{one} issue.
For me, I ’ve learned from my peer \textbf{volunteer}, who has hearing loss, that the climate crisis is very connected to the human rights issue as well. ” What \textbf{one} change would you most like to see?
I \textbf{’d} like to see more \textbf{people} \textbf{talking} about climate change, discuss and act upon it.
Individual action matters itself, but I believe that collective civic action can change a giant ’s action.
The government and corporations are the two giants we have to change in order to respond to the climate crisis.
As \textbf{citizens}, we need to monitor and pressure the government and businesses. \textbf{Today} ’s five \textbf{phone} calls from \textbf{people} can change the \textbf{meeting} agenda for the next \textbf{morning}.
Why don't you call congress \textbf{members} to do the right \textbf{thing}?
Why don't you call the company and say that there \textbf{will} be more chances in the green economy transition?
A single \textbf{phone} call can make more of a \textbf{difference} than just being a voter.
If you believe \textbf{something} is important, \textbf{start} \textbf{volunteering} and \textbf{start} making a change in your \textbf{society}.
It was while Neha Gupta was \textbf{researching} \textbf{ways} to lead a more sustainable \textbf{lifestyle} and to manage waste effectively that led her to Greenpeace.
During last \textbf{year} ’s lockdown when \textbf{schools} were closed, the \textbf{teacher} of environmental science and English took the opportunity to sign up as a \textbf{volunteer} with Greenpeace India. “ I wanted to join Greenpeace to \textbf{work} upon solutions at a community and organizational level, progressing further from the individual effort, ” she said, citing rising temperatures and its impact on food security as some of her biggest climate concerns.
In her short \textbf{time} with Greenpeace, Neha has proven to be a dedicated and valuable \textbf{member} of the \textbf{organisation}. \textbf{One} of her first activities was to lead a webinar on home composting.
Neha had been experimenting with various \textbf{ways} to compost at home and was able to \textbf{share} her \textbf{experience} and learnings with over 100 attendees from around India. “ I was nervous but had \textbf{good} support from the \textbf{team} in \textbf{terms} of how to organize the webinar.
It was a challenging but memorable \textbf{experience} for me, ” she said.
Later, she took part in the Clean Air for Blue Skies \textbf{campaign} to address Delhi ’s air pollution.
Dressed in ethnic wear, she \textbf{campaigned} for clean air at Connaught Place, speaking to the public about air pollution.
Neha also participated in the Power the Pedal \textbf{campaign}, empowering low-wage \textbf{women} labourers to use bicycles as a safe, sustainable and fun mode of transport.
Now, every \textbf{year}, as \textbf{summer} rolls around, Tya gets worried because that ’s when most forest fires occur, either naturally or as a result of land or plantation clearing. “ In my \textbf{opinion}, forest and land fires that most often occur in Indonesia, especially Riau, are the result of human activities of those who no longer think about clean air, ” she said.
With Greenpeace, Neha found an open and friendly community with a \textbf{shared} \textbf{mission} to protect the environment. “ Greenpeace gives you the opportunity to interact with like-minded \textbf{people} but also gives you a \textbf{sense} of community, ” she said. “ The \textbf{organisation} has taught me how to \textbf{work} in a community, how to ideate together, plant together, act together and at the same \textbf{time}, have fun together while fighting for a cause. ” If you could make \textbf{one} change to combat the climate crisis, what would it be?
I would stress upon respecting resources around \textbf{us}, be it the human resource, natural resource or man-made resource.
When we learn to respect and value the \textbf{people} and \textbf{things} that surround \textbf{us}, a \textbf{lot} of \textbf{things} \textbf{will} come into perspective.
Why is \textbf{volunteering} important to you? \textbf{Volunteering} not only gives me a window to express my gratitude and give back to \textbf{society} but it also encourages personal growth and it helps me learn.
Through \textbf{volunteering}, I have come to become a part of a community who believes in the similar cause and \textbf{works} together to achieve shared goals, mobilize and impact various \textbf{other} communities and stimulate a \textbf{hope} for the future.
Want to \textbf{volunteer} with Greenpeace?
Get in touch with your local Greenpeace \textbf{organisation} to find out about \textbf{volunteering} opportunities. \textbf{Stories} were made possible with support from Norika Maurin Abriana ( Greenpeace Indonesia ), Kristina Hernandez ( Greenpeace Philippines ), Sakeenah Omar ( Greenpeace Malaysia ), Anchalee Pipattanawattanakul ( Greenpeace Thailand ), Jane Yu ( Greenpeace Korea ) and Abhishek Kumar Chanchal ( Greenpeace India ). “ It ’s important for me to \textbf{volunteer} as it ’s \textbf{time} for \textbf{us} to take on any role, no matter how small or big.
Because when the fire is burning and the haze is everywhere, it ’s a sign that we ’ve lost but the real defeat is if we do not do \textbf{anything} to prevent it from happening. ” Before carrying out actual \textbf{tasks} in the field, she had to take a \textbf{volunteer} firefighting \textbf{course} conducted by Greenpeace Indonesia.
Along with \textbf{other} \textbf{volunteers}, she was given training on fire prevention in peatlands as well as navigation, safety and security protocols. “ Peat is a highly flammable soil and difficult to extinguish.
Where I live, almost 50 % of the land is peat soil, ” said Tya.
As a newly enrolled firefighter, Tya ’s first few duties have been to carry out \textbf{awareness} \textbf{campaigns} to \textbf{talk} about the dangers of forest fires, especially peat fires, to nature and to human health.
In order to protect nature, she encourages the public to reduce their use of single-use plastic and also to carry portable ashtrays to hold cigarette butts. “ It could be simple acts but the effect that is felt is extraordinary, especially when \textbf{people} around you are also slowly becoming more aware of climate issues, ” she said.
In 2016, Tya and her \textbf{team} spent two \textbf{weeks} extinguishing fires in the Bukit Timah area.
On \textbf{one} of the \textbf{days}, a strong wind had picked up, flaming the smouldering, underground fire back up to the surface where it quickly \textbf{started} getting more aggressive.
Back at camp, situated in the safe area, Tya and her \textbf{other} \textbf{colleagues} were worried for the safety of their teammates out in the field.
Fortunately they made it back safely, but “ their faces \textbf{looked} more tired than usual, ” said Tya, who had to be rushed to the hospital because of smoke inhalation. “ I learnt a \textbf{lot} in a very short \textbf{time}, ” she said.

% matched lemmas: age, anything, art, artist, awareness, background, bear, beginning, boat, campaign, child, citizen, colleague, content, course, daughter, day, desire, difference, everything, experience, family, fear, friend, good, great, group, history, hope, house, humanity, idea, information, kind, language, lesson, lifestyle, look, lot, man, medium, meeting, member, meter, mission, morning, mother, news, nothing, office, one, opinion, organisation, other, parent, passion, people, phone, platform, politician, reason, research, rest, school, sense, share, skill, society, someone, something, start, story, street, student, summer, talk, task, teacher, team, term, thing, time, today, topic, tour, train, us, volunteer, way, week, will, woman, word, work, world, year
\end{textsample}
