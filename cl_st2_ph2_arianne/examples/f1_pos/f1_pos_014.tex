\begin{textsample}{POS Dim 1 – human – Score 80.00 – t186\_human.txt}  \label{ex:f1_pos_014}
From Chile to Canada, \textbf{women} are at the forefront of the climate crisis, and are \textbf{one} of the most impacted \textbf{groups} by climate catastrophes.
Along with \textbf{other} minorities such as Black, Indigenous and \textbf{People} of Colour, LGBTQ+ folks and low-income communities, \textbf{women} are often part of those demographics as well, exposing the intersectionality of climate impacts, \textbf{gender} inequality and social injustice.
My \textbf{name} is Cláudia Farinha, I am a \textbf{woman} of the Land and of the Fight!
I am a \textbf{family} farmer and I live with my \textbf{family} in a Land Reform Settlement in Brazil.
I am an educator, communicator, ecofeminist, \textbf{lawyer} and socio-environmental \textbf{activist}.
I \textbf{worked} in the rural trade \textbf{union} \textbf{movement} for over two decades, which liberated the voice and \textbf{strength} of rural \textbf{women} the same \textbf{way} I forged myself in the struggle as farmer leader.
The rights of rural \textbf{women} and \textbf{men} are among the priorities in the fight for \textbf{better} living conditions.
In 2020, I joined a \textbf{course} to prepare \textbf{activists} to make \textbf{communication} an instrument of \textbf{women} ’s organizing.
In 2021, via the ECOAR project ( \textbf{Meetings} for \textbf{Content} on Anti-Deforestation in the Amazon and Cerrado ), I ’ve been in workshops to create socio-environmental \textbf{content} related to social \textbf{media}.
Since then, I have integrated my environmental activism on social \textbf{media} and on the \textbf{streets} in the fight for a \textbf{better} \textbf{world}.
It is not easy being a \textbf{woman}, an \textbf{activist}, a digital influencer and finding the \textbf{best} \textbf{way} to connect with \textbf{people} and raise \textbf{awareness} on climate issues.
We know the gravity of what we are \textbf{experiencing}, but the Brazilian Congress, the mass \textbf{media} and large capitalist corporations ignore the urgency of the environmental agenda.
The biggest challenge is to deepen contact with \textbf{people}, so that together, we can denounce and highlight the calamity we are \textbf{experiencing}.
The environmental, political, ethical and health crises require \textbf{communication} that shows the causes of these problems, for the collective construction of \textbf{awareness} about them, so we can undertake efforts to change this reality.
The only \textbf{way} is together.
I don't see any \textbf{other} alternative.
It is no use demanding the Congress, it is necessary to build a collective force to transform reality and save the planet.
This election \textbf{year} is very important in Brazil.
We need to unite and fight to occupy spaces of power and ensure we have voice and \textbf{strength} to have leaders who actively defend the environment and prioritize social justice, \textbf{gender} equality, the rights of \textbf{women} farmers, and the climate.
My \textbf{name} is Jackie Zamora from Mexico, for as long as I can remember I was taught to respect nature and when I learned from a very young \textbf{age} that our planet faced many threats, I wanted to somehow make a possitive \textbf{difference}.
In 2012 I joined Greenpeace Mexico and multitasked between being a College \textbf{student} and a \textbf{volunteer} coordinator.
This was just the \textbf{beginning} of my environmental \textbf{activist} journey. 10 \textbf{years} later I ’ve been in the Arctic, sailed in all 3 Greenpeace Ships and I ’m currently the Engagement lead for the European Mobility \textbf{Campaign}.
I grew up in Latin America, where environmental \textbf{activists} and human rights defenders face intimidation, threats and violence.
It ’s \textbf{one} of the most dangerous regions in the \textbf{world} to fight for climate justice and \textbf{gender} equality.
Although these \textbf{attacks} affect all defenders, \textbf{women} \textbf{activists} are specifically targeted and face additional obstacles and risks like gender-specific threats or barriers to accessing decision-making spaces and \textbf{platforms}.
Many brave environmental and human rights \textbf{activists} like Berta Cáceres from Honduras and Marielle Franco from Brazil have lost their lives in these fights.
These same \textbf{women} are showing the \textbf{world} that a planet centered on justice, resilience and \textbf{care} is not only possible but necessary.
Here are 8 \textbf{women} shaping the climate \textbf{conversation} in the Americas region : It ’s important to acknowledge the intersectionality that is inherent to how \textbf{people} \textbf{experience} the environment, to call for solutions that appreciate varied \textbf{experiences}.
Environmental justice is this intersection of both social justice and environmentalism.
As Berta Cáceres said “ \textbf{Let} ’s build \textbf{societies} that are able to coexist in a \textbf{way} that is dignified, just and protective of life ” My most memorable moment in my activism journey was the first \textbf{time} I saw icebergs and penguin colonies in the Antarctic and a few minutes later I saw plastic bottles in the very same sea.
Yes, there is a \textbf{lot} of plastic waste in the Antarctic.
It was heartbreaking to see the reality our oceans are facing. \textbf{Everyone} ’s fate is bound to the fate of our planet and we have a \textbf{shared} responsibility in the \textbf{mission} to protect our only home.
My \textbf{name} is Julialynne Walker and I ’m a land steward.
I like to put my \textbf{hands} in the \textbf{earth}.
I \textbf{started} a garden at my \textbf{mother} ’s \textbf{house}, and then \textbf{something} at her church.
In the process, I became reacquainted with the Columbus ( Ohio, US ) environment, in particular, the historic African-American section.
There just was a devastation that I found astonishing due to a number of different changes.
I wanted to be able to address that in some \textbf{way}, and that ’s how I got \textbf{started} with the Bronzeville growers market. \textbf{People} were coming to me and saying : I like to garden, but I don't know how ; this \textbf{looks} exciting but I live in a small place.
I approached Ohio State University and asked about doing a \textbf{class} there.
But then COVID hit, so it all got canceled.
I next found a young \textbf{person} to show me how to use zoom.
Then I got on my back porch, gathered plants and different \textbf{things} and did a 10 \textbf{week} free online introduction to agriculture \textbf{class}.
Everybody was excited, \textbf{people} came on and off as they chose, I wasn't prescriptive.
I realized there were \textbf{people} that really wanted to grow food and that I was developing a sort of vertical model of integration with this \textbf{work}.
I got funding to buy large plastic bins and we literally just dug holes in the bottoms, put about a third of leaves and small twigs and the soil.
We gave seeds and some starter plants, so \textbf{people} could have \textbf{one} to three bins and they could place them in their yard, on the porch, it was flexible.
If each of \textbf{us} can take responsibility for the little \textbf{piece} that we have and do it in a \textbf{way} that makes \textbf{sense}, that ’s how to \textbf{start} creating change in the \textbf{greater} whole.
I ’m afraid of \textbf{people} who are not open to the truth.
I ’m just dumbfounded at some \textbf{things} happening in the United States.
I do not understand how \textbf{people} don't see that their own interests are being sabotaged by this commitment to falsehood.
Every \textbf{one} of \textbf{us} has a responsibility to do \textbf{something} that ensures a future, because this is it, we got \textbf{one} planet.
I don't know why this is so difficult to understand.
There is no place else for \textbf{us} to go.
If you ’re not for zero waste, how much waste are you for?
After reduce, reuse, recycle, I \textbf{’d} add a fourth \textbf{one} : rethink.
I am a trans \textbf{woman} ( I also like the \textbf{term} “ travesti ”, which is Latin ), communicator, popular educator and poet, who lives in Rio \textbf{de} Janeiro, this land so marked by beauty and chaos, in Brazil.
Being a trans \textbf{person}, in the country that kills the most LGBT \textbf{people} in the \textbf{world}, and in such a violent city, brings to me the urgency to write my texts, my poetry and produce \textbf{content} for social \textbf{media}.
I think it is important to reflect on how violence occurs in different \textbf{ways}, including the lack of access to water, decent housing, green spaces in cities and several \textbf{other} factors that are also related to the environment and the social gap between \textbf{people}.
Brazilian trans \textbf{people}, for the most part, do not have the basics to survive.
And we are all the \textbf{time} stating that we need not only to survive, but above all to live.
We need to be seen as real \textbf{citizens}.
And this fundamentally involves building a planet that is socially and environmentally just, where all \textbf{people} can live their lives with guaranteed rights and in harmony with ecosystems.
We need to understand that the human \textbf{being} is also nature, not \textbf{something} apart from it.
Understanding the connections between nature and \textbf{humanity}, and that all humans must have their human rights guaranteed, is urgent.
I am physically based in Scotland, though most of my engagement with Greenpeace has taken place in Canada.
As a \textbf{volunteer}, I ran the KleerCut \textbf{campaign} in Thunder Bay, Ontario and supported a local \textbf{group} in Ottawa.
I am now a \textbf{member} of the Board of Directors of Greenpeace Canada, and currently serve as the Board Chair.
As a glaciologist, my “ \textbf{day} \textbf{job} ” consists of \textbf{research} on the deterioration of icebergs and retreating Antarctic glaciers.
Having participated in actions in Argentina, relating the debates on feminism, economics and environmentalism, marked me a \textbf{lot}.
I believe that our fight needs to break borders and solidarity needs to be international.
Whether in Rio \textbf{de} Janeiro or Vancouver, whether in Buenos Aires or Lisbon, our struggles are connected and the \textbf{answers} also need to come together.
I am Maria Paz Valenzuela, a Chilean mountaineer.
This is an activity that I have carried out since my adolescence in all continents of the \textbf{world}.
I currently carry out mountain and trekking activities, directing \textbf{groups} of \textbf{people} interested in outdoor activities.
The Magmandinas expedition to Ojos del Salado was a totally different \textbf{experience} to any \textbf{other} expedition for me.
The motivation and soul of this project is unique.
The \textbf{idea} was initially conceived as a \textbf{women} ’s expedition, since its fundamental plan was to bring together \textbf{women} from Latin America and create a space for reflection and coexistence with the environment to finally deliver a powerful message about the importance of \textbf{care} for our land, for ourselves and the \textbf{relationship} that we establish with our surroundings, as well as to empower \textbf{us} as mountaineer \textbf{women}, moving our own limits and achieving our own goals.
The role of all human \textbf{beings} cannot be postponed when we \textbf{talk} about protecting the planet.
This is not a \textbf{gender} issue, it transcends that.
It is a personal and \textbf{group} responsibility to take \textbf{care} of our environment.
Educate to protect, \textbf{care} for and \textbf{love} what surrounds \textbf{us}.
Educate to reconnect and \textbf{return} to the purest essence as an individual, recover the ability to wonder at each natural event, open your \textbf{eyes} and see, see with your heart.
Indigenous \textbf{woman} from the Sateré Mawé \textbf{People}, I am a biology \textbf{student} at the University of the State of Amazonas, Brazil.
Artisan, presenter, communicator and environmental \textbf{activist}, I \textbf{work} to simplify Indigenous issues online and as a defender of the environment.
When we Indigenous \textbf{People} are \textbf{born}, there is no moment when we decide to be \textbf{activists}, we are \textbf{born} \textbf{activists}, because we have always been fighting, for our culture, our \textbf{language}, identity and territory, and these struggles go \textbf{hand} in \textbf{hand} with the fight for the environment, since the protected Indigenous Lands are the \textbf{ones} with the most biodiversity.
As a biology \textbf{student}, much of the environmental activism comes from combining Indigenous Peoples ’ fight for our lands with the fight for climate.
And as \textbf{women}, our actions are often discredited, but in reality we are protagonists. \textbf{People} think that scientists are \textbf{men}, with glasses and a PhD, but \textbf{women} also do science and are more than ever in the daily fight for the protection of lands, along with all Indigenous Peoples.
I was able to participate in COP26, which was a historic milestone for me.
It was a unique \textbf{experience}, being an Amazonian, an Indigenous \textbf{woman} and being on another continent telling \textbf{people} and \textbf{other} countries how their actions impact our environment.
It felt unique to \textbf{talk} about how little \textbf{time} we have left and that the solution are Indigenous Peoples, the main defenders of the planet.
In addition, the \textbf{march} for the climate, made by the youth at COP26, and the first \textbf{march} of Indigenous \textbf{women} in Brazil were unique sensations of belonging and action.
It ’s hard \textbf{work} and sometimes we don't feel optimistic, but we have to keep going.
The current and impending impacts of glacier loss have local and global consequences, including endangering water resources for millions of \textbf{people} and altering coastlines worldwide.
Greenpeace has an important role to play in ensuring that \textbf{society} is making all efforts to mitigate climate change and address the inequalities that it \textbf{will} exaggerate.
We all have \textbf{something} – \textbf{strengths}, \textbf{skills}, perspectives, \textbf{experiences} – to contribute.
We all have \textbf{something} to learn from each \textbf{other} too.
I think our openness to new \textbf{ideas} and ways-of-thinking is key as we \textbf{look} to create and embrace change together.
I ’m Billie Lee and I ’m from Indiana, US.
I ’m a TV \textbf{writer}, producer and also an \textbf{activist} for food sustainability and transgender rights.
As a long-time vegan, creator and \textbf{author} of the popular blog She ’s So Vegan, I try to use my own \textbf{lifestyle} to educate and inspire \textbf{others} through the power of food.
I first gained notoriety on the hit Bravo series Vanderpump Rules, and now I am \textbf{working} on my first \textbf{book}, Why Are You So Sensitive?, which \textbf{will} be published by Andrew McMeel and released in 2023.
In addition, I continue to \textbf{work} for equity—in the workplace, in housing and in food accessibility/sustainability and in entertainment.
I believe we are here to be of \textbf{service} for our planet and those most vulnerable to environmental dangers.
As a trans \textbf{activist} I ’ve noticed we are fighting the same enemy.
Climate deniers and the \textbf{politicians} that support big oil are the same enemy we trans \textbf{people} face while fighting for our fundamental human rights.
My most memorable moment as an \textbf{activist} was protesting the big oil companies and for the new green deal, seeing all my trans/non-binary community on the \textbf{front} line fighting for our planet.
With all the discrimination trans and non-binary \textbf{people} \textbf{experience} we still know there is no life with or without rights if there is no planet to call home.

% matched lemmas: activist, age, answer, attack, author, awareness, bear, beginning, being, book, campaign, care, citizen, class, communication, content, conversation, course, day, de, difference, earth, everyone, experience, eye, family, front, gender, good, great, group, hand, house, humanity, idea, job, language, lawyer, let, lifestyle, look, lot, love, man, march, medium, meeting, member, mission, mother, movement, name, one, other, people, person, piece, platform, politician, relationship, research, return, sense, service, share, skill, society, something, start, street, strength, student, talk, term, thing, time, union, us, volunteer, way, week, will, woman, work, world, writer, year
\end{textsample}
