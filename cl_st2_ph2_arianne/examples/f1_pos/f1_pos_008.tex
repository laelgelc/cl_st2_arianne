\begin{textsample}{POS Dim 1 – human – Score 88.00 – t272\_human.txt}  \label{ex:f1_pos_008}
From 1974 to 1982, I served as \textbf{photographer} on Greenpeace \textbf{campaigns}.
Here are a \textbf{dozen} photographs from those \textbf{years} and some \textbf{memories} that they evoke : The \textbf{whales} were carved up on the factory ship deck, and a pipe protruding from the hull ran continually with blood.
Sharks followed the factory ship, and the stench from the floating slaughterhouse made \textbf{us} all nauseous.
Taking a photograph from a moving Zodiac, even in a mildly choppy sea, is extremely difficult, and we struggled with this at first.
We learned in 1975 that the most effective method was to stand in the bow, with a rope from the prow around \textbf{one} ’s waist.
With this method, the \textbf{photographer} could lean back, legs and rope creating a tripod, and remain stable with two \textbf{hands} free.
This photograph was taken on the James Bay, the Canadian minesweeper used for the 1976 and 1977 \textbf{whale} \textbf{campaigns}.
There was no internet, and we possessed no means to distribute \textbf{news} and photographs from the ships during this \textbf{time}, except by marine radio.
Hunter and I would call our respective newspapers on the radio, and read \textbf{stories} to our \textbf{editors}, who would record and transcribe them.
To send photographs, we had to reach land, process the \textbf{film}, and send photographs on the wire \textbf{services}.
Hunter was a splendid \textbf{writer} and a \textbf{media} prodigy.
He was a \textbf{student} of \textbf{media} analyst Marshall McLuhan, who predicted the internet in the 1960s and showed how \textbf{communications} technologies affect human cognition and therefore influence social \textbf{organization}.
Hunter ’s now-famous “ mind-bomb ” theory of social change, suggested that the fastest \textbf{way} to change \textbf{society} involved launching \textbf{images} and \textbf{stories} — “ \textbf{mind} \textbf{bombs} ” — that would “ explode in \textbf{people} ’s \textbf{heads} all over the planet. ” Greenpeace, he suggested, should \textbf{let} \textbf{others} sort out the details ; our goal was to infect the entire human \textbf{family} with an \textbf{idea} : We are relatives with all living \textbf{beings}, we are \textbf{children} of our Mother Earth, and we have a responsibility to \textbf{care} for her.
Bree was a seasoned \textbf{activist}, had saved a stand of cottonwood trees in Vancouver, helped organize the 1976 harp seal \textbf{campaign}, and coordinated the \textbf{engine} \textbf{room} \textbf{watch} schedule during the 1976 \textbf{whale} \textbf{campaign}. \textbf{Engine} \textbf{room} \textbf{watch} is a serious \textbf{job} on any ship.
It took about an \textbf{hour}, twice a \textbf{day}, to \textbf{check} all the oil pressure valves, fuel pumps, coolant and temperature gauges, to \textbf{check} for deterioration or cracks, to operate the bilge pump, and keep the \textbf{engine} \textbf{room} clean. \textbf{One} of the \textbf{greatest} threats on a ship is an \textbf{engine} \textbf{room} fire, which can ignite from \textbf{engine} sparks striking pools of untended oil or fuel.
The popular \textbf{media} \textbf{images} from Greenpeace \textbf{campaigns} highlight \textbf{activists} engaged in visible confrontation with the opposition.
However, for every \textbf{activist} climbing a tower or blockading a whaling ship, hundreds of \textbf{volunteers} perform the less visible, less glamorous \textbf{jobs} that are essential for any successful \textbf{campaign}.
Six \textbf{days} earlier, Mel Gregory, Caroline Keddy, and I had traveled across the bay to the Keystone Club in Berkeley and \textbf{talked} our \textbf{way} into the backstage area, where we met with Garcia, who agreed to do the benefit.
In the next five \textbf{days}, we secured the permits, drove 50 miles \textbf{north} of San Francisco to meet with a concert producer, created a perimeter around the pier, built the \textbf{stage} and speaker \textbf{platforms}, built a sound technician ’s hut, decorated the ship itself as a backstage area, put tickets on sale, visited radio \textbf{stations} to promote the show, and sold out all the tickets.
We were always broke in those \textbf{days}, raising \textbf{money} as we traveled on \textbf{campaigns}.
The show raised \$20,000, enough for \textbf{us} to refuel and provision the ship and go back out after the whaling fleets.
Garcia was joined by John Kahn on bass, Ron Tutt on drums, Keith Godchaux on keyboards, and Donna Jean Godchaux and Maria Muldaur singing vocals with Garcia. \textbf{Members} of the Jefferson Airplane band can be seen in the \textbf{background}, on the ship.
Without Captain John Cormack, there probably is no Greenpeace \textbf{today}.
He agreed to take the original 1971 \textbf{crew} to the Aleutian Islands nuclear \textbf{test} \textbf{site} in his 66-foot fishing \textbf{boat}, and in 1975 and 1976 he agreed to take his \textbf{boat} on the \textbf{whale} \textbf{campaigns}.
Cormack was an ex-wrestler, 40 \textbf{years} a fisherman, strong and self-assured, but with a modest bearing.
He did not drink alcohol, never used foul \textbf{language}, and commanded his ship with experienced authority.
If any among the \textbf{crew} violated ship protocol — standing in a doorway, opening a tin upside-down, sitting in the skipper ’s spot at the galley \textbf{table} — the teaching often came not with \textbf{words}, but with a sharp elbow.
Here, Hunter playfully fights back.
Cormack and his wife Phyllis never had \textbf{children}.
Bob lost his own \textbf{father} at \textbf{age} six.
The two \textbf{men} bonded during the Greenpeace \textbf{campaigns}, with Cormack becoming a sort of surrogate \textbf{father} to Hunter, who always treated the Captain with utmost respect.
Cormack was a master of tough \textbf{love}, holding the \textbf{crew} to high standards of \textbf{work} and behaviour.
We all learned from him and grew to \textbf{love} him dearly.
The very first Greenpeace \textbf{office} was in the home of Dorothy and Irving Stowe in Vancouver.
We used to meet at the Unitarian Church, in our kitchens, in coffee shops, and in pubs.
In 1975, leading up to the first \textbf{whale} \textbf{campaign}, we \textbf{shared} this small \textbf{office} space with the only \textbf{other} \textbf{ecology} \textbf{organization} in Vancouver, the Society for the Prevention of Environmental Collapse ( SPEC ).
Our tiny adjoining \textbf{office}, run by Bobbi Hunter, consisted of \textbf{one} telephone, a bulletin \textbf{board}, and a long wooden slab for a desk.
We often met at the pub across the \textbf{street}, upstairs by an open window, and when we got calls in the \textbf{office}, Bobbi would shout across the \textbf{street}, and the requested \textbf{party} would run across the \textbf{street} to the \textbf{phone}.
This photo was taken with my camera, by a local \textbf{friend}.
Three \textbf{years} earlier, I had left the United States as a military draft resister, so when we entered San Francisco after confronting the whaling fleet, I was concerned that I might be arrested.
Much to my surprise, rather than being met by federal agents, I was met by my maternal grandmother, Elizabeth Goodwin, who had been a \textbf{great} \textbf{inspiration} in my life, always encouraging me to follow my own heart and my values.
The Vietnam War had ended that \textbf{spring}, and since we had just confronted and embarrassed the Russians, who were illegally killing undersized \textbf{whales} in US territorial waters, the immigration authorities ignored my legal status as a draft resister and treated \textbf{us} like heroes.
Behind me on the ship is \textbf{film} cameraman Fred Easton.
The participants in this \textbf{picture}, from the left : Susi Newborn, Art Van Remundt, Hans Guyt, Jon Castle, Tim Mark, Martini Gotje, John Frizell, David McTaggart, Rémi Parmentier, David Moodie of the Fri leaning on pole in the back, Nancy Foote, Peter Balvers, Peter Woof, Louise Trussel, Bill Gannon, Alan Thornton, Glen Jonathans, and Naomi Petersen.
At this \textbf{meeting}, but absent from this \textbf{image} : Bob Hunter, Geert Drieman, Kay Treakle, Pete Wilkinson, and Campbell Plowden.
The building is 98 Damrak, the first Netherlands Greenpeace \textbf{office}, in the centre of Amsterdam.
On this \textbf{day}, the original Greenpeace Foundation entrusted all rights to the \textbf{name} and \textbf{organization} to an international council that included \textbf{representatives} from Canada, US, France, UK, Netherlands, Denmark, and New Zealand.
Australia and Germany joined soon thereafter. \textbf{Today} : 26 national/regional \textbf{organisations} in over 55 countries around the \textbf{world}.
Here, Bob Hunter, Greenpeace \textbf{president}, sits at the \textbf{head} of the \textbf{table} in the shared \textbf{meeting} \textbf{room} with papers in \textbf{hand}.
Bree Drummond, who had climbed into cottonwood trees to protect them from loggers, leans against the \textbf{wall}.
Leigh Wilks, a nurse at Vancouver Hospital, is taking notes, sitting beside Rod Marining, \textbf{one} of Vancouver ’s \textbf{ecology} visionaries, and \textbf{media} liaison during the \textbf{whale} \textbf{campaigns}.
There were no salaries. \textbf{Everyone} was a \textbf{volunteer}.
A \textbf{black} and white \textbf{version} of this iconic \textbf{image} from the 1975 \textbf{whale} \textbf{campaign} appeared in newspapers around the \textbf{world}.
When we planned the \textbf{whale} \textbf{campaign}, \textbf{one} of our goals was to replace the old Moby Dick ’ \textbf{image} of whaling — brave little \textbf{men} in tiny \textbf{boats} pursuing a ferocious leviathan — with the reality of modern whaling — giant steel ships with exploding harpoons decimating vulnerable \textbf{families} of cetaceans.
This \textbf{image} helped flip that perception, as we can visually see and viscerally feel the deadly imbalance of power.
This photograph was taken on the first \textbf{day} that we encountered the Russian whalers, June 27, 1975, over the Mendocino Ridge sea mounts, 50 miles off the \textbf{coast} of California.
We \textbf{worked} so feverishly that \textbf{day}, that I did not realize until later that evening how heartbroken and traumatized I felt after \textbf{witnessing} the carnage.
Taeko, from Japan, was possibly the most experienced environmental \textbf{activist} on the \textbf{crew} of the 1975 \textbf{whale} \textbf{campaign}.
She had \textbf{worked} with victims of mercury poisoning in Minamata, Japan, where over 2,000 \textbf{people} had died and thousands more suffered life-long afflictions caused by industrial wastewater from the Chisso Corporation chemical plant.
She had led a clean-air \textbf{campaign} in Tokyo, and a protest against the new Tokyo airport that devastated a rural community.
Mel was a Vancouver musician, who had a deep affinity with animals.
He would protect spiders from being killed by \textbf{others}, even by strangers, and he had a pet iguana \textbf{named} Fido.
During this voyage, he experimented with playing \textbf{music} to \textbf{whales} through under-water speakers, while recording their response.
Mel brought a “ \textbf{Dream} \textbf{Book} ” onto the ship for \textbf{crew} to record their \textbf{dreams}, which led to many fascinating discussions and to a historic dream/poem in the \textbf{book} from renowned poet Lawrence Ferlinghetti.
Between 1974 and 1978, Bobbi was the Greenpeace Foundation chief fundraiser and \textbf{office} \textbf{manager}.
In 1976, she served on the \textbf{whale} \textbf{campaign} \textbf{crew}.
This was the first \textbf{time} that we actually stopped a harpoon \textbf{boat}, dead in the water, ending its hunt.
During the 1970s, the \textbf{office} \textbf{teams} and \textbf{campaign} \textbf{teams} were interchangeable.
We believed that the \textbf{activists} should do a variety of \textbf{jobs}, in the \textbf{office}, with the public, and on \textbf{campaigns}.
We made every effort to make sure that willing \textbf{office} staff had opportunities to serve on the \textbf{front} lines of \textbf{campaigns}.
Bobbi was on the marine radio almost every \textbf{day}, with the staff in Vancouver, keeping track of budgets and \textbf{campaign} logistics.
We developed friendly relations with most of the whaling \textbf{crews} ( not necessarily with the \textbf{officers} ).
The first \textbf{time} we got close enough in a Zodiac to speak with them, \textbf{someone} asked \textbf{us} in English : “ Do you have LSD? ” We weren't able to satisfy this request, but we \textbf{returned} to our ship and brought back bottles of rum and \textbf{whale} pins, which they were pleased to receive.
I realized during various \textbf{ecology} \textbf{campaigns} that every ecological issue we addressed had an impact on \textbf{someone} ’s \textbf{job} or livelihood.
Clearly, part of the ecological transition of \textbf{society} would require efforts to support those employed in harmful industrial or military processes.
This challenge remains with \textbf{us} to this \textbf{day}.
Solving our ecological crisis probably means a substantial revision of our entire economic system.

% matched lemmas: activist, age, background, being, black, board, boat, bomb, book, campaign, care, check, child, coast, communication, crew, day, dozen, dream, ecology, editor, engine, everyone, family, father, film, friend, front, great, hand, head, hour, idea, image, inspiration, job, language, let, love, man, manager, media, medium, meeting, member, memory, mind, money, music, name, news, north, office, officer, one, organisation, organization, other, party, people, phone, photographer, picture, platform, president, representative, return, room, service, share, site, society, someone, spring, stage, station, story, street, student, table, talk, team, test, time, today, us, version, volunteer, wall, watch, way, whale, witness, word, work, world, writer, year
\end{textsample}
