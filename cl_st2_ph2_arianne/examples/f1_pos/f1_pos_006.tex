\begin{textsample}{POS Dim 1 – human – Score 91.00 – t999\_human.txt}  \label{ex:f1_pos_006}
In 1971, a small \textbf{group} of \textbf{activists} set sail to the Amchitka \textbf{island} off Alaska to try and stop a US nuclear \textbf{weapons} \textbf{test}.
The \textbf{money} for the \textbf{mission} was raised with a concert, their old fishing \textbf{boat} was called “ The Greenpeace ”.
This is where our \textbf{story} begins.
Afterwards, attendance at the \textbf{meeting} swelled, the \textbf{money} \textbf{started} to pour in.
By the end of October, the \textbf{group} had raised more than \$23,000.
Greenpeace was ready to go.
Yet, the voyage was a disaster.
The \textbf{boat} left the harbour at dusk on 15 September 1971, but internal tensions soon flared up. “ We never quite managed to go in the direction we wanted to go, or be in the place we wanted to be.
And we fought bitterly among ourselves about it. \textbf{Everything} we did or said got sucked into an overwhelming power struggle. ” “ Here we were, supposedly saving the \textbf{world} through our moral example, emulating the Quakers, no less, when in reality we spent most of our \textbf{time} at each \textbf{other} ’s throats, egos clashing, the \textbf{group} fatally divided from \textbf{start} to finish. ” Even worse, “ The Greenpeace ” was intercepted by the US navy, before it even got close to the Amchitka \textbf{testing} \textbf{site}.
Failure \textbf{looks} different, however.
The Amchitka voyage sparked a flurry of public interest.
The \textbf{media} went wild about the small \textbf{group} of \textbf{activist} who had sailed off in the face of \textbf{great} adversity – the first “ \textbf{media} mindbomb ”, as Bob Hunter conceived of those early Greenpeace actions, had been launched. “ As it turned out, all my angst was unnecessary, ” he later wrote. “ \textbf{Time} has proven my post-trip despair to be utterly mistaken.
The \textbf{trip} was a \textbf{success} beyond anybody ’s wildest \textbf{dreams}. ” The nuclear \textbf{bomb} the \textbf{group} had come to stop went off, but the \textbf{tests} planned for after that were cancelled.
Five \textbf{months} after the \textbf{group} ’s \textbf{mission}, the US stopped the entire Amchitka nuclear \textbf{test} programme.
The \textbf{island} was later declared a bird sanctuary. “ Whatever \textbf{history} decides about the big \textbf{picture}, the legacy of the voyage itself is not just a bunch of guys in a fishing \textbf{boat}, but the Greenpeace the entire \textbf{world} has come to \textbf{love} and hate. ” Today, Greenpeace is the \textbf{world} ’s most visible environmental \textbf{organisation}, with \textbf{offices} in more than 55 countries and over 2.9 million \textbf{members} worldwide.
Amchitka, it has turned out, was only the \textbf{beginning} of what would come to be a much bigger \textbf{story}. “ Why not sail a \textbf{boat} up there and confront the \textbf{bomb}? ” Bob Hunter sailed aboard the first Greenpeace voyage in 1971 to Amchitka in the Aleutian Islands to try and stop a U.S. nuclear \textbf{weapons} \textbf{test}.
When they were halfway to their destination, Richard Nixon announced a \textbf{month} ’s delay of the \textbf{test}.
Most of the \textbf{crew} were running out of \textbf{money} or vacation \textbf{time}, and an acrimonious debate broke out about whether to continue or turn back.
This is Bob ’s \textbf{story} about what happened.
When I got back from the expedition to Amchitka and sat down to write a \textbf{book} about it, I was convinced we had lost, and I was angry.
The \textbf{best} chance ever to actually interfere with nuclear \textbf{testing}, and we had blown it through sheer stupidity – and a failure of nerve, to put it kindly.
Cop-out on the \textbf{Way} to Amchitka was the title that loomed in my \textbf{mind}.
And my personal failure of \textbf{will} was a big factor in that cop-out.
Worse, I was afraid that I ’d subconsciously thrown the fight to carry on with the voyage.
I \textbf{’d} have to live with that until I died or the \textbf{world} blew up, whichever happened first.
I was also facing the most serious writing dilemma of my life.
Since childhood, when I had \textbf{started} writing science fiction in my \textbf{school} scribblers, I had been \textbf{looking} for “ \textbf{experience} ”.
Like all intense young \textbf{writers}, I had plenty to say, but rather little context in which to present my \textbf{thoughts}.
I \textbf{’d} read a \textbf{bit}, but there had been no plagues or crusades or recent \textbf{wars} on home ground.
Even when the Great Red River Flood hit in 1950, my \textbf{family} was evacuated before the dikes broke.
Real-life adventure had been hard to come by in working-class \textbf{south} Winnipeg after the \textbf{war}, a period during which Canada was at its dullest, if you can imagine.
Such adventures as I ’d managed to \textbf{experience} when I was growing up had been of the ordinary romantic or travel or childhood close-call variety.
I had done some solo camping in the boreal forest and some hitchhiking in the western Canada and Europe, had got married and \textbf{fathered} two \textbf{children}, had embarked on an interesting career in journalism and published three \textbf{books}, but until that fateful voyage in the fall of 1971, \textbf{nothing} had happened to me that leaped out as being absolutely essential to write about, if only for my own \textbf{understanding} of life.
And now that it had, I was obliged not to write about it – for the sake of the cause.
The problem was that I ’d joined.
What exactly I \textbf{’d} joined was not yet clear – it was still being defined – but I had definitely stopped \textbf{being} on the outside \textbf{looking} in and was instead on the inside \textbf{looking} out.
I ’d \textbf{started} out as a newspaper columnist, the ultimate Ishmaelian outsider, accustomed to being responsible for \textbf{nothing} except the authenticity of my insights and \textbf{words}. “ Tell it like it is ” was the creed of the counterculture scribe, and my personal mantra.
Suddenly I found myself in the inner circle of a nascent political \textbf{organization}, with a \textbf{bit} of potential power in my \textbf{hand}, which at the \textbf{time} seemed like the power to change the \textbf{course} of \textbf{history}.
All that had to happen was for the MV Phyllis Cormack, aka Greenpeace, to make it to Amchitka Island and \textbf{park} there under the nose of a nuclear \textbf{test} \textbf{bomb} code-named Cannikin.
How much simpler could it be?
Yet \textbf{everything} got fucked up.
We never quite managed to go in the direction we wanted to go, or be in the place we wanted to be.
And we fought bitterly among ourselves about it. \textbf{Everything} we did or said got sucked into an overwhelming power struggle.
Here we were, supposedly saving the \textbf{world} through our moral example, emulating the Quakers, no less, when in reality we spent most of our \textbf{time} at each \textbf{other} ’s throats, egos clashing, the \textbf{group} fatally divided from \textbf{start} to finish.
As every \textbf{writer} since Homer could tell you, this was the \textbf{story} : the conflict within.
But having agreed, early in the game, to the Unity Rule – \textbf{something} like : I Pledge to Stay On-Side With the Group No Matter What, which had seemed like a bold leap into solidarity with The \textbf{Movement} at the \textbf{time} – I had effectively gagged myself as a reporter and historian.
It was a trade-off : but I bought into it, so I couldn't complain.
I \textbf{’d} get to be part of the consensus – my own skinny \textbf{hand} on the wheel of decision-making over the \textbf{course} of the Greenpeace, and therefore destiny – but like any \textbf{other} \textbf{politician}, I \textbf{’d} have to agree not to disagree in public.
I disagreed, as it turned out, with just about \textbf{everything} that was done, but had to keep my mouth shut.
How, therefore, to write a \textbf{book}?
An authorized \textbf{book}, which followed the \textbf{party} line yet still told the awful truth – that we had screwed up.
Three decades later, his grey beard turned white, Jim Bohlen confided to me over a drink that he had been giving the sailing orders to our \textbf{captain} in secret throughout the voyage.
As the guy signing the cheques and as the chairman of the Don't Make a Wave Committee, which had chartered the \textbf{boat}, Bohlen had the legal authority to do that, but rather than say that he was the boss, and that the Greenpeace and the protest action were therefore being run as an old-fashioned hierarchical power structure, he played games to keep \textbf{us} radical young crewmen under control. \textbf{One} of them was the promise that the ship would be run by consensus – each of \textbf{us} would have the power of veto.
This was considered the ultimate hip form of \textbf{sharing} power at the \textbf{time}, and I, for \textbf{one}, respected it.
But it was all a sham.
Decisions were indeed made – Bohlen made them.
And he made them after the \textbf{rest} of \textbf{us} had gone back to our bunks.
At the \textbf{time} I wrote my manuscript, immediately after the voyage, I had no \textbf{idea} what Bohlen had been up to behind the \textbf{scenes}.
On any given \textbf{day} the actual \textbf{movements} of the \textbf{boat}, as opposed to the direction we ’d agreed to at our \textbf{meeting} the \textbf{night} before, remained a mystery to me.
Bohlen had \textbf{us} completely flummoxed.
I salute him now for his cunning and maturity and prudence.
We probably would have died if he hadn't assumed control.
But back then, I plotted and connived to overthrow him as leader because he was “ chickening out ”.
Ben Metcalfe, Bohlen ’s co-conspirator in the plot to bring \textbf{us} home alive, the \textbf{other} mature \textbf{war} veteran on \textbf{board}, and the mastermind of the \textbf{media} \textbf{campaign}, saw no \textbf{reason} to put \textbf{us} at risk of committing mass suicide, and I sneered at him for having “ lost it ”.
But this guy had fought in the Desert War against Rommel, had resisted raf orders to \textbf{bomb} Gandhi ’s followers, and was so far ahead of me in \textbf{terms} of that elusive \textbf{stuff} called \textbf{experience} that there was never any \textbf{doubt} that in matters of life or death he would outmanoeuvre the mutinous but naïve youth faction.
He was an old rogue survivor.
A genius, I now realize.
In the end, I studied at his \textbf{feet}.
The \textbf{man} who ultimately determined the fate of that first Greenpeace \textbf{trip} was John C.
Cormack, the \textbf{captain} and owner, who had accepted the \textbf{job} of sailing his fishing \textbf{boat} into a nuclear \textbf{test} zone only out of economic desperation, a \textbf{fact} that never got \textbf{talked} about much.
In hindsight it is interesting to remember what Cormack did and did not do at the critical moment.
He saved his \textbf{boat} and \textbf{us} along with it.
And we all saved face, at least enough to go home.
The key moment of the \textbf{trip} came a \textbf{day} before we limped back into Vancouver.
As we all sat slumped in the galley, burned out, Bohlen announced that he was going to shut down the Don't Make a Wave Committee as soon as he got the chance.
It was an ad hoc \textbf{group} and it had done its \textbf{thing}.
Don't do that, I told him.
Why waste all this hard-earned \textbf{media} capital?
Fold the \textbf{committee}, sure, but reconstitute it as the Greenpeace Foundation.
That was my main contribution, yet the moment did not find its \textbf{way} into my manuscript.
It was an element of \textbf{hope} for a future \textbf{revolution}, and I was not hopeful as I bobbed in the harbour at Steveston, heartsick and overmedicated, writing the \textbf{story} of our failure.
In the end I told the truth as I saw it, supposedly as it was, never \textbf{mind} loyalty to the cause.
As it turned out, all my angst was unnecessary. \textbf{Time} has proven my post-trip despair to be utterly mistaken.
The \textbf{trip} was a \textbf{success} beyond anybody ’s wildest \textbf{dreams}.
That \textbf{bomb} went off, but the \textbf{bombs} planned for after that did not.
The nuclear \textbf{test} program at Amchitka was cancelled five \textbf{months} after our \textbf{mission}, and some scholars argue that this was the \textbf{beginning} of the end of the Cold War.
Whatever \textbf{history} decides about the big \textbf{picture}, the legacy of the voyage itself is not just a bunch of guys in a fishing \textbf{boat}, but the Greenpeace the entire \textbf{world} has come to \textbf{love} and hate.
It was a fancy, at first.
Marie Bohlen casually expressed the \textbf{idea} over coffee \textbf{one} \textbf{morning}.
But the \textbf{people} around her – a loose \textbf{alliance} of Quakers, pacifists, \textbf{ecologists}, \textbf{journalists} and hippies – weren't known for shrugging off the really big \textbf{ideas}.
Excerpt from “ The Greenpeace to Amchitka ” By Bob Hunter Published by Arsenal Pulp Press, Canada ISBN Number : 1-55152-178-4 Reproduced with \textbf{kind} permission JUNE 2004 A few \textbf{weeks} later, the Don't Make a Wave Committee – as the \textbf{group} was still called then – had a plan. “ If the Americans want to go ahead with the \textbf{test}, ” Marie ’s husband Jim said, “ they ’ll have to tow \textbf{us} out. ” Leaving \textbf{one} of those heady first \textbf{meetings}, Irving Stowe flashed the \textbf{peace} sign – as was his custom – and said “ \textbf{Peace} ”.
On that occasion, the usually rather quiet Canadian \textbf{ecologist} Bill Darnell made the off-hand reply : “ Make it a green \textbf{peace}. ” When the \textbf{words} didn't fit onto buttons for the \textbf{group} ’s first fundraiser, they were simply merged : Green Peace became Greenpeace.
We had our \textbf{name}, but it was soon clear that selling 25-cent buttons wouldn't bring in the cash needed to buy a \textbf{boat}. \textbf{Someone} had the \textbf{idea} to put up a \textbf{rock} concert.
A few \textbf{phone} calls later, Joni Mitchell said she would be playing.
Chilliwack and Phil Ochs confirmed, before Joni called again to say she would bring a special \textbf{guest} : James Taylor.
None of them wanted any \textbf{money} for the \textbf{night}. “ The concert was a sell-out, the biggest counter-culture event of the \textbf{year}, ” Rex Weyler recalls in his Greenpeace biography.
The sixteen thousand that filled Vancouver ’s Pacific Coliseum left the concert entranced.

% matched lemmas: activist, alliance, beginning, being, bit, board, boat, bomb, book, campaign, captain, child, committee, course, crew, day, doubt, dream, ecologist, everything, experience, fact, family, father, foot, good, great, group, guest, hand, history, hope, idea, island, job, journalist, kind, look, love, man, media, medium, meeting, member, mind, mission, money, month, morning, movement, name, night, nothing, office, one, organisation, organization, other, park, party, peace, people, phone, picture, politician, reason, rest, revolution, rock, scene, school, share, site, someone, something, south, start, story, stuff, success, talk, term, test, testing, thing, thought, time, trip, understanding, us, war, way, weapon, week, will, word, world, writer, year
\end{textsample}
