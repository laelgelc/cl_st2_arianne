\begin{textsample}{NEG Dim 1 – human – Score 3.00 – t837\_human.txt}  \label{ex:f1_neg_004}
I can't imagine a world without bees.
These fantastic little insects are not only a vital part of natural ecosystems, they also play a crucial role in food production.
Worldwide, three out of four of our food crops depend on pollinators like bees, butterflies and other small creatures.
In Europe, 84 % of all cultivated plants are pollinated by insects – primarily bees.
Bees, bumblebees and other pollinators are vanishing at an alarming rate, in part due to the increased deployment of pesticides such as neonics.
Bees and other pollinators have a huge part to play in our food supply and the global economy.
Pollination affects both the quantity and quality of crops.
Unsurprisingly, inadequate pollination of certain crops results in lower yields.
The contribution of bees in global crop pollination is estimated at €265 billion.
But industrial agriculture threatens bees by depriving them of valuable food sources and exposing them to toxic chemicals.
As a result, bees and other pollinators are under serious threat.
This puts our food supply and ecological balance at risk.
We have a unique opportunity to change this.
In March, the European Commission proposed an almost complete ban on three bee-harming pesticides.
Our governments are voting on a full ban of harmful pesticides as soon as May.
We need to make sure that all these bee-harming pesticides are banned, now.
And we want all other chemical pesticides to be properly tested for their impact on bees, before they ’re put into industrial use.
Politicians need to hear our buzz and act.
They can't play with our food any longer.
Please act now to save the bees and other pollinators.
Tell our politicians to ban all bee-harming pesticides.
Luís Ferreirim is the ecological farming campaigner at Greenpeace Spain

% matched lemmas: 
\end{textsample}
