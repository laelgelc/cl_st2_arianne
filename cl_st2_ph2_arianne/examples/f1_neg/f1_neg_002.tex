\begin{textsample}{NEG Dim 1 – human – Score 1.00 – t468\_human.txt}  \label{ex:f1_neg_002}
Caros amigos, Todos entendemos que, ao enfrentar a aguda crise de saúde pública e os choques econômicos associados ao coronavírus, estão sendo feitas escolhas que terão um impacto profundo na emergência climática.
Devemos impedir o investimento em indústrias que já estavam destruindo o planeta a saúde.
E à medida que 2030 – uma data crucial para a luta contra a crise climática – se aproxima rapidamente, esses fundos devem ser investidos na saúde pública e do nosso planeta.
Os vastos fundos públicos anteriormente “ indisponíveis ” devem apoiar uma transição justa para um futuro melhor, em que as pessoas e a Terra estejam em harmonia : onde todos os seres vivos possam prosperar.
Nós devemos demandar responsabilidade por parte de nossos líderes.
Devemos estar atentos às tentativas de usarem os trabalhadores, que eles normalmente exploram com salários pífios ou expostos a substâncias letais, dia após dia, para capturar e desviar o apoio do público.
A chamada “ doutrina do choque ” está em jogo, com trilhões de dólares, euros e reais sendo bombeados para a economia para tentar inoculá-la do impacto do vírus.
Regras sem consentimento estão sendo adotadas.
Devemos pressionar por investimentos no futuro.
Em vez de olhar para o passado para explicar nossa situação atual, devemos olhar para o futuro para ver o que deve ser feito.
Um futuro aberto, cooperativo, igualitário, pacífico, em harmonia com a natureza e com o bem público como força motriz.
Esse vírus não tem nacionalidade, não possui agenda ou afiliação política, existe para se espalhar para onde, quando e como puder.
A única coisa que pode impedi-lo é a comunidade e a cooperação.
Não é hora de culpar ou dividir.
Existem muitas forças no mundo fazendo exatamente isso por poder e lucro.
Devemos dar o exemplo, estendendo nossos valores, plataforma e conhecimento a outras pessoas, especialmente às mais vulneráveis ​​em nossa sociedade.
Talvez isso signifique se unir a aliados incomuns neste momento sem precedentes, ou fazer coisas fora de nossas bolhas e zonas de conforto.
O que queremos, precisamos e merecemos deste novo capítulo da história de nosso planeta é que profundas lições sejam aprendidas rapidamente, raízes de problemas sejam abordadas por completo e uma verdadeira liderança política seja estabelecida.
Quando essa pandemia passar, nosso caráter coletivo e potencial futuro serão definidos pelas escolhas que fizemos para proteger os mais vulneráveis, não como protegemos as indústrias.
Será fortalecido pelas lições que aprendemos.
Cada um de nós temos uma peça para a construção do mundo que precisamos, onde compaixão e cooperação são as chaves para um futuro mais seguro e justo.
O futuro está sendo escrito hoje, vamos escrevê-lo juntos, com todos os nossos corações e nossa humanidade.
Jennifer e Anabella As consequências da pandemia de Covid-19 são – e serão – definidas por escolhas.
Essas escolhas devem ser baseadas em valores, não em valor monetário : compaixão, coragem e cooperação.
Esses são e sempre foram os nossos.
Vamos continuar nos apoiando neles agora.
Jennifer Morgan e Anabella Rosemberg são, respectivamente, diretora executiva e diretora de programas do Greenpeace International.
A luta para conter o coronavírus é nossa prioridade número um como pessoas e como organização.
As decisões de vida ou morte não estão sendo tomadas apenas por médicos e enfermeiros, mas por todos e cada um de nós enquanto praticamos o distanciamento físico.
Juntos, vamos fazer as escolhas certas.
O Greenpeace é uma família.
Como todos os membros desta família, estamos enfrentando nossos próprios desafios com o coronavírus.
Nós duas vivemos em países que nos adotaram, Alemanha e França, dos EUA e Argentina.
Nossa preocupação com nossos pais, irmãos e irmãs, sobrinhas e sobrinhos é agravada pela distância e pelos sistemas de saúde que podem não ser capazes de ajudá-los. É difícil encontrar equilíbrio na turbulência emocional.
Mas, como todos vocês, encontramos conexão.
Descobrimos que, em crise, nossas salas de estar se tornaram lugares para dançar com nossos filhos e parceiros, desfrutando do conforto de músicas antigas e favoritas!
Esperamos que vocês encontrem seu equilíbrio, conexões e formas de dar vazão.
Todos precisamos do tempo necessário para : colocar nossas famílias e nossas comunidades em primeiro lugar, cuidar de nossos filhos e pessoas doentes e, claro, cuidar de nós mesmos.
Estamos testemunhando muitos atos de coragem, compaixão e comunidade que fornecem inspiração e mostram o poder das pessoas.
Podemos ver ao redor do mundo um desejo resoluto de não apenas sobreviver, mas prosperar.
Vamos continuar a nos juntar ao coro de colaboração e celebração do melhor da humanidade diante das adversidades.
Vamos garantir que as histórias que contamos sejam de compaixão pelos mais vulneráveis, de união : combatendo o medo e a culpa.
Enquanto lideramos com compaixão, também sejamos vigilantes.
Em crise, como sabemos, o impossível se torna possível.
Para melhor ou pior.

% matched lemmas: 
\end{textsample}
