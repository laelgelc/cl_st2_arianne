\begin{textsample}{NEG Dim 1 – human – Score 4.00 – t942\_human.txt}  \label{ex:f1_neg_008}
…or maybe it ’s a Whark?
Whatever you want to call it, today is International Whale Shark Day!
But before you start running away screaming “ Jawwwwws! ” don't be alarmed.
With a face like a whale and a body like a \textbf{shark}, these seemingly frightening creatures are actually gentle giants.
Found in tropical oceans in areas like the Maldives, Philippines and Mexico they feed mainly on plankton and are by far the largest living non mammalian vertebrate.
But despite being docile ( they pose absolutely no threat to divers ) they ’re also unfortunately hunted for their highly prized fins and meat.
As a vulnerable species we need to protect the whale \textbf{shark} and their ocean home.
Check out these facts about whale \textbf{sharks} … And then raise a glass of plankton and celebrate whale \textbf{shark} day!
Each whale \textbf{shark} has a unique pattern, much like humans ’ fingerprints.
This allows researchers to run visual analytics to correctly identify and track each whale \textbf{shark}.
The International Union for Conservation of Nature ( IUCN ) Red List regards the species as one of the most vulnerable marine animals in the world.
Indonesia, through its Ministry of Maritime Affairs and Fisheries, enacted a law for whale \textbf{shark} conservation.
Unfortunately, law \textbf{enforcement} much like the whale \textbf{sharks}, has little to no teeth.
Like most \textbf{sharks}, whale \textbf{sharks} breed slowly which make them dangerously vulnerable to overfishing.
Most of which, are contributed due to the world ’s insatiable appetite towards \textbf{shark} fins.
Sumardi Ariansyah is an Oceans Campaigner in Indonesia, with Greenpeace Southeast Asia.

% matched lemmas: enforcement, shark
\end{textsample}
