\begin{textsample}{NEG Dim 4 – human – Score -1.00 – t470\_human.txt}  \label{ex:f4_neg_005}
Cher·e·s ami·e·s, Nous comprenons toutes et tous que, pour adresser cette crise de santé publique intense et les chocs économiques qui l’accompagnent, des choix sont faits – des choix qui auront \textbf{un} impact profond sur l’urgence climatique chronique.
Nous devons empêcher les investissements dans les industries ayant un historique de destruction de la planète et de la santé à leur actif.
Alors que le point de basculement climatique de 2030 approche à grand pas, ces fonds doivent être investis dans le bien-être humain et la santé de la planète.
Les fonds publics nouvellement débloqués doivent être mis au service d’une transition juste vers \textbf{un} futur meilleur, dans lequel les gens et la planète seront en harmonie : ( un environnement ) où chaque être vivant peut s’épanouir.
Nous devons demander des comptes à nos dirigeants.
Nous devons être vigilant·e·s contre les tentatives d’utiliser les travailleuses et travailleurs, habituellement exploité·e·s et sous-payé·e·s ou exposé·e·s à des substances toxiques, à des en vue d’accaparer et de détourner le soutien public.
La “ stratégie du choc ” est à l’oeuvre, des trillions de dollars, d’euros et de yens sont injectés dans l’économie pour tenter de l’immuniser contre l’impact du virus.
Des règles sont adoptées sans consentement.
Nous devons plaider pour un investissement dans l’avenir.
Plutôt que de regarder vers le passé pour expliquer la situation dans laquelle nous nous trouvons, regardons vers le futur pour voir ce qui doit être fait.
Un futur ouvert, coopératif, égalitaire, paisible, en harmonie avec la nature, guidé par l’intérêt général.
Ce virus n’a pas de nationalité, ni d’agenda ou d’affiliation politique.
Il existe seulement pour se répandre là où il le peut.
La seule chose qui peut l’arrêter est la solidarité et la coopération.
Le temps n’est pas au blâme ou à la division.
De nombreuses forces de notre monde agissent déjà en ce sens pour le pouvoir et le profit.
Nous devons donner l’exemple, en partageant nos valeurs, notre plateforme et notre connaissance à l’extérieur et avec les autres, surtout avec les personnes les plus vulnérables de notre société.
Peut-être que cette période sans précédent doit nous encourager à forger des alliances inhabituelles, à sortir de nos zones de confort et des chambres de résonance habituelles.
Nous voulons, avons besoin et méritons que ce nouveau chapitre dans l’histoire de notre planète soit l’occasion de leçons précieuses rapidement apprises, de causes systémiques pleinement adressées et de l’établissement d’une vraie direction politique.
Quand cette pandémie sera passée, notre personnalité collective et notre véritable potentiel futur seront définis par les choix que nous aurons fait pour protéger les personnes les plus vulnérables.
Pas la façon dont nous aurons protégé les industries.
Cette personnalité et ce potentiel seront renforcés par les leçons que nous aurons apprises.
Le futur est encore à écrire.
Faisons-le ensemble, de tout notre cœur et avec toute notre humanité.
Jennifer et Anabella Les conséquences de la pandémie covid-19 sont – et seront – définies par des choix.
Ces choix doivent être fait en fonction de valeurs : compassion, courage et coopération.
Ces valeurs ont toujours été les nôtres.
Appuyons nous sur elles.
La lutte pour contenir le coronavirus est notre priorité numéro un en tant que personnes et en tant qu’organisation.
Des décisions de vie ou de mort sont prises en ce moment, non seulement par les soignants, mais par chacun·e d’entre nous quand nous pratiquons la distanciation sociale.
Ensemble, faisons les bons choix.
Greenpeace est une famille.
Comme chaque membre de cette famille, nous faisons face à nos propres défis en ce qui concerne le coronavirus.
Nous vivons toutes les deux en Allemagne et en France, des pays qui nous ont adoptées alors que nous venons des États-Unis et d’Argentine.
Nos inquiétudes pour nos parents, nos frères et nos soeurs, nos nièces et neveux sont exacerbées par la distance et aggravées par des systèmes de santé que nous savons parfois incapables de les aider.
Il est difficile de trouver l’équilibre au milieu des turbulences émotionnelles.
Mais, comme vous toutes et tous, nous trouvons du lien.
Nous avons découvert qu’en ces temps de crise nos salons sont devenus des endroits où danser avec nos enfants et nos partenaires, où se réconforter au son de nos chansons préférées.
Nous espérons que vous parvenez à trouver votre équilibre, vos liens et vos exutoires.
Nous devrions toutes et tous prendre le temps nécessaire : pour faire passer nos familles et nos communautés avant tout.
Pour prendre soin de nos enfants et des malades.
Et, bien-sûr, pour prendre soin de nous.
Nous sommes témoins de nombreux actes de courage, de compassion et de solidarité qui nous inspirent et nous rappellent le pouvoir des gens.
Nous observons autour de nous \textbf{un} désir résolu non seulement de survivre, mais de s’épanouir.
Continuons à nous joindre à ce choeur de solidarité et de célébration de ce que l’humanité a de meilleur face à l’adversité.
Assurons nous que les histoires que nous racontons soient pleines de compassion pour les plus vulnérables et rappellent l’importance de se rassembler pour contrer la peur et le blâme.
Bien qu’avançant avec compassion, soyons aussi vigilant·e·s.
En temps de crise, comme nous le savons, l’impossible devient possible.
Pour le meilleur ou pour le pire.

% matched lemmas: un
\end{textsample}
