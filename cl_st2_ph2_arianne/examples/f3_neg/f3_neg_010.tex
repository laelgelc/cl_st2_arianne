\begin{textsample}{NEG Dim 3 – human – Score 1.00 – t351\_human.txt}  \label{ex:f3_neg_010}
There are few events in world history that make us ask ourselves : Where was I when that happened?
I still remember when the news broke about a plane crashing into the World Trade Centre in 2001 and the visuals of the giant waves hitting Indonesia and Thailand ’s coast in 2004.
Another shocking tragedy that affected so many of us was the tsunami hitting the nuclear power station on Fukushima ’s coast.
The images from these events are forever seared into memory.
It ’s been a decade since the disaster took place.
However, the trauma is still fresh, especially for the survivors who physically experienced the catastrophe.
We had seen Chornobyl exactly 25 years before this, and with Fukushima, we once again all witnessed the horror of another ​nuclear ​accident​. ​Like Chornobyl, hundreds and thousands of families had to be evacuated overnight as their homes were no longer safe.
Nuclear radiation was spreading every minute, contaminating everything on its way.
Ten years is a long journey.
Looking at the Greenpeace archives, beginning with the first team documentation from 2011 up until 2019, it reminds us that while a decade may seem like a long time, it is not enough to wash away the pains caused by the accident.
Here we present the visual documentations Greenpeace has done in Japan over the years, as we try to show the extent of radiation in various prefectures.
This is Greenpeace bearing witness, so we may never forget Fukushima ’s horrors and for us to continue campaigning for a truly nuclear-free world.

% matched lemmas: 
\end{textsample}
