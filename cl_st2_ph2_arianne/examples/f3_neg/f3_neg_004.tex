\begin{textsample}{NEG Dim 3 – human – Score -1.00 – t773\_human.txt}  \label{ex:f3_neg_004}
The Antarctic belongs to all of us, and is all of ours to protect.
Yet, until quite recently, an ice ceiling prevented women from joining expeditions to the Antarctic.
Louisa Casson is an Oceans Campaigner for Greenpeace UK and Samantha Wockner is a Mobilisation Campaigner for Greenpeace Australia Pacific.
A woman applying to join the British Antarctic Survey was rejected on the basis that “ women wouldn't like it in the Antarctic as there are no shops and no hairdressers. ” Sigh.
Other female applicants were told they wouldn't be able to handle the extreme temperatures, that there weren't any facilities for them, or they weren't up to the scientific tasks.
Some women still weren't allowed to work in the Antarctic until the early 1980s.
While we see \#everydaysexism in all walks of life, it ’s disheartening that even this icy wilderness had barriers up against female participation.
But because of the unwavering \textbf{determination} of women who wanted to learn more about the Antarctic and to protect this unique region, we ’ve overcome these barriers.
Women around the world are now playing an active role in conducting science and research that can help us protect the Antarctic.
Greenpeace ’s ship, the Arctic \textbf{Sunrise}, is currently on an expedition to the Antarctic Ocean as part of a global campaign to create the largest protected area on Earth by the end of this year.
We ’re lucky to be joined by female scientists, ship crew, campaigners, documentary makers and journalists on board – as well as women in Greenpeace offices all over the world – helping us create an Antarctic Ocean Sanctuary that will protect this vital ocean for all of us for generations to come.
To celebrate this, enjoy this video about some of the incredible women on board the Greenpeace ship in the Antarctic Ocean – and have a happy International Women ’s Day!

% matched lemmas: determination, sunrise
\end{textsample}
