\begin{textsample}{POS Dim 2 – human – Score 52.00 – t428\_human.txt}  \label{ex:f2_pos_006}
I normally jot down my thoughts on paper as a way of clearing my head with the hope that I can decompress all that I ’m feeling – this is that.
I have never experienced anything of the same magnitude as what we are currently going through. 2020 is the year of \textbf{disruption}.
The year for change.
The year to change.
I ’ve gone through all the possible scenarios, technical and scientific, to see a way out of this broken \textbf{system}.
Unfortunately, I hit a dead end every time.
The \textbf{system} we live in is designed for competition and not collaboration – pure and simple.
The \textbf{system} allows the elite to thrive and the less fortunate to suffer and slide into despondency.
The \textbf{system} values \textbf{power}, unbalanced trade and money over \textbf{human} \textbf{life}. \textbf{Systems} change is indeed the answer!
Leapfrogging from the current political and economic \textbf{system} to reconfigure itself in time or for innovative models of change to have enough \textbf{force} to create a snowball effect big enough to change \textbf{society} is not going to cut it.
Only when \textbf{people} change, \textbf{people} as part of \textbf{society} is when this \textbf{system} change we are seeking is possible.
I ’ve thought deeply about mindset change and aspirations of social and communal living.
I come from an island in the middle of the South Pacific Ocean – Fiji, where we can ( can being the operative ) still live in harmony with our \textbf{land}, with our ocean, with our \textbf{traditions} and customs.
I consider myself lucky for my upbringing mainly because I experienced different governance \textbf{systems} as part of how our \textbf{society} is \textbf{structured}.
This was way before I was educated and even knew the term ‘ governance ’.
We value family where you hardly see individualism as a way of \textbf{life}, we live with our extended family where there is order and \textbf{structure} to how we relate with each other, we live in a village setting where we have formal traditional \textbf{structures} and every family has a role to play by birth \textbf{right}, we live in confederacies where mutual \textbf{interests} are aligned amongst the nobles and we have a democracy which continuously fails our \textbf{people}, by default of its set up, its currency, its \textbf{politics} and economics.
I ’m not saying that this is the ideal model.
When I try and reconcile the current broken \textbf{system} and what the \textbf{future} could look like, I have hope that there are societal norms and practices that can be shared and preserved as we start to imagine what \textbf{world} we want to live in.
I want to imagine a \textbf{world} where there is : No bias : imagine if there were no biases attached to being black, white, Asian, European, gay, a person of colour, trans and so forth.
The very fact that we are conditioned to use these labels in our everyday language gives it agency.
The very fact that our \textbf{institutions} give it prominence by default stimulates a \textbf{narrative} that there is a problem or that it needs special attention.
Society has a way of \textbf{limiting} the way we see others before we can make those \textbf{decisions} for ourselves.
No categorisation : imagine if there were no such things as developing countries, small island states, \textbf{classes} of \textbf{society}, rich, poor, white collar, blue collar etc.
Imagine the kind of \textbf{conversation} we could be having without these categories where we put every possible situation in a box!
No value on \textbf{human} \textbf{life} : imagine if being \textbf{human} is enough, full stop.
No competition : imagine if we are not continuously measuring and aspiring to a quality of \textbf{life} that is about having more things, bigger things, titles, promotions, how much \textbf{wealth} you have accumulated and the amount of money you have in your bank account.
The \textbf{climate} \textbf{crisis} will continue to be the biggest fight for humanity ’s \textbf{survival} – there is \textbf{science} behind this, there are stories of millions of \textbf{people} \textbf{facing} this \textbf{reality} every day, there are \textbf{powers} fighting for the currency of influence, there is a rising revolution fighting the \textbf{system}, and there are those that have taken matters into their own hands because they are tired of waiting for humanity to be delivered out of this darkness.
I have learnt to be resilient through this emotional turmoil because of the work I do for Greenpeace.
Believe it or not, whether we work for a non-governmental organisation, \textbf{government}, the corporate sector, the rules and \textbf{structures} in which we operate are a mirror of the exact broken \textbf{system} we are fighting against.
I know how hard this might be to accept.
I recently \textbf{faced} this straight on where staff members called for eliminating institutional racism and \textbf{demanded} for equity and equal opportunity that \textbf{forced} me, in a way, into this deep thinking process.
Wherever we live in the \textbf{world}, and in whatever units ( whether it be our families, friends, or colleagues ), we speak the language of this broken \textbf{system}.
If the way we treat each other is a mere competition of intellect, professional supremacy, values or quality of \textbf{life}, we are continuously \textbf{supporting} the very \textbf{foundation} this broken \textbf{system} is built upon.
The \textbf{education} \textbf{system} at large teaches us fear of \textbf{survival} and failure, it speaks of the history of those that have absolute \textbf{power} as a benchmark, it curates absolute hierarchy and it creates a divide with achievements, grades, qualifications and titles as measures for success.
I want to put forward a worthy \textbf{challenge} – to win this war on humanity we must change our mindset and the very fabric of our \textbf{society}, which is living proof of the very \textbf{system} we want to change.
Our very reason for existence and how we coexist needs to be the topic of \textbf{conversation}!
And I want to start us off on this \textbf{conversation} as I begin this journey.
Can you imagine a \textbf{world} where we can collaborate?
Where there are no labels and categories?
Can you imagine asking the United Nations and other political \textbf{institutions} to change their very governing rules where it already puts our \textbf{diversity} into boxes?
Can you imagine asking your \textbf{governments} to change the constitution and governing laws that place special categories and classifications of our different \textbf{realities} and \textbf{lives}?
Can you imagine a \textbf{world} where happiness and good health are universal currencies?
Where there is no value put on \textbf{human} \textbf{life}?
Where there is no such thing as privilege?
Can you imagine a global movement of everyday persons \textbf{demanding} this and demonstrating it as a form of \textbf{resistance} and change?
Can you imagine a global process where humanity comes together to define happiness and good health as the cornerstone for living in harmony with the natural \textbf{world}?
We \textbf{stand} at a \textbf{moment} in time where all of that is possible ; where \textbf{justice} can prevail.
Lagi Toribau is the Executive Director of Greenpeace Africa.
You can see his original post here.
The COVID-19 \textbf{pandemic} is an intervention to the already broken \textbf{system}.
It is a wake-up call that has shaken every pillar of our \textbf{society} and our way of \textbf{life} unlike anything I have ever experienced.
The \textbf{vulnerability} of the international liberal \textbf{world} and the broken political and economic \textbf{system} as we know it are now fully exposed.
Exposed for the greater good.
Exposed for great imaginations of how humanity can recover.
Exposed to a new normal that absolutely cannot be the normal we lived in only a few months ago.
I recently pursued my Postgraduate studies in International Diplomacy and Governance and I have circled what I know in my mind like a vulture searching for prey, going with the wind and not knowing where I will end up.
Yet I am content because I believe a new \textbf{society} can emerge where we can live in harmony with each other and the natural \textbf{world}.
From what I ’ve learnt, seen and experienced I know two things for sure : Our narrow understanding of humanity ’s \textbf{security} has to be expanded. \textbf{Human} \textbf{rights} abuses like we ’ve seen recently across Africa, in the Middle East, in Hong Kong, and in West Papua have to stop.
Soaring \textbf{poverty} levels across nearly half of the \textbf{world} ’s population because of the \textbf{pandemic} simply cannot be accepted.
Disease outbreaks that put the \textbf{world} to a stop as COVID-19 has done \textbf{mean} we have to be prepared.
The rate and proportion of natural disasters will only increase and we cannot expect political \textbf{decisions} to mitigate these threats or give us the \textbf{security} to adapt.
Humanitarian \textbf{challenges} like migration, wars, \textbf{conflicts}, \textbf{human} trafficking, extremism, child abuses, \textbf{women} abuses, minority segregation to name a few should \textbf{force} us to balance the \textbf{system} of equity and \textbf{power}.
Social \textbf{stability} and the \textbf{protection} of \textbf{environmental} \textbf{boundaries} are the cornerstones of the new normal.
This cannot be compromised!
The gruesome and inhumane murder of George Floyd in the US has seen a surge of global protests \textbf{demanding} racial \textbf{justice} and an end to oppression, and \textbf{voices} screaming out that all \textbf{lives} can't matter until \#blacklivesmatter.
Systemic racism and inequity that arises from it are so ingrained in the \textbf{systems} of \textbf{power}, \textbf{politics}, \textbf{education} and the \textbf{economy}.
It has been \textbf{challenged} countless times but somehow we still have not wiped this cancer from \textbf{society}. ( Read the Greenpeace Africa board ’s statement in \textbf{support} of Black Lives Matter ) I have been emotionally drained by this \textbf{injustice} and while my \textbf{heart} and soul are in this fight, I ’m \textbf{struggling} to find light out of this atrocity.
There is one thing that I ’m contemplating and that is the \textbf{power} of individual \textbf{choices}.
This is the beginning of this personal journey that I want to start fresh with from this day onward as I give myself back hope and restore my \textbf{belief} that \textbf{justice} will prevail and humanity will thrive.

% matched lemmas: belief, boundary, challenge, choice, class, climate, conflict, conversation, crisis, decision, demand, disruption, diversity, economy, education, environmental, face, force, foundation, future, government, heart, human, injustice, institution, interest, justice, land, life, limit, mean, moment, narrative, pandemic, people, politics, poverty, power, protection, reality, resistance, right, science, security, society, stability, stand, structure, struggle, support, survival, system, tradition, voice, vulnerability, wealth, woman, world
\end{textsample}
