\begin{textsample}{POS Dim 2 – human – Score 55.00 – t193\_human.txt}  \label{ex:f2_pos_005}
The latest IPCC \textbf{climate} \textbf{science} report emphasises the importance of \textbf{climate} action \textbf{grounded} in \textbf{justice}.
In this post, Amrekha Sharma of the \textbf{Climate} Justice \& Liability Project reflects on the significance of some key findings for \textbf{climate} \textbf{justice} advocacy.
If the latest IPCC report legitimated known histories, and represents an apocalyptic “ atlas of \textbf{human} \textbf{suffering}, ” we must continue to elevate the stories of \textbf{courage}, creativity and \textbf{demands} for \textbf{justice} of the \textbf{people} experiencing \textbf{climate} change first and worst.
Among them are a rising number of protagonists in \textbf{courts} around the \textbf{world} who represent a tiny fraction of those who do not consent to being cast in a violent fantasy of “ overshoot ” and are trying to shape our \textbf{institutions} in line with the \textbf{demands} of \textbf{climate} \textbf{science} and \textbf{justice}.
The IPCC report may help to bolster their claims.
This much is clear : there is no room for denial, hubris and escapism now.
Whether we imagine this \textbf{moment} as a “ rapidly closing window of opportunity [ 13 ] ” or a portal to a liveable \textbf{future}, the last two weeks have taught us that \textbf{governments} and companies can move far and fast when they recognise a threat as real and imminent.
The prevailing mantra, “ not here, not now, not us, not me ” can be replaced overnight for \textbf{lives} we deem grievable.
As Barbados Prime Minister, Mia Mottley told her peers at COP 26 a few months ago, “ \textbf{leaders} must not fail those who elect them to lead, ” and ‘ we need to continue to encircle and to remind those who are not ready to lead that their \textbf{people} need them to get on board as soon as possible. ’ In 2022, \textbf{frontline} \textbf{communities} and \textbf{climate} activists worldwide are ready to remind them in \textbf{court}.
Amrekha Sharma is a Story Advisor for the \textbf{Climate} Justice \& Liability Project at Greenpeace International Mary Robinson, \textbf{Climate} Justice : Hope, \textbf{Resilience} and the Fight for a Sustainable Future ( Bloomsbury Publishing, 2018 ) 8.
Ibid.
IPCC WGII TS.E.3.4 at TS-83.
IPCC SPM-5.
IPCC WGII SPM D.5.2 at SPM-35 ; WGII SPM D.2.1 at SPM-32.
IPCC WGII TS.D.3.2 at TS-59.
Arundhati Roy, An Ordinary Person ’s Guide to Empire ( South End Press, 2004 ). “ The \textbf{weather}, which we had learned and predicted for centuries, had become uggianaqtuq — a Nunavut term for behaving unexpectedly, or in an unfamiliar way.
Our sea ice, which had allowed for safe travel for our hunters and provided a strong habitat for our marine mammals, was, and still is, deteriorating.
I described what we had already so carefully documented in the petition : the \textbf{human} fatalities that had been caused by thinning ice, the animals that may \textbf{face} extinction, the crumbling coastlines, the \textbf{communities} that were having to relocate—in other words, the many ways that our \textbf{rights} to \textbf{life}, health, property and a \textbf{means} of subsistence were being violated by a dramatically changing \textbf{climate}. ” IPCC WGII SPM B.2 at SPM-11.
IPCC TS-20 WGII TS.B.7.3.
IPCC WGII SPM.B.2.4 at SPM-12.
IPCC WGII, SPM D.2.1 at SPM-32.
IPCC WGII TS.D.3.4 at TS-59 ; WGII SPM.C.5.6 at SPM-30, D.2 at SPM-32.
IPCC WGII SPM D.5.3 at SPM-35.
In her compelling story, The Right to Be Cold, Inuk \textbf{leader}, Sheila Watt-Cloutier described her testimony to the Inter-American Commission of Human Rights.
She brought the Inuit petition in 2005, one of the \textbf{world} ’s first \textbf{climate} \textbf{justice} cases, on behalf of all Inuit of the Arctic regions of the United States and Canada who were experiencing \textbf{human} \textbf{rights} violations from \textbf{climate} change.
The petitioners believed that if you protect the \textbf{rights} of the Inuit hunter on the ice, you would protect the “ sentinels of \textbf{climate} change ” and the “ \textbf{world} ’s early warning \textbf{system}. ” They called on the \textbf{protection} of various sources of \textbf{human} \textbf{rights} law.
They summoned the authority of the \textbf{world} ’s scientists in the IPCC 2004 Arctic \textbf{Climate} Impact Assessment.
They detailed the \textbf{impacts} of \textbf{climate} change they had been experiencing.
But the Commission decided in 2006 that it was not possible to process the petition, because it could not determine whether there had been a violation of their \textbf{human} \textbf{rights} based on this information.
They were unable to ‘ see ’ these “ alleged facts ” as \textbf{human} \textbf{rights} violations.
The IPCC ’s reports on \textbf{climate} \textbf{science} have evolved since that time.
Mary Robinson said, “ if there is a \textbf{climate} change problem, it is in large part a \textbf{justice} problem [ 1 ]. ” She said that “ to deal with \textbf{climate} change we must simultaneously address the underlying \textbf{injustice} in our \textbf{world} and work to eradicate \textbf{poverty}, exclusion, and \textbf{inequality} [ 2 ]. ” Part of that \textbf{injustice} has been a refusal to listen to what \textbf{communities} living with \textbf{climate} \textbf{impacts} like the Inuit saw, felt and knew long before the IPCC reports confirmed it.
So, it is no small thing that the \textbf{world} ’s \textbf{climate} scientists choose to not only acknowledge, but to centre the “ \textbf{justice} problem ” in different ways in the recently released IPCC WGII report.
There are important implications for \textbf{climate} \textbf{justice} advocacy as a result of braiding other ways of knowing together with \textbf{climate} \textbf{science}.
The IPCC report centred the value of diverse forms of \textbf{knowledge}, including millennia of Indigenous \textbf{knowledge} on \textbf{environmental} adaptation [ 3 ] and local \textbf{knowledge}, to not only understand and evaluate adaptation actions that reduce \textbf{risks} from \textbf{climate} change [ 4 ], and enhance resilient development [ 5 ], but to avoid the pitfalls of maladaptive actions [ 6 ] such as deploying unproven technological ‘ solutions ’.
What counts as “ \textbf{knowledge}, ” “ \textbf{science} ” and “ truth, ” whose \textbf{knowledge} counts as such, and who decides are questions of \textbf{power} that sit at the \textbf{heart} of \textbf{climate} \textbf{justice}.
With its spotlight on a \textbf{diversity} of \textbf{knowledge}, the report reminds us that, to borrow from Arundhati Roy, there have never been “ voiceless ” \textbf{people}, only those “ preferably unheard [ 7 ]. ” The Inuit petitioners asserted that through their Qaujimajatuqangit, or IQ – a living, generational body of \textbf{knowledge} about their environment–they knew their \textbf{land}, and the \textbf{land} was changing.
The Black Summer bushfires of 2020 lifted the \textbf{knowledge} and fire practices of First Australians as part of a new paradigm in “ \textbf{caring} for country. ” The IPCC also recognises that \textbf{people} ’s \textbf{vulnerabilities} to \textbf{climate} change are driven by “ \textbf{patterns} of intersecting socio-economic development, unsustainable ocean and \textbf{land} use, inequity, marginalization, historical and ongoing \textbf{patterns} of inequity such as colonialism, and governance [ 8 ]. ” This is like looking through a wide-angle lens with a time-lapse function on \textbf{climate} \textbf{impacts}, and it reveals more inconvenient truths : \textbf{vulnerability} to \textbf{climate} change is not an unfortunate accident authored by an invisible hand in the sky.
It is the result of \textbf{choices} made again and again, by some with \textbf{power} over \textbf{generations}. \textbf{Climate} change–previously framed predominantly as a scientific problem, and an economic problem – when properly situated in its historical \textbf{context} as the IPCC recognises, is an international \textbf{justice} problem.
Different questions appear, and some are questions that advocates in \textbf{climate} and social \textbf{justice} movements around the \textbf{world} have long asked : why are some countries ‘ developing ’ and some ‘ developed ’?
How did the latter acquire such \textbf{wealth} and \textbf{power} to eventually destabilise Earth ’s \textbf{life} \textbf{support} \textbf{systems}?
Who is responsible for taking the boldest actions first? “ \textbf{Loss} and damage ” discussions, mired in a frame of charitable benevolence for developing countries, if tethered to historical truths, start to look rather like decades of receipts arriving on the doorstep of developed countries.
Celebrating their \textbf{resilience} but failing to pay is both immoral and unjust..
The IPCC report also underlined the intersectional \textbf{nature} of \textbf{people} ’s \textbf{vulnerability} to \textbf{climate} \textbf{impacts} because of many underlying factors that have led to imbalances in their \textbf{power} and agency, including their \textbf{gender}, race, \textbf{class}, ethnicity, sexuality, Indigenous \textbf{identity}, age, disability, income, migrant status, and geographical location [ 9 ].
Intersectionality is a term coined in 1989 by Kimberle Crenshaw, a Black feminist legal scholar and student of the civil \textbf{rights} movement.
She used it as a way to make visible the multiple, intersecting oppressions that Black \textbf{women} experienced in daily \textbf{life}, but were rendered invisible to the legal \textbf{system}. \textbf{Climate} activists have also used an intersectional lens to make visible the ways some \textbf{people} and groups, like Dalit \textbf{women}, are made more vulnerable to \textbf{climate} \textbf{impacts}.
The IPCC recognises that Indigenous \textbf{communities} and many across the Global South, who have been historically put in marginalised situations [ 10 ] and contributed the least to the problems suffer the worst consequences in a vicious cycle of \textbf{injustice}.
Intersectionality offers a higher resolution lens on the questions of who is \textbf{impacted} by \textbf{climate} change, how they and their \textbf{rights} are \textbf{impacted}, in what ways they are made vulnerable and why, and what capacities exist.
This may allow for more nuanced and tailored remedial strategies, and new advocacy \textbf{pathways} may open for addressing \textbf{people} ’s \textbf{vulnerabilities} between and within countries in more equitable ways.
An intersectional approach hones in on listening to \textbf{people} ’s experiences of \textbf{climate} change as key to finding solutions that work for them. i v ) \textbf{Human} rights-based approaches are needed The IPCC also emphasised the need to take \textbf{human} rights-based approaches from the outset that focus on building \textbf{communities} ’ capacities, ensuring meaningful \textbf{participation} in making the \textbf{decisions} that affect their \textbf{lives}, and providing access to financing and other \textbf{resources} \textbf{communities} need to adapt to \textbf{climate} \textbf{impacts} [ 11 ].
The Inter-American Commission on Human Rights recently said that “ the \textbf{human} \textbf{rights} approach with a \textbf{gender} perspective and intersectionality is essential to address \textbf{climate} change and the threat it poses to \textbf{people} in the most vulnerable situations. ” Human \textbf{rights} claims in \textbf{court} emphasise the \textbf{reality} of \textbf{climate} change on \textbf{people} ’s \textbf{lives}, the high \textbf{stakes} of \textbf{inaction}, and already play a key role in \textbf{climate} litigation for many activists whether in the Philippines, South Africa, Argentina or in Europe.
In sum, the IPCC confirmed what \textbf{frontline} \textbf{communities} have been saying for years : “ \textbf{climate} responses must centre \textbf{justice} [ 12 ]. ” They have introduced other \textbf{frameworks} of \textbf{knowledge}, weaving in historical \textbf{context}, traditional and local \textbf{knowledge} of place, intersectional experiences, and \textbf{human} rights-based responses to frame social \textbf{justice} as a key concern–as many plaintiffs in today ’s \textbf{climate} cases do.
As a powerful \textbf{institution}, this matters a great deal.

% matched lemmas: care, choice, class, climate, community, context, courage, court, decision, demand, diversity, environmental, face, framework, frontline, future, gender, generation, government, ground, heart, human, identity, impact, inaction, inequality, injustice, institution, justice, knowledge, land, leader, life, loss, mean, moment, nature, participation, pathway, pattern, people, poverty, power, protection, reality, resilience, resource, right, risk, science, stake, suffering, support, system, vulnerability, wealth, weather, woman, world
\end{textsample}
