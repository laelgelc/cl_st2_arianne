\begin{textsample}{POS Dim 2 – human – Score 55.00 – t387\_human.txt}  \label{ex:f2_pos_004}
The \textbf{crisis} signs couldn't be clearer : fires, \textbf{floods}, \textbf{droughts}, \textbf{pandemic}, species extinction … Earth is screaming with all her might.
We need to listen, and to \textbf{act}.
So-called “ fortress conservation ” which evicts \textbf{people} from \textbf{land} that ’s been home for Indigenous Peoples and local \textbf{communities} for \textbf{generations} is ethically deeply problematic and has had horrific consequences on the \textbf{ground}.
In the Congo Basin and elsewhere armed eco-guards \textbf{funded} by international donors and organisations have reportedly harassed, abused, raped and murdered local \textbf{people}.
These atrocities are not incidents, they are the \textbf{outcome} of a failed conservation model predicated on colonialism that treats marginalised and \textbf{forest} dependent \textbf{communities} as a threat to wildlife.
This outdated conservation approach must be discarded entirely.
It is not a solution to the planetary \textbf{crisis} we are \textbf{faced} with and needs ruling out before \textbf{governments} look for ‘ easy ways ’ to meet 30×30 targets.
This needs to happen fast or there ’s a very real \textbf{risk} we see a boom in colonial-style conservation that \textbf{pushes} millions of \textbf{people} off their \textbf{land}.
A worst-case \textbf{outcome} of \textbf{land} \textbf{protection} targets would be a rush of offset or greenwashing projects that allow states and \textbf{corporations} with large ecological footprints and greenhouse gas emissions to retain their unsustainable business model by investing in top-down managed protected areas.
This would further exacerbate social \textbf{injustice}, infringe \textbf{rights}, and undermine \textbf{dignity} and avenues for prosperity for local and Indigenous \textbf{communities} who are \textbf{guardians} of \textbf{biodiversity}.
We will not live in “ harmony with \textbf{nature} ” as the Convention on Biological \textbf{Diversity} vision states if we throw \textbf{people} off their \textbf{land} and make it inaccessible for customary use.
We can't remedy the immense \textbf{destruction} from industrialised capitalist \textbf{exploitation} of \textbf{nature} by placing the burden of repairing on those least responsible.
To turn the tide on \textbf{nature} \textbf{destruction}, decision-makers must listen to, \textbf{support} and \textbf{respect} those who have lived close and long with healthy \textbf{ecosystems}. \textbf{Governments} must ensure local and Indigenous \textbf{rights} to \textbf{land}, and \textbf{leadership} in planning and managing protected areas.
And provide robust legal instruments to defend these \textbf{rights}.
Global \textbf{biodiversity} \textbf{decisions} affect us all.
Those shaping them must recognise the interconnected \textbf{crises} of \textbf{climate}, \textbf{biodiversity} and global \textbf{inequality} and promote a shift of \textbf{power} that enables the \textbf{guardians} of \textbf{biodiversity} to help protect us all.
Irene Wabiwa Betoko is Greenpeace Africa ’s International Project Leader for the Congo Basin \textbf{forest}.
Savio Carvalho is Global Campaign Leader on Food, Forest and Biodiversity for Greenpeace International.
This story was originally posted by Al Jazeera.
We must defend the \textbf{planet} ’s \textbf{life} \textbf{support} against relentless corporate \textbf{greed} and rediscover humanity as part of the natural \textbf{world}, for current and \textbf{future} \textbf{generations}.
Restoring balance requires \textbf{governments} to heed the \textbf{knowledge} of \textbf{communities} who have listened deeply and worked with Mother Nature for \textbf{generations} – and to recognise and \textbf{respect} the \textbf{rights} of Indigenous Peoples and local \textbf{communities}.
Business-as-usual backed by polluted \textbf{politics} is the problem.
The same destructive \textbf{systems} that are stripping our \textbf{forests} and oceans of \textbf{life} are killing \textbf{environmental} \textbf{defenders} and \textbf{pushing} \textbf{people} into peril.
To reset our \textbf{relationship} with \textbf{nature}, we need systemic change to the way we produce and consume food, energy and natural \textbf{resources}.
At every level of governance – from our local \textbf{community} to the UN \textbf{biodiversity} \textbf{summit} – political \textbf{decisions} can aid a green and just recovery from \textbf{crises}, build \textbf{resilience} against \textbf{future} epidemics, and allow \textbf{people} and the \textbf{planet} to thrive.
As \textbf{governments} collectively agree on next steps for global \textbf{nature} \textbf{protection}, it ’s clear we need a better plan.
Protecting at least 30 % of \textbf{land} and oceans by 2030 can be an important component in planetary recovery.
It ’s an ambitious, measurable target.
A global safety net preventing further \textbf{degradation} of critical eco-regions is essential and could halve the extinction \textbf{risk} for species.
But only if failed conservation models are discarded in favour of a global \textbf{push} to recognise customary \textbf{land} and reinforce \textbf{people} ’s \textbf{rights} – this is key if countries are to protect \textbf{biodiversity}, fight \textbf{inequality} and attain their \textbf{climate} goals.
While many of the most effective and highly protected ocean sanctuaries have been championed and won by local coastal \textbf{communities}, the 30×30 target for \textbf{land} has been met with deep concern from \textbf{environmental} and \textbf{human} \textbf{rights} groups and activists : designed without proper consultation and implemented wrongly, protected areas do not deliver \textbf{protection} but make matters worse for \textbf{people}, endangered species, and the \textbf{planet}.
The current draft post-2020 Global Biodiversity \textbf{Framework} ( GBF ) lacks credible guarantees against such an \textbf{outcome}.
Success depends on approaches that both promote \textbf{justice} and protect \textbf{biodiversity} and the GBF must reflect that.
We ’ve seen how \textbf{people} \textbf{power} can effectively \textbf{push} back on greedy companies.
In Mexico ’s Cabo Pulmo, local \textbf{communities} secured legal \textbf{protection} and are reviving marine \textbf{life} and \textbf{livelihoods}.
With legal \textbf{rights} and instruments to enforce them, Indigenous \textbf{People} have defended their \textbf{territories} from encroachment, invasion and \textbf{exploitation}.
Studies from Brazil show that this is the most effective way to \textbf{safeguard} \textbf{forest} and \textbf{biodiversity} in the Amazon.
There is a lot of potential in ensuring \textbf{people} have the \textbf{means} to resist industrial \textbf{expansion} that ’s \textbf{fuelling} species \textbf{loss}, \textbf{climate} \textbf{breakdown} and deepening \textbf{inequalities}.
For example, tedious bureaucratic processes must be simplified in Indonesia and Indigenous \textbf{people} fighting for \textbf{community} forestry management in Democratic Republic of Congo must be sufficiently \textbf{supported} as part of new and more ambitious global \textbf{protection} targets.
However, there is a much darker side of conservation that commits \textbf{human} \textbf{rights} violations and outright atrocities in its name.
This needs a reckoning as \textbf{governments} set higher targets for protected areas.

% matched lemmas: act, biodiversity, breakdown, climate, community, corporation, crisis, decision, defender, degradation, destruction, dignity, diversity, drought, ecosystem, environmental, expansion, exploitation, face, flood, forest, framework, fuel, fund, future, generation, government, greed, ground, guardian, human, inequality, injustice, justice, knowledge, land, leadership, life, livelihood, loss, mean, nature, outcome, pandemic, people, planet, politics, power, protection, push, relationship, resilience, resource, respect, right, risk, safeguard, summit, support, system, territory, world
\end{textsample}
