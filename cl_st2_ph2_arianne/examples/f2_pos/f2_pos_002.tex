\begin{textsample}{POS Dim 2 – human – Score 62.00 – t186\_human.txt}  \label{ex:f2_pos_002}
From Chile to Canada, \textbf{women} are at the forefront of the \textbf{climate} \textbf{crisis}, and are one of the most impacted groups by \textbf{climate} catastrophes.
Along with other minorities such as Black, Indigenous and \textbf{People} of Colour, LGBTQ+ folks and low-income \textbf{communities}, \textbf{women} are often part of those demographics as well, exposing the intersectionality of \textbf{climate} \textbf{impacts}, \textbf{gender} \textbf{inequality} and social \textbf{injustice}.
My name is Cláudia Farinha, I am a \textbf{woman} of the Land and of the Fight!
I am a family farmer and I live with my family in a Land Reform Settlement in Brazil.
I am an educator, communicator, ecofeminist, lawyer and socio-environmental activist.
I worked in the rural trade union movement for over two decades, which liberated the \textbf{voice} and \textbf{strength} of rural \textbf{women} the same way I forged myself in the \textbf{struggle} as farmer \textbf{leader}.
The \textbf{rights} of rural \textbf{women} and men are among the \textbf{priorities} in the fight for better living conditions.
In 2020, I joined a course to prepare activists to make communication an instrument of \textbf{women} ’s organizing.
In 2021, via the ECOAR project ( Meetings for Content on Anti-Deforestation in the Amazon and Cerrado ), I ’ve been in workshops to create socio-environmental content related to social media.
Since then, I have integrated my \textbf{environmental} activism on social media and on the streets in the fight for a better \textbf{world}.
It is not easy being a \textbf{woman}, an activist, a digital influencer and finding the best way to connect with \textbf{people} and raise awareness on \textbf{climate} issues.
We know the gravity of what we are experiencing, but the Brazilian Congress, the mass media and large capitalist \textbf{corporations} ignore the \textbf{urgency} of the \textbf{environmental} \textbf{agenda}.
The biggest \textbf{challenge} is to deepen contact with \textbf{people}, so that together, we can denounce and highlight the calamity we are experiencing.
The \textbf{environmental}, political, ethical and health \textbf{crises} require communication that shows the causes of these problems, for the collective construction of awareness about them, so we can undertake efforts to change this \textbf{reality}.
The only way is together.
I don't see any other alternative.
It is no use \textbf{demanding} the Congress, it is necessary to build a collective \textbf{force} to transform \textbf{reality} and save the \textbf{planet}.
This election year is very important in Brazil.
We need to unite and fight to occupy \textbf{spaces} of \textbf{power} and ensure we have \textbf{voice} and \textbf{strength} to have \textbf{leaders} who actively defend the environment and prioritize social \textbf{justice}, \textbf{gender} \textbf{equality}, the \textbf{rights} of \textbf{women} farmers, and the \textbf{climate}.
My name is Jackie Zamora from Mexico, for as long as I can remember I was taught to \textbf{respect} \textbf{nature} and when I learned from a very young age that our \textbf{planet} \textbf{faced} many threats, I wanted to somehow make a possitive difference.
In 2012 I joined Greenpeace Mexico and multitasked between being a College student and a volunteer coordinator.
This was just the beginning of my \textbf{environmental} activist journey. 10 years later I ’ve been in the Arctic, sailed in all 3 Greenpeace Ships and I ’m currently the Engagement lead for the European Mobility Campaign.
I grew up in Latin America, where \textbf{environmental} activists and \textbf{human} \textbf{rights} \textbf{defenders} \textbf{face} intimidation, threats and \textbf{violence}.
It ’s one of the most dangerous regions in the \textbf{world} to fight for \textbf{climate} \textbf{justice} and \textbf{gender} \textbf{equality}.
Although these attacks affect all \textbf{defenders}, \textbf{women} activists are specifically targeted and \textbf{face} additional obstacles and \textbf{risks} like gender-specific threats or \textbf{barriers} to accessing decision-making \textbf{spaces} and platforms.
Many brave \textbf{environmental} and \textbf{human} \textbf{rights} activists like Berta Cáceres from Honduras and Marielle Franco from Brazil have lost their \textbf{lives} in these fights.
These same \textbf{women} are showing the \textbf{world} that a \textbf{planet} \textbf{centered} on \textbf{justice}, \textbf{resilience} and \textbf{care} is not only possible but necessary.
Here are 8 \textbf{women} shaping the \textbf{climate} \textbf{conversation} in the Americas region : It ’s important to acknowledge the intersectionality that is inherent to how \textbf{people} experience the environment, to call for solutions that appreciate varied experiences. \textbf{Environmental} \textbf{justice} is this intersection of both social \textbf{justice} and environmentalism.
As Berta Cáceres said “ Let ’s build \textbf{societies} that are able to coexist in a way that is dignified, just and protective of \textbf{life} ” My most memorable \textbf{moment} in my activism journey was the first time I saw icebergs and penguin colonies in the Antarctic and a few minutes later I saw plastic bottles in the very same sea.
Yes, there is a lot of plastic waste in the Antarctic.
It was heartbreaking to see the \textbf{reality} our oceans are \textbf{facing}.
Everyone ’s fate is bound to the fate of our \textbf{planet} and we have a shared responsibility in the mission to protect our only home.
My name is Julialynne Walker and I ’m a \textbf{land} steward.
I like to put my hands in the earth.
I started a garden at my mother ’s house, and then something at her church.
In the process, I became reacquainted with the Columbus ( Ohio, US ) environment, in particular, the historic African-American section.
There just was a devastation that I found astonishing due to a number of different changes.
I wanted to be able to address that in some way, and that ’s how I got started with the Bronzeville growers market. \textbf{People} were coming to me and saying : I like to garden, but I don't know how ; this looks exciting but I live in a small place.
I approached Ohio State University and asked about doing a \textbf{class} there.
But then COVID hit, so it all got canceled.
I next found a young person to show me how to use zoom.
Then I got on my back porch, gathered plants and different things and did a 10 week free online introduction to agriculture \textbf{class}.
Everybody was excited, \textbf{people} came on and off as they chose, I wasn't prescriptive.
I realized there were \textbf{people} that really wanted to grow food and that I was developing a sort of vertical model of integration with this work.
I got \textbf{funding} to buy large plastic bins and we literally just dug holes in the bottoms, put about a third of leaves and small twigs and the \textbf{soil}.
We gave seeds and some starter plants, so \textbf{people} could have one to three bins and they could place them in their yard, on the porch, it was flexible.
If each of us can take responsibility for the little piece that we have and do it in a way that makes sense, that ’s how to start creating change in the greater whole.
I ’m afraid of \textbf{people} who are not open to the truth.
I ’m just dumbfounded at some things happening in the United States.
I do not understand how \textbf{people} don't see that their own \textbf{interests} are being sabotaged by this commitment to falsehood.
Every one of us has a responsibility to do something that ensures a \textbf{future}, because this is it, we got one \textbf{planet}.
I don't know why this is so difficult to understand.
There is no place else for us to go.
If you ’re not for zero waste, how much waste are you for?
After reduce, reuse, recycle, I ’d add a fourth one : rethink.
I am a trans \textbf{woman} ( I also like the term “ travesti ”, which is Latin ), communicator, popular educator and poet, who lives in Rio de Janeiro, this \textbf{land} so marked by beauty and \textbf{chaos}, in Brazil.
Being a trans person, in the country that kills the most LGBT \textbf{people} in the \textbf{world}, and in such a violent city, brings to me the \textbf{urgency} to write my \textbf{texts}, my poetry and produce content for social media.
I think it is important to reflect on how \textbf{violence} occurs in different ways, including the lack of access to water, decent housing, green \textbf{spaces} in cities and several other factors that are also related to the environment and the social \textbf{gap} between \textbf{people}.
Brazilian trans \textbf{people}, for the most part, do not have the basics to survive.
And we are all the time stating that we need not only to survive, but above all to live.
We need to be seen as real citizens.
And this fundamentally involves building a \textbf{planet} that is socially and environmentally just, where all \textbf{people} can live their \textbf{lives} with guaranteed \textbf{rights} and in harmony with \textbf{ecosystems}.
We need to understand that the \textbf{human} being is also \textbf{nature}, not something apart from it.
Understanding the connections between \textbf{nature} and humanity, and that all \textbf{humans} must have their \textbf{human} \textbf{rights} guaranteed, is urgent.
I am physically based in Scotland, though most of my engagement with Greenpeace has taken place in Canada.
As a volunteer, I ran the KleerCut campaign in Thunder Bay, Ontario and \textbf{supported} a local group in Ottawa.
I am now a member of the Board of Directors of Greenpeace Canada, and currently serve as the Board Chair.
As a glaciologist, my “ day job ” consists of research on the deterioration of icebergs and retreating Antarctic glaciers.
Having participated in actions in Argentina, relating the \textbf{debates} on feminism, economics and environmentalism, marked me a lot.
I believe that our fight needs to break borders and \textbf{solidarity} needs to be international.
Whether in Rio de Janeiro or Vancouver, whether in Buenos Aires or Lisbon, our \textbf{struggles} are connected and the answers also need to come together.
I am Maria Paz Valenzuela, a Chilean mountaineer.
This is an activity that I have carried out since my adolescence in all \textbf{continents} of the \textbf{world}.
I currently carry out mountain and trekking activities, directing groups of \textbf{people} interested in outdoor activities.
The Magmandinas expedition to Ojos del Salado was a totally different experience to any other expedition for me.
The motivation and soul of this project is unique.
The idea was initially conceived as a \textbf{women} ’s expedition, since its fundamental plan was to bring together \textbf{women} from Latin America and create a \textbf{space} for reflection and coexistence with the environment to finally deliver a powerful message about the importance of \textbf{care} for our \textbf{land}, for ourselves and the \textbf{relationship} that we establish with our surroundings, as well as to empower us as mountaineer \textbf{women}, moving our own \textbf{limits} and achieving our own goals.
The role of all \textbf{human} beings cannot be postponed when we talk about protecting the \textbf{planet}.
This is not a \textbf{gender} issue, it transcends that.
It is a personal and group responsibility to take \textbf{care} of our environment.
Educate to protect, \textbf{care} for and love what surrounds us.
Educate to reconnect and return to the purest essence as an individual, recover the ability to wonder at each natural event, open your eyes and see, see with your \textbf{heart}.
Indigenous \textbf{woman} from the Sateré Mawé \textbf{People}, I am a biology student at the University of the State of Amazonas, Brazil.
Artisan, presenter, communicator and \textbf{environmental} activist, I work to simplify Indigenous issues online and as a \textbf{defender} of the environment.
When we Indigenous \textbf{People} are born, there is no \textbf{moment} when we decide to be activists, we are born activists, because we have always been fighting, for our \textbf{culture}, our language, \textbf{identity} and \textbf{territory}, and these \textbf{struggles} go hand in hand with the fight for the environment, since the protected Indigenous Lands are the ones with the most \textbf{biodiversity}.
As a biology student, much of the \textbf{environmental} activism comes from combining Indigenous Peoples ’ fight for our \textbf{lands} with the fight for \textbf{climate}.
And as \textbf{women}, our actions are often discredited, but in \textbf{reality} we are protagonists. \textbf{People} think that scientists are men, with glasses and a PhD, but \textbf{women} also do \textbf{science} and are more than ever in the daily fight for the \textbf{protection} of \textbf{lands}, along with all Indigenous Peoples.
I was able to participate in COP26, which was a historic milestone for me.
It was a unique experience, being an Amazonian, an Indigenous \textbf{woman} and being on another \textbf{continent} telling \textbf{people} and other countries how their actions \textbf{impact} our environment.
It felt unique to talk about how little time we have left and that the solution are Indigenous Peoples, the main \textbf{defenders} of the \textbf{planet}.
In addition, the march for the \textbf{climate}, made by the \textbf{youth} at COP26, and the first march of Indigenous \textbf{women} in Brazil were unique sensations of belonging and action.
It ’s hard work and sometimes we don't feel optimistic, but we have to keep going.
The current and impending \textbf{impacts} of glacier \textbf{loss} have local and global consequences, including endangering water \textbf{resources} for millions of \textbf{people} and altering coastlines worldwide.
Greenpeace has an important role to play in ensuring that \textbf{society} is making all efforts to mitigate \textbf{climate} change and address the \textbf{inequalities} that it will exaggerate.
We all have something – \textbf{strengths}, skills, perspectives, experiences – to contribute.
We all have something to learn from each other too.
I think our openness to new ideas and ways-of-thinking is key as we look to create and embrace change together.
I ’m Billie Lee and I ’m from Indiana, US.
I ’m a TV writer, producer and also an activist for food sustainability and transgender \textbf{rights}.
As a long-time vegan, creator and author of the popular blog She ’s So Vegan, I try to use my own lifestyle to educate and inspire others through the \textbf{power} of food.
I first gained notoriety on the hit Bravo series Vanderpump Rules, and now I am working on my first book, Why Are You So Sensitive?, which will be published by Andrew McMeel and released in 2023.
In addition, I continue to work for equity—in the workplace, in housing and in food accessibility/sustainability and in entertainment.
I believe we are here to be of service for our \textbf{planet} and those most vulnerable to \textbf{environmental} dangers.
As a trans activist I ’ve noticed we are fighting the same enemy. \textbf{Climate} deniers and the politicians that \textbf{support} big oil are the same enemy we trans \textbf{people} \textbf{face} while fighting for our fundamental \textbf{human} \textbf{rights}.
My most memorable \textbf{moment} as an activist was protesting the big oil companies and for the new green deal, seeing all my trans/non-binary \textbf{community} on the front line fighting for our \textbf{planet}.
With all the discrimination trans and non-binary \textbf{people} experience we still know there is no \textbf{life} with or without \textbf{rights} if there is no \textbf{planet} to call home.

% matched lemmas: agenda, barrier, biodiversity, care, center, challenge, chaos, class, climate, community, continent, conversation, corporation, crisis, culture, debate, defender, demand, ecosystem, environmental, equality, face, force, fund, future, gap, gender, heart, human, identity, impact, inequality, injustice, interest, justice, land, leader, life, limit, loss, moment, nature, people, planet, power, priority, protection, reality, relationship, resilience, resource, respect, right, risk, science, society, soil, solidarity, space, strength, struggle, support, territory, text, urgency, violence, voice, woman, world, youth
\end{textsample}
