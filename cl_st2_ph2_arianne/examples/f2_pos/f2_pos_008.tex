\begin{textsample}{POS Dim 2 – human – Score 51.00 – t226\_human.txt}  \label{ex:f2_pos_008}
From organising peaceful protests and fighting \textbf{forest} fires, to painting banners and working on digital content to spread awareness, Greenpeace volunteers come from all ages and backgrounds but are united in their passion for \textbf{climate} \textbf{justice}.
Despite the \textbf{challenges}, the courageous young \textbf{woman} \textbf{stands} ready to do what she can to protect the environment. “ As a firefighting volunteer, I must be ready when called upon in the event of a fire and ready to be placed wherever I am needed. ” What do you hope for with your action?
I want to continue to see the sky as it is now, blue and without the haze, and we can live our \textbf{lives} as usual.
I want the children to be able to play happily outside the house and all of us do not have to worry about the threat of respiratory infections anymore.
It was early November in 2013 when Ronan Renz Napoto and his family in Eastern Visayas, Philippines, heard over the news that there was a typhoon coming. “ Living in the Pacific, we ’re used to having typhoons so we weren't very worried, ” he said.
When Typhoon Haiyan hit, they were unprepared for its ferocity as it ripped through the Philippines.
One of the most powerful typhoons in history, it caused widespread devastation and \textbf{loss} of \textbf{lives}.
For years after, Ronan would have nightmares of the day, often waking up with tears running down his \textbf{face}. “ I can still remember the sequence of everything, even \textbf{right} now everything keeps on flashing back to me.
It ’s still painful to remember those events, ” he said.
It was on his journey to process the trauma that led Ronan down the \textbf{path} of \textbf{climate} advocacy.
Already a \textbf{youth} \textbf{community} \textbf{leader} and trash crusader, the natural disaster drove his awareness of the \textbf{urgency} of the \textbf{climate} \textbf{crisis} and his need to \textbf{act}.
After taking part in a story sharing event organised by Greenpeace Philippines, he started to actively volunteer with Greenpeace, participating in brand audits and helping with administrative \textbf{duties}.
When the Rainbow Warrior was anchored in Tacloban as part of the \textbf{Climate} \textbf{Justice} tour, he volunteered to be a guide.
Ronan is also engaged in influencing \textbf{policy} makers in his \textbf{community} about creating effective \textbf{environmental} and plastic \textbf{policies}.
An eloquent orator and natural storyteller, he often speaks at Greenpeace events about \textbf{climate} \textbf{justice} and the \textbf{science} behind \textbf{climate} change. “ I also talk about food and agriculture, the \textbf{impact} of agriculture on \textbf{climate} change, and other topics revolving around \textbf{climate}, ” he said.
His most memorable activity with Greenpeace is also the one closest to his \textbf{heart}, and that is collecting stories from the different \textbf{communities} in the yearly commemoration of Typhoon Haiyan. “ It reminds me that behind the \textbf{science} of \textbf{climate} change, there are real \textbf{people} with real stories, ” he said. “ Statistics are important but we don't want to be just remembered as numbers, we want to be remembered and our stories to be remembered about who we are and how we \textbf{struggled}. ” Ronan, who works with different organisations on research about \textbf{community} engagement and building \textbf{resilience}, is also the founder of Balud, a youth-led organisation that promotes ecological consciousness in the Visayas. “ Coming from the provinces, I wanted to highlight the \textbf{youth} \textbf{leaders} from outside the big cities.
We want to create more opportunities for \textbf{people} who are considered to be minorities or coming from vulnerable \textbf{communities} so that everybody acknowledges that we also have powerful stories, ” he said.
Meet a few of our volunteers in Asia as they share their inspiring stories about volunteering with Greenpeace and how they are fighting to protect our \textbf{planet}.
The process has not been easy for the young man but Ronan ’s \textbf{determination} and passion keeps him focused on his advocacy. “ We have in our local language the word Padayon, which translates as ‘ to keep going ’.
Because advocacy can be very hard and sometimes, you never know if there is going to be a light at the end of the tunnel.
But it ’s worth it to keep on going, to keep on trying and moving forward, especially to bring your \textbf{agenda} forward.
Nothing will happen if we just stop. ” Lee Hui Ling was exposed to \textbf{environmental} and social issues at a young age.
Born into a family of artists, her mother had a strong \textbf{environmental} conscience, which she had expressed through her art and imparted on her daughter. “ As a child, I would worry about the ozone layer expanding, acid rain, rubbish pollution, flora and fauna going extinct.
Very serious topics for a little girl! ” After graduating from the Sarah Lawrence college in New York and moving back to Malaysia, Hui Ling ’s concerns for the environment grew, particularly after the Fukushima nuclear \textbf{crisis} in 2011, her mother ’s hometown. “ Greenpeace was very active in investigating the extent of the pollution at that time and I was very impressed by how transparent they were with the information, as opposed to a lot of cover ups, ” she said.
Hui Ling responded by setting up a Greenpeace Malaysia online \textbf{community} on various social media platforms.
This was instrumental in the eventual setting up of the Malaysian office in 2017 A committed volunteer and a natural \textbf{leader}, Hui Ling was involved from the very beginning.
She helped to organise and lead at meet-ups in cafes and \textbf{community} halls, as well as run workshops, training and retreats.
Taking direct action, she has participated in various campaigns such as Radioactive Ruse, Stop The Haze, and Break Free From Plastic. “ \textbf{Environmental} activism has taught me that doing good is not a sprint, but a marathon, and we need to develop the endurance and \textbf{resilience} to make it through the difficult times, ” she said. “ I think activism has been normalised, and with that kind of normalisation, it brings a level of safety.
It becomes a very effective way to speak about social issues and affect change.
An artist and educator, Hui Ling organised participatory art projects in line with Greenpeace Malaysia campaigns on deforestation, plastic pollution and consumerism.
One of them was the Wings of Paradise project, where she led a team of 30 \textbf{youth} volunteers in creating a 64-meter long mural as part of a global street art campaign against deforestation in Papua. “ We see the \textbf{inequalities} between the Global North and Global South exacerbated through \textbf{climate} change.
This is a \textbf{climate} \textbf{emergency}, ” said Hui Ling. “ However, there is a beacon of hope in the \textbf{youth} activism of the last few years.
The \textbf{youths} of today are well organised, articulate and passionate in expressing their desire for positive change and a green and sustainable \textbf{future} for all. ” What change would you most like to see in the \textbf{world}?
As a whole, I would like \textbf{society} to be more kind, generous and inclusive.
I would also like to see a more proactive sharing \textbf{economy} and sustainable models of entrepreneurship, production and consumption.
Dwi Agustya Ningrum, also known as Tya to her friends, started volunteering as a Greenpeace firefighter after the great \textbf{forest} fire in Riau in 2015. “ The sky was filled with smoke.
Visibility was only about 1 to 2 meters, ” remembered Tya.
What does it \textbf{mean} to be a Greenpeace volunteer?
I think, first and foremost, one comes with the idea of wanting to effect change.
It comes from a place of empathy and being passionate about \textbf{environmental} issues in a very deep and involved way.
Greenpeace volunteers are different in a sense that they ’re more involved in \textbf{environmental} issues, it ’s not just greenwashing or a PR thing.
I think that ’s what sets Greenpeace volunteers apart, that there ’s more direct action.
Amika Jamjansri was a first year student at Mahidol University, Thailand, when enticed by the promise of diving lessons, she signed up for a beach clean up activity where the volunteers had to collect, separate and group all the plastic on the beach.
As Amika set to work, she was \textbf{shocked} by the magnitude of the plastic problem. “ This trash that we collected is only just a small amount of the plastic, what about the rest of it that ’s out there in the ocean? ” That exposure to the plastic waste problem signalled the start of Mika ’s \textbf{environmental} crusade, and she started adopting plastic-free habits.
In her passion to take action to help the \textbf{planet}, she joined Greenpeace Thailand as a volunteer in 2020.
She was working on the Solar Generation campaign when the \textbf{pandemic} hit and the country went into lockdown.
Despite the lack of mobility and activities being confided online, Mika continued volunteering with Greenpeace, joining the meat and dairy campaign where she helped to conduct online surveys.
When restrictions for in-person activities were lifted, she participated in brand audits for the plastic campaign and was an emcee for a Facebook Live event.
Her favourite experience was training with the boat team and the camaraderie she experienced with other like-minded volunteers. “ It was so fun!
I ’ve never done anything like this before and I learnt a lot about driving boats by the end of it, ” she said.
Besides being passionate about going plastic free, Amika is most concerned about the \textbf{impact} of \textbf{climate} change on humanity and wildlife, and the \textbf{loss} of \textbf{biodiversity}. “ There are so many incidents now of flooding and fires, and it sets off a chain reaction.
To tackle the problem, it comes down to law and implementation.
Those in \textbf{power} are not serious enough despite the big conferences.
There is just not enough real action. ” What one change would you most like to see?
I would like to see international law being changed in favour of solving \textbf{environmental} issues more effectively and that businesses are made to take responsibility for their actions.
Minseop Kim ’s awakening to the \textbf{climate} \textbf{emergency} came early this year when his home in Seoul, the capital city of South Korea and home to 10 million, was \textbf{flooded} after abnormally heavy rains. “ Lots of single-person households in their 20s in South Korea live in places that are very vulnerable to extreme \textbf{weather} like \textbf{floods}, \textbf{storms}, and rainfall. \textbf{Climate} \textbf{crisis} is happening to \textbf{people} like me, who migrated to Seoul with less economic \textbf{power} to buy a house resilient to \textbf{floods}.
Based on the current situation with global warming, we ’re going to experience extreme \textbf{weather} more frequently and more severely, ” he said. “ I was scared whenever I heard of heavy rainfalls in the news.
But the fear doesn't change anything.
If I move from here to another place, someone will go through a similar situation as mine.
So I decided to find something I can do to make a difference. ” “ My family was my main reason to sign up.
My parents had a respiratory infection caused by the haze.
I had to see them use oxygen cylinders at home.
Small children, who are supposed to play with friends their age outside the house, were also \textbf{forced} to stay at home.
What is certain is that we are directly \textbf{impacted} by the \textbf{forest} and peat fires. ” Minseop joined Greenpeace as a volunteer, using his skills as a web designer to design effective, powerful visuals for Greenpeace Korea ’s \textbf{Climate} Suffrage campaign.
The campaign is aimed at changing \textbf{climate} \textbf{policy} in South Korea, including the Korean \textbf{government} ’s carbon neutral \textbf{policy}, the national law and the presidential candidates ’ commitment to \textbf{climate} action.
He also participated in the ‘ Green New Deal Civic Action ’, where he monitored the legislative action of congress members and made calls to the office of congress members to relate his personal experience regarding \textbf{climate} change and to urge them to take more ambitious \textbf{climate} action. “ I got inspired by volunteers abroad who called the local politicians \textbf{demanding} stronger \textbf{climate} action as a citizen lobbyist.
I feel empowered taking real \textbf{climate} action, more than voting, ” he said. “ Volunteering is important because \textbf{people} who are concerned about similar issues are connected and \textbf{act} together.
We can learn from each other, from their different perspectives on one issue.
For me, I ’ve learned from my peer volunteer, who has hearing \textbf{loss}, that the \textbf{climate} \textbf{crisis} is very connected to the \textbf{human} \textbf{rights} issue as well. ” What one change would you most like to see?
I ’d like to see more \textbf{people} talking about \textbf{climate} change, discuss and \textbf{act} upon it.
Individual action matters itself, but I believe that collective civic action can change a giant ’s action.
The \textbf{government} and \textbf{corporations} are the two giants we have to change in order to respond to the \textbf{climate} \textbf{crisis}.
As citizens, we need to monitor and pressure the \textbf{government} and businesses.
Today ’s five phone calls from \textbf{people} can change the meeting \textbf{agenda} for the next morning.
Why don't you call congress members to do the \textbf{right} thing?
Why don't you call the company and say that there will be more chances in the green \textbf{economy} transition?
A single phone call can make more of a difference than just being a voter.
If you believe something is important, start volunteering and start making a change in your \textbf{society}.
It was while Neha Gupta was researching ways to lead a more sustainable lifestyle and to manage waste effectively that led her to Greenpeace.
During last year ’s lockdown when schools were closed, the teacher of \textbf{environmental} \textbf{science} and English took the opportunity to sign up as a volunteer with Greenpeace India. “ I wanted to join Greenpeace to work upon solutions at a \textbf{community} and organizational level, progressing further from the individual effort, ” she said, citing rising temperatures and its \textbf{impact} on food \textbf{security} as some of her biggest \textbf{climate} concerns.
In her short time with Greenpeace, Neha has proven to be a dedicated and valuable member of the organisation.
One of her first activities was to lead a webinar on home composting.
Neha had been experimenting with various ways to compost at home and was able to share her experience and learnings with over 100 attendees from around India. “ I was nervous but had good \textbf{support} from the team in terms of how to organize the webinar.
It was a challenging but memorable experience for me, ” she said.
Later, she took part in the Clean Air for Blue Skies campaign to address Delhi ’s air pollution.
Dressed in ethnic wear, she campaigned for clean air at Connaught Place, speaking to the public about air pollution.
Neha also participated in the \textbf{Power} the Pedal campaign, empowering low-wage \textbf{women} labourers to use bicycles as a safe, sustainable and fun mode of transport.
Now, every year, as summer rolls around, Tya gets worried because that ’s when most \textbf{forest} fires occur, either naturally or as a result of \textbf{land} or \textbf{plantation} clearing. “ In my opinion, \textbf{forest} and \textbf{land} fires that most often occur in Indonesia, especially Riau, are the result of \textbf{human} activities of those who no longer think about clean air, ” she said.
With Greenpeace, Neha found an open and friendly \textbf{community} with a shared mission to protect the environment. “ Greenpeace gives you the opportunity to interact with like-minded \textbf{people} but also gives you a sense of \textbf{community}, ” she said. “ The organisation has taught me how to work in a \textbf{community}, how to ideate together, plant together, \textbf{act} together and at the same time, have fun together while fighting for a cause. ” If you could make one change to combat the \textbf{climate} \textbf{crisis}, what would it be?
I would stress upon \textbf{respecting} \textbf{resources} around us, be it the \textbf{human} \textbf{resource}, natural \textbf{resource} or man-made \textbf{resource}.
When we learn to \textbf{respect} and value the \textbf{people} and things that surround us, a lot of things will come into perspective.
Why is volunteering important to you?
Volunteering not only gives me a window to express my gratitude and give back to \textbf{society} but it also encourages personal growth and it helps me learn.
Through volunteering, I have come to become a part of a \textbf{community} who believes in the similar cause and works together to achieve shared goals, mobilize and \textbf{impact} various other \textbf{communities} and stimulate a hope for the \textbf{future}.
Want to volunteer with Greenpeace?
Get in touch with your local Greenpeace organisation to find out about volunteering opportunities.
Stories were made possible with \textbf{support} from Norika Maurin Abriana ( Greenpeace Indonesia ), Kristina Hernandez ( Greenpeace Philippines ), Sakeenah Omar ( Greenpeace Malaysia ), Anchalee Pipattanawattanakul ( Greenpeace Thailand ), Jane Yu ( Greenpeace Korea ) and Abhishek Kumar Chanchal ( Greenpeace India ). “ It ’s important for me to volunteer as it ’s time for us to take on any role, no matter how small or big.
Because when the fire is burning and the haze is everywhere, it ’s a sign that we ’ve lost but the real defeat is if we do not do anything to prevent it from happening. ” Before carrying out actual tasks in the field, she had to take a volunteer firefighting course conducted by Greenpeace Indonesia.
Along with other volunteers, she was given training on fire prevention in peatlands as well as navigation, safety and \textbf{security} protocols. “ Peat is a highly flammable \textbf{soil} and difficult to extinguish.
Where I live, almost 50 % of the \textbf{land} is peat \textbf{soil}, ” said Tya.
As a newly enrolled firefighter, Tya ’s first few \textbf{duties} have been to carry out awareness campaigns to talk about the dangers of \textbf{forest} fires, especially peat fires, to \textbf{nature} and to \textbf{human} health.
In order to protect \textbf{nature}, she encourages the public to reduce their use of single-use plastic and also to carry portable ashtrays to hold cigarette butts. “ It could be simple \textbf{acts} but the effect that is felt is extraordinary, especially when \textbf{people} around you are also slowly becoming more aware of \textbf{climate} issues, ” she said.
In 2016, Tya and her team spent two weeks extinguishing fires in the Bukit Timah area.
On one of the days, a strong wind had picked up, flaming the smouldering, underground fire back up to the surface where it quickly started getting more aggressive.
Back at camp, situated in the safe area, Tya and her other colleagues were worried for the safety of their teammates out in the field.
Fortunately they made it back safely, but “ their \textbf{faces} looked more tired than usual, ” said Tya, who had to be rushed to the hospital because of smoke inhalation. “ I learnt a lot in a very short time, ” she said.

% matched lemmas: act, agenda, biodiversity, challenge, climate, community, corporation, crisis, demand, determination, duty, economy, emergency, environmental, face, flood, force, forest, future, government, heart, human, impact, inequality, justice, land, leader, life, loss, mean, nature, pandemic, path, people, planet, plantation, policy, power, resilience, resource, respect, right, science, security, shock, society, soil, stand, storm, struggle, support, urgency, weather, woman, world, youth
\end{textsample}
