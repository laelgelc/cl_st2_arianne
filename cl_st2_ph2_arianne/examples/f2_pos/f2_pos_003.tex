\begin{textsample}{POS Dim 2 – human – Score 62.00 – t039\_human.txt}  \label{ex:f2_pos_003}
World ’s leading \textbf{climate} scientists have just released their \textbf{assessment} of the \textbf{climate} \textbf{emergency} and ways to deal with it.
It ’s deeply unfair : Those least responsible are hit the hardest Worse is to come : We ’re on track to very high \textbf{risks} and irreversible \textbf{losses} 1.5°C is still within reach : With urgent action, the Paris long-term goal can still be met We have the solutions : We can halve global emissions by 2030, on the way to net zero Fossil \textbf{fuel} exit is needed fast : The fossil infrastructure we already have is too much Real solutions, no \textbf{delays}.
Solutions must deliver in real \textbf{life}, not only in models Equity and social inclusion are fundamental, \textbf{finance} \textbf{gaps} must be closed From incremental to transformative.
It ’s all sectors and all hands on deck, NOW!
Want to learn more?
Keep reading.
Below you will find these 10 key takeaways explained in more detail.
Human-caused \textbf{climate} change is now affecting \textbf{weather} and \textbf{climate} extremes in every region across the globe. \textbf{Impacts} and related \textbf{losses} and damages to \textbf{nature} and \textbf{people} have been widespread.
It ’s a crucial report delivered at a crucial \textbf{moment} in time, when \textbf{governments} are taking stock of their action under the Paris Agreement.
And in short, the verdict of the scientists is this : Effects on \textbf{ecosystems} have been experienced earlier, are more widespread and with further-reaching consequences than anticipated.
Half of all species are already on the move, due to \textbf{climate} change affecting their environments.
Evidence of observed changes in extremes such as \textbf{heatwaves}, heavy rain, \textbf{droughts} and tropical cyclones, and, in particular, their attribution to \textbf{human} influence, has only strengthened.
Global aggregated \textbf{impact} and \textbf{risk} levels are now assessed to become high to very high at lower levels of warming than previously assessed ( AR5 ).
Currently, we are at around 1.1°C of average global warming, heading towards about 3°C.
Unique and threatened \textbf{ecosystems} are expected to be at high \textbf{risk} already in the very near term at 1.2°C, due to mass tree mortality, coral reef bleaching, large declines in sea-ice dependent species, and mass mortality events from \textbf{heatwaves}.
At just 1.5°C, up to 14 % of species assessed in terrestrial \textbf{ecosystems} will likely \textbf{face} a very high \textbf{risk} of extinction.
Reaching 1.5°C will bring more and worse heat extremes and dangerous heat-humidity conditions, extreme rainfall and associated flooding, tropical cyclones, wildfires and extreme sea-level events.
Between 1.5°C–2.5°C, \textbf{risks} associated with large-scale singular events or tipping points, such as ice sheet instability or \textbf{ecosystem} \textbf{loss} from tropical \textbf{forests}, transition to high \textbf{risk}.
At about 1.9°C warming, half of the \textbf{human} population could be exposed to periods of life-threatening climatic conditions arising from the coupled \textbf{impacts} of extreme heat and humidity by 2100.
At sustained warming levels between 2°C and 3°C, the Greenland and West Antarctic ice sheets will be lost almost completely and irreversibly.
Vulnerable \textbf{communities} who have historically contributed least to the \textbf{climate} \textbf{crisis} are most affected.
Nearly half of the \textbf{world} ’s population ( 3.3–3.6 billion \textbf{people} ) live in \textbf{contexts} that are highly vulnerable to \textbf{climate} change.
Between 2010 and 2020, \textbf{human} mortality from \textbf{floods}, \textbf{droughts} and \textbf{storms} was 15 times higher in highly vulnerable regions, compared to regions with very low \textbf{vulnerability}. “ There is a rapidly closing window of opportunity to secure a liveable and sustainable \textbf{future} for all ( very high confidence ).
At the same time, only 10 % of the wealthiest households contributed up to 45 % of global consumption-based household GHG emissions.
With \textbf{policies} implemented by the end of 2020, we are on track to about 3.2°C warming by 2100. ( Estimates that cover more recent \textbf{policies} ( NDCs announced prior to COP26 until 2030, with no increase in ambition ), result in a slightly better \textbf{assessment} of 2.8°C median warming. ) Instead of halving global emissions by 2030, which would be required for \textbf{respecting} the Paris Agreement warming \textbf{limit}, there would be no decline in global emissions by 2030.
With continued emissions, we ’re on track to reach around 1.5°C in the near term.
But it is still in our hands to halt warming there to avoid the most severe \textbf{impacts}.
By following the strongest IPCC emission reduction \textbf{pathways} ( C1 ), warming would peak between 1.4°C and 1.6°C and by the end of the century be below 1.5°C. ( See Table 3.1 ) So with urgent action, the Paris Agreement long-term temperature goal is still within reach.
This would require roughly halving global emissions by 2030, followed by net zero CO2 by around 2050, and achieving and sustaining net negative CO2 ( and GHG ) emissions globally thereafter, with annual rates of carbon dioxide removal ( CDR ) being greater than remaining CO2 emissions. \textbf{Limiting} as much as possible any overshoot of 1.5°C, for the shortest duration possible is essential, as the gradual cooling would not undo the irreversible \textbf{impacts} triggered by the peak warming ( such as species \textbf{loss} or ice sheet melt ).
Furthermore, while some carbon dioxide removal is necessary by now, it comes with many \textbf{uncertainties}, so the reliance on it should be \textbf{limited}.
As the IPCC concluded in its earlier AR6 reports : “ CDR deployed at scale is unproven, and reliance on such technology is a major \textbf{risk} in the ability to \textbf{limit} warming to 1.5°C.
CDR is needed less in \textbf{pathways} with particularly strong emphasis on energy efficiency and low \textbf{demand}. ” ( IPCC SR15 ) “ (…)prioritising early decarbonisation with minimal reliance on CDR decreases the \textbf{risk} of mitigation failure and increases intergenerational equity. ” ( IPCC SRCCL ) We have all the \textbf{tools} we need for at least halving global emissions by 2030.
Half of this mitigation potential is estimated to be low-cost ( less than 20 USD/tCO2-eq ), or even to come with cost savings.
The biggest contributions would come from solar and wind energy, \textbf{protection} and restoration of \textbf{forests} and other \textbf{ecosystems}, climate-friendly food \textbf{systems}, and energy efficiency in its many forms. “ Rapid and far-reaching transitions across all sectors and \textbf{systems} are necessary ” By 2050, demand-side measures can reduce global GHG emissions by 40–70 % compared to baseline scenarios.
These refer to societal \textbf{choices} about how we use technology and \textbf{resources} to meet our needs for food, shelter, mobility and products.
One of the measures with the greatest potential, and synergies with adaptation and \textbf{biodiversity} conservation and \textbf{human} health, is the shift towards low-meat diets, termed “ balanced sustainable healthy diets ” by the IPCC reports.
Overall, providing better services with less energy and \textbf{resource} input is consistent with providing well-being for all.
The fossil \textbf{fuel} infrastructure already in place is enough to exceed the 1.5°C warming \textbf{limit}, if allowed to be in use without further restrictions.
So there is no room for any new fossil \textbf{fuel} infrastructure, and what exists needs to be phased out early, as is illustrated by the graph below.
In other words, keep it in the \textbf{ground} : “ About 80 % of \textbf{coal}, 50 % of gas, and 30 % of oil reserves cannot be burned and emitted if warming is \textbf{limited} to 2°C.
Significantly more reserves are expected to remain unburned if warming is \textbf{limited} to 1.5°C. ” ( SYR longer report ) So there needs to be a major shift away from fossil \textbf{fuels}.
But just how fast?
It depends on many assumptions, and in the underlying mitigation report ( AR6 WG3 ) the IPCC provides more detail.
In \textbf{pathways} that \textbf{limit} warming to 1.5°C with greater than 50 % likelihood and no or limited overshoot, the global use of \textbf{coal} is projected to decline by up to 100 %, oil by up to 90 % and gas by up to 85 % by 2050, with median values being lower. ( See WG3, SPM C.3.2 ) The fastest reductions are required in \textbf{pathways} that aim at 1.5°C with little to no overshoot, low reliance on carbon dioxide removal, low pressure on \textbf{land} and \textbf{biodiversity} and high efficiency in \textbf{resource} use.
Such \textbf{pathways} are illustrated by the IMP-LD \textbf{pathway}, where the overall use of fossil \textbf{fuels} declines by about 85 % by 2050, from 2020 levels ( See WG3, Figure 3.6 and SPM C.3.6 ).
We have entered a critical decade, during which we must nearly halve global emissions while delivering on food \textbf{security} and protecting and restoring \textbf{nature} too.
Among the big game-changers since the previous \textbf{assessment} is the breakthrough of solar and wind, that are now reaching cost levels equal to or below those of fossil \textbf{fuels}, being ready to enable the decarbonisation of different sectors through electrification.
These developments have occurred much faster than anticipated by experts and modelled in previous mitigation scenarios.
It ’s a game-changer.
Carbon capture and storage ( CCS ), then again, has not made significant progress.
It plays a big role in many emission reduction models, but still fails to deliver at scale in real \textbf{life}.
As the IPCC mitigation report summed up : “ Deployment and development of CCS technologies ( with large-scale storage of captured CO2 ) have been much slower than projected in previous \textbf{assessments} ”. “ Implementation of CCS currently \textbf{faces} technological, economic, institutional, ecological-environmental and socio-cultural \textbf{barriers}. ” Technological carbon dioxide removal, where CO2 is captured directly from the atmosphere ( DACCS ), or from biomass energy ( BECCS ), also play a role in most mitigation models, but remain unproven at scale, and they come with many feasibility and sustainability constraints – as does large-scale afforestation. “ The \textbf{choices} and actions implemented in this decade will have \textbf{impacts} now and for thousands of years ( high confidence ). ” So, as we navigate into the \textbf{future}, where carbon dioxide removal will be needed – though substantially less so with urgent, near-term action – it is essential to look beyond simplified models.
Prioritising CDR solutions that maximise sustainability benefits and minimise \textbf{risks}, \textbf{means} prioritising options that work with \textbf{nature} and for \textbf{communities}, like reforestation, restoration of \textbf{ecosystems} or \textbf{soil} carbon sequestration in agriculture.
This is critical in order to avoid \textbf{conflicts} with other \textbf{land} use needs or creating large \textbf{environmental} \textbf{impacts} and \textbf{conflicts} with \textbf{human} \textbf{rights} and food \textbf{security}.
The scale and speed of \textbf{transformation} needed will not be possible without equity and social \textbf{justice}, both between and within countries.
According to the IPCC, integrating \textbf{climate} action with macroeconomic \textbf{policies} can \textbf{support} sustainable low-emission development \textbf{paths}, safety nets and social \textbf{protection}, and improve access to \textbf{finance} for low-emissions infrastructure access, especially in developing regions.
At the \textbf{heart} of equity considerations is \textbf{finance}.
There ’s enough money in the \textbf{world} to enact real change, if existing \textbf{barriers} are removed.
But today, public and private \textbf{finance} flows for fossil \textbf{fuels} are still greater than those for \textbf{climate} adaptation and mitigation. ( Indeed, while the IPCC points to the growing \textbf{finance} and adaptation \textbf{gaps} vulnerable \textbf{communities} are \textbf{struggling} with, the IEA reports that last year alone, the oil and gas industries earned a whopping 4 trillion USD with businesses \textbf{fueling} the \textbf{climate} \textbf{crisis}! ).
The annual investment requirements before 2030 are a factor of three to six greater than current levels, for mitigation alone, with the largest needs being in the developing \textbf{world}.
To change this, both \textbf{governments} and financial \textbf{institutions} will need to align their goals and \textbf{policies} with 1.5°C, and remove \textbf{barriers}.
On an international level, equitable solutions need to be found that meet the adaptation and mitigation needs and address \textbf{loss} and damage for those least responsible.
There ’s a rapidly narrowing window of opportunity to implement \textbf{climate} resilient development.
To achieve the Paris Agreement and other sustainability goals, we need to think beyond individual technologies, sectors and \textbf{actors} and adopt holistic, inclusive and transformative approaches that encompass both mitigation and adaptation.
Action that protects and restores our \textbf{biodiversity} is fundamental.
By taking \textbf{care} of \textbf{nature}, we are taking \textbf{care} of ourselves.
According to the IPCC, maintaining the \textbf{resilience} of \textbf{biodiversity} and \textbf{ecosystem} services at a global scale depends on effective and equitable conservation of approximately 30 % to 50 % of Earth ’s \textbf{land}, freshwater and ocean areas, including currently near-natural \textbf{ecosystems}.
To achieve “ rapid and far-reaching transitions across all sectors and \textbf{systems} ” we need strong laws and \textbf{policies} and international co-operation.
It ’s all hands on deck and those with most responsibility and capability need to lead the way.
This goes for \textbf{governments}, businesses, investors and high-income individuals alike.
Scientists have delivered their toolbox for \textbf{survival}.
Now it is our job to make sure the \textbf{science} is \textbf{acted} on, by \textbf{governments}, businesses, investors and citizens.
And that we make it personal.
The \textbf{life}, well-being or \textbf{suffering} of our daughters, granddaughters, great-granddaughters and many more after them will look back at us and feel grateful for what we did, or perhaps what we did not do.
Kaisa Kosonen is a Senior Policy Advisor with Greenpeace Nordic and Reyes Tirado is a Senior Scientist with Greenpeace Research Laboratories at University of Exeter.
Let that sink in for a \textbf{moment}.
The Synthesis of the IPCC Sixth Assessment Report forms a robust analysis of thousands of peer-reviewed research papers published over the past decade, with more details to be found in the underlying six reports.
In summary, our take on it all in 10 key messages is this : It ’s bad : Human-caused \textbf{climate} change is already widespread, rapid and intensifying It ’s worse than expected : \textbf{Impacts} and \textbf{risks} are getting more severe sooner

% matched lemmas: act, actor, assessment, barrier, biodiversity, care, choice, climate, coal, community, conflict, context, crisis, delay, demand, drought, ecosystem, emergency, environmental, face, finance, flood, forest, fuel, future, gap, government, ground, heart, heatwave, human, impact, institution, justice, land, life, limit, loss, mean, moment, nature, path, pathway, people, policy, protection, resilience, resource, respect, right, risk, science, security, soil, storm, struggle, suffering, support, survival, system, tool, transformation, uncertainty, vulnerability, weather, world
\end{textsample}
