\begin{textsample}{POS Dim 2 – human – Score 71.00 – t185\_human.txt}  \label{ex:f2_pos_001}
All around the \textbf{world}, \textbf{women} and girls are making enormous contributions to \textbf{climate} action.
They are vital agents of change for the \textbf{planet}, but their \textbf{voices} are often missing from the decision-making table.
Many disabled folks want to be more environmentally conscious and play their part in taking \textbf{climate} action.
Still, a \textbf{legacy} of non-engagement and eco ableism has created distrust between many disabled \textbf{communities} and \textbf{environmental} and \textbf{climate} organisations.
Building and restoring credibility, trust and \textbf{relationships} will take time.
But if anything will keep me going, it ’s hope, that \textbf{climate} movements *can* make shifts to be intersectionally centring the \textbf{voices} of those most marginalised, including disabled \textbf{people}, and campaigning in \textbf{solidarity} with us for a \textbf{future} of \textbf{care} and flourishing for both \textbf{planet} and \textbf{people}. ” Áine Kelly-Costello ( she/they ) is a proudly multiply disabled \textbf{climate} campaigner advocating in Aotearoa New Zealand \& internationally.
They are passionate about furthering \textbf{climate} \textbf{justice} and creating an accessible, inclusive \textbf{world}, especially by harnessing the perspectives of disabled \textbf{people}.
They are newly exploring identifying as genderqueer \& want to shout out to all the trans \& \textbf{gender} non-conforming folks who bear the brunt of transphobia \& \textbf{gender} inequities.
Explore further : Eco Ableism and the \textbf{Climate} Movement ( Catrina Randall, Young Friends of the Earth Scotland ) Reading list on disability \textbf{justice} and \textbf{climate} \textbf{justice} ( compiled by SustainedAbility – scroll down page for list ) The Missing \textbf{Conversation} about Disabled \textbf{Leadership} in \textbf{Climate} Justice ( Áine Kelly-Costello, Stuff ) “ In my Khadia tribe, my surname ‘ Soreng ’ \textbf{means} ‘ rock ’.
Many indigenous \textbf{communities} have surnames related to \textbf{nature} showing how intertwined they are with the natural \textbf{world}.
My grandfather was a pioneer of a community-led \textbf{forest} \textbf{protection} in our village in Odisha ’s Sundergarh district.
He was a staunch believer that we should have a sustainable \textbf{relationship} with \textbf{nature}, only then can there be sustainable living.
My father was an advocate of indigenous healthcare.
It was my parents who said to me that : “ If you really want to contribute back to \textbf{society}, you need to enter into \textbf{policy} making ”.
Through my studies in humanities and \textbf{environmental} \textbf{science}, I realized that it is important for indigenous \textbf{communities} to write and speak for ourselves, and to share our perspective and worldview through our own lens.
It is important to raise our \textbf{voice}, because our \textbf{voice} matters.
In the month of March, Greenpeace had highlighted the \textbf{voices} of diverse \textbf{women} who are leading on \textbf{climate}.
These are just 10 \textbf{women} from the Asia Pacific region who are raising their \textbf{voices} to help shape the \textbf{climate} \textbf{conversation} and to heal our \textbf{planet}.
My father ’s death in 2017 triggered in me an urgent need to learn traditional practices and their contribution towards \textbf{climate} action and \textbf{biodiversity} conservation, and to document them so that upcoming \textbf{generations} will know about these practices.
Through the years I have interacted with numerous tribal \textbf{communities} in the state.
Every tribe is unique but all their \textbf{cultures} are sustainable and respectfully intertwined with \textbf{nature}. \textbf{Climate} change affects all of us but not equally.
Tribal \textbf{communities} are the ones who have been in the \textbf{frontlines} of protecting the \textbf{forest} and \textbf{nature} through their way of living, traditional \textbf{knowledge} and practices, even at the cost of their \textbf{lives}, yet they are the ones who are most adversely affected due to the \textbf{impact} of the \textbf{Climate} \textbf{Crisis}.
We need to make tribal \textbf{communities} an integral part of the \textbf{decision} making process of \textbf{Climate} Action. ” Archana Soreng, a member of the Khadia tribe in Odisha, India, and a member of UN Secretary General ’s Youth Advisory Group on \textbf{Climate} Change “ I am Eunbin Kang, living in Seoul, South Korea.
I was interested in the Korean War and division issues, and I studied Political \textbf{Science} and Diplomacy at university.
It ’s been my all-time routine to be vigilant about waste management and meat-eating issues.
In September 2019, I took part in the \textbf{Climate} \textbf{Crisis} Emergency Action Parade in Seoul.
The slogan “ \textbf{Climate} \textbf{Crisis}, Not Global Warming ” touched me.
The word ‘ \textbf{climate} \textbf{crisis} ’ was a wake up call that the \textbf{reality} in front of us is a \textbf{crisis} that ordinary \textbf{people} must work through together, not just experts or \textbf{people} in \textbf{power} In 2020, I started the \textbf{climate} movement in Youth \textbf{Climate} Emergency Action.
In 2021, we took direct action against Doosan Heavy Industries \& Construction, a representative Korean company that is building \textbf{coal} \textbf{power} plants in various parts of Asia.
Currently, we are undergoing trials including a claim for damages of 18.4 million won from Doosan Heavy Industries \& Construction.
Rather than despair in the \textbf{face} of the \textbf{climate} \textbf{crisis}, we have chosen to \textbf{stand} up and resist those in \textbf{power}.
I remember the saying that our very existence itself gives someone hope and stimulation.
I want to continue the \textbf{climate} movement, remembering that hope is within us and that love overcomes everything. ” Eunbin Kang, co-founder of the Youth \textbf{Climate} Emergency Action group in South Korea “ Growing up, I have always fancied doing public service, inspired by the biographies of our national heroes who fought for our country ’s independence.
I became the first Muslim student President in a very conservative and Catholic-denominated all-girls school.
Being \textbf{woman} and relatively young were \textbf{challenging}, given our patriarchal and tribal \textbf{society}.
I am a Maranao — a Muslim tribe from a war-torn province, south of Philippines — carving a \textbf{space} in the “ imperial ” Manila.
In order to be heard and earn “ legitimacy, ” I tread the \textbf{path} of lawyering to earn a \textbf{voice} not only for myself, but for the \textbf{people} I seek to serve and represent—women, children, elderly, indigenous \textbf{peoples}, and other vulnerable sectors.
After almost 5 years of honing my skills as a general litigation lawyer in a firm mostly serving corporate clients, I embarked on a journey to realize my dream to serve.
Destiny led me to Greenpeace to work with 32 audacious petitioners who sued 47 big \textbf{coal}, oil, gas, and cement companies, a landmark national investigation against multinational \textbf{corporations} that dealt with the cross-cutting issues of \textbf{climate} change and \textbf{human} \textbf{rights}.
Five years working with different \textbf{communities} and vulnerable groups heightened my passion to even look for opportunities to mainstream \textbf{climate} \textbf{justice} with the hope that every Filipino will take the matter seriously and with the same level of \textbf{urgency} as the current \textbf{pandemic}.
We cannot just look the other way when it comes to the \textbf{realities} on the \textbf{ground}.
We must \textbf{demand} \textbf{accountability} and resist \textbf{injustice}. “ I ’m a Nakhi, native to the foothills around Lijiang, where I was born and raised.
Lijiang sits among the Himalaya and Hengduan mountain ranges in a region that houses the most \textbf{biodiversity} of anywhere in China.
It ’s a haven of alpine plants, scattered with wild orchids, rhododendron, and innumerable precious plant \textbf{life}.
They are remarkably sensitive to the \textbf{impact} of \textbf{climate} change.
As the snow line moves higher and higher up the mountains and glaciers melt, we may see the extinction of more and more species, including those that are still undiscovered or understudied.
There are many species we still don't understand but nonetheless do know that they are on the \textbf{brink} of extinction.
While awaiting for the impending release of the final report on our legal action, I hope we have opened up the \textbf{space} and built \textbf{power} for \textbf{communities} to reclaim their \textbf{rights} and take action to achieve a safe \textbf{climate} and a healthy environment.
Indeed, there is no such thing as David vs.
Goliath if you are fighting for a good cause.
And there is no better cause than the cause for the environment and \textbf{human} \textbf{rights}. ” Hasminah D.
Paudac-Tawano, Legal Advisor, \textbf{Climate} Justice and Liability Program at Greenpeace Southeast Asia ( Philippines ) “ I joined the Greenpeace \textbf{Climate} and Energy Project in 2017, after receiving my PhD from the Chinese Academy of Social Sciences.
As the Program Manager of \textbf{Climate} \& Energy ( GPEA ), I lead the Beijing Office \textbf{Climate} Risk Project and Research Unit where I am committed to making the public and stakeholders aware of \textbf{climate} change and to take active actions.
In 2018, I visited the glaciers in the western plateau of China together with scientists.
We observed the catastrophic \textbf{impact} of glacial flooding under the influence of \textbf{climate} change.
In 2021, my colleagues and I participated in a post-disaster rescue after the Henan \textbf{floods} due to heavy rains.
From what we have witnessed, the \textbf{climate} change \textbf{crisis} is looming.
At the same time, we are also noticing that \textbf{climate} change is widening the \textbf{inequality} \textbf{gap}, exposing vulnerable populations to more severe \textbf{risks}. \textbf{Climate} change is such a grand and complex issue, engulfing a series of \textbf{crises} in \textbf{politics}, \textbf{economy}, health, development, and fairness. \textbf{Climate} \textbf{crisis} is like an abyss.
It is incalculable and may even swallow our \textbf{courage} to \textbf{act}. ” “ The only way to escape the abyss is to look at it, gauge it, sound it out and descend into it. ” — Cesare Pavese Liu Junyan, Programme Manager of \textbf{Climate} \& Energy at Greenpeace East Asia “ Pua Lay Peng is a local activist leading a \textbf{grassroots} \textbf{environmental} group called Persatuan Tindakan Alam Sekitar Kuala Langat ( Kuala Langat Environmental Action Group ) in Malaysia.
The group was active in campaigning against the imported plastic waste problem that was affecting their small town, Jenjarom.
In 2018, the global plastic waste trade was disrupted when China banned most plastic waste imports.
Southeast Asian countries like Malaysia had picked up the slack, accepting the outpouring of plastic waste from high-income countries that led to a spike in the number of illegal dumpsites and burning facilities in the country.
I was a journalist in traditional media for more than ten years.
Now, I use social media to record and exhibit beautiful but fragile plant \textbf{life} to get more \textbf{people} to pay attention to alpine plants, as well as \textbf{climate} change and its \textbf{impact} on \textbf{biodiversity}.
We quickly find joy in the gifts of \textbf{nature}, but can also easily ignore the step-by-step progression of this \textbf{crisis}.
A chemist, Pua had moved back to Jenjarom where she witnessed the devastating \textbf{impact} that the plastic recycling industry was having on her \textbf{community}.
With over 40 illegal plastic factories emitting toxic gases into the air and polluting the local rivers and waterways, they were making \textbf{people} very sick.
Alarmed by the \textbf{environmental} and health \textbf{impacts} of the illegal industries, Pua took action.
She led the group as they organised meetings, conducted fieldwork and published reports exposing the plastic problem.
They also worked with the authorities to \textbf{act} against the illegal plastic waste trade, leading to the closure of several hundred illegal facilities.
Despite receiving death threats, Pua and the group continued with their courageous and relentless efforts, empowering other \textbf{communities} to \textbf{stand} together and fight against \textbf{environmental} pollution. ” “ My name is Sylvia Wu.
As a legal professional, I aspire to use my legal \textbf{knowledge} to bring about positive changes.
In February of 2021, we filed the first \textbf{climate} litigation in Taiwan, Greenpeace East Asia and others v.
Ministry of Economic Affairs..
I worked closely with the individual plaintiffs, local NGOs, and our project team to urge the \textbf{government} to be more ambitious in \textbf{climate} change law amendments and to \textbf{challenge} the rigid judicial \textbf{system}. \textbf{Climate} \textbf{justice} should not be a privilege.
It ’s a fundamental \textbf{human} \textbf{right}.
That is also our ultimate goal in the case–to create a \textbf{path} for ordinary \textbf{people} to access \textbf{climate} \textbf{justice}.
One of the biggest \textbf{challenges} we \textbf{faced} was to overcome the procedural \textbf{barriers} in a highly traditional \textbf{court} \textbf{system} like Taiwan ’s.
Moreover, most Taiwanese have never heard of \textbf{climate} litigation or \textbf{climate} \textbf{justice}.
Therefore, we endeavored to amplify the influence of our case and to ensure that our plaintiffs ’ \textbf{voices} got heard while the case was still pending.
We collaborated with local \textbf{environmental} law groups to initiate a petition among lawyers advocating for adding citizen suit provision, hold \textbf{forums} allowing more \textbf{people} to understand \textbf{climate} \textbf{justice}, and publish articles.
In addition to \textbf{climate} litigation, I devoted myself to \textbf{supporting} our Distant Water Fishery project.
Due to the discriminated-based two-tier employment \textbf{system}, many distant water fishery workers experienced \textbf{forced} labor or \textbf{human} trafficking in the dark ocean.
We assessed and developed legal and political advocacies to \textbf{safeguard} their \textbf{rights} and to help them break out from the chains of an unfair \textbf{system}.
The \textbf{people} I have been fighting for are the highlight of my story.
What makes me most proud of my work is that I get to \textbf{stand} with them, amplify their \textbf{voices}, and fight together against the flaws in the \textbf{system}. ” Sylvia Wu, Legal coordinator at Greenpeace Taiwan “ My activism is sustained by what I ’ve learnt from my experiences while working with \textbf{communities}.
Movements are created from the \textbf{ground} up and building a movement is building \textbf{people} \textbf{power} in the process.
It is built on values of shared responsibilities as much as a shared \textbf{leadership}.
As a \textbf{community} organizer, these lessons have become my \textbf{core} values too.
What matters most is \textbf{supporting} and enabling each other to \textbf{act} together so as to achieve our shared vision.
Taking action on \textbf{climate} change \textbf{means} each person gets involved.
So far, we haven't yet done enough. ” I hope that others would aspire for a better \textbf{life}, not just for \textbf{survival}.
That they realize that they have the \textbf{power} to achieve this through persistent collective activism.
We must be at the forefront of the \textbf{struggle} and confront the \textbf{system} that makes the \textbf{people} poor and miserable, thus perpetuating \textbf{injustices}.
Through our activism we have an opportunity to desconstruct this \textbf{system} that prioritizes \textbf{profit} over the \textbf{right} and welfare of the \textbf{people} and \textbf{planet}.
I hope to show that aspiring for a better \textbf{life} and \textbf{society} is nothing short of \textbf{demanding} for a systemic shift.
While we have managed to achieve small victories in the past, the better \textbf{society} that we aspire to is still a long way away. \textbf{Challenges} have been non-stop but the biggest one at this time is the \textbf{context} of the shrinking democratic \textbf{space} and the growing impunity in many countries.
Places like in the Philippines where activism is being criminalized.
Even so, as I immersed myself in \textbf{community} work, my commitment gets deeper as an activist.
Despite the difficult personal and work-related \textbf{challenges} along the way, my motto is to go back to the start and remember the reasons why I chose this \textbf{life}. ” Veronica Cabe is a Coordinator of the Nuclear and Coal-Free Bataan Movement in the Philippines, a community-based network of organizations and individuals that campaigns for the \textbf{protection} of \textbf{communities} against the perils of nuclear and fossil-fuel energies. “ My name is Yaewon Hwang, and I am an Equity, \textbf{Diversity} and Inclusion Partner for Greenpeace East Asia.
I believe that many transgender \textbf{women} are born activists as we ’ve had to fight so hard to fit into a Cisgender-centric \textbf{society} from a young age.
We are taught from the \textbf{moment} we were born that everything we are about and that everything we want to be is ‘ wrong ’.
We ’ve been made to feel very lonely and isolated from \textbf{society}.
What I have gone through has made me broaden my horizons and taught me what the \textbf{core} of my activism should be, and that is love.
I will overcome hate with love.
I have, and always will be, vocal about what I believe in and what is \textbf{right}, especially when it comes to trans \textbf{justice}.
I believe that \textbf{conversations} about transgender movement have only just started in the last few years.
We are nowhere near where it needs to be. \textbf{People} tend to fear what they do not know and transphobia comes from ignorance and not being educated on this topic.
Trans \textbf{people} are your neighbors, colleagues, friends, and family.
We are no different from you.
I believe that we can overcome \textbf{people} ’s fears with exposure and I am very glad to see so many incredible trans \textbf{women} on many different platforms these days.
It really helps cis \textbf{people} to understand that we live and love just like them.
Achieving trans \textbf{justice} starts from our daily \textbf{lives}.
If you are a cis person, please be our \textbf{ally} because the trans \textbf{community} needs more \textbf{allies} who will \textbf{stand} up for us and who will speak out for what is \textbf{right}.
For example, if one of your family members or friends are making discriminatory comments or jokes about the trans \textbf{community}, please correct them.
Let them know that their discriminatory comments are not acceptable, and that we all need to \textbf{respect} each other in order for us to move forward for a better \textbf{future}. ” Yaewon Hwang, Equity, \textbf{Diversity} and Inclusion Partner for Greenpeace East Asia Ai Ji is a Nakhi conservationist using documentation and public awareness activism to highlight the fragile beauty of her home, Lijiang, as well as \textbf{climate} change and its \textbf{impact} on \textbf{biodiversity}. “ Politicians, journalists and the \textbf{climate} movement still barely ever talk about the disproportionate ways in which \textbf{climate} \textbf{breakdown} and responses to it are leaving disabled \textbf{people} behind.
Multiple layers of marginalisation, including indigenous status and \textbf{gender}, compound the harms many disabled \textbf{communities} \textbf{face}.
Often if we do get mentioned, disabled \textbf{people} are relegated to a list of “ vulnerable groups ”, taking away our agency.
I ’m determined to help rewrite that story.
Disabled \textbf{people} are change agents well-adapted to a \textbf{world} not created with us in mind.
We have lived experience and expertise which is integral to shaping an accessible, inclusive, safe and climate-resilient \textbf{future}.
I ’d love for \textbf{environmental} and \textbf{climate} campaigners to learn about how eco ableism can show up, and figure out how their organisations can start to dismantle it.
That includes looking internally at organisational \textbf{systems}, processes and practices.
Engaging with disabled members and consultants to develop and embed accessibility and inclusion guidance and training is crucial.

% matched lemmas: accountability, act, ally, barrier, biodiversity, breakdown, brink, care, challenge, climate, coal, community, context, conversation, core, corporation, courage, court, crisis, culture, decision, demand, diversity, economy, environmental, face, flood, force, forest, forum, frontline, future, gap, gender, generation, government, grassroots, ground, human, impact, inequality, injustice, justice, knowledge, leadership, legacy, life, mean, moment, nature, pandemic, path, people, planet, policy, politics, power, profit, protection, reality, relationship, respect, right, risk, safeguard, science, society, solidarity, space, stand, struggle, support, survival, system, urgency, voice, woman, world
\end{textsample}
