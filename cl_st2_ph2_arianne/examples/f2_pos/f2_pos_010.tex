\begin{textsample}{POS Dim 2 – human – Score 50.00 – t296\_human.txt}  \label{ex:f2_pos_010}
Dear Western Indian Ocean island \textbf{leaders} and other \textbf{governments} with \textbf{interests} in our waters, Ban the use of fish aggregating devices in our waters and apply a polluter pays principle so that industrial fisheries not only stop destroying the marine \textbf{ecosystems} they depend on but begin to pay their fair share in restoring and protecting them ; Conduct national and island specific carrying capacities before allowing the development of more hotels.
Already our \textbf{economies} are overly relient on this fragile and impactful industry.
If we continue to build in ignorance we will lose the natural beauty and tranquility visitors are willing to pay for ; Keep fossil \textbf{fuels} in the \textbf{ground}.
Given our \textbf{vulnerability} to \textbf{climate} change and recent experience with an oil spill we must adopt a moratorium that prevents fossil \textbf{fuel} exploration and \textbf{extraction} in our waters.
Our homes cannot \textbf{support} an industry that promises to drown them and through international \textbf{cooperation} I am certain we can become global renewable energy examples ; Say no to deep seabed mining. \textbf{Push} the International Seabed Authority to have your citizen ’s best \textbf{interests} prioritised.
As it \textbf{stands} they are not.
Apply the precautionary principle and join moratoriums that \textbf{demand} that interested parties prove that deep seabed mining ’s \textbf{impact} will not be catastrophic to not only the incredible \textbf{biodiversity} we have only begun to discover, but also global fish stocks, \textbf{future} pharmaceutical drugs and \textbf{climate} processes our \textbf{lives} depend on ; Consolidate our existing marine protected areas and create new ones in the JMA.
The Saya de Malha Bank ’s incredible \textbf{biodiversity} and enormous blue carbon value within the JMA provides an excellent opportunity for Mauritius and Seychelles to again be \textbf{leaders} and protect their shared \textbf{futures}, further delivering on sustainable development goal 14 for 30 % of ocean being protected by 2030.
Submit the area to be a marine UNESCO World Heritage Site ; Join and \textbf{support} the movement for a strong Global Ocean Treaty to protect the \textbf{biodiversity} that is beyond our national jurisdictions, but intertwined with our islands and \textbf{planet} ’s \textbf{future}.
My final words : if you truly hope to leave us a \textbf{world} better than what you found, you will do the hard things such as the above.
If you won't or cannot, you will be replaced.
Jeremy Raguain is a conservationist from Seychelles Guest authors work with Greenpeace International to share their personal experiences and perspectives and are responsible for their own content.
My name is Jeremy Raguain, a 27-year-old who is beyond concerned with the state of affairs and trajectory of his home, Seychelles, and surrounding ocean.
I write this letter as a Seychellois and dependent of the Indian Ocean to sound an alarm that has already been rung, but isn't being paid the attention it deserves.
I ’ve put pen to paper to plea with you as a young conservationist who has been granted the most exceptional chances to not only visit but help protect the marine UNESCO World Heritage Site of Aldabra Atoll ; to state that unsustainable fishing, over-tourism, the exploration and \textbf{exploitation} of oil and natural gas, and consideration of deep seabed mining are becoming the bedrock of a false blue \textbf{economy} that threatens the places that are supposed to be protected, as well as our homes.
I am contacting you today as a Global Shaper, a Sustainable Ocean Alliance Young Ocean Leader and Youth Policy Advisory Council member and 2019 UN \textbf{Climate} Action Summit attendee who has regrettably become more weary of our politicians.
I have seen \textbf{leaders} fail to turn up and counted too many missed chances to enshrine pretty commitments into law and \textbf{reality}, while watching others deliberately mislead their electorates by double counting empty \textbf{climate} promises.
I write to you as a young African that sees the Covid-19 virus – borne from \textbf{nature} ’s mistreatment and set our global \textbf{society} back decades – be a horrible opportunity to do what is necessary.
The \textbf{science} states we are in a \textbf{climate} and ecological \textbf{crisis}.
The facts indicate that this unfolding catastrophe will not only dwarf the \textbf{pandemic} ’s cataclysmic consequences, but doom your citizens to a \textbf{future} that promises to be hell on Earth, whereby we lose our homes and \textbf{livelihoods}.
We know that unless clear \textbf{decisions} are made in your terms, the \textbf{lives} and aspirations of your children and grandchildren will be irreversibly hurt.
We already see \textbf{climate} change ’s effects in the fires, \textbf{floods}, \textbf{droughts} and other extreme \textbf{weather} events that have caused famine, \textbf{displacement} and \textbf{conflict} in the countries that surround the Indian Ocean.
We have the information to show that illegal and industrial ocean \textbf{extraction} in our Economic Exclusive Zones and on the high seas not only threatens already endangered \textbf{biodiversity} with pollution while depleting crucial fish stocks, but is done so in an inequitable and neo-colonial way by Global North states and international \textbf{corporations} that behave with impunity, flagrantly disregarding our laws and even their own jurisdictions.
Each of your states ’ constitutions and the international treaties ratified by your \textbf{governments} obligate you to protect your citizens and their \textbf{rights}.
While our Global South island states are among the least responsible and most vulnerable to \textbf{climate} change ’s effects, they make a \textbf{biodiversity} hotspot which attracts tourists and foreign direct investment as well as provide other incalculable \textbf{ecosystem} services.
As sworn protectors of your citizens and their \textbf{right} to a safe and healthy environment, I have to say your destructive actions as well as apathy deeply worry me.
While the \textbf{priorities} and lack of \textbf{support} from our international \textbf{allies} in the region, some of the \textbf{world} ’s most responsible and capable states, further depresses me.
Although our past shows we are capable of preserving ourselves and natural heritage, nothing about our present and \textbf{future} is certain and each year that passes without meaningful action further seals our doom.
I have seen our islands receive increasing numbers of tourists and continue to host more hotels, currently empty due to the \textbf{pandemic}.
Citizens watch as their \textbf{governments} side step \textbf{environmental} \textbf{impact} \textbf{assessments} and do not take steps to know their islands or \textbf{planet} ’s carrying capacity.
I have collected industrial fisheries pollution from one of the \textbf{world} ’s most remote and protected areas, followed the threads of ghost gear all the way to the can of worms that is the unsustainable European Union funded fishing.
I have sat down with scientists, \textbf{policy} experts and young \textbf{people} in Seychelles and around the \textbf{world} agreeing how the exploration and \textbf{exploitation} of natural gas and oil by Small Island Developing States is like drilling a hole in a boat while at sea in a \textbf{storm}.
I am speaking out against the opaque and unethical way in which the International Seabed Authority is operating and I ’m also raising awareness on the \textbf{impacts} of deep seabed mining, an activity certain states in the region are already preparing themselves to engage with and seemingly \textbf{support}.
All of this leads me to see that our Indian Ocean is the least explored, but already the second most polluted and increasingly ravaged.
The facts show that a resilient and healthy ocean protects and nourishes its \textbf{people}.
What you are allowing to take place reduces its health and \textbf{resilience} and, as such, ours.
Global issues such as \textbf{conflict}, economic \textbf{crisis}, \textbf{pandemics} and \textbf{climate} change will be felt harder if we allow the ocean to suffer and die.
We also lose our chance to be richer when we take \textbf{decisions} that only benefit the short term and few.
But, there are opportunities for hope.
Our Indian Ocean is also home to a marine UNESCO World Heritage Site that has seen over 50 years of research and \textbf{protection}, with it and many other marine protected areas showing how they \textbf{support} our \textbf{people}.
With the recent documentation of the \textbf{world} ’s largest seagrass meadow on Saya de Malha Bank, the implementation of the Seychelles Marine Spatial plan and Mauritius-Seychelles Joint Management Area ( JMA ), there are clear synergies and \textbf{paths} towards actions that can contribute to a meaningful blue \textbf{economy} that is sustainable and equitable.
If your \textbf{Governments} fulfill their Nationally Determined Contributions, renounce \textbf{interest} in the \textbf{extraction} of non-renewable ocean \textbf{resources} and join \textbf{forces} to take on unjust and destructive industrial fishing fleets we can \textbf{stand} a chance.
Working together within the Indian Ocean Commission and \textbf{supporting} a Global Ocean Treaty will allow the \textbf{biodiversity} that moves across our jurisdictions to continue sustaining our shared aspirations.
Ultimately, our fate is uncertain, and as I am sure you may remind me, there is only so much we small island states can do.
But ocean \textbf{protection} is \textbf{climate} action and we need such action now.
Moreover, a blue \textbf{economy} is a lie if it allows companies and \textbf{governments} new \textbf{grounds} to extract with impunity.
We must recognise ocean conservation within and outside of our borders as vital to economic development in not only allowing debt restructuring, but also the replenishment of our natural assets.
Make no mistake, the \textbf{world} and your \textbf{people}, and especially us young \textbf{people}, are watching, in increasing numbers and with increasinging anger.
Where there were a few there are more each day.
If you don't rise to be the champions we need, if you don't apply the precautionary principle, if you don't make polluters pay or if you keep accepting the pittance we are offered while our \textbf{future} slips away, we will find other \textbf{leaders} or become those we need.
I promise to fight for what ’s \textbf{right}, for what \textbf{science} tells us, what local \textbf{communities} need and those that have no \textbf{voice}, \textbf{future} \textbf{generations} and wildlife.
I am not here to oppose you, but rather to work with you against \textbf{climate} and ecological collapse.
To restore our faith in you, to secure prosperity for today and tomorrow, to build on milestones I offer six solutions and recommendations :

% matched lemmas: ally, assessment, biodiversity, climate, community, conflict, cooperation, corporation, crisis, decision, demand, displacement, drought, economy, ecosystem, environmental, exploitation, extraction, flood, force, fuel, future, generation, government, ground, impact, interest, leader, life, livelihood, nature, pandemic, path, people, planet, policy, priority, protection, push, reality, resilience, resource, right, science, society, stand, storm, support, voice, vulnerability, weather, world
\end{textsample}
