\begin{textsample}{POS Dim 4 – human – Score 40.00 – t637\_human.txt}  \label{ex:f4_pos_002}
Recent studies show human \textbf{fishing} \textbf{fleets} are out-competing seabirds for \textbf{fish}, leading to steep population \textbf{declines} and degraded marine \textbf{ecosystems}.
In the 19th century, albatross colonies were harvested for the fashion industry feather trade, leading to the near-extinction of some \textbf{species}.
In 1909, over 300,000 albatrosses were killed on Midway and Laysan Islands alone. \textbf{Thousands} of albatross are killed colliding with tall military air traffic towers on Midway.
Invasive \textbf{species} carried on \textbf{ships} — rats, mice, and cats — attack breeding colonies, taking chicks and eggs.
Albatross chicks require five to nine months to fledge, and remain extremely vulnerable during that time.
Meanwhile, plastic flotsam kills over a million seabirds per year.
Global heating, \textbf{ocean} acidification, oil \textbf{spills}, and toxic chemical pollution also contribute to seabird \textbf{decline}.
Albatross are particularly vulnerable to longline hooks and and trawling \textbf{nets}. \textbf{Thousands} of \textbf{birds} get snared in “ ghost \textbf{nets}, ” virtually invisible nylon netting lost from \textbf{ships}, drifting throughout the \textbf{oceans}.
We now know from the Grémillet, Paleczny and other studies, that seabirds are disappearing due to a lack of food, as industrial-scale \textbf{fishing} \textbf{fleets} over-harvest prey \textbf{fish} \textbf{species}.
The 2018 Grémillet research found that the prey \textbf{species} had been significantly depleted in 48 % of all seabird marine \textbf{habitats}, particularly in the Southern Ocean, Asian shelves, Mediterranean Sea, Norwegian Sea, and along the Californian \textbf{coast}. “ Seabirds are particularly good indicators of the health of marine \textbf{ecosystems}, ” says Michelle Paleczny in her seabird population study. “ When we see this magnitude of seabird \textbf{decline}, we can see there is something wrong with marine \textbf{ecosystems}. ” Furthermore, the loss of seabirds becomes a system feedback that “ causes a variety of impacts in coastal and marine \textbf{ecosystems}. ” Seabirds, for example, transport nutrients in their waste from the deep \textbf{sea}, back to the coastal \textbf{ecosystems} in which they breed, helping to fertilize entire food \textbf{webs}.
Over-fishing is only one of many human impacts on the \textbf{oceans}, along with plastic, oil, and chemical pollution.
Perhaps the most pervasive human impact comes from our carbon \textbf{dioxide} emissions.
Since the 1970s, the world ’s \textbf{oceans} have absorbed over 30 % of the carbon \textbf{dioxide} from the burning of oil, coal, and gas.
When \textbf{water} ( H 2 O ) reacts chemically with carbon \textbf{dioxide} ( CO 2 ), the process leaves an abundance of positive hydrogen ions ( H+ ), the criterion of acidity, measured on the pH scale.
Since the beginning of the industrial revolution in the 18th century, the average \textbf{ocean} pH has dropped from approximately 8.25 to less than 8.1, representing a 30 % increase in acidity.
The pH acid/base scale, like the Richter scale for measuring earthquakes, is logarithmic, so drop of only 0.1 pH units represents a 25 % increase in acidity.
If human society continues to burn petroleum as a primary energy source, the world ’s seawater pH could drop to 7.7 by 2100, creating an \textbf{ocean} more acidic than any seen since the Miocene heating, around 14 million years ago, long before hominids emerged in Africa, and when global \textbf{temperatures} had reached about 3°C warmer than today.
Acidification has already limited coral growth, and contributes to the \textbf{decline} of coral beds, which serve as nurseries for \textbf{thousands} of marine \textbf{species}, and which effects the entire \textbf{ocean} food \textbf{web}.
Increased acidity corrodes coral skeletons, slows growth, and can prevent coral larvae from maturing into adulthood.
Acidic \textbf{water} inhibits \textbf{shell} formation among clams, oysters, mussels, urchins and starfish, contributing to the \textbf{decline} of some \textbf{species}.
The \textbf{shells} of tiny forminifera zooplankton begin to dissolve in the more acid \textbf{sea} \textbf{water}.
A study by Sven Uthicke and others at the Australian Institute of Marine Science has predicted that tropical foraminifera — and entire class of amoeboid protists critical to the marine food \textbf{web} — will be extinct by the end of the century.
Researchers have observed that the \textbf{shells} of pteropods, free-swimming \textbf{sea} snails, are already dissolving in the more acidic \textbf{water} of the Southern \textbf{Oceans}.
Ocean acidification also changes the pH of some \textbf{fishes} ’ blood, a chemical reaction called acidosis.
A small pH change can make a large difference in health and survival.
In humans, a blood pH drop of 0.2 can cause seizures and even death.
Almost two decades ago, a study by Jeremy Jackson from Scripps Institution of Oceanography, with international colleagues, concluded that “ extinction caused by overfishing precedes all other pervasive human disturbance to coastal \textbf{ecosystems}, including pollution … and climate change. ” Ocean dead zones, along populated \textbf{coast} lines, are primarily caused by fertilizer runoff and fossil-fuel use.
Algal blooms, augmented by the nutrient increase, create hypoxic, oxygen-depleted \textbf{waters} that kill other marine \textbf{life}.
According to the 2017 “ World Scientists ’ Warning to Humanity, ” these coastal dead zones have increased by 75 % since the 1960s, with more than 600 coastal \textbf{ecosystems} affected.
That report also warns of unsustainable marine \textbf{fisheries} that have exceeded the maximum sustainable yield, are in steady \textbf{decline}, and “ on the verge of collapse. ” Global marine \textbf{fisheries} \textbf{catches} peaked in 1996 and have been \textbf{declining} ever since despite expanded industrial \textbf{fishing} \textbf{fleets} with improved technologies.
As a result, \textbf{fishing} \textbf{fleets} have turned to previously non-commercial \textbf{species}, which include the smaller prey of seabirds.
Jellyfish, which have more resistance to pH changes are already beginning to dominate some marine \textbf{ecosystems}, disrupting the food \textbf{web} by competing with \textbf{fish} for \textbf{declining} zooplankton, thereby effecting the prey of seabirds.
As we should know well enough by now : when human \textbf{activity} touches anything within a complex \textbf{ecosystem}, those actions necessarily affect the entire \textbf{web} of \textbf{life} that binds that \textbf{ecosystem} together.
The picture emerging from these studies has inspired interest in marine \textbf{reserves} and protected \textbf{areas}.
For seabird populations to recover, this political work needs to expand.
In 2017, Chile stopped the massive Dominga open-pit copper and iron \textbf{mining} project near Coquimbo, to \textbf{safeguard} the Humboldt \textbf{Penguin} Marine Reserve, located off the \textbf{coast}.
The \textbf{reserve} supports the rare Humboldt \textbf{penguin}, blue \textbf{whales}, humpback and sperm \textbf{whales}, bottlenose dolphins, \textbf{sea} \textbf{turtles}, \textbf{sea} lions, albatross, and many \textbf{species} of \textbf{fish}.
In 2011, Belize permanently banned trawling in all its \textbf{waters}, the third nation to do so, partially to protect the Belize Barrier Reef System.
Venezuela and Palau have also banned trawling completely.
Large \textbf{areas} in the US, Indonesia, Philippines and other Pacific \textbf{islands} are closed to trawling.
Under president Obama, the US banned \textbf{bottom} trawling in a 23,000 square \textbf{mile} \textbf{area} off the Southeast Atlantic \textbf{coast} and reinstated a ban on offshore \textbf{drilling} in the eastern Gulf of Mexico and the Atlantic \textbf{coast}.
Morocco and Turkey recently banned illegal drift-net \textbf{fishing}.
In 2004, 13 countries ratified the Agreement on the Conservation of Albatrosses and Petrels, to reduce bycatch and remove invasive \textbf{species} from nesting \textbf{islands}.
However, 2016 paper by Avigdor Abelson in BioScience, concludes that “ current practices are insufficient to reverse \textbf{ecosystem} \textbf{declines}. ” The Ableson study shows that restoration ecology must become an integral part of marine conservation efforts.
The Worm research showed that active restoration of marine biodiversity could increase \textbf{ocean} productivity fourfold.
There remains an urgent need for increased international seabird and marine conservation effort internationally. “ Persisting Worldwide Seabird-Fishery Competition Despite Seabird Community \textbf{Decline}, ” David Grémillet et al., Current Biology, v. 28, \# 24, Dec. 17, 2018 Five years later, in 2006, research by Boris Worm from Dalhousie University in Canada, and others, \textbf{documented} an “ accelerating loss ” of biodiversity in marine \textbf{ecosystems} as “ recovery potential, stability, and \textbf{water} quality decreased exponentially. ” “ Historical overfishing and the recent collapse of coastal \textbf{ecosystems}, ” Jackson, J.B., et al., Science 293, 629–637, pdf, 2001. “ Impacts of biodiversity loss on \textbf{ocean} \textbf{ecosystem} services, ” Worm, B., et al.
Science 314, 787–790. pdf, 2006. “ Starving seabirds : unprofitable foraging and its fitness consequences in Cape gannets competing with \textbf{fisheries} in the Benguela upwelling \textbf{ecosystem}, ” David Grémillet, et al. ; Marine Biology, 2016. “ Population Trend of the World ’s Monitored Seabirds, 1950-2010, ” Paleczny M, Hammill E, Karpouzi V, Pauly D, PLoS ONE 10(6) : e0129342, 2015 “ State of World \textbf{Fisheries} and Aquaculture, ” US Food and Agriculture Organization, 2018 pdf. “ \textbf{Ecosystem} consequences of \textbf{bird} \textbf{declines} ; Sekercioglu CH, Daily GC, Ehrlich PR ; PNAS, 101(52):18042–7. pmid:15601765, 2004. “ The food consumption of the world ’s seabirds, ” Brooke Mde.
L., The Royal Society Biology Letters,271:S246–S8, pmid:15252997, PubMed/NCBI, 2004. “ A Global \textbf{Map} of Human Impact on Marine \textbf{Ecosystems}, ” Benjamin S.
Halpern, Science, pdf, 2008. “ Adverse Effect of Ocean Acidification on Marine Organisms, ” Alessandra Gallo, Elisabetta Tosti ; Journal of Marine Science, 2016 “ Ocean Acidification, ” Smithsonian, 2018 In 2016, an international team led by David Grémillet at the Centre d’Ecologie Fonctionnelle et Evolutive in Montpellier, France, published a paper in Marine Biology, presenting evidence that over-fishing was “ starving seabirds ” along the southwest \textbf{coast} of Africa.
They studied Cape gannets, large diving seabirds that compete with purse-seiners for small pelagic \textbf{fish}.
They found that 80–95 % of the \textbf{birds} ’ feeding dives were unsuccessful, and that both adult health and chick growth rates had \textbf{declined} significantly. “ \textbf{Ocean} acidification to hit levels not seen in 14 million years, ” Julia Short, Cardiff University, Phys Org, 2018. “ High risk of extinction of benthic foraminifera in this century due to \textbf{ocean} acidification, ” Uthicke, P.
Momigliano \& K.
E.
Fabricius, Nature, 2013.
Plastic garbage patches, NOAA Marine Debris project, revised 2019. “ \textbf{Oceans} in shocking \textbf{decline}, ” Richard Black, BBC, 2011 World Scientists ’ Warning to Humanity : A Second Notice, William J.
Ripple, et al. with 15,364 scientist signatories from 184 countries, BioScience, 2017. “ The biomass distribution on Earth, ” Yinon M.
Bar-On, et al., Weizmann Institute of Science, Rehovot, Israel ; PNAS : May 21, 2018,, National Academy of Sciences, US, Article \#17-11842 : Improving Ocean Management through the Use of Ecological Principles and Integrated \textbf{Ecosystem} Assessments, Melissa M. et al.
BioScience, 2013. “ Upgrading Marine \textbf{Ecosystem} Restoration Using Ecological‐Social Concepts, ” Avigdor Abelson, et al., BioScience, 2016 “ Chile Stops Billion-Dollar Mining Project To Protect Humboldt \textbf{Penguins}, ” IFL Science, 2017.
Trawling Banned In Belize, Ambergris, 2010 Concerned by these results, the Grémillet team reviewed forty years of global seabird \textbf{monitoring} and \textbf{catch} statistics for all \textbf{fisheries} targeting seabird prey.
Their study, published in Current Biology last year, found that the global \textbf{catch} of seabird prey \textbf{fish} had increased by 10 % during that period, while global seabird food consumption had decreased by 19 %.
They concluded that global \textbf{fishing} “ constrains a vanishing seabird community. ” Seabirds are among the most devastated and threatened \textbf{species} of wild \textbf{animals} in the world.
Furthermore, since seabirds perform critical food-web functions, their \textbf{decline} contributes to the destabilization of marine \textbf{ecosystems} worldwide.
According to a 2015 \textbf{bird} population trend study by Michelle Paleczny and colleagues at the University of British Columbia, in Canada, seabirds suffered a “ 70 % community-level population \textbf{decline} between 1950 and 2010. ” The largest \textbf{declines} appear among \textbf{species} that \textbf{feed} on deep-sea \textbf{fish}, severely reduced by industrial \textbf{fishing} \textbf{fleets}.
The famous and beloved albatross, closely related to the gannet and booby, provides a good example.
Albatross have the longest wingspan on Earth, up to four \textbf{metres}.
They live for decades, and one female Laysan albatross, banded in 1954, is still alive.
Albatross rely on remote \textbf{island} breeding grounds and dive for squid, \textbf{fish}, and krill.
These magnificent \textbf{birds}, however, have suffered substantial \textbf{declines}.
A 2017 study by Deborah Pardo and colleagues with the British Antarctic \textbf{Survey} \textbf{Ecosystems} Programme found that three major groups — the Wandering, Black-browed, and Grey-headed albatrosses — had \textbf{declined} by 40–60 % in 35 years.
Of the 22 known \textbf{species}, all appear on the International Union for Conservation of Nature ( IUCN ) Red List, 17 “ Threatened with extinction. ”

% matched lemmas: activity, animal, area, bird, bottom, catch, coast, decline, dioxide, document, drilling, ecosystem, feed, fish, fishery, fishing, fleet, habitat, island, life, map, metre, mile, mining, monitoring, net, ocean, penguin, reserve, safeguard, sea, shell, ship, specie, species, spill, survey, temperature, thousand, turtle, water, web, whale
\end{textsample}
