\begin{textsample}{POS Dim 4 – human – Score 29.00 – t883\_human.txt}  \label{ex:f4_pos_007}
I ’m writing this in the high Arctic at 78º North Latitude in early July, aboard the Greenpeace \textbf{ship} Arctic \textbf{Sunrise} where I ’m a guest for a few days, with 24-hour daylight and gleaming glaciers in the valleys of snow-capped coastal mountains.
We ’re here because shrinking \textbf{sea} \textbf{ice} and warming \textbf{ocean} \textbf{water} is moving \textbf{fish} farther \textbf{north}, and \textbf{fishing} \textbf{vessels} are coming with them. 5.
Destroying anemones, sponges, \textbf{sea} pens, urchins, and other fine, fragile-bodied \textbf{animals}.
A lot of the seafloor harbors delicate upstanding \textbf{creatures}.
Woe unto them ; they shall be felled. 6.
Crushing \textbf{life} within the \textbf{seabed}.
Trillions of \textbf{shelled} or soft-bodied \textbf{animals} like worms, amphipods, clams, crabs, lobsters, and many others live in the seafloor in their quiet burrows, minding their own business and hiding.
Quite crushable.
This fauna is also food for \textbf{fish} and crabs.
So even if you don't care, even if you just want to \textbf{catch} or eat fish—if your \textbf{method} of \textbf{catching} \textbf{fish} kills the food of \textbf{fish} and ruins the places where \textbf{fish} live and hide, there won't be as many \textbf{fish} to \textbf{catch}.
In that sense, trawling can be like sawing off the tree-limb you ’re standing on.
So where trawlers trawl and what trawlers do makes a big difference to our \textbf{ocean} and our food supply.
That ’s why we need trawling-free \textbf{areas}. 7.
Justice for all.
Shocking perhaps, but the world wasn't made just for those of us who happen to be here right now.
The world was here and doing just fine for millions of years before we showed up.
These trawling \textbf{ships} have been around for just a few decades.
There are many people alive who were alive when the first big trawlers went to \textbf{sea}.
And there will be many people alive in the future who will get what we leave and won't get what we ruin.
We can take care of the place, or we can wreck it.
It ’s really a deeply moral consideration.
But there ’s nothing that says the world owes us all the \textbf{fish} in the \textbf{sea}.
Leaving some space in the \textbf{sea} is the smart—and the decent—thing to do.
Carl Safina is a writer and conservationist, and founder of The Safina Center.
He is currently a guest on \textbf{board} the Arctic \textbf{Sunrise} off the \textbf{coast} of Svalbard, Norway.
A version of this article was first published by National Geographic, 4th July 2016.
These are big trawling \textbf{ships}, and in other \textbf{regions} trawl-fishing has harmed—in some cases ruined—vast \textbf{areas} of seafloor.
Here there ’s still a chance to get it right by letting trawlers work in some \textbf{areas} and designating other \textbf{areas} as trawl-free zones.
We ’re here to \textbf{document} the trawling and help advance the discussion.
Trawling at its most basic it ’s a \textbf{boat} pulling a \textbf{net} through the \textbf{water}.
Sometimes that \textbf{net} is midway between \textbf{surface} and seafloor.
Sometimes—most of the time, actually—it ’s dragged across the seafloor.
Trawls have been called “ bulldozers of the \textbf{ocean}. ” Recently some big retailers like McDonald ’s and the major \textbf{fishing} companies of Norway and Russia have entered into an agreement with Greenpeace to not expand further until an agreement can be reached to put some big \textbf{areas} here aside, safe from trawling.
Trawling is one of the most basic and most effective ways of \textbf{catching} \textbf{sea} \textbf{life}.
If you ’ve eaten \textbf{fish}, most were probably \textbf{caught} by trawling.
Here are some major issues : 1.
Overfishing.
Millions of tons of \textbf{sea} \textbf{life} find themselves engulfed in trawl \textbf{nets} each year.
Trawling has been done so intensively that it ’s depleted many kinds of \textbf{fish} in many parts of the world. \textbf{Catches} must be strictly managed or in a few years there ’ll be little left. 2.
Untargeted, unwanted \textbf{catch}, or “ bycatch. ” Regardless of different variations in \textbf{method}, the one thing all trawlers have in common is that they basically core a hole through the \textbf{ocean}, so they \textbf{catch} a lot of things they ’re not trying to catch—unmarketable \textbf{fish}, marine \textbf{mammals}, even seabirds.
In some \textbf{fisheries} the \textbf{catch} is pretty “ clean. ” But in many, more than half of what trawls \textbf{catch} is unwanted.
Virtually all of a trawl ’s \textbf{catch} comes up dead or fatally injured, and if it ’s unwanted it ’s just shoveled back.
Shrimp \textbf{fishing} can be some of the worst, because small mesh also \textbf{catches} small \textbf{fish}.
And large \textbf{fish}.
At times, they can \textbf{catch} 10 \textbf{fish} for each single shrimp.
Many are babies of large \textbf{species}, and have no market.
Out come the shovels.
I ’ve seen it many times. 3.
Destabilization of the seafloor.
If the \textbf{net} is dragged, it is weighted.
It plows heavily along the seafloor.
Most of the deeper \textbf{ocean} seafloor has extremely stable natural \textbf{conditions}.
Stable currents, stable \textbf{temperature} ( it ’s cold ; things grow slowly ).
Not much happens to disturb the peace.
Enter : disturbance-trawlers. 4.
Coral damage.
Corals aren't just for tropical \textbf{reefs}.
Many coral \textbf{species} have specialized to grow in deep, cold \textbf{water}.
Those corals often continue growing for centuries ( I ’ve read that they can be \textbf{thousands} of years old)—until the moment a trawl snaps and crushes them.
Off Florida and New Zealand, deep corals have been 97-99 percent destroyed by trawling ( Allsopp et al.
State of the World ’s \textbf{Oceans}, 2009, Springer ).
This is where \textbf{fish} live and hide ; it ’s their \textbf{habitat}.
These deep \textbf{reefs} and coral groves are among the oldest old-growth on Earth.
And there are many kinds of soft corals too.
That word “ soft ” can help you guess what happens when a heavy trawl \textbf{net} comes plowing through.

% matched lemmas: animal, area, board, boat, catch, coast, condition, creature, document, fish, fishery, fishing, habitat, ice, life, mammal, method, net, north, ocean, reef, region, sea, seabed, shell, ship, specie, sunrise, surface, temperature, thousand, vessel, water
\end{textsample}
