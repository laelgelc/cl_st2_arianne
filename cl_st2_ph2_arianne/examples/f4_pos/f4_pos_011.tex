\begin{textsample}{POS Dim 4 – human – Score 28.00 – t577\_human.txt}  \label{ex:f4_pos_011}
The \textbf{depths} of our \textbf{oceans} hide a unique living world that we barely understand – but these mysteries are already under \textbf{threat} from a controversial new industry : deep \textbf{sea} \textbf{mining}.
The deep \textbf{sea} is an incredibly important \textbf{store} of ‘ blue carbon ’, the carbon that is naturally absorbed by marine \textbf{life}, and which remains \textbf{stored} in deep \textbf{sea} sediment for \textbf{thousands} of years after these \textbf{creatures} die – helping to slow climate change.
But by impacting on natural processes that \textbf{store} carbon, deep \textbf{sea} \textbf{mining} could even make climate change worse.
The disruption caused by the machines may release carbon \textbf{stored} in deep \textbf{sea} sediments, and the wider impacts could disrupt the processes that \textbf{store} carbon in those sediments.
We know that we are facing a climate emergency.
Why would we make it worse for ourselves?!
This widespread disruption to marine \textbf{life} would impact the whole \textbf{ocean} food chain.
A Greenpeace investigation revealed that companies hoping to be involved in the supply chain are fully aware of this risk – a \textbf{document} circulated at a deep \textbf{sea} \textbf{mining} stakeholder meeting acknowledges “ the potential extinction of unique \textbf{species} which form the first rung of the food chain. ” So far, we ’ve only explored 0.0001 % of the deep seafloor to see what lives there.
We have so much more to learn about the deep \textbf{ocean} ’s \textbf{wildlife} and \textbf{ecosystems}.
How can companies properly risk manage something that we are yet to fully understand?
There could be many more risks than they are aware of, from opening a new industrial \textbf{frontier} in the largest \textbf{ecosystem} on Earth.
Without proper \textbf{protection} of the deep \textbf{sea}, we could destroy \textbf{species} and \textbf{ecosystems} yet to even be discovered.
No minerals or \textbf{metals} are worth destroying \textbf{ecosystems} we don't even understand yet.
Companies that use these materials for smartphones and renewable energy should be investing in recycling and new technology instead of threatening marine \textbf{life} for profit.
And the corporations that want to get going with this destructive practice know the risks.
Just as the oil industry spent years downplaying the environmental risks of their product while convincing politicians it was essential for the economy, so now deep \textbf{sea} \textbf{mining} companies are trying to convince politicians they offer a “ green solution ”.
This isn't true.
Let ’s not repeat our mistakes and let this risky industry get a grip on our deep \textbf{seas}.
With the nature and climate crises that we are facing, the \textbf{stakes} are simply too high.
We shouldn't be asking ‘ how much new harm will we allow? ’ – we should be making ambitious plans to help our \textbf{oceans} recover to health.
To help make this happen we are building an unstoppable global movement for healthy \textbf{oceans}, to protect nature and tackle climate change.
So far over 900,000 people globally have signed our petition, calling on governments to agree a Global Ocean Treaty next year, which can put vast \textbf{areas} off-limits to exploitation and raise environmental standards for any industrial \textbf{activity} in international \textbf{waters}, putting \textbf{protection} at the heart of how we collectively manage our global \textbf{oceans}.
A strong Global Ocean Treaty can help protect the hidden treasures of the deep \textbf{sea} from reckless exploitation.
The deep \textbf{sea} is the largest \textbf{ecosystem} on Earth.
We should protect it and learn from it, not mine it.
A handful of companies and governments are planning to send monster machines deep beneath the waves, disrupting sensitive and unique \textbf{habitats} to extract \textbf{metals} and minerals.
While licences have been granted to explore for deep \textbf{sea} \textbf{mining} in over a million square \textbf{kilometres} of our global \textbf{oceans}, no deep \textbf{sea} \textbf{mining} is happening – yet.
Sending gigantic \textbf{mining} machines designed to bulldoze and churn up the \textbf{seabed} is clearly a very bad idea.
Want to know how bad?
Here ’s five reasons why deep \textbf{sea} \textbf{mining} will only get our planet into deep trouble.
Scientists are warning that plundering the seafloor with monster machines risks inevitable, severe and irreversible environmental damage to our \textbf{oceans} and marine \textbf{life}.
You only have to look at some of the names of recent research papers : ‘ Deep-Sea \textbf{Mining} with No \textbf{Net} Loss of Biodiversity – An Impossible Aim ’.
In the deep \textbf{sea}, we find underwater mountains that are oases for \textbf{sea} \textbf{creatures}, ancient coral \textbf{reefs} and \textbf{sharks} that can live for hundreds of years.
These are among the longest living \textbf{creatures} on Earth, which makes them particularly vulnerable to physical disturbance because of their slow growth rates.
Researchers estimate that harm to \textbf{wildlife} from \textbf{mining} “ is likely to last forever on human timescales ”.
As if the total destruction of their homes wasn't bad enough, machines cutting the seafloor will create sediment plumes, which could smother deep \textbf{sea} \textbf{habitats} for \textbf{kilometres}.
The \textbf{ships} on the \textbf{surface} for the \textbf{mining} \textbf{operation} could also release toxic vapours into the \textbf{water}, harming many \textbf{ocean} \textbf{species} for hundreds or even \textbf{thousands} of \textbf{kilometres}.
And it ’s not just pollution \textbf{wildlife} have to worry about.
Noise generated by churning machinery risks harming and disturbing marine \textbf{mammals} like \textbf{whales}, while floodlighting \textbf{areas} of the dark deep \textbf{ocean} could cause permanent disruption to \textbf{sea} \textbf{creatures} adapted to very low levels of natural light.
The \textbf{creatures} of the deep \textbf{sea}, specially adapted to live in the extreme alien environment of the \textbf{ocean} \textbf{depths}, almost look like something from another planet.
From ‘ zombie worms ’ discovered in 2002, to a transparent anemone that can eat worms six times its own mass, the deep \textbf{seas} are full of weird and wonderful \textbf{creatures}.
At one of the target sites for \textbf{mining}, 85 % of the \textbf{wildlife} living around hydrothermal vents are found nowhere else in the \textbf{oceans}.
Shockingly, licences have already been granted to explore for \textbf{mining} potential at vents, including the incredible Lost City in the middle of the Atlantic. \textbf{Mining} machines grinding up and destroying the \textbf{habitats} that these \textbf{creatures} are specifically designed to live in means they could never recover – risking the extinction of unique \textbf{species}.
The physical recovery of the nodules that \textbf{mining} companies want to extract takes millions of years, and we do not know if \textbf{creatures} dependent on nodules can recover after their removal.

% matched lemmas: activity, area, creature, depth, document, ecosystem, frontier, habitat, kilometre, life, mammal, metal, mining, net, ocean, operation, protection, reef, sea, seabed, shark, ship, specie, stake, store, surface, thousand, threat, water, whale, wildlife
\end{textsample}
