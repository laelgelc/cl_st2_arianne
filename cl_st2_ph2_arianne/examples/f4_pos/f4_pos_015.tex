\begin{textsample}{POS Dim 4 – human – Score 26.00 – t638\_human.txt}  \label{ex:f4_pos_015}
What do pink dolphins and leafy \textbf{sea} dragons have in common?
They both benefit from two amazingly unique and little-known \textbf{reefs} on opposite sides of the world.
Despite their geographic differences, the Amazon Reef and the Great Southern Reef have some striking similarities.
The same thing is happening down under, too.
For years, oil \textbf{giants} have been lurking around, getting ready to begin exploratory deepwater \textbf{drilling} off the southern \textbf{coast} of Australia, right beside the Great Southern Reef.
State-owned Norwegian oil \textbf{giant} Equinor ( formerly Statoil ) wants to sink its drills into the Great Australian Bight as early as summer of 2019/2020, while other oil companies could be blasting their seismic cannons at the \textbf{whales} and \textbf{reef} dwellers as soon as this Spring.
Two million : this is the number of people around the world that have banded together as Amazon Reef Defenders to protect this stunning \textbf{area} from the ravages of the oil industry.
This is truly a people-powered movement.
From scientists to local communities, from Malaysia to Paris, from actors and actresses to climbers and samba players ; we ’ve stayed united.
And it ’s worked!
Just last month, at the end of 2018, the Brazilian environmental agency ( Ibama ) denied French oil company Total a license to drill for oil near the Amazon Reef.
This wouldn't have happened without the passion and dedication of the Amazon Reef Defenders – people like you fighting to protect some of the last untouched places on earth.
Similarly down in Australia, Greenpeace has been working with First Nations, local communities, scientists, surfers, divers and people like you to keep oil companies out of the Great Australian Bight and away from the Great Southern Reef.
We ’re making sure Australians and people around the world know just what ’s at risk if we allow oil \textbf{drilling} and seismic blasting to happen in these wild and beautiful \textbf{waters}.
The same month that Total had their Amazon Reef \textbf{drilling} application rejection, Equinor asked to delay their \textbf{drilling} plans for another year – maybe because they ’ve seen the huge community opposition they ’re facing.
But that ’s not all they ’d be up against – extreme \textbf{depth}, unknown pressure, wild weather and the remote location all make \textbf{drilling} in the Great Australian Bight extremely risky.
We ’re for a world beyond oil, coal and gas.
We ’re for clean energy and better ways of getting around that don't rely on climate-wrecking, air-polluting fossil fuels.
We love our \textbf{oceans} and know the work that our precious \textbf{reefs} do cleaning the \textbf{water} and keeping whole \textbf{ecosystems} healthy.
Will you join the movement to protect \textbf{reefs} around the world?
Zoë Deans is a Digital Campaigner at Greenpeace Australia Pacific The Amazon Reef is located at the mouth of the Amazon River \textbf{basin}, where the river meets the \textbf{sea} and fresh \textbf{water} mixes with salt \textbf{water}.
Silt from the river washes out through the river mouth, meaning the \textbf{water} there is often muddy.
Most \textbf{reefs} grow in warm, clear, salty tropical \textbf{waters} – not murky river mouths.
Scientists were amazed to discover the unexpected Amazon Reef was flourishing in these unusual \textbf{conditions}, and officially announced the \textbf{reef} ’s existence in a 2016 paper.
In 2017, Greenpeace worked with scientists on \textbf{board} the Esperanza \textbf{ship} to capture the first ever images of this newly-discovered \textbf{reef} : The Great Southern Reef lies off the southern \textbf{coast} of Australia in the Great Australian Bight, a pristine and remote stretch of \textbf{ocean} that reaches from Tasmania in the east to Western Australia in the \textbf{west}.
It ’s known for its wild \textbf{waters} and unpredictable weather, which makes researching the Great Southern Reef extra challenging.
Unlike the tropical coral of the Great Barrier Reef, the coral thriving at the Great Southern Reef is cold-water coral – and it ’s incredible.
Scientists estimate the Amazon Reef ’s \textbf{size} could span 56,000 km2 near the mouth of the Amazon River.
Not bad for a \textbf{reef} system that nobody expected to be there!
Because of its unique nature, the critters that live amongst the Amazon Reef are very precious.
Between 2010 and 2014, scientists undertook three \textbf{surveys} of the \textbf{area}, and they believe they have found new \textbf{species} of \textbf{fish} and sponges.
The Great Southern Reef is a massive series of \textbf{reefs} with extensive kelp seaweed forests that extend around Australia ’s southern \textbf{coastline}, covering around 71,000sqkm from Brisbane to Kalbarri.
It ’s an amazing \textbf{life} support for the incredibly diverse \textbf{wildlife} of the Great Australian Bight – like the unique leafy \textbf{sea} dragon.
In fact, 85 % of the \textbf{species} in the Bight live nowhere else in the world – and they wouldn't be there if not for the Great Southern Reef.
The Amazon Reef \textbf{region}, at the mouth of the Amazon River \textbf{basin}, is a migratory route for different \textbf{species} of \textbf{whales}.
But along with the incredible \textbf{sea} \textbf{life}, the Amazon River itself is home to some incredible marine \textbf{mammals}, like the \textbf{giant} river otter.
These sleepy-looking \textbf{animals} live mostly in and along the Amazon River.
When we say \textbf{giant}, we mean it : \textbf{giant} river otter can reach up to 1.7m in length.
And that ’s not all : the Amazon River \textbf{region} is home to endangered pink dolphins, often known as boto.
The Great Australian Bight ’s wild \textbf{waters} are basically a maternity ward for \textbf{whales} – like southern right \textbf{whales}, which journey up from the chilly \textbf{waters} of the Antarctic every year to have their babies.
The Bight is also home to curious and playful Australian \textbf{sea} lions, well-known for their propensity to get up-close and personal – they ’re also the smallest, possibly the rarest, and very arguably the cutest pinnipeds in the world.
Multinational oil \textbf{giants} have had their eye on both of these pristine \textbf{regions}, greedy for the oil they think sits below these magical \textbf{reefs}.
You know as well as we do that \textbf{drilling} for oil carries an inherent risk of an oil \textbf{spill} which could cause irreparable damage to these \textbf{reefs}.
With some of the oil block offers at the Amazon River mouth \textbf{basin} actually on top of the Amazon Reef, it ’s outrageous that oil companies are even considering \textbf{drilling} in this \textbf{region}.
To make things worse, the risks of oil exploration in the \textbf{area} are greater due to the strong currents and sediment carried by the Amazon River.
A \textbf{spill} could cause irreparable damage to the \textbf{reef}.

% matched lemmas: animal, area, basin, board, coast, coastline, condition, depth, drilling, ecosystem, fish, giant, life, mammal, ocean, reef, region, sea, ship, size, specie, species, spill, survey, water, west, whale, wildlife
\end{textsample}
