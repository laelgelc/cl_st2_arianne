\begin{textsample}{POS Dim 4 – human – Score 26.00 – t527\_human.txt}  \label{ex:f4_pos_016}
Seamounts are large submarine volcanic mountains, formed through volcanic \textbf{activity} and submerged under the \textbf{ocean}.
Though they were once seen as nothing more than a nuisance by sailors, scientists have discovered that the structures of seamounts form \textbf{wildlife} hotspots.
The steep slopes of seamounts carry nutrients upwards from the \textbf{depths} of the seafloor towards the sunlit \textbf{surface}, providing the \textbf{sea} \textbf{life} with nutrient-rich food.
Seamounts like Mount Vema are often found \textbf{miles} from countries ’ national \textbf{waters}, far out on the high \textbf{seas}.
That makes it difficult to give them proper \textbf{protection}, as the gaps in existing regulations can be easily exploited by destructive industries.
This is why we are campaigning for a global treaty to protect the high \textbf{seas}, so that unique \textbf{ecosystems} like Vema ’s can finally be protected effectively.
Greenpeace is going from \textbf{pole} to \textbf{pole} to show the biodiversity, \textbf{threats} and possible solutions to protect our \textbf{oceans}, and Mount Vema is the next stop!
Joins us to cover our planet in \textbf{ocean} \textbf{sanctuaries}.
Helena Kowarick Spiritus is an \textbf{oceans} campaigner with Greenpeace Germany We are currently on an epic journey, sailing from the Arctic to the Antarctic.
We are stopping along the way to do research that will help us understand the complexity of the massive amounts of \textbf{water} that surrounds us.
Our next stop is the Mount Vema seamount and here are four things you need to know about it : 1/ Mount Vema is as high as 767 giraffes piled on top of each other The Vema seamount was discovered in 1957 ( some sources say 1959 ) by an Oceanographic Research \textbf{vessel} with the same name.
From the \textbf{ocean} floor, it stretches 4 600m high.
That is 4,5 times higher than the iconic Table Mountain in South Africa, or as high as 767 giraffes piled on top of each other.
Which also means that the peak of Mount Vema is just 26m below the \textbf{ocean} \textbf{surface}, so it will be possible for Greenpeace to go there with human divers and show the amazing biodiversity of the \textbf{region}. 2/ The first explorers of Mount Vema were on a hunt for diamonds The discoverers initially hoped to find large diamond deposits on Vema.
Instead they found another kind of wealth : the Tristan rock lobster or Jasus tristani, a lobster \textbf{species} that is otherwise found only on the Tristan da Cunha archipelago about 1,000 nautical \textbf{miles} away.
This kind of lobster enjoyed great fame among \textbf{seafood} lovers and sold for a good price, before it became virtually extinct at Mount Vema due to overfishing.
The population of Tristan lobsters still hasn't recovered to this day 3/ Mount Vema is littered with abandoned \textbf{fishing} gear Now, instead of Tristan lobsters, \textbf{surveys} in the \textbf{area} only find old discarded \textbf{fishing} \textbf{equipment}, a deadly trap for numerous \textbf{animals}.
Abandoned \textbf{fishing} gear, called “ ghost gear ” continues to \textbf{catch} \textbf{sea} \textbf{creatures} as if they were still being used, snaring and entangling \textbf{species} that cannot free themselves and end up dying.
This damages both marine \textbf{life} and the \textbf{fisherman} who lose part of their potential \textbf{catch}. 4/ A Global Ocean Treaty could help protect this place

% matched lemmas: activity, animal, area, catch, creature, depth, ecosystem, equipment, fisherman, fishing, life, mile, ocean, pole, protection, region, sanctuary, sea, seafood, specie, species, surface, survey, threat, vessel, water, wildlife
\end{textsample}
