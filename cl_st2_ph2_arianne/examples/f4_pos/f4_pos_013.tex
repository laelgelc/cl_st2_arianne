\begin{textsample}{POS Dim 4 – human – Score 26.00 – t988\_human.txt}  \label{ex:f4_pos_013}
For many people the Antarctic is little more than a far-away frozen \textbf{region}, literally at the \textbf{edge} of the world ; with sterile glaciers, icebergs and colonies of not-so ‘ Happy Feet ’ \textbf{penguins}, buffeted for much of their \textbf{lives} in the extreme Antarctic wind.
The ice-covered \textbf{waters} of Antarctica are actually bursting with \textbf{life}.
Magnificent \textbf{whales}, orcas, \textbf{seals}, \textbf{fish} and soaring seabirds come here to forage on krill-rich \textbf{waters}.
Below the icy \textbf{ocean} \textbf{surface}, the seafloor is covered with a carpet of \textbf{creatures} of different shapes, colours and \textbf{sizes}, many of which are found nowhere else on Earth.
Every year scientists find yet more \textbf{species}.
The Antarctic is the world ’s last wild \textbf{frontier}.
And it is one that we need to protect before it ’s too late.
The Antarctic peninsula is one of the most rapidly warming \textbf{regions} on Earth.
Changes in \textbf{sea} \textbf{ice} has had an impact on krill – the basis of the Antarctic food \textbf{web}, whose depletion might cause potentially devastating effects on \textbf{whales} and other \textbf{ocean} \textbf{life}.
After depleting over 80 % of \textbf{fish} populations that live close to \textbf{shore}, greedy industrial \textbf{fishing} \textbf{fleets} are moving to the remote Antarctic \textbf{regions} to hoover up \textbf{fish} and to suck up krill to \textbf{feed} the growing \textbf{seafood} demand on the other side of the world.
The future of the Antarctic is in the hand of a group of countries that are members of the ‘ Commission for the Conservation of Antarctic Marine Living Resources ’ ( CCAMLR ).
They have put in place a unique set of rules to protect \textbf{waters} around Antarctica.
This system contains a mechanism to establish marine \textbf{reserves}, national parks at \textbf{sea} where \textbf{fishing} and other destructive \textbf{activities} are prohibited.
In 2009, CCAMLR members agreed to protect key Antarctic \textbf{waters} by 2012.
The clock is still ticking, but progress has been very slow.
Today the Antarctic Ocean Alliance, which includes Greenpeace and partner groups, is launching a new publication, ‘ Antarctic Ocean Legacy : A Vision for Circumpolar \textbf{Protection} ’ which makes the case for the creation of a network of 19 large no-take marine \textbf{reserves} and marine protected \textbf{areas} covering over 40 % of the Southern Ocean and the precious \textbf{habitats} and marine \textbf{life} within it.
There is no time to waste.
We need to act now before it is too late.
At the next CCAMLR meeting in October 2012, CCAMLR members will have the unique opportunity to create this largest network of marine \textbf{reserves}.
A good outcome – with our pressure – will demonstrate to governments and politicians that we need a global network of marine \textbf{reserves} in the rest of the high \textbf{seas}, which are still largely unprotected.
We need your help to save the Antarctic \textbf{oceans}!
Sign the AOA petition and let governments know that we are watching and we expect them to make the right decision, to protect our \textbf{oceans} for the future we need.
Veronica Frank is a Greenpeace International \textbf{oceans} campaigner based in London.

% matched lemmas: activity, area, creature, edge, feed, fish, fishing, fleet, frontier, habitat, ice, life, ocean, penguin, protection, region, reserve, sea, seafood, seal, shore, size, specie, surface, water, web, whale
\end{textsample}
