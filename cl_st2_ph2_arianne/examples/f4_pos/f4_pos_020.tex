\begin{textsample}{POS Dim 4 – human – Score 25.00 – t886\_human.txt}  \label{ex:f4_pos_020}
Last week I visited the Svalbard archipelago in the \textbf{northern} Barents Sea to bear \textbf{witness} to the rapid changes occurring in the Arctic.
In many ways, the Arctic is the frontline of dramatic environmental changes that will impact everyone.
For Svalbard, it is estimated that about 80 percent of the trash washed ashore comes from industrial \textbf{fishing}.
In annual Barents Sea \textbf{fisheries} \textbf{surveys} the highest litter counts coincide with \textbf{areas} of intensive \textbf{fishery} and shipping.
These items are not only a blight on the \textbf{landscape}, they are dangerous.
Reindeer, polar bears and other \textbf{animals} get entangled in old \textbf{fishing} \textbf{nets} and suffocate. \textbf{Birds} consume toxic particles of plastic.
Plastics can carry pollutants that, if ingested, may also accumulate through the food chain.
As part of our ‘ Protect What You Love ’ campaign, the team decided to organize a \textbf{beach} clean up to raise awareness about the consequences of \textbf{bottom} trawling and the trash left behind by the \textbf{fishing} industry.
According to UNESCO, globally, up to 1 million seabirds and 100,000 marine \textbf{mammals} and \textbf{sea} \textbf{turtles} die each year from eating plastic or getting entangled in plastic debris.
This unequivocal destruction of the underwater ecologies is legal.
The current protected \textbf{areas} cover just a small part of unique marine \textbf{habitats}, and the rest is open for destructive oil production and industrial \textbf{fishing}.
For now, some of the world ’s largest \textbf{seafood} and \textbf{fishing} companies have committed to not expand their search for cod into large, previously un-fished \textbf{areas} in the \textbf{northern} Barents Sea in the Norwegian Arctic.
The initiative, brokered by Greenpeace, marks the first time an entire industry has collectively called for Arctic \textbf{protection} – in the absence of legal \textbf{protection} in these \textbf{areas}.
The agreement states that any \textbf{fishing} company expanding into the agreed to Arctic \textbf{waters} will not be able to sell their cod to a number of major \textbf{seafood} brands and retailers.
But the agreement, though in good faith, is voluntary.
Our environment – whether it is the Arctic or the Amazon – needs protecting.
I believe the Arctic \textbf{waters} are crucial for the \textbf{life} of our planet and should be protected.
The Norwegian government needs to act fast to ensure these previously untouched \textbf{areas} of the \textbf{northern} Barents Sea are protected and any expansion of \textbf{fishing} is stopped.
Until recently, the \textbf{northern} Barents Sea, one of Earth ’s last pristine environments, had been safely protected with \textbf{ice} cover.
However, this is no longer the case.
Due to the effects of climate change, \textbf{sea} \textbf{ice} is melting, the \textbf{area} is changing dramatically, and a new \textbf{ocean} is opening up.
This enables the oil and \textbf{fishing} industries to expand into the far North.
Jennifer Morgan is Executive Director at Greenpeace International.
This story first appeared on The Huffington Post.
This far away land and its \textbf{seas} are no longer what they used to be and are surprisingly not protected.
Norway, due to its role in deciding what happens in these \textbf{waters}, has a key part to play.
If we are to retain what we have left, and make the \textbf{area} more able to be resilient in the face of the changing climate, the Norwegian government must make the \textbf{waters} a marine protected \textbf{area}.
Up to now, however it has shown a lack of political will to do so.
Having spent 20 plus years as a climate campaigner, I didn't travel to the Arctic with rose-colored glasses.
I knew the \textbf{ice} was melting.
The Earth ’s 2015 \textbf{surface} \textbf{temperatures} were the warmest since modern record keeping began in 1880, according to independent analyses by NASA and the National Oceanic and Atmospheric Administration ( NOAA ).
NASA reports that spring 2016 was the hottest recorded in the \textbf{Northern} hemisphere.
With the Arctic warming twice as fast as the rest of the world, the ancient glaciers of Svalbard are retreating. \textbf{Witnessing} the \textbf{ice} melt – with my own eyes in my new role as Executive Director, Greenpeace International – I imagined the day in just a short few decades when the Arctic will have ice-free summers and it petrified me.
Climate change is opening up the \textbf{northern} Barents Sea and the \textbf{fishing} industry is advancing further to the North each year.
Which means previously untouched \textbf{areas} are turning into a new hunting ground.
Dozens of \textbf{giant} trawlers are already above the 78th parallel, a historic development.
This industry is using \textbf{bottom} trawling as part of their practice. \textbf{Bottom} trawling does not mean just taking cod ; it is like clear-cutting forests but underwater.
The trawlers carry 100-metres long \textbf{nets} weighing with heavy \textbf{metal} rollers that smash everything in their path.
They are like bulldozers of the \textbf{sea}.
Fragile \textbf{seabed} communities of \textbf{sea} pens and corals that need decades to grow can be wiped out in seconds.
And with the industry, comes trash.
Probably like most people who don't live here, I imagined this part of the Arctic as a land of \textbf{ice} and snow with limited traces of human beings.
I expected to see more glaciers and more majestic ranges covered in white.
Instead our Greenpeace \textbf{crew}, campaign team and the three winners of our poster competition, found trash.
I wasn't sure how much trash we would find, but at the end of day one we had collected a large mound of trash – buoys, \textbf{fishing} \textbf{nets}, rope, glass, plastic bottles and more.
The recurring piece of trash though was shoes – was this a message about our human footprint?

% matched lemmas: animal, area, beach, bird, bottom, crew, fishery, fishing, giant, habitat, ice, landscape, life, mammal, metal, net, northern, ocean, protection, sea, seabed, seafood, surface, survey, temperature, turtle, water, witness
\end{textsample}
