\begin{textsample}{POS Dim 4 – human – Score 28.00 – t489\_human.txt}  \label{ex:f4_pos_008}
Whether you eat it, love swimming in the \textbf{ocean} with it, or enjoy \textbf{catching} it, \textbf{fish} and other types of \textbf{seafood} play a large role in our \textbf{lives}.
It ’s nourishing, wonderful to observe in its natural \textbf{habitat}, and who doesn't love Finding Nemo?
But for those who enjoy tucking into a tuna steak now and then, have you ever wondered where it came from? \textbf{Fishing} is labour intensive and literally requires all hands on \textbf{deck}.
Whilst things such as fuel costs are unavoidable, labour accounts for 30-50 % of total \textbf{fishing} operational costs and a rising tide of evidence suggests that migrant workers, mainly from Southeast Asia, are often used to cut corners.
After being lured in with the promise of an attractive salary and an opportunity to provide for their families, well-documented cases have been found of migrant \textbf{fishers} being made to live in appalling \textbf{conditions}, work inhumane hours for little or no pay, and often face abuse.
A practice called transshipment, in which \textbf{catch} ( some of which can include illegal \textbf{catches} ) is offloaded from one \textbf{vessel} to another to enable the main \textbf{fishing} \textbf{vessel} to stay out longer without entering \textbf{port}, means that these workers are completely isolated at \textbf{sea} for months, even years.
This is still happening now, and Greenpeace Southeast Asia has testimonials and reports to show it.
The current climate emergency poses one of the worst \textbf{threats} to our fragile marine \textbf{ecosystems}.
A warmer planet means warmer \textbf{oceans}, and marine \textbf{life} are finding it harder and harder to survive.
In addition, the \textbf{ocean} is becoming more acidic as carbon pollution from burning coal, oil and gas, dissolves in seawater.
But actually, having a thriving \textbf{ocean} could help protect us from climate change – carbon is naturally absorbed by plants and \textbf{animals} and captured in the bodies of living \textbf{creatures} like \textbf{whales} and \textbf{fish}.
Around the world, the number of critical \textbf{fisheries} have \textbf{declined} due to climate change, and overfishing is not only making it worse, but also fueling illegal, unreported, and unregulated ( IUU ) \textbf{fishing}.
IUU \textbf{fishing}, which refers to \textbf{activities} that contravene national or international laws, are all too easy to get away with on the lawless high \textbf{seas}.
It ’s a practice of low risk and high reward – IUU \textbf{fishing} reportedly accounts for up to \$23.5 billion worth of \textbf{seafood} a year, and cracking down on the major players is not easy.
Like other foods we consume, what you see packed in a can, in the frozen food aisle, or at your local fresh \textbf{seafood} market hides a story.
How it got there, who \textbf{caught} it, and where it was \textbf{caught} is often overlooked.
Getting an entire industry to change its practices is not easy, but by arming yourself with the knowledge of what to look for in sustainable \textbf{seafood}, you can be an ocean-friendly consumer.
To go that step further, ask representatives in your government what they ’re doing to ensure sustainable \textbf{fisheries} management.
With enough pressure, we can all help protect the “ little \textbf{fish} in the big pond ” – from the workers to the marine \textbf{life} and local \textbf{fishing} communities – to ensure an ethical, sustainable \textbf{seafood} industry and a thriving \textbf{ocean}.
Shuk-Wah Chung is the Communications Lead for the global \textbf{fisheries} campaign with Greenpeace Southeast Asia.
The truth is, the journey from \textbf{sea} to supermarket is both complicated and eye-opening.
Behind the lucrative \textbf{seafood} industry is a whole lot of fishy business tainted with modern slavery, environmental destruction, and government collusion, not to mention the \textbf{threat} to small scale fisherfolk and their families.
Here is what you need to know about the impacts the \textbf{fishing} industry has on our planet. “ There are plenty more \textbf{fish} in the \textbf{sea} ” goes the old adage.
But an ever-growing population coupled with an ever-growing demand for \textbf{seafood} means that the \textbf{oceans} are not as plentiful as we might think.
From 1950 to 2015, the number of \textbf{fishing} \textbf{fleets} around the world has doubled to 3.7 million.
The continuous expansion of industrial \textbf{fleets} has resulted in an enormous pressure on \textbf{fish} populations, and the practices employed by the \textbf{fishing} industry are often aimed to take as much \textbf{fish} as possible, often disregarding the impacts on other marine \textbf{life} and the marine \textbf{ecosystem}. \textbf{Bottom} trawlers, which drag heavy \textbf{nets} over the \textbf{sea} floor, have caused irreparable damage to fragile \textbf{habitats} ; industrial longliners, which consists of setting \textbf{thousands} of drifting baited hooks, often stretching for dozens of \textbf{kilometres}, have decimated some \textbf{shark} and seabird populations ; and purse seine \textbf{fishing} \textbf{vessels}, which use \textbf{giant} \textbf{nets} that encircle and \textbf{catch} vast amounts of tuna with the help of FADs ( \textbf{fish} aggregating devices ), in which large schools of \textbf{fish} and marine \textbf{life} are drawn to a type of \textbf{fish} magnet to be scooped up, are just some of the practices that threaten our \textbf{ocean} ’s biodiversity and minimises the chance for more \textbf{fish} to thrive and survive.
Millions of coastal communities around the world rely on \textbf{fishing} as a source of income.
In developing countries, \textbf{fishing} and related \textbf{activities} such as \textbf{boat} building or \textbf{fish} processing employ millions, and \textbf{seafood} is one of their main sources of protein.
For others, \textbf{fishing} is more than just food – it plays a strong part in their identity, especially for indigenous communities.
Food security and access is vital to coastal communities, but with industrial \textbf{fishing} \textbf{vessels} working the \textbf{ocean} like a factory floor, millions of local and small scale \textbf{fishers} who have depended on the \textbf{ocean} for generations are left not only with a narrow and depleted zone in which to \textbf{fish} sustainably but also, at some point in the not-too-distant future, a total loss of \textbf{livelihood}.
A cheap can of tuna might seem like a bargain, but an industry that carelessly plunders the \textbf{ocean} to bring second-rate \textbf{seafood} to our tables not only comes with a great cost to the environment but another cost that ’s often hidden – cheap and abusive labour.

% matched lemmas: activity, animal, boat, bottom, catch, condition, creature, deck, decline, ecosystem, fish, fisher, fishery, fishing, fleet, giant, habitat, kilometre, life, livelihood, net, ocean, port, sea, seafood, shark, thousand, threat, vessel, whale
\end{textsample}
