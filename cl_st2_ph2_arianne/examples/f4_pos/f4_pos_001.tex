\begin{textsample}{POS Dim 4 – human – Score 50.00 – t558\_human.txt}  \label{ex:f4_pos_001}
We ’ve journeyed from Pole to Pole, on an epic voyage to reveal the wonders that lie beneath the \textbf{surface} of our \textbf{oceans} and confront the \textbf{threats} they face.
Our mission – to secure a Global Ocean Treaty, agreed at the UN, to protect the \textbf{oceans} that lie outside national \textbf{waters}. \textbf{Penguin} numbers in the Antarctic have dropped by almost 60 % – with some colonies losing as much as 77 % of their population since they were last counted in the early seventies.
Why does a warmer world mean fewer \textbf{penguins}?
It ’s mostly about food.
Like lots of Antarctic \textbf{animals}, these \textbf{penguins} live on krill, which can get harder to find as the \textbf{ice} becomes less predictable.
This pressure on food supplies from the \textbf{ocean}, combined with changes to the land where they nest and raise their chicks, makes climate change a huge \textbf{threat} to the Antarctic ’s best-loved residents.
But hope for the Antarctic is far from lost.
While the \textbf{oceans} are being hit hard by the climate crisis, they ’re also one of our best allies in fighting it.
Our expedition in the Antarctic was the last stop on our Pole to Pole voyage, but the campaign for \textbf{ocean} \textbf{protection} isn't over.
Take action today and follow our journey on Instagram.
Turtle Journey We ’ve worked with Aardman Animations ( the creators of Wallace and Gromit and Shaun the Sheep ) to produce a powerful short film about why our \textbf{oceans} need \textbf{protection}.
While we know that scientific research helps build the case for protecting our \textbf{oceans}, we also need millions of people around the world to help spread the word.
This means we need the message of \textbf{ocean} \textbf{protection} to reach as many people as possible.
This film will help us reach far beyond people who already follow Greenpeace, as well as introducing a whole new generation to this issue.
The film features special guests Jim Carter, Olivia Colman, David Harbour, Giovanna Lancellotti, Helen Mirren, Bella Ramsey and Ahir Shah Hundreds of \textbf{miles} offshore, beyond the reach of national laws or public attention, the open \textbf{ocean} can be a lawless place.
And that ’s especially true in the south \textbf{west} Atlantic.
More than 40 \textbf{species} in this \textbf{area} are under \textbf{threat}, and it ’s the home of the beautiful Southern Right \textbf{Whale}.
But it ’s also a mostly unregulated \textbf{fishing} ground, where \textbf{giant} trawlers sweep up \textbf{sea} \textbf{life}.
Lots of these \textbf{fishing} \textbf{boats} drop off their \textbf{catch} in the \textbf{port} of Montevideo in Uruguay.
Activists from Greenpeace Andino confronted them here with a 25m banner reading ‘ Ocean Looters ’.
Scientists, media, and marine experts have joined the \textbf{crew} of the Greenpeace \textbf{ship} Esperanza, to conduct scientific research on \textbf{ocean} \textbf{life} and \textbf{document} destructive human \textbf{activities} like overfishing, plastic pollution, and the impacts of climate change.
It sounds like the plot for a sci-fi fantasy story, but far off the \textbf{coast} of South Africa there really is an underwater mountain.
It ’s littered with ‘ ghost \textbf{nets} ’ abandoned by industrial \textbf{fishing} \textbf{boats} over the years.
Mount Vema is still beautiful and full of \textbf{life}, but \textbf{fish} and other \textbf{animals} are still getting tangled up ( and even killed ) in this abandoned \textbf{fishing} gear.
The team on \textbf{board} the Greenpeace \textbf{ship} Arctic \textbf{Sunrise} will be investigating the role this mountain of the \textbf{sea} and others like it play in maintaining the balance of our \textbf{oceans}, and \textbf{documenting} the impact destructive overfishing may have had on this \textbf{habitat} rich in marine \textbf{life}.
In late August, the Esperanza returned to the beautiful Amazon Reef, which has been at the heart of a long-running campaign against \textbf{giant} oil companies planning to drill near this unique \textbf{habitat}.
Following on from the UN meeting in New York, the pressure is on to make world leaders pay attention to \textbf{ocean} \textbf{protection}.
At the Amazon Reef, on-board scientists quickly made an important new discovery – after spotting a mother \textbf{whale} and her calf, they ’ve confirmed that the Amazon Reef \textbf{region} is a breeding \textbf{area} for \textbf{whales}.
This expedition also saw the first ever human deep-dive in the Amazon Reef.
Unencumbered by the mini-submarine used on previous trips, scuba divers were able to gather incredible, pin-sharp images of the \textbf{reef} ’s fascinating \textbf{wildlife}.
As part of their efforts to understand the \textbf{wildlife} in this \textbf{area}, the team is also tracking a group of leatherback \textbf{sea} \textbf{turtles} from their nesting ground in French Guyana, at the \textbf{northern} end of the Amazon Reef.
This is a long-term study so we won't have final results for a while, but one thing is clear : those \textbf{turtles} can really swim!
As this \textbf{map} shows, they ’ve crossed \textbf{thousands} of \textbf{miles} of open \textbf{ocean} in just a few months.
In August 2019, government officials from around the world met at the UN in New York to negotiate a Global Ocean Treaty.
Millions of people have called for world leaders to protect our \textbf{oceans} and Oscar-winning actor Javier Bardem delivered the message straight to the UN ( via Times Square! ) But the outcomes of the meeting were disappointing – the negotiations are going too slowly and countries are still not showing enough ambition.
But we aren't giving up!
The next set of negotiations should happen in 2021, so the pressure is on to make sure that world leaders get the message – our \textbf{oceans} urgently need \textbf{protection}.
This meeting may be over but the journey continues.
We want a Global Ocean Treaty because it could open the door to a network of \textbf{ocean} \textbf{sanctuaries} around the world.
Scientific studies have shown that when large \textbf{ocean} \textbf{areas} are protected, marine \textbf{life} and their \textbf{habitats} quickly begin to recover.
Not only will this mean \textbf{turtles}, \textbf{sharks} and \textbf{whales} will be given space in which they are safe from many of the human dangers facing them, but it will also help our fight against climate breakdown.
Healthy \textbf{oceans} are essential to keep our climate stable, so we need to do everything we can to protect them.
In the run up to the UN meeting in New York, the Esperanza journeyed through another incredible \textbf{ecosystem}.
North of the Caribbean, stretching across a vast \textbf{area} of the North Atlantic, is the Sargasso Sea.
These \textbf{waters} have one defining characteristic : floating mats of Sargassum algae.
This brown, frondy seaweed is kept in place by a gyre ; \textbf{ocean} currents which encircle the Sargasso like a whirlpool and push the Sargassum into \textbf{giant} entangled clumps.
The seaweed mats provide shelter and food for a vast array of \textbf{species}, including \textbf{fish}, seabirds and \textbf{turtles}.
But the sad fact is that the same currents keeping the seaweed in place are also collecting plastic – any plastic rubbish floating into the currents will become trapped amongst the weed.
We need action on land to stop the flow of plastic into our \textbf{oceans}, and we need a network of \textbf{ocean} \textbf{sanctuaries} to \textbf{wildlife} space to recover.
The \textbf{crew} of the Esperanza used a range of \textbf{methods} to collect images and scientific data on a range of Sargasso \textbf{species} ; from simply snorkelling to \textbf{survey} the Sargassum mats, to collecting DNA from seawater to see which \textbf{species} had passed by.
They also used blackwater photography which uses special lighting to capture images of \textbf{creatures} which usually only appear at night.
The result was a breathtaking set of photographs revealing the wonders hidden within the Sargasso Sea.
The \textbf{crew} were also lucky enough to see green \textbf{turtles} which, along with loggerhead \textbf{turtles}, rely on the Sargassum.
Baby \textbf{turtles} hatch from eggs laid on \textbf{beaches} and make the perilous journey to the Sargasso Sea – less than 1 in 1,000 hatchlings survive that journey.
There, the survivors shelter from predators amidst the weed, \textbf{feeding} away as they grow.
But like all marine \textbf{creatures}, \textbf{turtles} are under pressure from destructive \textbf{fishing}, overheating and acidifying \textbf{oceans}, as well as plastic pollution which they mistake for food.
Actor Shailene Woodley joined the \textbf{crew} to help spread the word about the impacts of plastic pollution and the need for \textbf{ocean} \textbf{protection}.
Throwaway plastic is made on land and it ’s here that we need to see action from companies to reduce the amount of single-use plastic they are pumping out onto the market.
But out at \textbf{sea} we need to create vast \textbf{ocean} \textbf{sanctuaries}, which provide safe havens for \textbf{wildlife} to recover free from human interference.
The latest stage of the expedition saw the Esperanza sail across the Atlantic Ocean, to discover the wonders of the Lost City and highlight the \textbf{threat} of deep \textbf{sea} \textbf{mining}, before heading to Jamaica to take this message to an international \textbf{seabed} conference.
About 20 years ago, scientists made an astonishing discovery.
Deep in the Atlantic Ocean lies a network of hydrothermal vents, pumping scalding \textbf{water} from the \textbf{depths} of the Earth.
These vents resemble cathedral spires, so the Lost City was an obvious name to choose. \textbf{Catch} up on the voyage below, and join over three million people from around the world in demanding a global treaty to protect the \textbf{oceans}.
And around these vents live diverse and unique \textbf{creatures} – crabs, anemones and \textbf{giant} worms have adapted to the extreme \textbf{conditions}, creating a thriving \textbf{ecosystem} where few other \textbf{creatures} can survive.
Scientists even think that vents like these could have hosted the origins of \textbf{life} on earth.
While most of us will never see the Lost City in person, artists from around the world used it as the inspiration for artworks depicting its \textbf{beauty} and the \textbf{threats} it faces.
Inevitably, where many people see natural marvels, big companies see something else : resources to be exploited.
The \textbf{waters} gushing from the vents are rich in minerals and \textbf{mining} companies are keen to send in monster machinery to rip them open.
They ’re particularly interested in rare earth \textbf{metals} which are crucial for phones and tablets, and claim deep-sea \textbf{mining} is the only way to keep us all online ( this is rubbish, and so are their other claims. ) But there ’s no two ways about it : deep-sea \textbf{mining} will obliterate these fragile communities and even make the climate crisis worse.
The good news is that deep-sea \textbf{mining} hasn't started, at least not yet, so we have a chance to protect the Lost City and other \textbf{areas} on the \textbf{sea} bed. \textbf{Thousands} of people have sent messages to big tech companies – Google, Microsoft, Apple and Hewlett Packard – in the style of their own adverts, asking them to promise they ’ll never use materials from the deep \textbf{sea} in their products.
If tech companies won't use deep-sea minerals, the \textbf{mining} companies ’ main argument falls away.
Meanwhile, the Esperanza \textbf{crew} took this message to the annual meeting of the International \textbf{Seabed} Authority.
You ’ve probably never heard of this obscure organisation, but it ’s supposed to regulate deep-sea \textbf{mining}.
Instead of proceeding with caution, they ’ve so far approved all \textbf{mining} licence applications and barely consider environmental impacts.
Even worse, they ’re lobbying for a weaker Global Ocean Treaty.
You can't run a mammoth \textbf{ocean} expedition and not get involved in World \textbf{Oceans} Day.
On 8 June ( mark it in your calendar each year! ), people across the globe celebrated our blue planet with face paint and human waves.
Together, we sent a strong message to governments – to create a treaty that will protect and heal our \textbf{oceans}. \textbf{Shark} numbers are plummeting and one of the main reasons is \textbf{shark} finning.
This brutal and wasteful practice – in which only the highly-prized fins are taken, leaving the \textbf{shark} dead or dying – is a prime example of how our \textbf{oceans} need more \textbf{protection}.
The Esperanza passed through a \textbf{shark} \textbf{fishing} ground and monitored a \textbf{fishing} \textbf{vessel} in action.
The results weren't pretty, but \textbf{documenting} destructive \textbf{fishing} is essential to make the case for a Global Ocean Treaty.
Our new research showing the impact on endangered \textbf{shark} populations like mako \textbf{sharks}, travelled around the globe, featuring in the New York Times, France24 and ABC.
Our epic voyage began among the \textbf{ice} floes of the high Arctic – although the frozen \textbf{north} is becoming distinctly less frozen each year.
The Arctic is warming twice as fast as the global average.
This means the \textbf{region} is on the frontline of the climate crisis and changing fast.
Most \textbf{life} there – such as polar bears, narwhals and walruses – depends on the \textbf{ice} covering the Arctic Ocean.
As the \textbf{ice} shrinks each year, these \textbf{creatures} are finding it harder to survive.
The \textbf{ice} also reflects the \textbf{sun} ’s energy but the darker \textbf{ocean} absorbs this heat, accelerating climate breakdown.
The Antarctic Incredibly, oil and gas companies see the retreating \textbf{ice} as an opportunity to extract even more fossil fuels.
At time when we should be cutting our use of these fuels to zero, this is extremely reckless.
Oil \textbf{spills} would devastate the \textbf{ocean}, and cleaning up a \textbf{spill} in the face of icebergs and winter \textbf{ice} would be impossible.
To protect the Arctic, we need to know as much as possible about the changes taking place.
The Esperanza ’s \textbf{crew} included a team of scientists who took \textbf{ice} cores and \textbf{ocean} \textbf{water} samples for analysis.
By examining factors such as the amount of nutrients and acidity levels, they aim to understand how the melting \textbf{sea} \textbf{ice} is affecting Arctic \textbf{life}.
See more about our Arctic voyage on CNN, the Guardian and National Geographic.
Science isn't the only way to appreciate the Arctic.
Music can also describe this frozen – but very much alive – expanse.
To highlight its \textbf{beauty} and fragility, we organised a unique concert, with chimes, horns, and a cello carved from \textbf{ice}.
But the piece was almost never performed.
Ironically, the above-average \textbf{temperatures} meant that the instruments began melting as soon as they were finished.
Add your name to the petition and tell world leaders to create a Global Ocean Treaty to protect our \textbf{oceans}.
To follow the expedition check this blog for updates, and follow us on Instagram and Facebook.
In January this year, we arrived in the Antarctic to begin the final stage of this epic year-long expedition.
Bursting with \textbf{life} and as remote as it gets, it ’s no wonder this extraordinary part of the world has captured our imagination for centuries.
But today the Antarctic is being threatened by the climate crisis, and the incredible array of \textbf{wildlife} that calls it home is at risk.
A team of scientists joined the expedition to \textbf{survey} \textbf{penguin} colonies that hadn't been counted for decades.
This involved counting tens of \textbf{thousands} of \textbf{penguins} ( by hand! ) You can have a go at some \textbf{penguin} counting yourself in this video : By investigating \textbf{penguin} colonies, the scientists can better understand how climate change and other \textbf{threats} are impacting them.
Sadly, it ’s not looking good.

% matched lemmas: activity, animal, area, beach, beauty, board, boat, catch, coast, condition, creature, crew, depth, document, ecosystem, feed, fish, fishing, giant, habitat, ice, life, map, metal, method, mile, mining, net, north, northern, ocean, penguin, port, protection, reef, region, sanctuary, sea, seabed, shark, ship, specie, spill, sun, sunrise, surface, survey, temperature, thousand, threat, turtle, vessel, water, west, whale, wildlife
\end{textsample}
