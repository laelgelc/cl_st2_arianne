\begin{textsample}{POS Dim 4 – human – Score 29.00 – t015\_human.txt}  \label{ex:f4_pos_006}
Tuna sandwiches, tuna tartare, tuna poke, tuna salad—the ways to enjoy this \textbf{fish} seem endless.
Within the last decade, more people have been consuming this versatile, nutritious, and “ affordable ” \textbf{fish}.
Recently, its popularity has soared, boosted by the pandemic, economic uncertainty, and its perceived sustainability.
It even has its own TikTok aficionados and a trendy hashtag, \#TinnedFishDateNight.
As if all of the environmental impacts weren't devastating enough, the distant-water \textbf{fishing} industry also has a long history of exploiting not just our \textbf{oceans} but also people.
They serve to reinforce each other as the diminishing \textbf{fish} \textbf{stocks} mean \textbf{vessels} must travel further out to \textbf{sea} for increasingly longer periods, where the isolated \textbf{fishers} are more likely to experience human rights abuses.
Transshipment, wherein a refrigerated \textbf{vessel} meets the \textbf{fishing} \textbf{vessel} at \textbf{sea}, collects the \textbf{fishing} \textbf{vessel} ’s cargo ( \textbf{catch} ), and returns it to \textbf{port}, extending the voyages at \textbf{sea} for months, worsening the isolation and ripening the \textbf{conditions} for abuse.
Forced labor, a type of modern slavery, is prevalent on distant-water \textbf{fishing} \textbf{vessels}.
Greenpeace East Asia and Greenpeace Southeast Asia investigations have examined the cases of dozens of \textbf{fishers} from 40 different \textbf{vessels}.
Some of their most disturbing findings include \textbf{fishers} working an average of 20 hours a day and receiving insufficient nutrition or inedible food, such as expired and/or moldy food and rusty-colored drinking \textbf{water}.
In the Choppy Waters report, published in 2020, one \textbf{fisher}, who worked on \textbf{board} Taiwanese “ longliner A ” reported : “ We only got to sleep for five hours if and when we \textbf{caught} some \textbf{fish}.
If we didn't \textbf{catch} anything, we ’d just have to keep working, even for 34 hours straight.
If it were possible, I ’d like to change how much time we have to work and rest, to meet the needs of human bodies.
There ’s got to be a way to make it more balanced, just like how people who work on land do it. ” With \textbf{vessels} far out at \textbf{sea}, most \textbf{fishers} do not have access to the internet or phone service for months at a time. \textbf{Fishers} in these situations cannot verify if their salary has been sent to their family, and a recent Greenpeace East Asia report found that over two-thirds of \textbf{fishers} interviewed had had their wages withheld.
Additionally, it ’s common practice for captains to confiscate passports or other identity \textbf{documents}, which limits their freedom of movement.
Given the prevalence of suspected forced labor in the tuna supply chain, there ’s a chance that the person who \textbf{caught} that tuna on the shelf in your neighborhood grocery \textbf{store} is a victim of forced labor.
In fact, in 2022, Greenpeace USA staff picked up a can of Bumble Bee tuna at a grocery \textbf{store}, Harris Teeter in Arlington, Virginia, only to find it contained \textbf{fish} sourced from the Da Wang – a \textbf{fishing} \textbf{vessel} confirmed by US Customs \& Border Protection to have employed forced labor.
One worker onboard this \textbf{vessel} even died after an accident where he was reportedly hit on the back of the head.
Like other foods we consume, what you see packed in a can, in the frozen food aisle, or at your local fresh \textbf{seafood} market hides a story.
How it got there, who \textbf{caught} it, and where it was \textbf{caught} is often overlooked.
Getting an entire industry to change its practices is not easy, but by arming yourself with the knowledge of what to look for in sustainable \textbf{seafood}, you can be an ocean-friendly consumer.
To go that step further, ask representatives in your government what they ’re doing to ensure sustainable \textbf{fisheries} management.
With enough pressure, we can all help protect the “ little \textbf{fish} in the big pond ” – from the workers to the marine \textbf{life} and local \textbf{fishing} communities – to ensure an ethical, sustainable \textbf{seafood} industry and a thriving \textbf{ocean}.
Marilu Cristina Flores is the Senior \textbf{Oceans} Campaigner at Greenpeace US While an appetite for tuna may seem endless, what isn't endless are the numbers.
Many tuna \textbf{stocks} have dramatically decreased over the last few decades as consumption has trended upward.
In some countries, a consumer can walk into any grocery \textbf{store} and find hundreds, if not \textbf{thousands}, of cans and pouches of tuna on the shelves.
Some even carry the little blue tick meant to reassure us that the product we are about to purchase is sustainably sourced.
But as the demand is expected to grow even more within the next few years, we ask, how sustainable is tuna?
Can tuna populations survive at this rate of consumption?
And what is the human cost of cheap tuna?
The answers may leave a bad taste in your mouth.
The value of the global tuna industry is expected to reach \$49 billion by 2029, with the U.S. accounting for the largest tuna market globally.
There are fifteen \textbf{species} of tuna, and these highly migratory \textbf{animals} can be found in \textbf{oceans} around the world.
In the wild, tuna can take up to five years to reach breeding maturity, and while a female can lay up to 10 million eggs per year, only two in every 30 million will reach adulthood.
With all this growth in the industry led by a seemingly insatiable appetite for tuna, numerous reports and studies have raised concerns about global tuna populations.
The latest estimates by the “ FAO ’s Status of World \textbf{Fisheries} and Aquaculture Report ” show that as of 2019, about a third of the principal commercial \textbf{stocks} of tuna were being \textbf{fished} at biologically unsustainable levels.
This report also noted that tuna \textbf{fishing} \textbf{fleets} “ continue to have significant overcapacity. ” In 2020, the International Commission for the Conservation of Atlantic Tunas ( ICCAT ) reported that the \textbf{stock} of Atlantic bluefin tuna had plummeted to 13 % of its levels from 70 years prior.
Bycatch—fish or other marine \textbf{species} that are \textbf{caught} unintentionally while \textbf{fishing} for specific \textbf{species} or \textbf{sizes} during commercial \textbf{fishing} operations—is another pressure point on tuna and other marine \textbf{life} populations.
Most bycatch, including other \textbf{species} of \textbf{fish}, \textbf{sea} \textbf{turtles}, \textbf{sharks}, rays, dolphins, and even seabirds, is not wanted, cannot be sold or kept, and the carcasses are often disposed of at \textbf{sea}, turned into fishmeal for \textbf{fish} farms, or turned into pet food.
Longlining is one of the most commonly used \textbf{methods} for \textbf{fishing} tuna.
It has an astounding 20 % bycatch rate.
On the other hand, purse seining is often used to \textbf{catch} Skipjack, the smallest of the tuna \textbf{species}.
Skipjack tuna often shoal together with young bigeye or yellowfin tuna.
Consequently, the purse-seining \textbf{method} used to collect skipjack also results in the capture of these other \textbf{species}, contributing to population \textbf{decline} as the young \textbf{fish} are removed from the \textbf{ecosystem} before they can breed. \textbf{Fish} Aggregation Devices ( FADs ), often used in tandem with purse-seining, exacerbate the problem by further increasing the amount of bycatch.
With such a high rate of bycatch from just these two \textbf{species}, the economic and ecological consequences of these detrimental \textbf{activities} are easy to extrapolate.
Bycatch stymies efforts to recover overfished \textbf{stocks}, endangers protected \textbf{species} like \textbf{whales} and \textbf{sea} \textbf{turtles}, and harms key \textbf{fish} \textbf{habitats} by removing coral.
Bycatch can potentially alter the availability of prey, affecting marine \textbf{ecosystems} and \textbf{fisheries} production.
Bycatch can also have significant economic and social consequences for \textbf{fishers} and their communities, such as the premature closure of a \textbf{fishery} due to the high bycatch of a non-target \textbf{species}.

% matched lemmas: activity, animal, board, catch, condition, decline, document, ecosystem, fish, fisher, fishery, fishing, fleet, habitat, life, method, ocean, port, sea, seafood, shark, size, specie, species, stock, store, thousand, turtle, vessel, water, whale
\end{textsample}
