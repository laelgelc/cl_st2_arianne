\begin{textsample}{POS Dim 4 – human – Score 28.00 – t329\_human.txt}  \label{ex:f4_pos_009}
In the heart of the Indian Ocean, there is a hidden underwater bank teeming with \textbf{life}.
The Saya de Malha Bank is part of the Mascarene Plateau, a continuous shallow ridge connecting Seychelles in the \textbf{north} to Mauritius and Réunion in the south.
Curious \textbf{creatures} such as pygmy blue \textbf{whales} breed in the \textbf{area} and the deep \textbf{waters} surrounding the bank are rich in nutrients, supporting sperm \textbf{whales}, flying \textbf{fish} and tuna.
But the bank can be difficult to get to that observations and studies of this \textbf{life} are few, and from long ago.
The Saya de Malha Bank is known for supporting the world ’s largest seagrass meadow – and therefore one of the biggest carbon sinks in the \textbf{ocean}.
Seagrass meadows account for less than 0.2 % of the world ’s \textbf{oceans}, but take up approximately 10 % of the carbon buried in \textbf{ocean} sediment each year.
On one \textbf{hectare}, seagrasses can \textbf{store} up to twice as much carbon as terrestrial forests.
By keeping carbon safely locked in the \textbf{seabed}, seagrass meadows help slow down climate breakdown.
Worldwide, they are critical \textbf{feeding} and breeding grounds to a wealth of \textbf{wildlife} from the easy-going dugong to sleek tiger \textbf{sharks} and a colourful assemblage of \textbf{fish}.
Governments around the world have recognized Saya de Malha as an Ecologically and Biologically Significant \textbf{Area}.
The \textbf{seabed} is under shared governance of Seychelles and Mauritius, while the \textbf{water} flowing through the seagrass meadows is international \textbf{waters}.
The Greenpeace \textbf{ship} Arctic \textbf{Sunrise} has been at the Saya de Malha Bank to \textbf{map} and research the \textbf{wildlife} of the \textbf{region} with an international team of scientists.
With the help of binoculars and hydrophones, they ’ve been looking for \textbf{whales}, \textbf{sharks}, seabirds and \textbf{turtles}.
The team also collected \textbf{water} samples for environmental DNA \textbf{monitoring}.
This novel \textbf{method} helps trace \textbf{fish}, \textbf{sharks} and \textbf{whales} by the skin, poo, scales and other stuff they leave behind.
Mapping \textbf{animals} by their traces complements the visual and acoustic \textbf{surveys} by detecting \textbf{species} that are elusive, prefer the \textbf{depths} of the \textbf{ocean}, or are otherwise not easy to spot.
The global \textbf{oceans} are under unprecedented pressure – and this \textbf{region} is no exception.
For example, \textbf{shark} populations in the global \textbf{oceans} have \textbf{declined} by a staggering 71 % in just a few decades.
Long lines studded with hundreds of hooks, or enormous purse seine \textbf{nets}, often ensnare \textbf{sharks} as bycatch, and their use has doubled in the last half century, while the number of oceanic \textbf{sharks} \textbf{caught} in them has approximately tripled.
We urgently need a vast network of \textbf{ocean} \textbf{sanctuaries}, free from destructive human \textbf{activity}, where \textbf{sharks} and other marine \textbf{life} can recover.
During our voyage to the Indian Ocean, we will show what ’s at \textbf{stake} and call on governments around the world to agree on a strong Global Ocean Treaty.
Laura Meller is an Ocean policy advisor with Greenpeace Nordic.

% matched lemmas: activity, animal, area, catch, creature, decline, depth, feed, fish, hectare, life, map, method, monitoring, net, north, ocean, region, sanctuary, seabed, shark, ship, specie, stake, store, sunrise, survey, turtle, water, whale, wildlife
\end{textsample}
