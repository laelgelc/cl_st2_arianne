\begin{textsample}{POS Dim 4 – human – Score 27.00 – t313\_human.txt}  \label{ex:f4_pos_012}
The Saya de Malha Bank is a place that few of us have heard of, let alone been to – and as I sit here on the Greenpeace \textbf{ship} Arctic \textbf{Sunrise}, I feel privileged to say that I ’m one of the few people who has.
A remote \textbf{area} in the middle of the Indian Ocean, between the Seychelles and Mauritius, this underwater plateau the \textbf{size} of Belgium is home to the world ’s largest seagrass meadow, some of the few shallow \textbf{water} corals so far away from land and an abundance of marine \textbf{life}.
It ’s clear to us at Greenpeace, as well as to millions of people around the world, that the current rules and laws are not only failing to protect the \textbf{oceans}, but hastening their \textbf{decline}.
That ’s why we need a strong Global Ocean Treaty.
A treaty that would enable the creation of a network of \textbf{ocean} \textbf{sanctuaries} in international \textbf{waters} : \textbf{areas} free from destructive \textbf{fishing} and other \textbf{threats} to marine \textbf{life}.
Governments are negotiating this treaty at the UN already – we have an opportunity to make history if we get this right.
Join us in calling for governments around the world to take action to protect our \textbf{oceans} : join over 3.5 million people and sign the petition.
Daniel Bengtsson is \textbf{Oceans} campaigner for Greenpeace Nordic on \textbf{board} the Arctic \textbf{Sunrise} in the Indian Ocean The bank provides feeding \textbf{habitats} for endangered \textbf{turtles} and breeding grounds for majestic sperm \textbf{whales} and pygmy blue \textbf{whales}.
Seagrass meadows occupy a vanishingly small \textbf{area} of our \textbf{oceans} but capture up to twice as much carbon \textbf{dioxide} as forests on land, making them extremely important for our climate and the balance of the \textbf{ocean}.
This applies to the cold-water eelgrass meadows in Sweden, where I come from, as well as seagrass meadows in tropical \textbf{waters}.
They function as nurseries for vulnerable cod spawn in the \textbf{north} and feeding grounds for \textbf{sea} \textbf{turtles} and dugongs in the Indian Ocean.
But these \textbf{ecosystems} are in \textbf{decline} across the globe due to human \textbf{activity}, making it crucial to protect the remaining \textbf{areas}.
The Saya de Malha Bank has been identified as an ecologically and biologically significant \textbf{area} of global interest by scientific experts.
Places like these could be safe havens for marine \textbf{life}, protected in a vast network of \textbf{ocean} \textbf{sanctuaries} across our blue planet.
But the rich \textbf{wildlife}, especially the shoals of tuna that pass by on their journey through the high \textbf{seas}, attracts the real predators : the \textbf{fishing} industry.
A few powerful \textbf{fishing} nations are depleting marine \textbf{life} around the world, and this hotspot in the heart of the Indian Ocean is no exception.
Industrial \textbf{fishing} \textbf{vessels} from the EU, mainly Spain and France, \textbf{fish} for yellowfin tuna, a population that has been classified as overfished for several years.
These \textbf{vessels} use huge \textbf{fishing} \textbf{nets} that can stretch for 2km and reach 200m deep.
The \textbf{net} is placed in a ring around a school of \textbf{fish} and pulled together from below – scooping up pretty much anything that gets in its way.
Turtles, \textbf{whales} and \textbf{sharks} can be \textbf{caught} up in the \textbf{net} as ‘ bycatch ’ and young yellowfin tuna are trapped before they have a chance to reproduce.
The other type of destructive \textbf{fishing} that takes place around Saya de Malha Bank is longline \textbf{fishing}.
This \textbf{method} is used by around 500 \textbf{ships}, from distant \textbf{water} \textbf{fleets}, using a single long line, anywhere from 50 to 120km long, with \textbf{thousands} of hooks.
Can't imagine that?
It ’s like 1,000 football pitches, laid end to end.
The biggest problem with this \textbf{fishing} \textbf{method} is that it also \textbf{catches} and kills many other \textbf{animals} as bycatch, especially already endangered \textbf{sharks}.
We must ensure that rich companies and nations stop this destruction of \textbf{life} in the \textbf{oceans}, which not only impoverishes \textbf{wildlife} but also the coastal communities that are truly dependent on small-scale \textbf{fishing} for survival.
It ’s critical that we protect important \textbf{areas} of the \textbf{ocean} to give marine \textbf{life} a chance to recover and thrive.
Today, there is no legal mechanism for creating \textbf{ocean} \textbf{sanctuaries} in international \textbf{waters} and less than 1 % of the global \textbf{ocean} is actually protected from destructive \textbf{fishing}.
Due to a widespread failure to adhere to scientific advice, the \textbf{fishing} industry is free to continue destroying marine \textbf{ecosystems} through destructive \textbf{fishing} \textbf{methods} and overfishing in sensitive \textbf{areas}.

% matched lemmas: activity, animal, area, board, catch, decline, dioxide, ecosystem, fish, fishing, fleet, habitat, life, method, net, north, ocean, sanctuary, sea, shark, ship, size, sunrise, thousand, threat, turtle, vessel, water, whale, wildlife
\end{textsample}
