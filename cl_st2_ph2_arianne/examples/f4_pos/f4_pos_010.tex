\begin{textsample}{POS Dim 4 – human – Score 28.00 – t911\_human.txt}  \label{ex:f4_pos_010}
You probably know that climate change is melting Arctic \textbf{ice} with astonishing speed.
And while some hear a warning bell, others see a business opportunity.
As Arctic \textbf{ice} disappears, oil companies and \textbf{fishing} \textbf{fleets} are moving further \textbf{north} than ever before, keen to exploit the unexplored \textbf{ocean} opening up at the top of the world.
You can act now.
Tell the Norwegian government to protect the pristine Arctic \textbf{ecosystem} around Svalbard from destructive \textbf{fishing}.
Emily Buchanan is a creative producer for Greenpeace UK.
This blog was originally posted by Greenpeace UK.
All rights \textbf{reserved}.
You probably also know how wrong it is to take advantage of melting \textbf{ice} to drill for more of the stuff that caused the problem in the first place.
But did you know that industrial \textbf{fishing} presents its own set of risks?
Here are some of the lesser-known ways destructive \textbf{fishing} \textbf{fleets} threaten the Arctic Ocean : \textbf{Bottom} trawlers are a kind of heavy \textbf{fishing} gear that gets dragged along the \textbf{seabed}, smashing up \textbf{creatures} like \textbf{sea} pens, soft corals and basket stars.
Greenpeace has \textbf{documented} an increasing number of \textbf{bottom} trawlers in the sensitive \textbf{waters} around Svalbard in the Norwegian Arctic.
When \textbf{bottom} trawling happens in a pristine and largely undiscovered \textbf{ecosystem}, the consequences can be deadly : unknown \textbf{species} could be lost forever, with only a trail of destruction left behind.
Many Arctic \textbf{fish} live near the \textbf{seabed}, so when \textbf{bottom} trawlers hunt for cod or haddock, they scoop up everything in their path, including other \textbf{species} that are then needlessly \textbf{caught} and killed.
One example is the mysterious Greenland \textbf{shark}, a slow and mostly blind \textbf{creature} that grows only a centimeter a year.
Currently classified as ‘ near threatened, ’ this unique \textbf{animal} must not be lost as a side effect of industrial \textbf{fishing}. \textbf{Fishing} \textbf{vessels} throw a lot of debris into the \textbf{oceans}, especially plastic.
Old \textbf{nets}, lines and traps are dumped overboard when they can't use them anymore – with untold impacts on marine \textbf{life}.
In the summer of 2014, there were two incidents of polar bears becoming entangled in plastic \textbf{fishing} gear on the \textbf{coast} of Svalbard.
Luckily they were freed, but tens of \textbf{thousands} of \textbf{nets} are thought to have been discarded in these fragile \textbf{waters} in the last few years.
More industrial \textbf{fishing} also means more underwater noise.
The noise of \textbf{fishing} \textbf{vessels}, amplified underwater, can deeply affect marine \textbf{mammals}. \textbf{Species} like narwhals and beluga \textbf{whales} use sound to communicate and sense the world around them – to find food, or sound the alarm.
But noise pollution can disrupt their normal behaviour, and force them to flee to quieter \textbf{areas}.
What ’s more, chronic noise pollution can make it difficult for these near threatened \textbf{mammals} to breed.
We know the \textbf{web} of \textbf{life} is all interconnected, but not much has been researched into how land and \textbf{water} \textbf{ecosystems} affect each other in the Arctic.
What we do know though is that they are very interdependent.
To cite just one example, \textbf{bottom} trawling crushes the soft \textbf{species} on the \textbf{seabed} like clams and worms.
That means there ’s less food available for walruses, which eat those clams.
At the same time, walruses are already suffering the loss of \textbf{sea} \textbf{ice} that causes them to crowd on land, risking deadly stampedes.
Adding the impact of industrial \textbf{fishing} into the already stressed Arctic \textbf{ecosystem} could be devastating.

% matched lemmas: animal, area, bottom, catch, coast, creature, document, ecosystem, fish, fishing, fleet, ice, life, mammal, net, north, ocean, reserve, sea, seabed, shark, specie, species, thousand, vessel, water, web, whale
\end{textsample}
