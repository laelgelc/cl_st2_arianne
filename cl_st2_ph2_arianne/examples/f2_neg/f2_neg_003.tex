\begin{textsample}{NEG Dim 2 – human – Score 1.00 – t951\_human.txt}  \label{ex:f2_neg_003}
Today, Greenpeace is throwing the biggest \textbf{clothes} swap party ever seen in Austria and Germany : In over 40 cities, from the Danube to the Danish border, more than 10,000 expected participants will exchange some 50,000 trousers, skirts, t-shirts and evening tops.
The interactive map is part of a new Greenpeace campaign encouraging alternatives to fast \textbf{fashion} consumption : Re-sell your \textbf{clothes} instead of buying new ones, repair them instead of dumping them, buy certified organic fabrics instead of cheap, mass-produced \textbf{clothes}.
Everyone can join the movement and become a consumer ambassador ( in German ). “ Everyone holds the reins in their hands and can get active with unconventional \textbf{fashion} ideas, ” added Brodde.
Clothing consumption doubles every ten years.
We increasingly buy ever-cheaper clothing ; a t-shirt for three euros or kids jeans for eight.
These prices are what we have come to expect, but this drives the cycle of consumption even faster.
The cheaper we buy \textbf{clothes}, the easier we throw them away : On average, we wear an evening top 1.7 times before we discard it ( in German ).
After their short lifespan, three out of four garments will end up in landfills or be incinerated.
Only a quarter will be recycled.
The consumption campaign is part of the Greenpeace Detox campaign : fighting for clean textile production and a responsible approach to \textbf{fashion}.
Carolin Wahnbaeck is a Communications Specialist at Greenpeace Germany.
There ’s no need to buy new \textbf{clothes} if you want to change your look.
Rather than buying the latest fast \textbf{fashion} it -pieces, try this : rent your \textbf{clothes}, share them, exchange them.
Today, Greenpeace is inviting people to discover these alternatives with a mass clothing exchange in Germany and Austria.
For information and local updates, please check our Facebook events ( in German ). “ \textbf{Clothes} swap parties are the answer to our ever-increasing \textbf{fashion} consumption.
They satisfy our desire for a new look without producing huge piles of garbage or poisoning the water, ” says Kirsten Brodde, textile expert at Greenpeace Germany.
As the global textile industry continues to increase production, ever-greater quantities of liquid waste products are finding their way into our environment, poisoning precious freshwater resources.
In China, the world ’s largest textile producer, some two-thirds of freshwater resources are already contaminated with hazardous chemicals.
The textile industry is one of the main sources of this pollution.
REVOLUTIONISE your wardrobe!
About 40 % of the \textbf{clothes} in our closets are rarely or never worn ( in German ).
Get active and give your forgotten pieces a new life!
You can get started by downloading our wardrobe tags ( in German ).
Donate, repair or sell your rarely worn garments.
Or simply exchange them – \textbf{clothes} swapping events are now everywhere!
Secondhand and vintage stores, clothing repair and eco-fashion stores are spreading fast, with more than 250 eco-fashion shops and nearly 500 secondhand stores in Germany and Austria alone.
These alternatives can be easily found on our interactive, frequently updated map ( in German ).
Think bigger than fast \textbf{fashion}.
Here ’s how :

% matched lemmas: clothes, fashion
\end{textsample}
