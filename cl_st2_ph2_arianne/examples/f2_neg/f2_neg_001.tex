\begin{textsample}{NEG Dim 2 – human – Score 1.00 – t927\_human.txt}  \label{ex:f2_neg_001}
Good news!
It ’s getting a little easier to find \textbf{clothes} produced by environmentally conscious discounters.
Our German office did the research and announced which \textbf{supermarket} chains are “ Detox Trendsetters ” and who made the “ Detox Losers ” list.
Rewe/Penny Tchibo Coop Edeka/Netto Norma Metro/Real Interspar Migros ( *None of these companies have committed to detoxing their supply chains until 2020. ) Brian Adams is a Global Communications Strategist at Greenpeace East Asia. ( Don't worry, if you ’re on the run you can just scroll down to see which companies are making the grade and who fell far short of our Detox goals. ) Affordable \textbf{clothes} can also be clean – that ’s now proven by several large international \textbf{retailers} like Aldi and Lidl.
Thanks to pressure by the Greenpeace Detox campaign these major \textbf{retailers} are in the process of banning extremely harmful chemicals from textile production, publishing discharge data, and starting recycling programs.
They are tightly following their Detox commitments to clean up production by 2020.
Now it ’s time for premium \textbf{supermarkets} like Edeka to join the Detox trend.
We are also happy to announce that \textbf{supermarket} chain Kaufland publicly commits to banning toxic chemicals from its textile production by 2020.
Kaufland is the 33rd international \textbf{brand} joining the Greenpeace Detox movement, representing roughly 15 percent of the global fabric production.
Kaufland has 1,300 shops in Germany and Eastern Europe and is quickly heading for the top of the Detox list by also committing to increase its share of high quality clothing – which will last longer and is easier to recycle.
When you consider the increasing amount of old \textbf{clothes}, this is more important than ever.
Durability and repairable designs are the future of the industry.
Several Detox trendsetters are already investing resources to take on responsibility for the whole life cycle of their products.
This includes recycling-friendly designs without toxic chemicals, in-store take back programs to receive recyclable fabrics, and new designs that use fabrics recovered from recycling.
The reuse of fabrics helps to reduce the consumption of fresh fibres and protect the environment.
Now it ’s your turn.
Take your \textbf{clothes} back to the \textbf{supermarkets}.
The more you do that the faster \textbf{supermarket} chains will start recycling.
Aldi Lidl

% matched lemmas: brand, clothes, retailer, supermarket
\end{textsample}
