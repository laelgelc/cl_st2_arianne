\begin{textsample}{POS Dim 3 – human – Score 48.00 – t393\_human.txt}  \label{ex:f3_pos_007}
Today is the International Day of Awareness of Food Loss and Waste.
Many of us instinctively feel guilty about throwing out food, especially when “ 690 million people in the world went hungry in 2019 – up by 10 million from 2018, an \textbf{increase} by nearly 60 million in five years, ” according to the World Health Organisation ( WHO ).
And, COVID-19 \textbf{will} likely make the situation worse, according to the Food and Agricultural Organisation of the United Nations ( FAO ).
Some farmers and farm workers simply had restricted access to farm fields due to various lockdown \textbf{measures}.
Some have faced harsh working conditions that were inadequate to prevent COVID-19 transmission in food manufacturing plants.
Labour shortages have \textbf{meant} that there \textbf{may} not be enough farm workers at harvest \textbf{time}.
For example, farmers in Canada reportedly had difficulties recruiting farm workers to harvest their \textbf{products} on \textbf{time}, unfortunately leaving crops rotting in the fields and generating food losses.
The economic impacts of the COVID-19 have also reportedly left many people and communities in a fragile state with \textbf{increased} poverty, reduced income, and, in some cases, more food insecurity than before.
In these difficult moments, food \textbf{waste} and losses generated by the industrial food \textbf{system} are even more egregious and unacceptable.
The wasting of over 30 % of the food produced globally each year shows that there is enough food in the world to feed everyone, but it does not always get to those who \textbf{need} it.
Instead, it is \textbf{wasted}.
Such \textbf{waste} translates into more deforestation, more irrigation and therefore greater \textbf{use} of water, more pressure on \textbf{soil}, more air pollution from fertilizers and \textbf{pesticides}, and higher \textbf{greenhouse} \textbf{gas} \textbf{emissions}.
Food loss and \textbf{waste} accounts for roughly 8 % of global \textbf{emissions}, which is comparable to global \textbf{emissions} from road \textbf{transport}, and 4 \textbf{times} global \textbf{emissions} from aviation.
Also, the industrial and global food \textbf{system} has lengthened the \textbf{supply} \textbf{chain}, resulting in more potential environmental damages due to \textbf{waste} and losses.
Food \textbf{waste} and loss is part of an industrialised food \textbf{system} \textbf{model} that continues to justify \textbf{increased} \textbf{production} to ‘ feed people ’ while dumping a third of the food in the trash.
Food insecurity is in \textbf{fact} due to inequality, not lack of \textbf{production}.
Therefore, the true \textbf{solution} is to change the food \textbf{system} \textbf{model}, which \textbf{will} minimise food loss and \textbf{waste}.
Instead of an industrialised \textbf{commodity} trade \textbf{model}, we must \textbf{start} to relocalise \textbf{production} and \textbf{consumption} of ecologically produced food.
Individually, we can do our part to reduce food loss and \textbf{waste} by paying more attention to how we treat the food we eat.
Some actions we can take are planning our meals before grocery shopping, buying no more than what we \textbf{need}, and being creative with our food leftovers.
We \textbf{need} to avoid being seduced by bulk offers which might \textbf{end} up as \textbf{waste}.
Another tip is to buy fresh seasonal produce directly from local farmers such as at farmer ’s \textbf{markets} or Community Supported Agriculture and avoid extraneous steps along the \textbf{supply} \textbf{chain} where food can be \textbf{wasted}.
The \textbf{business} \textbf{model} drive to continually produce ‘ cheaper ’ food has established a dangerous mindset that our food isn't really valuable, so it is unsurprising that food loss and \textbf{waste} is a central byproduct of their \textbf{business} \textbf{model}.
Instead, we \textbf{need} to appreciate our food, the natural environment in which it grows, and all of the work that goes into growing and preparing it.
Instead of a ‘ cheap food ’ \textbf{business} plan we \textbf{need} to convince our governments to adopt a fair \textbf{price} food \textbf{system} for farmers and farm workers while respecting our planet to continue to produce healthy and nutritious food.
A just and green recovery can help to reduce food loss and \textbf{waste} through policy adoption and community \textbf{investments}.
For example : Shorten \textbf{supply} \textbf{chains} to relocalise food \textbf{systems} by encouraging and expanding direct-to-consumer \textbf{systems} like farmers \textbf{markets}.
And, it is some of the same systemic \textbf{problems} behind hunger that lead to food \textbf{waste} and its impact on the environment.
The FAO estimates that if food \textbf{waste} were a country, it would be the 3rd largest \textbf{greenhouse} \textbf{gas} emitter in the world.
There is much that we as individuals can do to tackle food \textbf{waste} at home, and each action matters.
But the systemic \textbf{problems} behind the large \textbf{volumes} of food \textbf{waste} also \textbf{need} rectifying.
To fix these \textbf{problems} we \textbf{need} to act together.
Support cities and municipalities to enable residents to reconnect with food through community projects such as allotment gardens, garden rooftops, collective community kitchens, and \textbf{zero} food \textbf{waste} \textbf{programs}.
Provide \textbf{better} harvesting and storage facilities at farm and rural \textbf{level} to minimise food losses.
Support the adoption of ecological farming practices that protect crops without harming the environment and the \textbf{health} of farmers and farm workers.
Ensure a fair pricing \textbf{system} for farmers.
Let ’s not \textbf{waste} our opportunity!
In 2021, the United Nations \textbf{will} organise a Food System Summit.
It could be an opportunity at global, national, and local \textbf{levels} to put the issue of food loss and \textbf{waste} on the \textbf{public} agenda.
We can change our current failing food \textbf{system} into a green and just one that is not only more respectful of ecological boundaries, but also ensure food justice for all.
The corporate \textbf{business} as usual approach must be exposed, challenged, and replaced with food sovereignty to ensure the \textbf{needs} of producers, \textbf{consumers}, and nature are at the heart of our food \textbf{system}.
Let ’s not \textbf{waste} such an opportunity!
Tell us what you are doing to reduce food \textbf{waste} in your daily lives.
What are some local groups and organisations doing to create a \textbf{better} and just food \textbf{system}?
Share your \textbf{good} \textbf{news} to inspire us all. Éric Darier \& Monique Mikhail are Senior \textbf{Strategists} at Greenpeace International based in Montreal and San Francisco respectively.
Generally speaking, food \textbf{waste} describes food we throw away at home, as well as food lost at various points along the food \textbf{chain}, including on farms, in manufacturing, during \textbf{transport}, in storage and at retail centres.
Experts who look at the overall issue from the field to our \textbf{plates} \textbf{use} the \textbf{term} ‘ food loss and \textbf{waste} ’ ( FLW for short ).
While exact \textbf{data} about food loss and \textbf{waste} is hard to obtain, an \textbf{increasing} \textbf{number} of \textbf{studies} are shedding light on the \textbf{problem}.
The FAO has estimated that at least 30 % of the food produced globally – worth around \$940 \textbf{billion} USD – is \textbf{wasted} annually.
The situation is worse in wealthier countries like in North America where food \textbf{waste} has been calculated to \textbf{cost} \$278 \textbf{billion} USD each year and could feed 260 million people!
Large agribusiness and food and beverage ( F\&B ) \textbf{companies} who \textbf{control} and shape the industrial food \textbf{system} play a major role in the \textbf{scale} of food loss and \textbf{waste}.
Both global \textbf{retailers} and agribusiness conglomerates exercise enormous influence over the \textbf{supply} \textbf{chain}, forcing farmers – many in low-income nations – to shoulder the global \textbf{burden} of food loss and \textbf{waste} in international \textbf{supply} \textbf{chains}.
In high-income nations, large \textbf{companies} \textbf{using} the industrial farming \textbf{model} disproportionately benefit from \textbf{public} financial support and \textbf{tax} \textbf{breaks}.
Big agribusiness and F\&B \textbf{companies} further contribute to systemic food losses at the farm \textbf{level} because food \textbf{commodity} \textbf{prices} are kept too low.
Their drive to lower the \textbf{cost} of raw \textbf{commodities} has the perverse \textbf{effect} of making food loss and \textbf{waste} a common side \textbf{effect} of our \textbf{broken} industrial food \textbf{system}.
Sometimes, entire fields of crops go to \textbf{waste} if they are not cost-effective to harvest and \textbf{transport}.
Growers also sometimes plant more crops than \textbf{market} demand to hedge against weather and pests, further lowering \textbf{prices} in bumper crop years and \textbf{increasing} \textbf{waste} in-field because it ’s not lucrative to harvest.
Despite lacking detailed \textbf{data}, most \textbf{studies} acknowledge that food loss and \textbf{waste} tends to take place more at the \textbf{consumer} \textbf{level} in richer countries and more up-stream at the \textbf{production} \textbf{level} in poorer countries.
For example, lack of adequate farm storage for harvested \textbf{products} in poorer countries is one of the causes of food loss.
For wealthier countries, marketing by big food \textbf{companies} encourages \textbf{consumers} to over-purchase bulk food items, which then tend to generate food \textbf{waste} at home.
Such phenomenon is the result of how food is \textbf{marketed} and sold by \textbf{retailers}.
Let ’s remember that food loss and \textbf{waste} is one of the symptoms of a sick food \textbf{system}, not the root \textbf{problem}.
The Covid-19 pandemic has compounded some of the previously existing systemic fragility, inequalities and failures of the industrial food \textbf{system}.
Overall, food \textbf{waste} and loss most likely has \textbf{increased} during COVID-19.

% matched lemmas: billion, break, burden, business, chain, commodity, company, consumer, consumption, control, cost, datum, effect, emission, end, fact, gas, good, greenhouse, health, increase, investment, level, market, may, mean, measure, model, need, news, number, pesticide, plate, price, problem, product, production, program, public, retailer, scale, soil, solution, start, strategist, study, supply, system, tax, term, time, transport, use, volume, waste, will, zero
\end{textsample}
