\begin{textsample}{POS Dim 3 – human – Score 45.00 – t472\_human.txt}  \label{ex:f3_pos_015}
In 1961, when I was 14, working on a science fair project, my father – a geologist and petroleum engineer – explained oil depletion to me.
To grow \textbf{production}, oil \textbf{companies} were drilling deeper and deeper wells, developing \textbf{technologies} to extract more oil from spent fields, and would one day tap into shale rock and the Canadian tar sands to extract the dregs.
I recall my father estimating that the peak might be around 90 mb/d, which now looks close, perhaps slightly optimistic.
About \textbf{half} of global oil \textbf{production} depends on the world ’s \textbf{top} three producer nations ; the US, Russia and Saudi Arabia.
World \textbf{production} outside these three peaked in 2017 at about 52 mb/d, and has since declined by 6 %. “ Not every nation other than the big three have peaked, ” Patterson \textbf{reports}, “ but cumulatively they have peaked. ” Most of Russia ’s oil comes from aging fields in Western Siberia that are in decline, and Minister of Energy, Alexander Novak, has warned that Russia ’s oil \textbf{production} could drop by 40 % by 2035.
Saudi Arabia – in spite of threatening to \textbf{increase} \textbf{production} – also appears to be in decline.
According to Bloomberg, the giant Ghawar oil field in Saudi Arabia is “ fading faster than anyone guessed. ” Last year, Saudi Aramco oil \textbf{company} published financial figures, revealing that Ghawar ’s historic \textbf{production} has declined by 24 % in six years.
Aramco \textbf{reports} a natural decline \textbf{rate} of 8 %, which \textbf{means} their \textbf{production} would fall by \textbf{half} in less than nine years, without investing \textbf{billions} annually into new wells and new \textbf{technology} on marginal sites.
In 2005, Saudi Arabia \textbf{increased} its operating rig count by 144 %, to \textbf{increase} oil \textbf{production} by 6.5 %.
According to a 2019 Geological Survey of Finland \textbf{report}, the world \textbf{average} decline \textbf{rate} on post-peak \textbf{production} is 5 to 7 %, \textbf{meaning} that oil \textbf{production} could plummet to \textbf{half} its current \textbf{volume} in the next 10 to 14 years.
Over the past decade, only a massive, expensive, noxious, water-and-chemical-intensive fracking campaign in US shale fields has kept the “ all liquids ” petroleum peak at bay, at least until 2018.
However, even this ‘ shale boom ” is a ruse, made possible by massive debt and unpaid environmental \textbf{costs}.
Between 2010 and 2018, \textbf{average} fracking \textbf{production} in the US \textbf{increased} by 28 %.
That sounds impressive.
At the same \textbf{time}, water, sand, and \textbf{chemical} injections \textbf{increased} by 118 % – four \textbf{times} faster – leading to massive water and sand depletion, toxic pollution, and mounting debt.
The Finland \textbf{report} points out that “ most oil producers in the U.S. tight oil fracked \textbf{sector} have a negative cash \textbf{flow}, ” \textbf{meaning} that the \textbf{investment} and operating \textbf{costs} to drill new wells exceed revenues.
That \textbf{study} was completed when oil was selling for \$55 per barrel.
Now, under \$30 per barrel, the US fracking \textbf{industry} is unprofitable and crashing.
The US shale \textbf{industry} is essentially a Ponzi \textbf{scheme}, whereby the insiders create \textbf{companies} with massive debt, borrow at cheap \textbf{rates}, operate on a negative cash \textbf{flow}, sell shares in their “ booming ” \textbf{companies} to a naive \textbf{public}, then get out, making millions in profits while the \textbf{companies} go bankrupt.
Between 2012 and 2017, \textbf{companies} in the Permian oilfield in west Texas – Pioneer, Concho, Cimarex, and others – collectively operated at a \$40 \textbf{billion} cash \textbf{flow} deficit.
Nearby Eagle Ford field, where \textbf{production} peaked in 2015 and has declined over 25 % since, loses about \$1 \textbf{billion} per year.
Anadarko, a typical \textbf{company} in the Niobrara shale field in Colorado, has been operating on debt while hemorrhaging cash.
Anadarko ’s stock \textbf{price} grew from pennies to \$112 per share in 2014, making a few insiders filthy rich.
Then, \textbf{burdened} by debt, the \textbf{company} stock collapsed.
In 2018, Occidental Oil \textbf{company} bought Anadarko for \$65 per share, but falling oil \textbf{prices} and the weight of Anadarko ’s debt have since reduced Occidental ’s value by 85 %.
Oil, he explained, was a finite \textbf{store} of condensed organic matter from the bottoms of ancient seas.
The \textbf{industry} had been extracting the highest \textbf{quality} and least expensive oil, but over \textbf{time}, the \textbf{quality} of oil would decline, the \textbf{cost} of finding it would \textbf{increase}, and decades in the future, perhaps in my lifetime, oil would no longer be economic to produce.
According to Robert Rapier at Oil Price, over the last five years, 208 oil and \textbf{gas} producers and 224 oil service \textbf{companies} – most linked to shale oil \textbf{schemes} – have filed for bankruptcy, abandoning some \$209 \textbf{billion} of debt.
Chevron, a notorious climate change denier, recently had to write off \$11 \textbf{billion} linked to its unproductive shale oil assets.
Chevron ’s share \textbf{price} has plummeted over 30 % in six months.
Since 2016, Exxon Mobil ’s stock \textbf{price} has declined by over 50 %, a loss of over \$200 \textbf{billion} in value.
Tar sands \textbf{companies} are not doing much \textbf{better}.
Canada ’s Suncor recently announced a \$2.8 \textbf{billion} write-down, and in the last 18 months their value has been cut in \textbf{half}.
Canada ’s Teck Corporation suffered a billion-dollar write-down and then cancelled a \$20 \textbf{billion} tar sands project in Alberta.
According to a 2019 Ozy / Financial Times \textbf{report}, “ Investments in oil and \textbf{gas} are tanking, ” and \textbf{energy} investors are experiencing a “ crisis of faith. ” The Wall Street Journal stated that oil \textbf{companies} are “ falling out of favor with investors. ” Goldman Sachs announced it \textbf{will} no longer finance Arctic oil development over concern for climate change and the \textbf{fact} that drilling and exploring in the far north is risky and expensive.
International petroleum engineer, Jean Laherrère, who worked for 37 years with France ’s \textbf{Total} oil \textbf{company}, wrote in 2012, “ Technology cannot change the geology of the reservoir, ” and Chemist Chris Rhodes wrote in Chemistry World in 2014, “ Fracking won't plug the gap in crude oil ’s falling figures.
Oil ’s exhaustion is inevitable. ” The oil \textbf{industry} is in decline for completely natural reasons.
The peak days of cheap, high-quality oil are behind us, and extracting the dregs – shale and tar sands – is expensive, dirty, and catastrophic for Earth ’s climate.
Ecologists and environmental activists \textbf{may} cheer the demise of oil, since we know that humanity has to reduce its carbon \textbf{emissions}.
Nevertheless, there \textbf{will} be a \textbf{cost} to the global \textbf{economy}.
According to the Finland \textbf{report}, “ approximately 90 % of the \textbf{supply} \textbf{chain} of all industrially manufactured \textbf{products} depend on the availability of oil derived \textbf{products}, or oil derived services. ” The \textbf{report} warns that the unsustainable \textbf{economics} of the oil \textbf{industry} \textbf{may} crash the entire global financial \textbf{system}.
The oil \textbf{companies} failed to recognize that they should have become diversified \textbf{energy} \textbf{companies} 60 years ago.
The \textbf{price} of that failure \textbf{will} now be paid by every human and every other species on Earth.
The transition to lower \textbf{consumption} lifestyles and renewable \textbf{energy} \textbf{systems} remains more urgent than ever.
Finland Report : “ Oil from a Critical Raw \textbf{Material} Perspective, ” Dr.
Simon Michaux, Ore Geology and Mineral Economics, Geologian tutkimuskeskus, Geological Survey of Finland, December 12, 2019 : pdf. “ Exploring Hydrocarbon Depletion, ” Jean Laherrère, Peak Oil News, 2012. “ The \textbf{end} of cheap oil, ” C.J.
Campbell, J.F.
Laherrère, Scientific American, JSTOR, 1998.
Although we would not technically see the \textbf{end} of all oil on Earth, the cost/benefit ratio would begin to favour other forms of \textbf{energy}.
He told me then, in 1961, that oil \textbf{companies} should be developing other \textbf{energy} sources, that they should consider themselves in the “ \textbf{energy} \textbf{business}, ” not just the oil \textbf{business}. “ The case for peak oil,”Ron Patterson, Peak Oil Barrel, February 13, 2020. “ Peak oil is not a myth, ” Chris Rhodes, Chemistry World, 2014. “ U.S. shale has already peaked for major service \textbf{companies}, David Wethe, World Oil, January 22, 2020. “ Peak oil, 20 years later : Failed prediction or useful insight? ” Ugh Bardi, Energy Research \& Social Science, February 2019. “ A simple interpretation of Hubbert ’s \textbf{model} of resource exploitation, ” U.
Bardi, A.
Lavacchi, \textbf{Energies}, 2009.
Resource depletion curve \textbf{analysis} : J.
Forrester, World Dynamics, ( 1971 ) “ Limits to \textbf{Growth} ” \textbf{report}, D.H.
Meadows, D.L.
Meadows, J.
Randers, W.I.
Bherens, pdf version. “ The Biggest Saudi Oil Field Is Fading Faster Than Anyone Guessed, ” Javier Blas, Bloomberg, April 2019. “ Government Agency Warns Global Oil Industry Is on the Brink of a Meltdown, ” Nafeez Ahmed, Vice / Motherboard, Feb 4 2020.
Since my father knew all this 60 years ago, I suspect that virtually every engineer and manager in the oil \textbf{industry} knew the same \textbf{facts}.
They knew oil was a finite resource, and would eventually run out.
They also knew that burning oil created carbon \textbf{emissions}, which would heat the planet.
In 1965, the American Petroleum Institute warned that CO2 pollution could “ cause marked changes in climate ” with “ catastrophic consequence. ” “ Oil Bankruptcies Are Reaching Worrying Levels, ” Robert Rapier, Oil Price, February 1, 2020. “ Energy Bankruptcy Reports and Surveys, Haynes and Boone, December 31, 2019. “ Why US Energy Investors are experiencing a crisis of faith, ” Harry Dempsey, Ozy / Financial Times, 1, 2019 “ The Hidden Signs That the Oil Industry Is Heading for a Reckoning, ” Geoff Dembicki, Vice, Jan 3, 2020 “ Goldman Sachs to stop financing new drilling for oil in the Arctic, ” Stephanie Kirchgaessner, Guardian, December 16, 2019. “ Chevron takes-10-billion-charge, ” Wall Street Journal, 2020 “ Canadian mining giant withdraws plans for C\$20bn tar sands project, ” Guardian, Feb 24, 2020. “ This 5-year bear \textbf{market} in \textbf{energy} stocks could turn into forever, ” Howard Gold, \textbf{Market} Watch, Feb 19, 2020. “ BP warns of third-quarter charges as it spurs \$10 \textbf{billion} divestment target, ” Reuters, October 11, 2019. “ Schlumberger takes \$12 \textbf{billion} charge as CEO charts new course, Shariq Khan, Reuters, October 18, 2019 Now, 60 years later, all these events have come to pass.
The year of peak oil discoveries is behind us ( 1962 ), the peak of conventional oil \textbf{production} is behind us ( 2005 ), most major oil fields are in decline, oil \textbf{quality} and net \textbf{energy} are in decline, extraction \textbf{costs} are rising, oil \textbf{companies} have gone after the dregs in shale rock and tar sands, and – no surprise to anyone – carbon \textbf{emissions} are heating the planet to catastrophic \textbf{effect}. “ United States to lead global oil \textbf{supply} \textbf{growth}, while no peak in oil demand in sight, ” IEA, 11 March 2019 No Peak Oil For America Or The World, ” James Conca, Forbes, Mar 2, 2017 “ Shale Reality Check ”, David Hughes, Post Carbon Institute, 2018. “ \textbf{Technology} cannot change the geology of the reservoir, but \textbf{technology} ( in particular horizontal drilling ) can help to produce faster, but no more, ” Jean Laherrere, Peak Oil, 2012. “ \textbf{Top} oil firms spending millions lobbying to block climate change policies, ” Sandra Laville, Guardian, Fri 22 Mar 2019 “ Fossil Fuel Giants Claim To Support Climate Science, Yet Still Fund Denial, ” Paul Thacker, Huffington Post, December 18, 2019.
On March 8th this year, Saudi Arabia slashed oil \textbf{prices} after Russia refused to cut \textbf{production} in response to the Coronavirus economic recession.
Saudi Arabia threatened to \textbf{increase} \textbf{production} which \textbf{will} lower \textbf{prices} and undermine both Russian and American \textbf{companies}.
However, this bluster is actually a minor tantrum in a larger story : The natural decline of global oil \textbf{production}.
We live in the era of peak oil, but just as the modern oil \textbf{industry} attempts to deny the \textbf{effects} of carbon \textbf{emissions}, the \textbf{industry} has also found it convenient to deny that oil is a finite resource that \textbf{will} peak and decline.
Since 1956, when Shell geophysicist Marion King Hubbert predicted ( accurately ) the peak of US conventional oil, \textbf{industry} cheerleaders have mocked anyone who dares to speak about the natural peak and decline of oil.
Last year, the International Energy Agency claimed that, “ The United States is increasingly leading the expansion in global oil \textbf{supplies}, ” and promised that a “ second wave of the US shale \textbf{revolution} is coming. ” Forbes magazine, a relentless denier of oil limits, stated in 2017 that “ Peak oil is not in sight, ” and that any impact of limits on \textbf{energy} policy or \textbf{production} “ \textbf{will} not be a force for some \textbf{time}. ” This sort of bravado bolsters oil \textbf{company} stock \textbf{prices}, but ignores geo-physical reality.
Although the natural peak and decline of oil remains inevitable, we won't know the precise moment of maximum oil \textbf{production} until a few years after the event.
However, recent \textbf{production} \textbf{data} suggests that the all-time peak \textbf{production} \textbf{may} have already occurred.
Ron Patterson, a computer engineer, who worked with Saudi Aramco oil \textbf{company}, suggests that the all-time peak \textbf{may} prove to be November 2018, when the world was producing oil at a \textbf{rate} of 84.7 million barrels per day ( mb/d ). \textbf{Production} has declined ever since, and with the Coronavirus pandemic slowing \textbf{economies} everywhere, \textbf{production} \textbf{may} never recover.

% matched lemmas: analysis, average, billion, burden, business, chain, chemical, company, consumption, cost, datum, economics, economy, effect, emission, end, energy, fact, flow, gas, good, growth, half, increase, industry, investment, market, material, may, mean, model, price, product, production, public, quality, rate, report, revolution, scheme, sector, store, study, supply, system, technology, time, top, total, volume, will
\end{textsample}
