\begin{textsample}{POS Dim 3 – human – Score 46.00 – t049\_human.txt}  \label{ex:f3_pos_012}
Nate Hagens is a former editor of the Oil Drum website that discusses the implications of resource limits, he founded the Institute for the \textbf{Study} of Energy \& our Future, taught a popular class at the University of Minnesota, “ Reality 101, ” on the human ecological predicament, and now produces two of the most popular environmental podcasts : The Great Simplification, in which he interviews knowledgeable ecologists and resource analysts, and Frankly, his personal reflections based on what he has learned from the various experts.
Hagens was born in Milwaukee, Wisconsin in 1969.
Since his family moved frequently, following his father ’s military doctor career, young Nate often felt isolated and took solace in his walks in nature.
He became a Vice President at Salomon Brothers \textbf{investment} firm, and then experienced an ecological epiphany, left Wall Street, and began his career in \textbf{energy} and ecology \textbf{analysis}.
I caught up with Hagens at his home in Minnesota to talk about his personal journey and environmental insights.
My next epiphany arose from learning about “ externalities, ” that environmental damages by human activity were not even included in our economic theories, nor in the \textbf{prices} we pay for \textbf{things}.
My interest in oil and human \textbf{systems} became an obsession.
I realized that the \textbf{energy} crisis would also be an economic crisis.
I spent more \textbf{time} \textbf{studying} \textbf{energy} than I did finding new clients or \textbf{studying} the \textbf{markets}, so in 2002 I gave my clients their \textbf{money} back, left Wall Street, and began educating myself on ecology and \textbf{energy} full \textbf{time}.
Two years later, I decided if I was going to spend 40-50 hours a week learning about \textbf{energy} and human \textbf{systems} I might as well get a PhD, so I enrolled at the University of Vermont and got my doctorate in Natural Resources.
That is a curious flip.
Now you ’ve created a career for yourself as a \textbf{public} intellectual, addressing our ecological crisis and \textbf{energy} future.
You call your podcast “ The Great Simplification? ” Explain that idea : The story we face can be understood by a sixth-grader.
In \textbf{fact}, the more education and social conformity one undergoes, the more the consensus trance pulls one away from the simple truth of our situation.
Here is what my \textbf{studies} have revealed to me : We are a social species that self organizes around \textbf{energy} and \textbf{material} surplus.
Of all the inputs to our \textbf{economy}, \textbf{energy} is the largest and most important.
To operate the human \textbf{economy} for a single year, we \textbf{use} about 100 \textbf{billion} boe [ barrels of oil, \textbf{equivalent} ] of fossil carbon and hydrocarbons.
Energy and GDP are over 99 % correlated.
Materials and GDP are 100 % correlated.
The hope that we could globally decouple \textbf{energy} and \textbf{materials} from our \textbf{economy} is delusional.
Some regions, the US and Europe, are decoupling \textbf{energy} and GDP but have achieved this by exporting their energy-intensive \textbf{industries} to Asia.
Globally, \textbf{energy} \textbf{use} and carbon \textbf{emissions} continue to rise.
The \textbf{energy} surplus, primarily fossil fuels, but any \textbf{energy} surplus, enables complexity – global \textbf{supply} \textbf{chains}, trade, electronics, gadgets, \textbf{growth}, networks – all requiring not only cheap \textbf{energy} but more \textbf{energy}.
However, the complexity built up over past centuries \textbf{will} reverse in the coming century.
That was my critical realization.
This historic reversal is what I call “ The Great Simplification. ” Now, we want to replace oil for global warming reasons and embrace a transition to renewables. \textbf{Will} this offset the reversal of fossil \textbf{energy}? : Not so far.
As we have built new \textbf{energy} \textbf{infrastructure}, we have not slowed fossil hydrocarbon extraction, and our extraction \textbf{technologies} are evermore destructive.
We ’re now fracking source rock and digging into the Canadian tar sands, creating more ecological destruction.
Historically, \textbf{energy} transitions take centuries, and we have never stopped \textbf{using} any \textbf{energy} source ; we only add new \textbf{energy} sources, as we are doing now.
The evidence suggests we \textbf{need} to embrace a world of less \textbf{energy}, not just of different \textbf{energy}.
How did you get into the financial \textbf{business}?
Conventional oil has already peaked and is in decline.
Furthermore, hydro dams, solar panels, and wind turbines require the entire global fleet of fossil \textbf{energy} machinery to make steel and cement, to mine and \textbf{transport} the necessary \textbf{minerals}.
This \textbf{may} one day change but as of now producing solar panels with solar panels is not something that ’s in sight.
To avoid an economic crash, governments and banks print \textbf{money} and financial guarantees to keep our \textbf{consumption} at high \textbf{levels}.
However, \textbf{money} is ultimately a claim on \textbf{energy}, and debt is a claim on future \textbf{energy}.
As a global culture, we have amassed over 350 % of our annual GDP in debt, a debt still growing, which ultimately \textbf{will} not be paid back in the era of less \textbf{energy}.
Meanwhile, even though many people know or intuit these \textbf{facts}, we primarily self-organize — as individuals, families, small \textbf{businesses}, corporations, and nation states — to accumulate profits, 99 % tethered to \textbf{energy}, 100 % tethered to \textbf{materials}, 80 % tethered to carbon.
In \textbf{effect}, we have become a blind, hungry “ amoeba ” – an unseeing uncaring superorganism that is out of the \textbf{control} of politicians, billionaires, voters, or institutions.
Explain why debt is a claim on future \textbf{energy}?
All global governments and institutions expect approximately 3 % annual \textbf{growth} into the future.
This \textbf{growth} is necessary to pay the interest on the world debt.
At 99 % \textbf{energy} tether, even allowing for unprecedented \textbf{efficiency} gains, this \textbf{means} the global \textbf{economy} — and \textbf{energy} demand — \textbf{will} double in size in the next 30 years and \textbf{will} double again in the following 30 years.
If those \textbf{growth} expectations are to be realized, if the debts are to be paid, \textbf{energy} \textbf{will} be required at \textbf{increasing} \textbf{levels}, and a 15-year-old human today would expect a quadrupling in the energy/material \textbf{footprint} of humanity by the \textbf{time} she reaches 75.
My work suggests that even the next doubling \textbf{will} not occur in the face of very real \textbf{energy}, \textbf{material}, and environmental limits.
The moment we are no longer able to grow sufficiently to service debt and financial claims, there \textbf{will} be a musical chairs moment in global financial \textbf{systems}.
Debt or “ credit ” allows us to consume resources today that without credit would still be available in the future.
We are producing \textbf{things} that we \textbf{’d} not be able to produce without debt.
The implication is that once the \textbf{markets} no longer trust that governments can be fiscally responsible or that \textbf{growth} can't continue, we have a sharp \textbf{reduction} in economic output akin to the 1930s.
This \textbf{will} be the \textbf{beginning} of the Great Simplification, the largest event ever encountered by a global human culture.
To some people, that \textbf{will} sound depressing.
We can still take meaningful action at many \textbf{scales}.
This simplification is a likely characteristic of the human future.
If we accept this and prepare for it, we can mitigate some of the harm to society.
If we ignore reality, our progeny, and other species ’ progeny, \textbf{will} suffer more.
We \textbf{need} to wake up to Earth ’s ecological limits.
We \textbf{need} to \textbf{start} shedding frivolous \textbf{consumption}, especially, of course, in wealthy nations.
This won't likely happen at national \textbf{scales} but at individual/local \textbf{scales} it \textbf{will} lead to resiliency.
Globally, most people, even as they learn about these issues, are not really concerned about climate change, \textbf{energy} depletion, etc.
Most people are concerned about their livelihood, concerned about their built identities, their personal circumstances.
The Great Simplification story is threatening to almost all built identities, making it unpopular and difficult to \textbf{scale} up preparations or even to talk about.
To many, the ecological crisis feels complex, threatening, abstract, in the future, and has no easy \textbf{answers}, so few people even talk about it.
After high school, I wanted a high paying job to get a \textbf{better} apartment, nicer \textbf{car}, all that superficial \textbf{stuff}, so I \textbf{studied} \textbf{business} in university, with a minor in Chinese language.
Before grad school, I went to China for a year and experienced first hand that most of the world does not live as we do in the US.
In China, I felt the first seed of recognition that perhaps our environmental predicament was linked to \textbf{energy} \textbf{consumption}.
I ’ve learned to pull back on telling the whole story as there is a tradeoff between being accurate and being helpful to people.
I ’ve also learned that there is a ‘ glass ceiling ’ on passing this \textbf{information} upwards in the social hierarchy.
High status humans don't like to hear this message, mostly because they were elected or got rich during the era of \textbf{growth} expectations.
Since there are no easy or profitable \textbf{solutions}, most high status humans don't want to tackle it.
In your podcasts, you talk with other global analysts and activists who share some of your assessment.
What have you learned?
I recently learned that US oil \textbf{production} is in more trouble than I thought.
In the rush to make the US “ \textbf{energy} self-sufficient, ” which has failed, we ’ve effectively been “ draining America first. ” We ’re now fracking to get tight oil from source rock, a process that depletes at 80 % in the first 2 years.
Thus, the \textbf{industry} continually \textbf{needs} to find new places to drill.
The \textbf{amount} of resource hasn't changed, we ’ve just developed a larger, wider straw for sucking the oil from the Earth.
We keep getting liquid out for now, but we \textbf{will} get short warning before we hear that slurping sound.
The serious ecologists I ’ve talked with have helped me understand that climate change isn't the fundamental \textbf{problem}.
It is a symptom of a much larger dysfunction connected to our cultural addiction to \textbf{growth}.
We won't solve or reduce climate disruption with \textbf{technology} unless we pair that with different societal goals.
We \textbf{need} to value the well being of people and of Earth ’s ecosystems rather than just \textbf{material} metrics.
You talk about “ \textbf{energy} blindness ” among politicians, the \textbf{public}, and even activists.
How does a deeper \textbf{energy} awareness change how we might respond to our challenges?
We swim in \textbf{energy} like a fish swims in water : We only notice it if it ’s suddenly unavailable.
Here are some core \textbf{facts} about \textbf{energy} that might change our approach to change itself : \textbf{Energy} primacy : \textbf{Energy} is central in nature and in human \textbf{systems}.
Every living \textbf{thing} on Earth, even every river and climate \textbf{system}, processes \textbf{energy}.
Every single \textbf{good} in our global economic \textbf{system} \textbf{needs} \textbf{energy} to imagine, mine, create, deliver, maintain, run, repair, and dispose of that \textbf{good}.
There are no exceptions.
Every \textbf{product} resulting in GDP \textbf{started} somewhere on Earth with a small fire.
This is true even for our emails and websites, solar panels and windmills. \textbf{Energy} benefits : A barrel of oil provides the same \textbf{total} work as a human working full \textbf{time} for 11 years.
Since humans can be more efficient than mechanical \textbf{systems} at turning muscle labor into useful work, we discount the oil by about 60 %, so a barrel of oil represents about 4.5 years of human work.
Since every year we \textbf{use} 100 \textbf{billion} barrels-equivalent of fossil fuels, we get effectively 400-500 \textbf{billion} human workers added to our economic \textbf{system}, to the 4-5 \textbf{billion} real human workers.
This \textbf{energy} pulse over the last few centuries has led to massive profits, wage \textbf{increases}, and inexpensive \textbf{goods}.
The hydrocarbon \textbf{energy} pulse has goosed the human \textbf{economy} — human activity — to over 1000-times larger than it was five centuries ago.
However, this brings us to : \textbf{Energy} depletion : We are alive during the carbon pulse, our economic \textbf{system} treats this huge yearly benefit as if it were interest, when actually, we are drawing down the principle, Earth ’s natural principle, all the \textbf{energy} and \textbf{materials} we drain from the Earth and \textbf{use} for human \textbf{economy}.
We are drawing down the fossil hydrocarbon \textbf{store} 10-million-times faster than it was trickle-charged by daily photosynthesis millions of years ago. \textbf{Energy} properties : All joules aren't equal.
Different \textbf{energy} \textbf{technologies} have different properties, such as power density, transportability, safety, spatial distribution, and so forth.
We can't just plug and play one kind of \textbf{energy} for another.
For instance, solar and wind are great, mature, and effective, but they produce \textbf{electricity}, only about 20 % of global \textbf{energy} \textbf{use} today.
That share can grow, but many industrial activities — extreme and large-scale heat, ocean barges, heavy trucks and equipment, mining, etc. — cannot easily switch to \textbf{electricity}.
We don't yet have an \textbf{answer} for replacing over \textbf{half} our industrial \textbf{use} of hydrocarbon \textbf{energy}.
Furthermore, it takes \textbf{energy} to get \textbf{energy}.
That ’s true for every living organism.
Globally, our net \textbf{energy} is already declining.
More and more \textbf{energy} is going into extracting, creating, transforming, \textbf{storing}, and transmitting \textbf{energy}.
Then, I got my MBA at the University of Chicago.
I must admit, \textbf{studying} arcane economic formulas felt narrow and lifeless.
My first job at Salomon Brothers was to call anyone that had over \$100 million in assets, and sign them up as \textbf{investment} clients.
Other than great food, I hated life in New York.
I found stocks boring, so I focused on bonds, currencies, \textbf{commodities}, swaps, and derivatives. \textbf{Energy} and well-being : We feel entitled to current large quantities of \textbf{energy}, but after basic \textbf{needs} are met ( about 100 Giga-Joules per capita globally ) there is very little benefit derived from more \textbf{energy} \textbf{use}.
Citizens in the US and Canada \textbf{use} over 300GJ per capita, so there is lots of room to simplify.
We can reduce our \textbf{energy} demands without significantly reducing the general well-being of humanity.
Who should we be listening to, learning from?
People who \textbf{use} an ecology lens rather than \textbf{technology} to seek \textbf{solutions}, people who are skeptical of popular narratives and who don't watch too much TV.
I just talked with Vandana Shiva about the simplification challenges.
She ’s doing amazing work in India, getting communities off the high-energy, industrial \textbf{agriculture} treadmill, getting people back on the land, creating their own local food security — she called it ‘ Yoga of the Earth ’.
A decent human future is possible, but we \textbf{will} all have to do a bit more physical work, and live more simply. ========== • Selected Great Simplification interviews : Agroecology and The Great Simplification : Vandana Shiva Sensemaking, Meaning and Purpose : Daniel Schmachtenberger The Devil is in the Diesel : Art Berman East Africa and the Poly-Crisis : Ayan Mahamoud What happened?
You had some kind of epiphany.
How did you realize that physical limits and ecology were the big issues?
AI, The Shape of Language, and Earth ’s Species : Aza Raskin Peak Fish and Other Ocean Realities : Daniel Pauly • Selected Nate Hagen talks and “ Frankly ” podcasts : The Great Simplification – Full Movie Earth Day Myth and Reality 2021 The Ecological Tarot Deck 2022 More on the Nate Hagens ’ YouTube channel I had a love for the natural world from childhood, but I came across the work of Herman Daly [ Steady State Economics ], and then met him.
He just passed away recently.
He had a giant mind and a bigger heart.
As an economist, he dedicated his life to solving human \textbf{problems}, he more or less invented ecological \textbf{economics}, and he inspired me to look more deeply into \textbf{energy} and environmental limits as keys to all \textbf{economy}.
One of my clients was interested in oil futures, so I researched oil and \textbf{energy} in general.
During two years of grad school in finance/economics, I don't recall the word “ \textbf{energy} ” ever being mentioned.
As I dug into oil and its role in our \textbf{economy}, I experienced three critical realizations : One, I realized how powerful the oil resources were ; secondly, I saw how our entire economic \textbf{system} relied on \textbf{energy}, even though no one talked about it ; and finally, I understood that fossil hydrocarbons would peak and decline in my lifetime.
Connecting these three dots was a gut punch for a 27-year-old who had chosen a Wall Street career.
Our entire human \textbf{economy} depended on a resource that was soon going to decline.

% matched lemmas: agriculture, amount, analysis, answer, beginning, billion, business, car, chain, commodity, consumption, control, economics, economy, effect, efficiency, electricity, emission, energy, equivalent, fact, footprint, good, growth, half, increase, industry, information, infrastructure, investment, level, market, material, may, mean, mineral, money, need, price, problem, product, production, public, reduction, scale, solution, start, store, study, stuff, supply, system, technology, thing, time, total, transport, use, will
\end{textsample}
