\begin{textsample}{POS Dim 3 – human – Score 45.00 – t022\_human.txt}  \label{ex:f3_pos_016}
This \textbf{time} last year I was standing on a riverbank that was literally made of \textbf{clothes} \textbf{waste}, sifting through the rags with fellow \textbf{researchers} in Nairobi, Kenya.
Something caught my eye in the homogenous mass of \textbf{waste} – a label which said ‘ Conscious ’.
A sad \textbf{end} for an item of clothing, which could have been made better, loved more, passed on and repaired – and when it was really no \textbf{good} anymore, taken apart so that the precious resources it was made of could be recovered and recycled.
Is this a matter of financial resources?
Does the \textbf{fashion} \textbf{industry} lack capable and willing engineers and sustainability teams?
No, the profits are huge and during our campaign we have met many dedicated people working for \textbf{fashion} \textbf{brands} or other organisations, who want to change the world as much as we do ; but the \textbf{money} gets funnelled up to careless shareholders and what ’s left mostly gets spent on recolouring Hell on Earth in green.
Where are we today?
A snowstorm of communication with little connection to reality, glossy collections, or tiny, niche activities.
Involve some famous designers!
It ’s so much more sexy than talking about the hazardous \textbf{chemical} \textbf{burden}, non-recyclable \textbf{waste} and invasive microplastic pollution, or the conditions for workers and \textbf{waste} pickers.
So in the \textbf{end} the status quo remains the same, on target for 206 \textbf{billion} pieces by 2030. \textbf{Brands} even \textbf{use} these initiatives to sell their ‘ green ’ \textbf{products} – promotion tools that hide the truth about the destructive fast \textbf{fashion} \textbf{system}, with \textbf{consumers} as the target.
We decided to investigate some of these self-assessed marketing labels, to see whether the \textbf{terms} such as ‘ Sustainable ’, ‘ Green ’ or ‘ Fair ’ can be justified or backed up by \textbf{better} \textbf{production} conditions in the \textbf{fashion} \textbf{supply} \textbf{chain}.
Our \textbf{report} – out today – finds that the majority did not meet the criteria for credibility, and are in a ‘ Greenwash danger zone ’.
But we did find some true gems among the shiny baubles and fakery – and some hints for a \textbf{better} \textbf{direction}.
Credible \textbf{brand} labels from Vaude, Coop, with German \textbf{retailer} Tchibo also close to hitting the mark – back their claims with the \textbf{best} \textbf{standards} available today, and offer more \textbf{options} to \textbf{consumers} that want to be part of the \textbf{solution} – \textbf{better} \textbf{quality}, durable \textbf{clothes}, and choices to rent, to buy second hand, and to repair them.
We also detected the \textbf{beginnings} of a true movement towards traceability, through \textbf{product} labels and webshop \textbf{information} for \textbf{consumers} that shows where the \textbf{clothes} are made and how – right back to the cotton fields.
These signs are not only from the leaders ; the \textbf{brand} H\&M, a by-word for fast \textbf{fashion}, has the \textbf{beginnings} of a traceability \textbf{system}, along with a newcomer on Greenpeace ’s \textbf{fashion} radar, Italian \textbf{brand} Calzedonia.
While there ’s a lot to do to make this traceability more credible, it ’s our dream that the wastewater \textbf{data} from the thousands of \textbf{suppliers} that work with Detox-committed \textbf{brands} \textbf{will} be part of this traceability \textbf{system} too – just as we at Greenpeace already do with our own textiles procurement.
Change won't happen in a day – it is not easy for big \textbf{companies} to \textbf{shift} to a completely different \textbf{business} \textbf{model}.
But in the meantime \textbf{fashion} \textbf{needs} to cut out the greenwash and concentrate on the fundamentals – \textbf{facts} and figures, \textbf{transparency} and traceability – the true bottom line when it comes to tackling the multiple planetary crises and the only possible basis for credible change.
This is the least that is owed to the victims and survivors of Rana Plaza.
Rana Plaza survivors, including some who lost limbs or were disabled in the \textbf{factory} collapse, laid wreaths today for those lost and repeated their demands for compensation and legal redress during protests in Dhaka marking 10 years since one of the world ’s worst industrial disasters.
Viola Wohlgemuth is a Circular Economy and Toxics Campaigner at Greenpeace Germany.
People in Kenya and other countries of the Global South are being overwhelmed by the growing flood of \textbf{used} and rejected \textbf{clothes} from wealthier countries, at \textbf{volumes} well beyond the \textbf{needs} and demands of local \textbf{markets}.
Over the years they have become less and less useful ; while the imported \textbf{clothes} \textbf{used} to be of \textbf{good} \textbf{quality}, now they are poorly made, or stained, \textbf{broken} and unusable.
They are nothing less than plastic \textbf{waste} dumping, bringing a series of disasters – polluting \textbf{waterways} with microplastic fibres, catching on fire in the huge landfills and poisoning the air, or blocking the drains during downpours and aggravating flooding.
But \textbf{fashion} is no stranger to disaster.
Ten years ago today, 1134 people died in the collapsed clothing \textbf{factory} Rana Plaza in Bangladesh.
Considered the biggest modern catastrophe in the \textbf{fashion} \textbf{industry}, these people died making \textbf{clothes} for Western \textbf{consumers}.
This tragedy has come to symbolise the devastating impacts of fast \textbf{fashion}, not only on workers in \textbf{supply} \textbf{chains} but from the whole life \textbf{cycle} of \textbf{fashion} which exploits people and nature from the cradle to the grave.
It led to the creation of Fashion Revolution, now the world ’s largest \textbf{fashion} activism movement, which has the engagement of many non-governmental organisations ( NGOs ), including Greenpeace.
Garment workers, most of whom are women ( making up approximately 80 % of the workforce in the garment \textbf{industry} ) have been raising their voices and mobilising for years to ask for change, better working conditions, greater union rights and fair and living wages.
Survivors of the Rana Plaza collapse are also speaking out and calling out \textbf{brands} that haven't contributed to compensation or changed their \textbf{business} practices.
Our contribution was the successful Detox My Fashion campaign, which unveiled the insidious pollution of \textbf{waterways} and toxic exposure of workers in the global \textbf{supply} \textbf{chain}.
We confronted global \textbf{brands} and \textbf{retailers} with this reality – and together with Detox supporters, activists and NGOs from around the globe and their creative protests, petitioning and advocacy, we \textbf{broke} the silence around hazardous \textbf{chemicals} in the manufacture of clothing – and convinced 29 \textbf{brands} to sign a ‘ Detox commitment ’ to achieve \textbf{zero} discharges of hazardous \textbf{chemicals} into \textbf{waterways} and eliminate their \textbf{use} at \textbf{supply} \textbf{chain} \textbf{factories}.
We sowed a seed that is still growing – there is definitely a before and after Detox in the \textbf{fashion} \textbf{industry}.
However, we quickly realised that it was not enough, we were tackling the ‘ \textbf{fashion} ’ but not yet the ‘ fast ’, this ever-accelerating \textbf{flow} that extracts resources to make garments that are then consumed and discarded as fast as possible, and finally sent far away to become someone else ’s \textbf{problem}.
So we \textbf{started} focusing on this destructive linear \textbf{business} \textbf{model} and called on \textbf{brands} to ‘ close the loop ’ on the throughput of \textbf{material} resources, but first of all to ‘ slow the \textbf{flow} ’.
By then, ‘ circularity ’ had already become a buzzword.
With the EU finally giving value to the idea of optimising the \textbf{use} of depleting resources, \textbf{fashion} \textbf{brands} jumped on the bandwagon.
Yet they took the path of least resistance, making magical promises about closing the loop \textbf{using} non-existent \textbf{recycling} \textbf{technologies} for their plastic fossil-fuel reliant polyester apparel, while the flood of \textbf{clothes} \textbf{waste} from wealthy countries to the Global South continued, disguised as charitable donations.
As for slowing the \textbf{flow}, this was a job for green marketing.
Growing cotton with the tiniest environmental and social improvements became ‘ sustainable cotton ’.
Recycling plastic \textbf{bottle} \textbf{waste} ( what a relief for the food \textbf{industry} ) became ‘ responsible \textbf{materials} ’ – and a great way for those microplastic fibres to speed their way into the ocean.
And to hell with the complications of growing fair-trade organic cotton and textile-to-textile \textbf{recycling}!
The ‘ \textbf{growth} ’ show must go on!

% matched lemmas: beginning, billion, bottle, brand, break, burden, business, chain, chemical, clothes, company, consumer, cycle, datum, direction, end, fact, factory, fashion, flow, good, growth, industry, information, market, material, model, money, need, option, problem, product, production, quality, recycling, report, researcher, retailer, shift, solution, standard, start, supplier, supply, system, technology, term, time, transparency, use, volume, waste, waterway, will, zero
\end{textsample}
