\begin{textsample}{POS Dim 3 – human – Score 43.00 – t269\_human.txt}  \label{ex:f3_pos_018}
From fires made worse by heatwaves and droughts, to floods and mudslides tied to increasingly severe storms, there is no \textbf{doubt} that the climate emergency is here — and everywhere.
As the IPCC \textbf{report} made clear, the \textbf{time} for action to combat fossil fuel-driven climate change is now.
To achieve the green and just future we \textbf{need}, we must work together to hold those responsible accountable.
Despite abundantly clear evidence of the \textbf{need} to reduce plastic \textbf{production} and \textbf{use} to stem the plastic pollution crisis and prevent the worst \textbf{effects} of climate change, Exxon is reportedly planning to massively expand its plastic \textbf{production} capacity and Coke continues to defend its reliance on single-use plastic peddling the illusion that \textbf{recycling} can somehow capture its plastic that so often \textbf{ends} up polluting the natural environment.
This is despite the \textbf{fact} they know that of all of the plastic \textbf{waste} ever produced only 9 % has actually been recycled.
The fossil fuel \textbf{industry} and the \textbf{consumer} \textbf{goods} \textbf{sector} are both working to preserve a \textbf{broken} \textbf{system} that maximizes corporate profits by externalizing environmental and social \textbf{costs}.
Graham : The plastic pollution crisis harms people and this planet.
The \textbf{effects} of plastic in our environment on wildlife, livelihoods and local \textbf{economies}, food \textbf{systems} and culture connected to nature are increasingly known.
But lesser known impacts of plastic, such as those related to climate, human \textbf{health} and on communities living near plastic \textbf{production} facilities, have equal or greater consequences.
Despite the vast challenges ahead, there is reason to be optimistic about the future.
It is clear that the climate and plastic crises are systemic failures that require a \textbf{systems} change.
And for the first \textbf{time}, we are beginning to better understand the intersection between these two crises and the \textbf{companies} responsible.
This knowledge is power and enables a new perspective where we can see the intersections between these issues step back and reimagine the \textbf{system} more broadly.
Taking a more systemic approach is instructive and empowers us to stop having the wrong conversations that delay meaningful action, and begin having meaningful conversations that can move us towards the more just and sustainable \textbf{models} that are desperately \textbf{needed}.
There is maybe no \textbf{better} example of this type of incremental, counterproductive debate than the one being pushed by the fossil fuel \textbf{industry} and \textbf{consumer} \textbf{goods} \textbf{sector} on plastic \textbf{recycling}.
As long as these \textbf{industries} continue to keep \textbf{recycling} at the heart of our conversations around plastic pollution, they protect that status quo by distraction.
The conversation we \textbf{need} to be having is how we \textbf{shift} the \textbf{system} towards \textbf{models} of \textbf{product} delivery that are based on \textbf{reusing} finite, sustainable resources that grow local \textbf{economies}, empower communities and build a new kind of resilience.
And this conversation is already underway with \textbf{reuse} based concepts being proven by small \textbf{business}, adopted by people and increasingly encouraged by governments.
Marian : People, small \textbf{businesses}, civil society and local governments are responding with policies and \textbf{solutions}.
Nearly 500 local governments in the Philippines have issued bans or \textbf{regulations} on plastic \textbf{use}, while Greenpeace Philippines, youth groups, and other allies call for a national plastics ban and the release of non-environmentally acceptable \textbf{products} including major plastic \textbf{products}.
To complement policies and plastic \textbf{reduction}, \textbf{reuse} \textbf{systems} are being implemented by small \textbf{businesses} and communities.
They demonstrate \textbf{reuse} ’s viability in developing nations and people ’s acceptance of such \textbf{models}, challenging the claims of \textbf{consumer} \textbf{goods} \textbf{companies}.
Marian : National, regional and global policies that address the entire life \textbf{cycle} of plastic \textbf{products} \textbf{need} to be enacted.
Southeast Asia and regions with developing \textbf{economies} can mandate plastic \textbf{reduction}, incentivize \textbf{reuse} and packaging-free \textbf{systems}, ban single-use plastic and promote slow circular \textbf{economies}. \textbf{Reuse} and plastic \textbf{reduction} efforts \textbf{need} to be \textbf{scaled} up, too.
Large \textbf{companies} can no longer be allowed to evade accountability for the climate emergency and plastic pollution created and perpetuated by their \textbf{industries}.
They must have concrete, time-bound commitments to reduce single-use plastic \textbf{production} and transition to reusables.
If small \textbf{businesses} and communities can phase out plastic and adopt \textbf{reuse}, multinationals have no excuse not to follow suit.
At every step of its life \textbf{cycle}, plastic \textbf{production} accelerates climate change by releasing \textbf{greenhouse} \textbf{gases} and places vulnerable populations further at risk.
Plastic comes from fossil fuels and the \textbf{consumer} \textbf{goods} \textbf{companies} pushing plastic on our communities from the United States to the Philippines are making the climate crisis worse.
In addition to creating a global pollution \textbf{problem}, plastic also contributes to deteriorating \textbf{health} of people living near petrochemical facilities, and to injustices that disproportionately affect marginalized and low-income groups.
Graham : In the US, we \textbf{need} to immediately stop the expansion of petrochemical facilities to produce plastic and stop exporting oil and \textbf{gas} to make plastic.
The US must fully adopt the Basel Convention and stop exporting its plastic \textbf{waste} abroad, pass the \textbf{Break} Free From Pollution Act and advocate for a strong, legally-binding global plastic treaty that addresses the entire lifecycle of plastic.
And corporations must stop their self-interested focus on \textbf{recycling} and help advance real conversations about \textbf{shifting} their \textbf{business} \textbf{model} towards \textbf{reuse}, including what they \textbf{need} from governments and people to make this happen.
What might comprehensive \textbf{reuse} \textbf{infrastructure} look like in a community?
How can \textbf{reuse} be incorporated into other efforts at the local \textbf{level} to reduce dependence on fossil fuels and build climate resilience?
What would it look like for Coca-Cola to \textbf{end} its dependence on single-use plastic and become a leader in advancing new \textbf{systems} of \textbf{reuse}?
We don't have \textbf{time} to keep having the wrong conversations on plastic.
With a new understanding of how the entire \textbf{system} of plastic \textbf{production} and \textbf{use} works, we are well equipped to begin having the right conversations that put science and the \textbf{public} interest first.
Take action now!
Demand Coke, Pepsi and Nestlé ditch single-use plastics and switch to \textbf{reuse} and refill \textbf{solutions}.
Marian Ledesma is a Zero Waste Campaigner at Greenpeace Philippines Graham Forbes is the Global Plastic Project Lead at Greenpeace USA A \textbf{report} released by Greenpeace USA The Climate Emergency Unpacked : How Consumer Goods \textbf{Companies} are Fueling Big Oil ’s Plastic Expansion, reveals how \textbf{consumer} \textbf{goods} \textbf{companies} like Coca-Cola, PepsiCo, and Nestlé are driving the expansion of plastic \textbf{production} and threatening the global climate and communities around the world.
To learn more about the global and local impacts of the plastic pollution crisis, we ’ve checked in with Marian Ledesma of Greenpeace Philippines and Graham Forbes of Greenpeace USA.
Marian : In the Philippines, plastic \textbf{use} and \textbf{production} has grown steadily, producing more \textbf{greenhouse} \textbf{gas} \textbf{emissions} and plastic pollution in a region already severely impacted by the climate crisis. \textbf{Consumer} \textbf{goods} \textbf{companies} keep releasing single \textbf{use} plastic sachets and smaller disposable \textbf{packaging} to capture low-income \textbf{markets}, displacing \textbf{reuse} \textbf{systems} and resourceful local \textbf{packaging} which were common in the past.
All the while, these corporations ignore calls to make reusable \textbf{alternatives} or \textbf{reuse} \textbf{systems} available to \textbf{consumers}.
Online shopping, food delivery and the influx of imported \textbf{waste} are also adding to plastic \textbf{waste}, and \textbf{companies} are exploiting this to promote false \textbf{solutions} like failed \textbf{recycling} \textbf{schemes} and waste-to-energy initiatives.
Instead of providing people-and-planet-centered \textbf{solutions}, corporations are lobbying to institutionalize \textbf{technologies} which compound plastic pollution ’s impacts with more \textbf{greenhouse} \textbf{gases} and toxic \textbf{emissions}.
Graham : The climate emergency is being felt in every community in the United States, from fires and drought in the West to increasingly severe storms in the East.
And the pain caused by this increasingly severe weather is not experienced evenly across the population with low-income and communities of color bearing the brunt of these impacts.
Plastic comes from fossil fuels and is making the climate crisis worse.
The US has been found to be the largest per capita generators of single-use plastic \textbf{waste} in the world, second only to Australia, and a major producer of plastics and petrochemicals.
The Irving, Texas headquarters for Exxon, the world ’s largest producer of single-use plastic, is just a short-two hour flight from the Atlanta, Georgia headquarters for Coca-Cola, which \textbf{uses} nearly 3 millions \textbf{tons} of plastic a year, sending tens of \textbf{billions} of single-use plastic \textbf{bottles} across the globe, securing the \textbf{company} ’s position as the biggest plastic polluter for 3 years running, according to global \textbf{brand} audits.

% matched lemmas: alternative, billion, bottle, brand, break, business, company, consumer, cost, cycle, doubt, economy, effect, emission, end, fact, gas, good, greenhouse, health, industry, infrastructure, level, market, model, need, packaging, problem, product, production, public, recycling, reduction, regulation, report, reuse, scale, scheme, sector, shift, solution, system, technology, time, ton, use, waste
\end{textsample}
