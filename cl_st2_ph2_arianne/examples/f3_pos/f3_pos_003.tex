\begin{textsample}{POS Dim 3 – human – Score 50.00 – t183\_human.txt}  \label{ex:f3_pos_003}
The world ’s \textbf{top} climate scientists just delivered their rescue plan for humanity, directly to our governments.
The Working Group III Report contribution to Sixth Assessment Report from the Intergovernmental Panel on Climate Change ( IPCC ) focuses on how to reduce further warming and follows the first two \textbf{reports} of the assessment, on the physical science and climate change impacts.
It ’s a thick \textbf{report} on climate \textbf{solutions} that can and must be put into action right now.
And here ’s where you and I come in : we \textbf{need} to make sure this \textbf{report} is not shelved.
It \textbf{needs} to be talked about in every corner of the world, and most importantly : acted on.
The challenges that \textbf{need} to be overcome, overall, are not small.
Meeting the Paris Agreement goals would strand fossil fuel assets, with the economic impacts \textbf{amounting} to trillions of \textbf{dollars}.
Hence, countries, \textbf{businesses} and individuals, who stand to lose wealth, \textbf{may} resist change.
Therefore, ensuring the decision-making process is not unduly influenced by actors with much to lose is key to managing transformation.
Societal awareness and support for climate action have been on the rise.
And so are cases of climate litigation against states, the private \textbf{sector} and financial institutions, as citizens are increasingly turning to courts to access justice and exercise their right to a healthy environment.
In just three years since 2017 the \textbf{number} of climate litigation cases nearly doubled.
And the IPCC finds that “ there is now \textbf{increasing} academic agreement that climate litigation has become a powerful force in climate governance. ” These were some of our highlights from the IPCC \textbf{report}.
But there ’s much, much more!
And it ’s all highly recommended reading.
But then what?
It ’s a unique moment to be alive.
Both the \textbf{problems} and the \textbf{solutions} and bigger than ever before.
But so is the power of determined people who unite for change.
We have eight years to halve global \textbf{emissions}.
And the decisions that either enable or prevent those \textbf{emission} cuts \textbf{will} be made much earlier.
We have already achieved one key milestone, with the breakthrough of solar and wind.
Now we must up our \textbf{game}, big \textbf{time}, to push fossil fuels out of the way, to heal our food \textbf{system}, to protect our forests and land, and to fight for a future that meets the rights and \textbf{needs} of all rather than the greeds of the few.
For a longer Greenpeace briefing on Key takeaways, with references to the Working Group III IPCC \textbf{report}, and Greenpeace calls to action, see here.
Kaisa Kosonen is a Senior Policy Advisor with Greenpeace Nordi c.
The \textbf{starting} point you already know : the climate action our governments and the financial \textbf{sector} have taken by now keeps being too little too late, and we \textbf{need} much, much more, and fast.
Not a single country is yet doing enough.
And it ’s a critical decade when we make or \textbf{break} this.
So then, what is the action \textbf{needed} right now?
Here are our 6 takeaways from the IPCC \textbf{report} on mitigation that we think you should know : This is the \textbf{best} \textbf{news} : we have the \textbf{solutions} to slash more than \textbf{half} of global \textbf{emissions} in just eight years, and to continue from there towards net \textbf{zero} \textbf{emissions}, as is \textbf{needed} to meet the Paris Agreement goal of limiting warming to 1.5°C.
In this critical decade by 2030, the biggest contributions to net \textbf{emission} \textbf{reductions} would come from solar and wind \textbf{energy}, conservation and restoration of forests and other natural ecosystems, climate-friendly \textbf{agriculture} and food, and \textbf{energy} \textbf{efficiency}.
More than \textbf{half} the potential by 2030 comes with low \textbf{costs} ( below 20 USD/tonne ) or even negative \textbf{costs}! \textbf{Costs} below \textbf{zero} \textbf{means} that investing in \textbf{solutions}, like solar and wind, \textbf{will} bring \textbf{cost} savings compared to continuing current ways.
By 2050, huge potential exists, overall, in demand-side strategies that could cut \textbf{emissions} by 40-70 % compared to current policies.
This \textbf{means} \textbf{designing} and repurposing \textbf{infrastructure}, advancing \textbf{technology} adoption and enhancing socio-cultural \textbf{factors} that enable and reward sustainable ways of life from walkable and bikeable cities and shared and electrified mobility to self-sustaining homes, healthy plant-based \textbf{diets}, avoided flights, and to \textbf{consumption} requiring less \textbf{material} input as we \textbf{reuse}, repair and improve \textbf{recycling}.
Rather than leaving it to individuals and their choices, we \textbf{need} \textbf{systems} approaches that advance climate-friendly choices for all, while prioritising the rights and \textbf{needs} of those who are yet to reap the benefits of development.
The poorest quarter of the population worldwide lack decent homes, mobility and food and \textbf{will} \textbf{need} additional \textbf{energy}, capacity and resources for human wellbeing.
To achieve the \textbf{needed} \textbf{emission} cuts, annual \textbf{investment} \textbf{flows} towards clean \textbf{energy}, \textbf{efficiency}, \textbf{transport}, \textbf{agriculture} and forests \textbf{will} \textbf{need} to \textbf{increase} at least 3-6 fold up to 2030.
There is sufficient global capital and liquidity to close these \textbf{investment} gaps, but it ’s not heading the right way.
As of today, more private and \textbf{public} \textbf{money} still \textbf{flows} to fossil fuels than to climate \textbf{solutions}, due to misaligned incentives both outside and within the financial \textbf{sector}.
Removing fossil fuel subsidies could, alone, reduce \textbf{emissions} by up to 10 % by 2030.
Access to finance remains a big barrier especially for developing countries, and the promised \textbf{levels} of climate finance ( 100 \textbf{billion} USD/year ) from developed countries have not been met.
While many countries have improved their climate plans, not a single country is yet reducing \textbf{emissions} at a speed required by the 1.5°C goal.
Misaligned policies lead to misaligned financial \textbf{flows}, into the fossil fuel \textbf{economy}, when in reality there ’s no room for any new fossil fuel \textbf{infrastructure}.
There are already enough coal plants and other fossil fuel \textbf{infrastructure} in place to take us above 1.5°C, if allowed to be in full \textbf{use} until the \textbf{end} of their projected lifetime.
Instead, global fossil fuel \textbf{use} \textbf{needs} to cease to about one tenth by 2050, if we are to follow a \textbf{pathway} that avoids overshooting 1.5°C and doesn't bet on sucking large \textbf{amounts} of extra carbon back from the \textbf{atmosphere}.
Avoiding short-term action by relying on long-term plans that assume that somehow, somewhere, somebody \textbf{will} remove our \textbf{emissions} back from the \textbf{atmosphere} in large \textbf{amounts}, sometime in the future, is a risky plan.
Such carbon \textbf{dioxide} removals, at the \textbf{scale} assumed by many \textbf{pathways}, are an uncharted territory and come with many uncertainties and risks.
Some \textbf{amount} of carbon \textbf{dioxide} removal \textbf{will} be necessary, to compensate for those \textbf{emissions} that can't be avoided, but the \textbf{need} for it can be limited with urgent \textbf{emission} cuts. \textbf{Households} with income in the \textbf{top} 10 % contribute about 36-45 % of global \textbf{emissions}.
Two thirds of them live in developed countries and one third in other \textbf{economies}.
Those with high \textbf{emissions} have a higher potential for \textbf{emissions} \textbf{reductions} too, while maintaining \textbf{good} living \textbf{standards} and well-being.
Equity and justice are, overall, essential considerations for effective climate policy and for securing national and international support for deep decarbonisation, given the differences in current and historical \textbf{emissions} contributions, degree of vulnerability and impacts, as well as capacities within and between nations.
Accelerated international cooperation, including on finance, is a critical enabler of low-carbon and just transitions. \textbf{Shifting} to a sustainable future \textbf{will} require transformative changes that disrupt existing \textbf{trends}.
It takes technological, systemic, and cultural changes, for which we \textbf{need} both consistent action from politicians and other decision makers, as well as \textbf{public} pressure and social movements.
That solar, wind and storage \textbf{solutions} have now made a disruptive breakthrough in \textbf{costs}, performance and adoption, much faster than anticipated by experts and earlier mitigation \textbf{models}, can be a gamechanger.
Together these \textbf{solutions} could now, through electrification, \textbf{start} pushing fossil fuels out of the \textbf{system} in \textbf{energy}, \textbf{transport}, buildings and \textbf{industry} at a speed and \textbf{scale} once considered unthinkable, If enabled by further determined action.
This breakthrough did not happen by a coincidence.
It was driven by policy, \textbf{innovation} and \textbf{public} pressure for change ( thanks to people like you! ).

% matched lemmas: agriculture, amount, atmosphere, billion, break, business, consumption, cost, design, diet, dioxide, dollar, economy, efficiency, emission, end, energy, factor, flow, game, good, half, household, increase, industry, infrastructure, innovation, investment, level, material, may, mean, model, money, need, news, number, pathway, problem, public, recycling, reduction, report, reuse, scale, sector, shift, solution, standard, start, system, technology, time, top, transport, trend, use, will, zero
\end{textsample}
