\begin{textsample}{POS Dim 3 – human – Score 62.00 – t157\_human.txt}  \label{ex:f3_pos_002}
We know that oil \textbf{companies} hid knowledge of global \textbf{heating} for decades, but the captains of petroleum also \textbf{schemed} to turn the ecological crisis into a profit centre.
The \textbf{industry} devised a plan to swindle \textbf{money} from the \textbf{public} purse by pretending to address the climate issue while \textbf{using} subsidies to \textbf{increase} oil \textbf{production}.
If one had no moral compass, one might say their scam was a stroke of genius.
In Australia, the \textbf{companies} promised to capture millions of \textbf{tonnes} of carbon, beginning in 2016, but for the first four years they captured \textbf{none}, and in 2019 the Gorgon CCUS project clogged up with sand and had to shut down for repairs.
To date, Gorgon has captured about 30 % of its target for “ processing \textbf{emissions}, ” but this \textbf{term} hides the \textbf{fact} that the \textbf{companies} have only captured about 2 % of the target for \textbf{total} \textbf{emissions}.
However, the one \textbf{thing} that Chevron did capture and \textbf{store} was 100 % of the \$60 million in \textbf{public} hand-outs. “ Managing \textbf{greenhouse} \textbf{gas} \textbf{emissions}, ” Chevron declared, “ is an integral part of how Chevron plans and executes its \textbf{business}. ” The corporate strategy appears to be : Socialize \textbf{costs} and privatize profits.
However, carbon capture added an additional strategy : Socialize risk.
Since carbon \textbf{emissions} would accelerate global \textbf{heating}, and since the hydrogen produced is highly explosive, the \textbf{companies} faced severe liability risks.
No \textbf{problem} : In Australia, Chevron and Shell convinced the government, the taxpayers, to accept liability for the hazardous Gorgon project.
The swindle appears simple : Pretend to help solve a \textbf{problem}, while making the \textbf{problem} worse, socialize the \textbf{costs} and liabilities, and privatize the profits.
Clever.
However, the unscrupulous \textbf{scheme} began to show signs of unravelling.
In 2006, the German Federal Ministry of the Environment determined that there was “ no direct \textbf{cost} advantage for \textbf{technologies} \textbf{using} fossil fuels [ i.e. carbon capture ] … compared to advanced renewable \textbf{energy} \textbf{technologies}, ” and a year later, the Australian Environmental Protection Agency recommended that the Gorgon project should not proceed due to environmental risks.
EnergyWashington Week revealed, as \textbf{reported} by Oil Change International and the US Environmental Protection Agency, that “ A power plant equipped with a CCS \textbf{system} … would \textbf{need} roughly 10 to 40 % more \textbf{energy} than a plant of \textbf{equivalent} output without CCS. ” More \textbf{energy} \textbf{consumption} yields more CO 2 \textbf{emissions}, not less.
These warnings and recommendations were ignored.
The American Petroleum Institute continued to promote carbon capture, although their own consultant \textbf{report} on “ Carbon \textbf{Dioxide} Enhanced Oil Recovery ” warned that “ the \textbf{amount} of \textbf{infrastructure} necessary to perform geologic storage on a meaningful \textbf{level} is \textbf{equivalent} to the existing worldwide \textbf{infrastructure} associated with current oil and \textbf{gas} \textbf{production}. ” To reverse global \textbf{heating}, CCUS would require doubling the world ’s petroleum \textbf{infrastructure}, built up over the previous century, a near impossibility with \textbf{costs} running into the trillions.
Furthermore, that \textbf{infrastructure} would require massive mining, \textbf{transport} of \textbf{materials}, cement, steel, and carbon-intensive fabrication, yielding more \textbf{emissions}.
In 2007, as these nagging \textbf{problems} surfaced, BP scrapped a £500-million carbon capture \textbf{scheme} in Scotland.
In the US, a “ clean coal ” CCS project in Mississippi, behind schedule and \textbf{billions} over \textbf{budget}, closed, and the Petra Nova CCS plant in Texas—promising to capture 1.6 million \textbf{tonnes} of CO2 annually—missed its targets over three years of operation and shut down in 2020.
The Carbon Capture and Sequestration Technologies \textbf{program} at MIT closed due to the \textbf{technology} ’s ecological damage and unviable \textbf{economics} in 2016.
By the \textbf{end} of 2020, more than 80 % of US CCUS projects had failed.
Meanwhile, Western Australia ’s Environmental Protection Authority concluded in 2019 that Chevron should be held accountable for venting \textbf{gas} from the Gorgon project and for failing to capture and \textbf{store} the project ’s \textbf{emissions} as promised and required.
According to a January 2022 \textbf{study} by Global Witness, Shell ’s Quest plant in Canada ’s tar sands, is emitting more carbon than it is capturing, with the same annual carbon \textbf{footprint} as 1.2 million gas-powered vehicles.
Shell ’s \textbf{scheme}, one of the biggest boondoggles of carbon capture chicanery, \textbf{uses} the hydrogen produced to refine thick, toxic bitumen into synthetic crude, creating more carbon \textbf{emissions}.
The project also emits methane, a much more potent \textbf{greenhouse} \textbf{gas}.
Since the oil \textbf{industry} — Shell, Chevron, and others — were not prepared to actually slow oil \textbf{production} to halt global \textbf{heating}, and since they had no intention of aiming for \textbf{zero} carbon \textbf{emissions}, they invented “ net \textbf{zero}. ” The “ net ” requires that we subtract some carbon from \textbf{total} \textbf{emissions} to create the illusion of “ \textbf{zero} ” \textbf{emissions}.
Thus, the patriarchs of petroleum profiteering came up with “ carbon capture, ” a deception that has netted them \textbf{billions} of \textbf{dollars} and euros in \textbf{public} \textbf{money}.
Global Witness found that although Shell ’s Quest plant was capturing 4.81-million \textbf{tonnes} of carbon annually ( Mt/yr ), it was emitting 12.47 Mt/yr in \textbf{greenhouse} \textbf{gases} from on-site and \textbf{supply} \textbf{chain} \textbf{emissions} and from the power required to operate the CCS \textbf{system}.
The plant therefore annually is responsible for some 7.66-million \textbf{tonnes} of \textbf{greenhouse} \textbf{gases}, even after the CCUS bookkeeping tricks.
Shell originally promised to capture 90 % of \textbf{emissions}, had to admit failure, and changed their target to 65 %, but according to the Institute for Energy Economics and Financial Analysis, the Quest plant failed to reach its target every year from 2015 to 2020.
Upon awarding the Quest project a Canadian-dollar \$834-million subsidy ( US\$654-million, €571-million ) Canada ’s Ministry of Natural Resources Joanna Sivasankaran claimed that CCS was “ an important tool on the \textbf{pathway} to reaching Canada ’s ambitious climate goals, ” to reach “ net-zero by 2050. ” However, since the Quest project emits more than it captures and \textbf{increases} tar sands \textbf{production}, the dirtiest, most carbon-intensive petroleum \textbf{product} on Earth, these “ ambitious climate goals, ” remain unattainable and appear preposterous.
Four hundred international scientists, academics, and \textbf{energy} analysts signed a letter to the Canadian government asking that they halt the subsidy scam. “ Deploying CCUS at any climate-relevant \textbf{scale}, ” they wrote, “ carried out within the short timeframe we have to avert climate catastrophe without posing substantial risks to communities on the front lines of the buildout, is a pipe dream. ” The letter warned that CCUS is “ not a negative \textbf{emissions} \textbf{technology}, ” with \textbf{billions} of taxpayer \textbf{dollars} \textbf{used} to boost oil \textbf{production}.
The scientists and scholars warned of the \textbf{health} impacts to local communities, that the \textbf{tax} subsidies would tie Canada to “ dependence on dirty tar sands, ” and that the project would add some 50 million metric \textbf{tons} CO 2 \textbf{emissions} annually by 2035.
According to Lubicon Cree citizen Melina Laboucan-Massimo, the tar sands project yields “ elevated \textbf{rates} of cancers, as well as elevated \textbf{rates} of respiratory illnesses … contamination to the water, destruction and complete fragmentation of the Boreal forest. ” According to Reuters, 26 commercial CCS facilities around the world capture about 40 million \textbf{tonnes} of CO 2 each year.
To put that in perspective, the world emits about 36.4-billion \textbf{tonnes} of CO2 each year.
That \textbf{means} that after 50 years of CCS development ; after \textbf{billions} of \textbf{dollars} in subsidies ; after all the hype, deceits, \textbf{tax} \textbf{breaks}, and guarantees ; the oil \textbf{industry} captures about 0.1 % of annual CO 2 \textbf{emissions}.
The other 99.9 % pollutes the \textbf{atmosphere} and heats Earth.
Meanwhile, most of this captured CO 2 is \textbf{used} to produce more oil.
Since that first CCS project began in 1972, world CO 2 \textbf{emissions} have almost tripled from 14.68 to 36.4 \textbf{billion} \textbf{tonnes} per year, not exactly the “ net \textbf{zero} ” we were promised.
Carbon capture was a scam from the \textbf{beginning}, and remains so today.
Carbon Capture : Five Decades of False Hope, Hype, and Hot Air, ” Andy Rowell and Lorne Stockman, Oil Change International, June 2021.
Even the Intergovernmental Panel on Climate Change ( IPCC ) has enabled the scam, since most IPCC climate \textbf{models} require carbon capture and storage ( CCS ) to balance the carbon books, always of course, at some \textbf{time} in the distant future.
Shell ’s fossil hydrogen plant in Canada emitting more \textbf{greenhouse} \textbf{gasses} than it is capturing : GlobalWitness \textbf{report} : “ Hydrogen ’s Hidden \textbf{Emissions}, ” January 2022 ; sourcesand methodology : Pembina “ Carbon Intensity of Blue Hydrogen ; ” Shell, Alberta \textbf{data} set ; UK Dept.
Transport ; and Nimblefins Insurance ; calculations shown in annex. “ The Western Australian government rules against the oil and \textbf{gas} \textbf{company} over \textbf{emissions} at the Gorgon LNG project, Guardian, 2020. “ Between a rock and a hard place : The science of geosequestration, ” Standing Committee on Science and Innovation, House of Representatives, The Parliament of the Commonwealth of Australia, 2007, PDF. “ Chevron ’s Gorgon \textbf{emissions} rise after sand clogs \$3.1B C02 injection \textbf{system}, ” Peter Milne, Boiling Cold, Jan 12, 2021.
American Petroleum Institute : CCS \textbf{used} to “ enhance oil \textbf{production}, ” Platts Energy Economist, “ Carbon Capture and Storage : panacea or an expensive red herring? ” November 1, 2006, \textbf{reported} in Oil Change International, June 2021. “ Summary of Carbon \textbf{Dioxide} Enhanced Oil Recovery ( CO2EOR ) Injection Well Technology, ” James P.
Meyer PhD, Contek \textbf{Solutions}, Plano, Texas, for the American Petroleum Institute, web archive Western Australia ’s Environmental Protection Authority concluded that Chevron should be held accountable for venting \textbf{gas} from the Gorgon project : “ Chevron Faulted for Gorgon \textbf{Emissions}, ” September 30, 2019 ; cited in “ Carbon Capture : Five Decades of False Hope, Hype, and Hot Air, ” A.
Rowell and L.
Stockman, Oil Change International, June 2021.
Over 80 % of U.S.
CCUS projects have failed : “ Explaining successful and failed \textbf{investments} in U.S. carbon capture and storage, ” Abdulla et al., Environmental Research Letters, 2021 ; Science IPO. “ Honest Government Ad, Carbon Capture \& Storage, ” The Juice Media ; “ Australien Government ” Sept 1, 2021 : “ A power plant equipped with a CCS \textbf{system}.. would \textbf{need} roughly 10 to 40 % more \textbf{energy} than a plant of \textbf{equivalent} output without CCS. ” EnergyWashington Week, “ International Panel Finds Carbon Sequestration Has High \textbf{Price} Tag ”, October 12, 2005 ; JSTOR.
Oil \textbf{industry} geologists knew in the 1950s that all oil fields would deplete over \textbf{time}, as pressure dropped in rock formations and the oil would no longer \textbf{flow}.
They developed certain “ enhanced oil recovery ” \textbf{technologies} to extend the life of depleted oil fields, by fracking and by pumping carbon \textbf{dioxide} ( CO 2 ) into old wells.
However, these \textbf{technologies} were expensive and reduced their gargantuan profit margins.
Furthermore, by 1965, even the American Petroleum Institute had anticipated the “ catastrophic consequences ” of carbon \textbf{dioxide} \textbf{emissions}.
Thus the Great Carbon Capture Scam was born. “ IEEFA : Carbon capture goals miss the mark at SaskPower ’s Boundary Dam coal plant, ” IEEFA. “ The World ’s Only Coal Carbon Capture Plant Is Regularly \textbf{Breaking}, ” Audrey Carleton, Vice, 2022.
SaskPower ; has never met this goal ( spglobal ), as of the \textbf{end} of 2021. “ Are Canada ’s carbon capture plans a ‘ pipe dream? ” John Woodside, National Observer, Canada, January 20, 2022 400 Canadian scientists ’ letter urges Canadian government to avoid rewarding \textbf{companies} who \textbf{use} carbon capture \textbf{technology}. “ Shell ’s Massive Carbon Capture Plant Is Emitting More Than It ’s Capturing, ” Anya Zoledziowski, Vice, January 2022. “ Comparison of carbon capture and storage with renewable \textbf{energy} \textbf{technologies} regarding structural, economic, and ecological aspects in Germany, ” Peter Viebahn, et al., International Journal of Greenhouse Gas Control, April 2007, p.121-133. “ Fossil Fuel Racism : How phasing out oil, \textbf{gas} and coal can protect communities, ” Donaghy, T. \& Jiang, C., 2021, Greenpeace. 26 commercial CCS facilities globally, capture about 40 million tonnesCO2/year : “ Global CCS capacity grew by a third, ” Reuters, Dec. 2020. 2021 global CO2 \textbf{emissions}, 36.4-billion tonnes/year : “ Global carbon \textbf{emissions} rebound to near pre-pandemic \textbf{levels}, ” Andrea Januta, Reuters, Nov. 2021 ; from University of Exeter \textbf{study}. \textbf{Using} carbon \textbf{dioxide} for enhanced recovery : “ The Scurry Area Canyon Reef Operating Committee ( SACROC ) \textbf{unit}, Scurry County, Texas, ” over a \textbf{billion} barrels of oil produced, Global CCS Institute, 2016.
Carbon Capture, Centre for Climate and \textbf{Energy} \textbf{Solutions}, C2ES.
Exxon predicts Greenhouse \textbf{Effect}, CO 2 build-up, and global \textbf{heating} : Exxon internal Engineering Report, 1982. \textbf{Industry} insiders publicly claimed that they could capture and \textbf{store} the dangerous CO 2, \textbf{using} \textbf{public} \textbf{money} of course, while secretly planning to \textbf{use} this captured CO 2 for enhanced oil recovery, which would create more carbon \textbf{emissions}.
It might take decades for the \textbf{public} to figure out that they had been filched. “ Peak oil and the low-carbon \textbf{energy} transition : a net-energy perspective, ” Delannoya, Murphy, ASPO France, 2021. “ Grand Transitions : How the Modern World Was Made, ” Vaclav Smil, \textbf{amazon}. “ Hydrogen : The dumbest \& most impossible renewable, ” Alice Friedmann, Energy Skeptic, 2019 Energy Mix over \textbf{time} : Our world in Data Energy Timeline, Alternative Energy “ COP-26 : Stopping Climate Change and Other Illusions, ” William E.
Rees ( Professor Emeritus, University of British Columbia ), Buildings and Cities, October 2021.
Leaks Show Attempts to Weaken UN Climate Report, Greenpeace Says​, Deutsche Welle, Eco Watch, Oct. 21, 2021.
Anderson, K. \& Peters, G. ( 2016 ) The trouble with negative \textbf{emissions} : science.org “ A Review of the Role of Fossil Fuel Based Carbon Capture and Storage in the Energy System, ” Garcia Freites, S. \& Jones, C. ; Friends of the Earth Scotland, 2020.
In 1948, Chevron discovered a promising field in Scurry County, Texas, which showed signs of depletion by 1951.
In 1972, they began the world ’s first CCS project, \textbf{using} \textbf{waste} carbon \textbf{dioxide} from a \textbf{gas} field 400 kilometers away, near the Mexican border, shipping it north through a pipeline, and \textbf{using} the \textbf{gas} to extend the life of their Scurry field.
After \textbf{using} the CO 2, they vented the \textbf{gas}, so there was no real climate advantage.
However, the \textbf{technology} worked to produce more oil.
Since the \textbf{companies} intended to \textbf{use} captured CO 2 for enhanced oil recovery, the \textbf{technology} was then called “ Carbon Capture, Use, and Storage, ” ( CCUS ).
In 1992, international oil \textbf{companies} held the first CCUS conference in the Netherlands.
In 1998 Chevron, Exxon-Mobil, Shell, and the Australian government began promoting carbon capture and \textbf{use} for the huge Gorgon \textbf{gas} field in Australia that had two \textbf{public} relations \textbf{problems} : It was in a nature reserve and it produced a relatively dirty, climate-wrecking \textbf{gas} with 14 % carbon \textbf{dioxide} \textbf{waste}.
Since the carbon had to be captured anyway, to meet export \textbf{regulations}, the oil \textbf{companies} lobbied to have Australian citizens pay for it.
Chevron and their partners received a \$60-million grant from the Australian government, and in 2003 Chevron claimed that CCUS was “ a vital \textbf{technology} to ensure a safe, reliable \textbf{supply} of \textbf{energy} to meet the world ’s \textbf{needs}. ” Meanwhile, an API promotional campaign confirmed that CCS was primarily \textbf{used} to “ enhance oil \textbf{production}. ”

% matched lemmas: amazon, amount, atmosphere, beginning, billion, break, budget, business, chain, company, consumption, cost, datum, dioxide, dollar, economics, effect, emission, end, energy, equivalent, fact, flow, footprint, gas, greenhouse, health, heating, increase, industry, infrastructure, investment, level, material, mean, model, money, need, none, pathway, price, problem, product, production, program, public, rate, regulation, report, scale, scheme, solution, store, study, supply, system, tax, technology, term, thing, time, ton, tonne, total, transport, unit, use, waste, zero
\end{textsample}
