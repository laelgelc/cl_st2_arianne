\begin{textsample}{POS Dim 3 – human – Score 48.00 – t386\_human.txt}  \label{ex:f3_pos_005}
I remember thinking, in the 1970s, that once people became aware of the ecological crisis — disappearing species, polluted rivers, poisoned air — that the necessary changes would be simple to achieve.
Humanity only had to curb industrial \textbf{waste} and destruction, preserve wilderness for other species, put limits on our \textbf{consumption}, stabilize human population, and just be smart about how to live on Earth without destroying it.
Nevertheless, not all \textbf{effects} are immediately observable ; some are delayed by centuries.
Long lags in the Earth \textbf{system} are difficult for human observers to notice in a single lifetime.
Meanwhile, some \textbf{effects} build up slowly over \textbf{time} and eventually reach a biological or physical threshold, a tipping point at which point tiny changes can yield a large and sudden response.
In a society, in an ecosystem, and even in one ’s own body, the members or parts rely on each other.
This interdependence \textbf{may} render diverse \textbf{systems} more stable than simple \textbf{systems}.
In addition to all of this, even random events can disrupt complex \textbf{systems}.
The asteroid that hit Earth 65 million years ago was a random incident that had a huge influence on who perished ( many large reptiles ), who survived ( our ancestor mammals ), and who changed ( small raptor reptiles became birds ).
Complex \textbf{systems} are vulnerable to random events.
Sometimes random events or tipping points \textbf{will} trigger other tipping points in a cascade of \textbf{effects} similar to a nuclear \textbf{chain} reaction.
A cascade can cause a complete, irreversible state \textbf{shift} in a complex \textbf{system} and actually cause key \textbf{information} held within the \textbf{system} to be lost.
Complex bio-physical \textbf{systems} collect and \textbf{store} \textbf{information} about their own patterns.
Genetic codes, habits, instincts, and cook books are examples.
If not protected, this \textbf{information} can be lost.
Human \textbf{design} and planning \textbf{models} are only beginning to glimpse how we might interact with these dynamic features of ecosystems and social \textbf{systems}, with, for example, flexibility and adaptive preparedness.
In response to some human/ecological dilemma, perhaps you ’ve heard someone say, “ If everyone would just \_\_\_\_\_\_\_\_\_ ” ( fill in the blank ).
Yes, we could solve so many \textbf{problems} if everyone would just … be nice to each other, be reasonable, rely on solar power, lose their ego, consume less \textbf{stuff}, and so forth.
Or perhaps you ’ve read an engineering \textbf{report} about how we could — technically — scrub all the carbon from industrial exhaust, or build enough windmills to power the world, or solve the \textbf{problem} of human \textbf{consumption} by adopting vegan \textbf{diets}.
Each one of these alleged \textbf{solutions} \textbf{may} contain some truth, but \textbf{none} of them tells the whole story.
The natural world is not so simple.
Corporate apologists also like to simplify our ecological dilemma, claiming ecological progress, while actively working against real \textbf{solutions}.
Urban \textbf{design} offers countless examples of engineering \textbf{solutions} gone wrong.
During the last century, transportation designers responded to traffic congestion by building more roads, bigger roads, smoother roads, freeways, roundabouts, and so forth.
This approach led to cities \textbf{designed} for \textbf{cars} and produced greater congestion.
The engineering mindset failed to consider the deeper, systemic context. \textbf{Cars} were a dubious idea in the first place, and the \textbf{car} culture was promoted by profiteers, not by a wise assessment of transportation \textbf{options}.
In the 1930s, Standard Oil, General Motors, and Firestone Tire created a U.S. \textbf{company}, National City Lines, that bought \textbf{public} transportation \textbf{systems} and sabotaged them.
They literally tore out light rail tracks and lobbied and bribed government officials around the world to build roads at \textbf{public} expense.
Much of the world adopted a private \textbf{car} culture because that \textbf{system} benefited a few \textbf{business} elites, who wanted to \textbf{increase} profits.
The engineering \textbf{may} have been brilliant, but the fundamental assumptions were wrong, or at least incomplete.
Our industrial food boom spawned similar failures of systemic context.
Our inventive, so-called green \textbf{revolution} of food \textbf{solutions} produced a lot of food and made profits for large corporations, but depleted \textbf{soil} nutrients, spread toxins, severed mycelial networks, and disrupted nutrient \textbf{cycles}, creating eutrophic lakes and dead zones in our oceans.
Oops!
Maybe we should have thought that through better.
Of course, I was naive to think any of that would be easy.
Since that \textbf{time}, human population has doubled, \textbf{consumption} of \textbf{material} resources has quadrupled, biodiversity collapse has accelerated, and after 34 international climate meetings, we are emitting more carbon than ever before.
Meanwhile, we have not exactly \textbf{ended} war, vanquished racism, nor achieved gender or economic parity.
Even worse, giant corporate interests actively work to halt and reverse any ecological \textbf{regulation} on industrial activity.
Thousands of years ago, ancient Taoists and Indigenous cultures appear to have understood the interacting, co-evolving \textbf{qualities} of living \textbf{systems}.
Poets have long felt that reciprocity with wild nature.
By the 1950s, certain \textbf{researchers} in western academic cultures began to grasp the challenges of human interaction with these large, complex living \textbf{systems}, giving rise to \textbf{systems} theory and ecology.
In the 1970s, architectural \textbf{design} professor Horst Rittle, at the University of Stuttgart, described “ wicked \textbf{problems} ” that defied linear \textbf{solutions}, contained contradictions, and implied \textbf{solutions} that \textbf{may} create or aggravate other \textbf{problems}.
Today, we face many wicked \textbf{problems}, including social injustice, the Covid-19 pandemic, and the climate emergency.
Each of these influences the others, making them all even more wicked.
Earlier this year, engineer Oz Sahim, ecologist Shannon Rutherford and their colleagues at Griffith University in Australia created a “ Causal Loop Diagram ” to describe the “ wicked complexity ” of the Covid pandemic.
The graphic representation provides a \textbf{good} impression of what a wicked \textbf{problem} looks like : The climate crisis presents another wicked \textbf{problem}, beset with social, political, ecological, economic, thermodynamic, and \textbf{chemical} complexity, involving lags, thresholds, and feedbacks.
Here are just the most urgent tipping points that could cascade into a runaway state \textbf{shift} for the entire Earth : Amazon Rainforest Dieback : Evaporation and transpiration move water between \textbf{soil} and \textbf{atmosphere}, creating clouds and generating about \textbf{half} of the Amazon ’s rainfall.
Industrial logging and \textbf{agriculture} as well as global \textbf{heating} disrupt this process and \textbf{shift} the climate into a drier state that \textbf{may} not be able to support a rainforest.
Atlantic Ocean Circulation brings warm water north from the tropics.
Ice melt from global \textbf{heating} dilutes salty seawater with freshwater, slowing the current, which could stop or reverse, and dramatically change the climate throughout Europe, Scandinavia, the Arctic, and elsewhere.
Monsoons in Africa and India bring essential rainfall to semi-arid regions.
Warmer ocean temperatures have reduced the land-ocean temperature gradient, reinforced by slower vegetation \textbf{growth}, causing less evapo-transpiration of water to the \textbf{atmosphere}, and thus less rainfall, resulting in more drought, famine, and thousands of deaths.
Boreal Forest \textbf{Shift} : The conifer forests of Canada, Europe, and Asia, which sequester carbon, are already depleted by industrial logging. \textbf{Studies} in 2012 and 2014 reveal that warmer temperatures have \textbf{increased} fires and disease and allowed bud worms to move north, reducing these forests.
Some declining boreal forests are now net carbon \textbf{emission} sources, rather than sequestering sinks, adding to the warming.
Melting Permafrost releases methane, a \textbf{greenhouse} \textbf{gas}.
A \textbf{total} melt would triple Earth ’s atmospheric carbon content.
A 2019 NOAA Report concluded that thawing permafrost \textbf{may} already be releasing “ 300-600m \textbf{tonnes} of carbon/year to the \textbf{atmosphere}. ” Coral Reef Die-off : Warmer, more acidic oceans cause coral reefs to expel the algae that provide them with \textbf{energy}.
Dying reefs eliminate habitat for a quarter of all marine fish species, which has a direct impact on over 500 million people worldwide, who rely on those fish for sustenance.
Our emotional responses to crisis evolved over millennia, primarily to meet immediate \textbf{needs}, perhaps to benefit our tribe or community, not necessarily to solve complex, multi-dimensional, long-term dilemmas.
Our ideas about “ \textbf{solutions} ” tend to be linear, short-term, and linked to a perception of simple cause and \textbf{effect}.
Our educational institutions encourage this linear thinking about \textbf{problems} and \textbf{solutions}.
Meanwhile, our social and ecological challenges are systemic, multidimensional, and complex.
Melting Ice Sheets in Antarctica, the Arctic, and Greenland are changing ocean currents and raising sea \textbf{levels}.
A \textbf{total} melt of the most vulnerable ice sheets — West Antarctica and Greenland — would raise sea \textbf{levels} over ten meters, devastating every coastal city on Earth.
Albedo \textbf{Shift} : Melting ice \textbf{increases} Earth ’s heat absorption ( decreases reflectivity, or albedo ), thus adding to the warming.
The Jet Stream, influenced by global temperature gradients, appears to be slowing, which can lead to extreme weather events, heatwaves, floods, and droughts.
All these \textbf{effects} influence each other, reduce biodiversity, and \textbf{increase} human \textbf{health} and economic impacts.
This is the nature of multiple wicked \textbf{problems} among complex, interacting \textbf{systems}.
We desperately \textbf{need} a social tipping point, a political tipping point, to help us overcome our outdated awareness of and response to these crises.
Hall, N. ed.1994, Exploring Chaos : A Guide to the New Science of Disorder.
W.W.
Norton, 1994.
Waldrop, M. 1992, Complexity : The emerging science at the edge of order and chaos, Simon / Schuster. “ As Climate Change Worsens, A Cascade of Tipping Points Looms, ” Fred Pearce, Yale E360, 2019.
Nine ‘ tipping points ’ that could be triggered by climate change, Infographic : Carbon Brief, 2019 “ Computer \textbf{model} predicts likelihood of crossing several climate change thresholds, ” Kevin Roark, Los Alamos National Laboratory, phys.org, 2017. “ Breaching a carbon threshold could lead to mass extinction, ” Jennifer Chu, MIT News, 2019.
Living ecosystems are dynamic, always changing, and possess \textbf{qualities} such as thresholds, cascades, feedback loops, tipping points, lags, and generally unintended consequences to input.
Maybe we \textbf{need} to learn more about how change actually occurs in nature, not just in our imaginations or in our engineering dissertations. “ A Degree of Concern : Why Global Temperatures Matter, Alan Buis, NASA, 2019. “ Crossing the Paris Agreement thresholds, ” Arctic News, 2020. “ Tipping \textbf{Elements} – the Achilles Heels of the Earth System,, Potsdam Institute.
Schellnhuber, H.J ; Lenton T.M., et al., “ Tipping \textbf{elements} in the Earth ’s Climate \textbf{System}, Proceedings of the National Academy of Sciences, PNAS, 2008.
Gunderson, L.H. and C.S.
Holling ; 2009, Panarchy : Understanding Transformations in Human and Natural Systems, Island Press, 2009.
Walker, B. and D.
Salt. 2006, Resilience Thinking : Sustaining Ecosystems and People in a Changing World, Island Press, Washington.
Marshall, A. ( 2010 ), “ Climate Change : The 40 Year Delay between Cause and \textbf{Effect} ”, skeptical science.
Hansen, J. ( 2018 ), “ Climate Change in a Nutshell : The Gathering Storm ”, U Columbia, 2018.
Steffen, W., J.
Rockström, et al., “ Trajectories of the Earth System in the Anthropocene ”, PNAS, 2018.
Korzybski, Alfred, “ A Non-Aristotelian System and its necessity for Rigour in Mathematics and Physics, ” American Mathematical Society, 1931.
In a 1931 \textbf{paper} on the language of mathematics and physics, Polish linguist Alfred Korzybski wrote the now famous line, “ A map is not the territory ”, acknowledging that all \textbf{models}, maps, and perceptions of the world are “ \textbf{reductions} ” of natural complexity.
When we presume to \textbf{design} projects to “ solve ” ecological \textbf{problems}, we should keep in mind that living \textbf{systems} possess characteristics that \textbf{may} not feel intuitive to us.
Oz Sahin, Hengky Salim, Emiliya Suprun, Shannon Rutherford, et al., “ Developing a Preliminary Causal Loop Diagram for Understanding the Wicked Complexity of the COVID-19 Pandemic, ” Griffith University, Australia Systems, May 2020. “ It Could Be Decades Before \textbf{Emissions} Cuts Slow Global Warming, Scientists Warn, ” Harlowe Hood, ScienceAlert, 2020.
Timothy M.
Lenton, Johan Rockström, et al., “ Climate tipping points — too risky to bet against : The growing threat of abrupt and irreversible climate changes must compel political and economic action on \textbf{emissions}. ”, Nature, 2019.
Horst Rittel and “ Wicked Problems, ” Management Science, December 1967 ; and “ Dilemmas in a General Theory of Planning ” ( PDF ).
Policy Sciences. 4 ( 2 ) : 155–169.
Arctic Report Card, US National Oceanic and Atmospheric Administration ( NOAA ), 2019.
O.
Hoegh-Guldberg1, et al., “ Coral Reefs Under Rapid Climate Change and Ocean Acidification, Science, 2007.
Lenton, T.M.
Arctic Climate Tipping Points.
AMBIO 41, 2012 Wolfgang Buermann, Bikash Parida, et al. ; “ Recent \textbf{shift} in Eurasian boreal forest greening response \textbf{may} be associated with warmer and drier summers, ” AGU, 2014.
Petoukhov, Vladimir, et al. “ Quasiresonant amplification of planetary waves and recent Northern Hemisphere weather extremes. ” Proceedings of the National Academy of Sciences, PNAS, 2013.
Palmer, T.
N. “ Climate extremes and the role of dynamics. ” Proceedings of the National Academy of Sciences 110.14 : 5281-5282, PNAS, 2013.
Fifty years before Korzybski, French physicist Henri Poincaré noticed that multiple orbiting bodies travelled paths that were neither random nor entirely predictable.
This appeared at the \textbf{time} as a paradox.
In the 1920s, German physicist Werner Heisenberg demonstrated that the more precisely one determines the velocity of a particle, the less one knows about its position, a realization that came to be known as the “ uncertainty principle. ” Ruangpan, L., Vojinovic, Z., et al., “ Nature-based \textbf{solutions} for hydro-meteorological risk \textbf{reduction}, ” Natural Hazards Earth System Science, 20, 243–270, 2020.
Meanwhile, many naturalists observed that living \textbf{systems} — biomes, herds, flocks, organisms, societies — are never in fixed states, but always in process, what science historian James Gleick described as “ becoming rather than being. ” Thus, arose what we now call “ chaos theory. ” Living \textbf{systems} exhibit both patterns and chaos, simultaneously, always changing, but change is neither continuous nor purely chaotic.
Rather, change in complex \textbf{systems} appears to fluctuate among long periods of relative stability, punctuated by bursts of rapid change.
These \textbf{systems} have no central \textbf{control}, and abrupt \textbf{shifts} can be triggered by a random input, such as an asteroid hitting Earth, or by an accumulation of small changes that reach a tipping point.
Complex \textbf{systems} respond to input, and those responses can become new inputs into the \textbf{system}.
Cause and \textbf{effect} in this case are neither simple nor linear.
When forests first grew on Earth, they sequestered carbon from the \textbf{atmosphere}, cooling the Earth, limiting forest \textbf{growth}, and altering the species of trees that flourished.
As human carbon \textbf{emissions} heat Earth, we trigger feedbacks that both heat and cool the Earth.
Some feedbacks self-regulate, some self-amplify.
Events in such \textbf{systems} can stabilize or run out of \textbf{control}.

% matched lemmas: agriculture, atmosphere, business, car, chain, chemical, company, consumption, control, cycle, design, diet, effect, element, emission, end, energy, gas, good, greenhouse, growth, half, health, heating, increase, information, level, material, may, model, need, none, option, paper, problem, public, quality, reduction, regulation, report, researcher, revolution, shift, soil, solution, store, study, stuff, system, time, tonne, total, waste, will
\end{textsample}
