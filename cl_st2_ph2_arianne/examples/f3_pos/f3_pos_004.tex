\begin{textsample}{POS Dim 3 – human – Score 49.00 – t039\_human.txt}  \label{ex:f3_pos_004}
World ’s leading climate scientists have just released their assessment of the climate emergency and ways to \textbf{deal} with it.
It ’s deeply unfair : Those least responsible are hit the hardest Worse is to come : We ’re on track to very high risks and irreversible losses 1.5°C is still within reach : With urgent action, the Paris long-term goal can still be met We have the \textbf{solutions} : We can halve global \textbf{emissions} by 2030, on the way to net \textbf{zero} Fossil fuel exit is \textbf{needed} fast : The fossil \textbf{infrastructure} we already have is too much Real \textbf{solutions}, no delays. \textbf{Solutions} must deliver in real life, not only in \textbf{models} Equity and social inclusion are fundamental, finance gaps must be closed From incremental to transformative.
It ’s all \textbf{sectors} and all hands on deck, NOW!
Want to learn more?
Keep reading.
Below you \textbf{will} find these 10 key takeaways explained in more detail.
Human-caused climate change is now affecting weather and climate extremes in every region across the globe.
Impacts and related losses and damages to nature and people have been widespread.
It ’s a crucial \textbf{report} delivered at a crucial moment in \textbf{time}, when governments are taking stock of their action under the Paris Agreement.
And in short, the verdict of the scientists is this : \textbf{Effects} on ecosystems have been experienced earlier, are more widespread and with further-reaching consequences than anticipated. \textbf{Half} of all species are already on the move, due to climate change affecting their environments.
Evidence of observed changes in extremes such as heatwaves, heavy rain, droughts and tropical cyclones, and, in particular, their attribution to human influence, has only strengthened.
Global aggregated impact and risk \textbf{levels} are now assessed to become high to very high at lower \textbf{levels} of warming than previously assessed ( AR5 ).
Currently, we are at around 1.1°C of \textbf{average} global warming, heading towards about 3°C.
Unique and threatened ecosystems are expected to be at high risk already in the very near \textbf{term} at 1.2°C, due to mass tree mortality, coral reef bleaching, large declines in sea-ice dependent species, and mass mortality events from heatwaves.
At just 1.5°C, up to 14 % of species assessed in terrestrial ecosystems \textbf{will} likely face a very high risk of extinction.
Reaching 1.5°C \textbf{will} bring more and worse heat extremes and dangerous heat-humidity conditions, extreme rainfall and associated flooding, tropical cyclones, wildfires and extreme sea-level events.
Between 1.5°C–2.5°C, risks associated with large-scale singular events or tipping points, such as ice sheet instability or ecosystem loss from tropical forests, transition to high risk.
At about 1.9°C warming, \textbf{half} of the human population could be exposed to periods of life-threatening climatic conditions arising from the coupled impacts of extreme heat and humidity by 2100.
At sustained warming \textbf{levels} between 2°C and 3°C, the Greenland and West Antarctic ice sheets \textbf{will} be lost almost completely and irreversibly.
Vulnerable communities who have historically contributed least to the climate crisis are most affected.
Nearly \textbf{half} of the world ’s population ( 3.3–3.6 \textbf{billion} people ) live in contexts that are highly vulnerable to climate change.
Between 2010 and 2020, human mortality from floods, droughts and storms was 15 \textbf{times} higher in highly vulnerable regions, compared to regions with very low vulnerability. “ There is a rapidly closing window of opportunity to secure a liveable and sustainable future for all ( very high confidence ).
At the same \textbf{time}, only 10 % of the wealthiest \textbf{households} contributed up to 45 % of global consumption-based \textbf{household} GHG \textbf{emissions}.
With policies implemented by the \textbf{end} of 2020, we are on track to about 3.2°C warming by 2100. ( Estimates that cover more recent policies ( NDCs announced prior to COP26 until 2030, with no \textbf{increase} in ambition ), result in a slightly \textbf{better} assessment of 2.8°C median warming. ) Instead of halving global \textbf{emissions} by 2030, which would be required for respecting the Paris Agreement warming limit, there would be no decline in global \textbf{emissions} by 2030.
With continued \textbf{emissions}, we ’re on track to reach around 1.5°C in the near \textbf{term}.
But it is still in our hands to halt warming there to avoid the most severe impacts.
By following the strongest IPCC \textbf{emission} \textbf{reduction} \textbf{pathways} ( C1 ), warming would peak between 1.4°C and 1.6°C and by the \textbf{end} of the century be below 1.5°C. ( See Table 3.1 ) So with urgent action, the Paris Agreement long-term temperature goal is still within reach.
This would require roughly halving global \textbf{emissions} by 2030, followed by net \textbf{zero} CO2 by around 2050, and achieving and sustaining net negative CO2 ( and GHG ) \textbf{emissions} globally thereafter, with annual \textbf{rates} of carbon \textbf{dioxide} removal ( CDR ) being greater than remaining CO2 \textbf{emissions}.
Limiting as much as possible any overshoot of 1.5°C, for the shortest duration possible is essential, as the gradual cooling would not undo the irreversible impacts triggered by the peak warming ( such as species loss or ice sheet melt ).
Furthermore, while some carbon \textbf{dioxide} removal is necessary by now, it comes with many uncertainties, so the reliance on it should be limited.
As the IPCC concluded in its earlier AR6 \textbf{reports} : “ CDR deployed at \textbf{scale} is unproven, and reliance on such \textbf{technology} is a major risk in the ability to limit warming to 1.5°C.
CDR is \textbf{needed} less in \textbf{pathways} with particularly strong emphasis on \textbf{energy} \textbf{efficiency} and low demand. ” ( IPCC SR15 ) “ (…)prioritising early decarbonisation with minimal reliance on CDR decreases the risk of mitigation failure and \textbf{increases} intergenerational equity. ” ( IPCC SRCCL ) We have all the tools we \textbf{need} for at least halving global \textbf{emissions} by 2030. \textbf{Half} of this mitigation potential is estimated to be low-cost ( less than 20 USD/tCO2-eq ), or even to come with \textbf{cost} savings.
The biggest contributions would come from solar and wind \textbf{energy}, protection and restoration of forests and other ecosystems, climate-friendly food \textbf{systems}, and \textbf{energy} \textbf{efficiency} in its many forms. “ Rapid and far-reaching transitions across all \textbf{sectors} and \textbf{systems} are necessary ” By 2050, demand-side \textbf{measures} can reduce global GHG \textbf{emissions} by 40–70 % compared to baseline \textbf{scenarios}.
These refer to societal choices about how we \textbf{use} \textbf{technology} and resources to meet our \textbf{needs} for food, shelter, mobility and \textbf{products}.
One of the \textbf{measures} with the greatest potential, and synergies with adaptation and biodiversity conservation and human \textbf{health}, is the \textbf{shift} towards low-meat \textbf{diets}, \textbf{termed} “ balanced sustainable healthy \textbf{diets} ” by the IPCC \textbf{reports}.
Overall, providing \textbf{better} services with less \textbf{energy} and resource input is consistent with providing well-being for all.
The fossil fuel \textbf{infrastructure} already in place is enough to exceed the 1.5°C warming limit, if allowed to be in \textbf{use} without further \textbf{restrictions}.
So there is no room for any new fossil fuel \textbf{infrastructure}, and what exists \textbf{needs} to be phased out early, as is illustrated by the graph below.
In other words, keep it in the ground : “ About 80 % of coal, 50 % of \textbf{gas}, and 30 % of oil reserves cannot be burned and emitted if warming is limited to 2°C.
Significantly more reserves are expected to remain unburned if warming is limited to 1.5°C. ” ( SYR longer \textbf{report} ) So there \textbf{needs} to be a major \textbf{shift} away from fossil fuels.
But just how fast?
It depends on many assumptions, and in the underlying mitigation \textbf{report} ( AR6 WG3 ) the IPCC provides more detail.
In \textbf{pathways} that limit warming to 1.5°C with greater than 50 % likelihood and no or limited overshoot, the global \textbf{use} of coal is projected to decline by up to 100 %, oil by up to 90 % and \textbf{gas} by up to 85 % by 2050, with median values being lower. ( See WG3, SPM C.3.2 ) The fastest \textbf{reductions} are required in \textbf{pathways} that aim at 1.5°C with little to no overshoot, low reliance on carbon \textbf{dioxide} removal, low pressure on land and biodiversity and high \textbf{efficiency} in resource \textbf{use}.
Such \textbf{pathways} are illustrated by the IMP-LD \textbf{pathway}, where the overall \textbf{use} of fossil fuels declines by about 85 % by 2050, from 2020 \textbf{levels} ( See WG3, Figure 3.6 and SPM C.3.6 ).
We have entered a critical decade, during which we must nearly halve global \textbf{emissions} while delivering on food security and protecting and restoring nature too.
Among the big game-changers since the previous assessment is the breakthrough of solar and wind, that are now reaching \textbf{cost} \textbf{levels} equal to or below those of fossil fuels, being ready to enable the decarbonisation of different \textbf{sectors} through electrification.
These developments have occurred much faster than anticipated by experts and \textbf{modelled} in previous mitigation \textbf{scenarios}.
It ’s a game-changer.
Carbon capture and storage ( CCS ), then again, has not made significant progress.
It plays a big role in many \textbf{emission} \textbf{reduction} \textbf{models}, but still fails to deliver at \textbf{scale} in real life.
As the IPCC mitigation \textbf{report} summed up : “ Deployment and development of CCS \textbf{technologies} ( with large-scale storage of captured CO2 ) have been much slower than projected in previous assessments ”. “ Implementation of CCS currently faces technological, economic, institutional, ecological-environmental and socio-cultural barriers. ” Technological carbon \textbf{dioxide} removal, where CO2 is captured directly from the \textbf{atmosphere} ( DACCS ), or from biomass \textbf{energy} ( BECCS ), also play a role in most mitigation \textbf{models}, but remain unproven at \textbf{scale}, and they come with many feasibility and sustainability constraints – as does large-scale afforestation. “ The choices and actions implemented in this decade \textbf{will} have impacts now and for thousands of years ( high confidence ). ” So, as we navigate into the future, where carbon \textbf{dioxide} removal \textbf{will} be \textbf{needed} – though substantially less so with urgent, near-term action – it is essential to look beyond simplified \textbf{models}.
Prioritising CDR \textbf{solutions} that maximise sustainability benefits and minimise risks, \textbf{means} prioritising \textbf{options} that work with nature and for communities, like reforestation, restoration of ecosystems or \textbf{soil} carbon sequestration in \textbf{agriculture}.
This is critical in order to avoid conflicts with other land \textbf{use} \textbf{needs} or creating large environmental impacts and conflicts with human rights and food security.
The \textbf{scale} and speed of transformation \textbf{needed} \textbf{will} not be possible without equity and social justice, both between and within countries.
According to the IPCC, integrating climate action with macroeconomic policies can support sustainable low-emission development paths, safety nets and social protection, and improve access to finance for low-emissions \textbf{infrastructure} access, especially in developing regions.
At the heart of equity considerations is finance.
There ’s enough \textbf{money} in the world to enact real change, if existing barriers are removed.
But today, \textbf{public} and private finance \textbf{flows} for fossil fuels are still greater than those for climate adaptation and mitigation. ( Indeed, while the IPCC points to the growing finance and adaptation gaps vulnerable communities are struggling with, the IEA \textbf{reports} that last year alone, the oil and \textbf{gas} \textbf{industries} earned a whopping 4 trillion USD with \textbf{businesses} fueling the climate crisis! ).
The annual \textbf{investment} requirements before 2030 are a \textbf{factor} of three to six greater than current \textbf{levels}, for mitigation alone, with the largest \textbf{needs} being in the developing world.
To change this, both governments and financial institutions \textbf{will} \textbf{need} to align their goals and policies with 1.5°C, and remove barriers.
On an international \textbf{level}, equitable \textbf{solutions} \textbf{need} to be found that meet the adaptation and mitigation \textbf{needs} and address loss and damage for those least responsible.
There ’s a rapidly narrowing window of opportunity to implement climate resilient development.
To achieve the Paris Agreement and other sustainability goals, we \textbf{need} to think beyond individual \textbf{technologies}, \textbf{sectors} and actors and adopt holistic, inclusive and transformative approaches that encompass both mitigation and adaptation.
Action that protects and restores our biodiversity is fundamental.
By taking care of nature, we are taking care of ourselves.
According to the IPCC, maintaining the resilience of biodiversity and ecosystem services at a global \textbf{scale} depends on effective and equitable conservation of approximately 30 % to 50 % of Earth ’s land, freshwater and ocean areas, including currently near-natural ecosystems.
To achieve “ rapid and far-reaching transitions across all \textbf{sectors} and \textbf{systems} ” we \textbf{need} strong laws and policies and international co-operation.
It ’s all hands on deck and those with most responsibility and capability \textbf{need} to lead the way.
This goes for governments, \textbf{businesses}, investors and high-income individuals alike.
Scientists have delivered their toolbox for survival.
Now it is our job to make sure the science is acted on, by governments, \textbf{businesses}, investors and citizens.
And that we make it personal.
The life, well-being or suffering of our daughters, granddaughters, great-granddaughters and many more after them \textbf{will} look back at us and feel grateful for what we did, or perhaps what we did not do.
Kaisa Kosonen is a Senior Policy Advisor with Greenpeace Nordic and Reyes Tirado is a Senior Scientist with Greenpeace Research Laboratories at University of Exeter.
Let that sink in for a moment.
The Synthesis of the IPCC Sixth Assessment Report forms a robust \textbf{analysis} of thousands of peer-reviewed research \textbf{papers} published over the past decade, with more details to be found in the underlying six \textbf{reports}.
In summary, our take on it all in 10 key messages is this : It ’s bad : Human-caused climate change is already widespread, rapid and intensifying It ’s worse than expected : Impacts and risks are getting more severe sooner

% matched lemmas: agriculture, analysis, atmosphere, average, billion, business, cost, deal, diet, dioxide, effect, efficiency, emission, end, energy, factor, flow, gas, good, half, health, household, increase, industry, infrastructure, investment, level, mean, measure, model, money, need, option, paper, pathway, product, public, rate, reduction, report, restriction, scale, scenario, sector, shift, soil, solution, system, technology, term, time, use, will, zero
\end{textsample}
