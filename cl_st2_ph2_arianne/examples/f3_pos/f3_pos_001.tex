\begin{textsample}{POS Dim 3 – human – Score 70.00 – t572\_human.txt}  \label{ex:f3_pos_001}
A popular bumper sticker in the United States – typically seen on large vehicles, with giant wheels and vibrating chrome muffler pipes – reads : “ My carbon \textbf{footprint} is bigger than yours. ” This appears as a banner for the culture of extravagant indulgence.
And wherever \textbf{consumption} is encouraged and admired, \textbf{waste} follows.
Mining for these \textbf{minerals} tends to be ecologically destructive and exploitive of human labourers.
Due to \textbf{increasing} demand and low \textbf{rates} of electronics \textbf{recycling}, mining \textbf{companies} are now proposing strip mines on the ocean floor, a practice that ocean biologists say would permanently damage unique and biodiverse ocean ecosystems.
As computer chips got smaller, more powerful, and more \textbf{energy} efficient, the \textbf{material} and \textbf{energy} intensity of those chips \textbf{increased} exponentially.
Since our computers require so little \textbf{energy} to operate, we \textbf{may} believe they are “ efficient, ” but we are \textbf{measuring} the wrong metric.
To understand the high \textbf{cost} of high tech, we must consider the embodied \textbf{energy} built into our devices, our telecom \textbf{infrastructure}, server networks, and \textbf{data} centres.
We also have to consider the sheer \textbf{growth} of \textbf{consumption} and the acceleration of \textbf{waste}.
According to Statisa, about 4 million cell phones are sold every day, over 1.5 \textbf{billion} per year.
About 250 million computers are sold each year.
The \textbf{average} lifetime of these devices is now about two and a \textbf{half} years.
Manufacturers \textbf{design} in obsolescence, changing critical parts and \textbf{marketing} more fashionable, “ improved ” devices.
We \textbf{may} marvel at social media and connectivity, but this \textbf{level} of \textbf{consumption} leaves behind a massive, toxic, and destructive \textbf{waste} stream.
Apple Corporation has become notorious for \textbf{designing} smartphones, tablets, and laptops that are difficult to repair or upgrade.
These policies are not an accident or a necessity of technological advance.
They are \textbf{marketing} decisions, \textbf{designed} specifically, like the three-month light bulb, to sell more \textbf{products}.
Between June 29, 2007 and November 3, 2017, Apple introduced 14 new iPhone \textbf{models}, one every 37 weeks.
The \textbf{company} stopped supporting the first generation phones within three years, and continues to make previous phones obsolete and unsupported.
According to Jason Koebler at Motherboard, “ Apple is trying to kill legislation that would make it easier for normal people to fix iPhones. ” Apple \textbf{designs} \textbf{products} with proprietary parts that cannot be easily repaired and the \textbf{company} has actively lobbied against right-to-repair legislation in the US.
According to a Repair.org \textbf{study}, both Apple and Sony have blocked environmental electronics \textbf{standards} that would support repair, upgrade, and \textbf{recycling}.
However, Apple Corporation is not alone.
According to a 2017 Greenpeace \textbf{report}, other \textbf{consumer} electronics \textbf{companies} are lagging far behind.
Although Apple has made progress in the \textbf{use} of renewable \textbf{energy} they are “ moving in the wrong \textbf{direction}, ” along with Microsoft and Samsung, by shortening the useful life of devices.
Samsung, Amazon, Oppo, Vivo, and Xiaomi receive failing grades in every category, \textbf{using} toxic \textbf{chemicals} and dirty \textbf{energy}, making short-lived \textbf{products} that are difficult to recycle, and hiding the \textbf{data} about their practices.
On the other hand, HP, Dell, and Fairphone are leaders in producing \textbf{products} that are repairable and upgradable.
Electronic \textbf{waste} has now reached over 65 million \textbf{tonnes} per year.
Computers, screens, and small hand devices comprise about 22 % of that \textbf{waste}, 14 million \textbf{tonnes} annually.
According to a 2014 UN Report, Europe produced the highest per-capita electronic \textbf{waste}, over 15 kilograms per person every year.
Asia generated the most e-waste, 16 million metric \textbf{tonnes}, followed by the Americas, 11.7 million \textbf{tonnes} per year.
Since 2014 those \textbf{volumes} have \textbf{increase} by about 50 %.
As with most of our ecological challenges, there are \textbf{solutions}, but the response requires more than marginal change.
According to Deishin Lee, at MIT ’s Sloan School of Management, “ most \textbf{waste} is generated on purpose, ” built into modern \textbf{business} \textbf{models}.
Lee criticizes “ output-oriented, ” \textbf{production} \textbf{systems} that only consider the \textbf{product}. “ Every output-oriented process, ” she writes, “ is \textbf{designed} to produce \textbf{waste}. ” We can overcome this by \textbf{shifting} to input-oriented \textbf{production}, considering the value of all resources, how to conserve, and how to \textbf{use} resources effectively, with a minimum of \textbf{waste}.
Economist Tim Cooper, at Nottingham Trent University believes that a transformation away from planned obsolescence \textbf{will} require a “ radical, systemic change. ” In his book, “ Longer Lasting Products, ” Cooper suggests the change could be accomplish with economic policies to encourage minimum \textbf{standards} of durability, repairability, and upgradeability.
The world ’s rich cultures are all wasteful, and not just because of excessive fossil fuel \textbf{use}.
Even our modern electronic devices represent a massive \textbf{waste} stream.
Last year, electronic \textbf{waste} reached an all-time record of 65 million \textbf{tonnes}. \textbf{Quality} \textbf{goods}, robust repair-and-servicing, and secondhand \textbf{markets} would result in more jobs and more economic activity for a given \textbf{amount} of resources.
Cooper calculates that when \textbf{consumers} spend less on throwaway \textbf{products}, they \textbf{will} spend more for other services and \textbf{investments}.
In “ Culture of Waste, ” Julian Cribb, a fellow of the Australian Academy of Technological Sciences and Engineering, describes how we could reverse the \textbf{trends} toward food \textbf{waste} with government \textbf{regulation} to limit wasteful practices, full-cost pricing and \textbf{taxing}, subsidies for \textbf{good} stewardship \textbf{production}, and with education.
The 2017, Greenpeace Report, advocates similar actions to create closed loop, circular \textbf{production}, beginning at the \textbf{design} stage, with all \textbf{companies} required to \textbf{design} recyclable parts, easy repair, and a take-back \textbf{program} for all \textbf{products}.
Everything we build requires \textbf{energy}.
Wasteful practices \textbf{waste} \textbf{energy}.
Although we are witnessing an unprecedented effort to develop renewable \textbf{energy}, we are failing to keep pace with \textbf{growth} in demand.
Unless we address the \textbf{growth} of human \textbf{numbers} and human enterprise, we are destined for the natural results of ecological overshoot.
We also \textbf{need} to phase out fossil fuels and redouble efforts to build renewable \textbf{energy} \textbf{infrastructure}.
The following chart – prepared by Canadian \textbf{energy} engineer David Hughes, \textbf{using} \textbf{data} from the 2019 BP Energy Review – shows the annual \textbf{growth} in renewable \textbf{energy} compared to the annual \textbf{growth} in \textbf{electricity} demand.
A great \textbf{deal} of this demand is due to wasteful manufacturing and \textbf{sales} practices.
Two-thirds of the \textbf{growth} is met with fossil fuels.
Furthermore, this only accounts for \textbf{electricity}. 83 % of the world ’s \textbf{energy} \textbf{consumption} is non-electric.
The only year that renewable \textbf{energy} \textbf{growth} exceeded demand \textbf{growth} occurred in 2009 during an economic recession.
This chart reveals two critical pieces of our \textbf{waste} and \textbf{energy} challenge : ( 1 ) Renewable \textbf{energy} \textbf{growth} is not keeping pace with \textbf{total} \textbf{energy} demand, and ( 2 ) The way to turn this around is to \textbf{end} the expectation of endless economic \textbf{growth}.
Some \textbf{companies}, such as Fairphone and Patagonia, have \textbf{business} \textbf{models} that account for slowing \textbf{growth}.
The idea that we should keep \textbf{businesses} growing by creating \textbf{waste} is no longer valid – and never was.
We can employ more people by building \textbf{quality} \textbf{products} and repairing them.
To reverse the \textbf{trend} of wasteful \textbf{production}, biodiversity collapse, carbon \textbf{emissions} that cause global \textbf{heating}, and general ecological overshoot, humanity has to embrace modest \textbf{consumption} and put an \textbf{end} to the era of extravagant indulgence. “ E-waste World Map Reveals National \textbf{Volumes}, International \textbf{Flows}, ” StEP Initiative, 2013, Quoted in Greenpeace E-Waste \textbf{report}, 2016.
E-waste : The Escalation of a Global Crisis, TCO certified “ Made to \textbf{Break} : \textbf{Technology} and Obsolescence in America, ” Giles Slade, Harvard University Press, 2007 ; and excerpts at Google Books. “ A Culture of Waste, ” Julian Cribb, Ecology Today, 2012.
Even modern LED light bulbs, for example, do not last as long as incandescent bulbs made a century ago.
One carbon filament light bulb, at a fire station in Livermore, California, is still burning continuously after 120 years.
Building \textbf{things} that last, and consuming modestly, \textbf{used} to be common human values.
But that all changed with the advent of contemporary \textbf{business} \textbf{models} and modern marketing.
Guide to Greener Electronics 2017, Greenpeace Reports, October 17, 2017 “ Overcoming the culture of \textbf{waste} ” Deishin Lee, MIT, Sloan School of Management, 2017.
Power-hungry gadgets endanger \textbf{energy} \textbf{efficiency} gains, \textbf{review} of The International Energy Association \textbf{analysis}, John Timmer, 2009, ARS Technica.
The Global E-waste Monitor, 2014 : UN University, 2014. “ Electronic Waste ( E-Waste ) : How Big of a \textbf{Problem} is it? ” Rubicon, 2018 \textbf{Facts} and Figures on E-Waste and Recycling, Electronics Takeback Coalition, 2014 “ The monster \textbf{footprint} of digital \textbf{technology}, ” Kris de Decker, Low-Tech Magazine, “ Electronics Standards Are In \textbf{Need} of Repair, ” Mark Schaffer, Repair.org, August 2017. “ Apple is against your-right to repair i-Phones New York state records confirm, ” Jason Koebler, Motherboard, 2017. “ Longer Lasting Products : \textbf{Alternatives} to the Throwaway Society, ” Tim Cooper, Gower Books, 2010.
In 1924, three \textbf{companies} – Dutch Philips, German Osram, and US General Electric – formed a cartel, Phoebus, to shorten the life of light bulbs.
Making light bulbs that could last 100 years limited their \textbf{sales} \textbf{growth}.
They agreed on a thousand-hour \textbf{standard}, about three or four months of normal \textbf{use}, the historic \textbf{beginning} of planned obsolescence.
Culture and Waste : The Creation and Destruction of Value, Edited by Gay Hawkins and Stephen Muecke, Rowman \& Littlefield, 2002. “ The L.E.D.
Quandary : Why There ’s No Such \textbf{Thing} as ‘ Built to Last ’, ” J.
B.
MacKinnon, New Yorker, 2016. “ Patagonia ’s Anti-growth Strategy, ” J.B.
MacKinnon, New Yorker, 2015.
During World War I, the U.S.
Treasury Department launched a frugality campaign to save resources for the war effort.
Merchants, however, opposed the initiative.
According to Giles Slade in Made to \textbf{Break}, US \textbf{stores} displayed signs such as, “ Beware of Thrift, ” and “ Business as Usual. ” New York \textbf{retailers} formed the “ National Prosperity Committee, ” with slogans like, “ Full Speed Ahead! ” and “ Clear the Track for Prosperity! ” During the global economic depression in 1932, New York manufacturers circulated a pamphlet : “ \textbf{Ending} the Depression through Planned Obsolescence, ” the first known printed \textbf{use} of this phrase.
An article in Printer ’s Ink journal warned that the idea of durability was “ outmoded, ” claiming that, “ If merchandise does not wear out faster, \textbf{factories} \textbf{will} be idle, people unemployed. ” Paul Mazur, a partner at Lehman Brothers, declared that obsolescence, \textbf{designing} \textbf{products} to fail or wear out, was the “ new god ” of \textbf{business} philosophy.
In 1950s America, advertising firms learned that they could sell \textbf{products} not based on function, \textbf{quality}, or durability, but on novelty. \textbf{Products} were sold as “ new, ” “ modern, ” and “ innovative, ” whether or not the “ \textbf{innovations} ” offered any genuine value.
The throwaway \textbf{fashion} \textbf{industry} was born on the notion that clothing “ styles ” allegedly changed every year, and that to appear “ modern, ” one must repeatedly buy new clothing.
Ad agencies convinced popular journals to publish \textbf{fashion} sections to inform, or manipulate, the \textbf{public} regarding the latest styles.
Thus, the idea of well-made, durable \textbf{products} died away in rich nations, replaced by \textbf{products} that \textbf{break}, wear out, become obsolete, or go out of \textbf{fashion}.
This \textbf{trend} has now seized the modern electronics \textbf{industry}.
Since the 1980s, computers and electronic devices have made lives in rich countries more convenient and entertaining.
Some observers expected that modern electronics would also make society more “ efficient, ” that computers would save \textbf{paper} and other resources.
Those hopes, however, encountered what is known in \textbf{economics} as the “ rebound \textbf{effect} “ : \textbf{Efficiency} often leads to more resource \textbf{use}, not less.
Human enterprise now \textbf{uses} six \textbf{times} more \textbf{paper} than we \textbf{used} at the dawn of the computer age, six \textbf{times} more lithium, five \textbf{times} more cobalt, more iron, copper, and more rare earth metals.

% matched lemmas: alternative, amount, analysis, average, beginning, billion, break, business, chemical, company, consumer, consumption, cost, datum, deal, design, direction, economics, effect, efficiency, electricity, emission, end, energy, fact, factory, fashion, flow, footprint, good, growth, half, heating, increase, industry, infrastructure, innovation, investment, level, market, material, may, measure, mineral, model, need, number, paper, problem, product, production, program, public, quality, rate, recycling, regulation, report, retailer, review, sale, shift, solution, standard, store, study, system, tax, technology, thing, time, tonne, total, trend, use, volume, waste, will
\end{textsample}
