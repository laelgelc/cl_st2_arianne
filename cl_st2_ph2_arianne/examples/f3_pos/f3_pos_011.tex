\begin{textsample}{POS Dim 3 – human – Score 46.00 – t582\_human.txt}  \label{ex:f3_pos_011}
The \textbf{average} global temperature between March and May this year reached 1.5°C above midrange temperatures in the early 19th century.
Global \textbf{heating} deniers claim the variation in temperature since the industrial \textbf{revolution} is not unusual for temperature fluctuations during the Holocene – the 11,700-year era since the last glaciation.
They claim, therefore, that anthropogenic carbon \textbf{emissions} cannot be considered a major contribution to the \textbf{heating} of Earth ’s oceans and \textbf{atmosphere}.
Human global \textbf{heating} denialists \textbf{will} conjure up charts that show a sweeping arc of high temperatures since the last ice age that dwarf the current rise in global temperature.
However, the serious and rigorous scientific research shows that today ’s temperature is higher than most or all of the Holocene, with the possible exception of some past temperature spikes.
Furthermore, todays temperature \textbf{increase} has \textbf{broken} through the \textbf{trend} line of the Holocene.
All fluctuating curves – climate, populations, stock \textbf{markets} – reveal a \textbf{trend} line, a running \textbf{average} of a variable indicating a general progression of change.
Compared to Earth ’s \textbf{average} 1961-1990 temperature, the Holocene temperature \textbf{trend} line shows a rise after the last glaciation, from about -0.2°C to +0.4°C around 5,000 BC, followed by a decline to about -0.4°C in about 1650 AD.
These changes appear as a smooth trend-line curve.
After 1800, human fossil fuel burning contributed a growing source of carbon, and after 1900, the temperature shoots up, \textbf{breaking} through this \textbf{trend} line.
From a 12,000-year perspective, Earth ’s temperature appears to rise almost straight up.
This \textbf{break} through the historic range of variability ( through the “ upper Bollinger bands ” of volatility in \textbf{investment} theory language ) — is unprecedented in the Holocene.
Holocene Earth temperature variations compared to historic \textbf{average} ( 1961-1990 ), showing modern spike that has \textbf{broken} through the \textbf{trend} lines.
Graphic courtesy of NOAA, with \textbf{data} from S.
Marcott, Science, 2013.
All earlier Holocene fluctuations plateaued and turned around in a repetitive fluctuation pattern.
Denialists typically presume the modern temperature \textbf{increase} is over or nearly over.
Some denialist sites even draw graphs showing the modern temperature line curving over into a plateau to simulate a typical Holocene anomaly.
But no evidence indicates the modern Earth temperature \textbf{increase} is over, and latent heat \textbf{energy} inertia from human carbon \textbf{emissions} suggests the rise is not even remotely over.
Thermodynamic theory predicts the \textbf{increase} \textbf{will} have to stop rising eventually, but the current steep rise without a plateau remains unprecedented in the Holocene.
Heat changes in a biophysical \textbf{system} require what physicists call \textbf{energy} “ forcings, ” \textbf{measured} in watts per square-meter ( w/m 2 ).
Throughout Earth ’s history, natural forcings that influenced Earth ’s temperature included solar radiation ; cloud cover ; the reflective \textbf{quality} of land, water, or ice ; and of course, the molecular components of the \textbf{atmosphere}.
Throughout the industrial age humanity has added several new forcings : fuel \textbf{waste} carbon, methane, CFCs, nitrous-oxide, aerosols, and land \textbf{use} changes.
The net human forcings ( about 1.7 w/m 2 ) is now the largest forcing \textbf{effecting} Earth ’s climate.
The \textbf{growth} of forests 300-million years ago produced a significant cooling \textbf{effect}, and the rise of land animals produced a warming \textbf{effect}, but during the Holocene, no single species has ever produced an \textbf{energy} forcing remotely on the \textbf{scale} of the human forcing.
Heat can trigger tipping points that add more heat.
Such “ positive feedbacks ” have always existed in natural \textbf{systems}, but human activity has augmented natural feedbacks and introduced new feedbacks.
As oceans take up human carbon, they begin to reach a limit, and thus are less effective as carbon sinks.
Human forest destruction has reduced forests as a carbon sink, leading to more \textbf{heating}.
The \textbf{scale} of human fishing and disruption of nutrient \textbf{cycles}, leading to ocean dead zones.
The full \textbf{effect} of these feedbacks remains uncertain, but tends toward \textbf{increased} \textbf{heating}.
In the extreme, human forcings and feedbacks could lead to runaway \textbf{heating} and what physicists call a “ state-shift ” on Earth.
These human-influenced feedbacks did not exist previously in the Holocene.
All successful species tend to grow beyond the sustainable limits of their habit.
We can witness this in our own gardens.
Evolution teaches species to consume, grow, and reproduce, but does not teach species to stop.
Predators overshoot prey, algae overshoot lake nutrients, and humans are now the first animal in Earth ’s history to overshoot the entire Earth.
The Global \textbf{Footprint} Network calculates that humans overshot Earth ’s capacity by 1970, about the \textbf{time} Greenpeace was founded, and now have overshot Earth ’s carrying capacity by at least 75 percent. “ Existing overshoot estimates, scary as they \textbf{may} be, are under-estimates, ” says William Rees, who wrote the Global \textbf{Footprint} \textbf{model} with Mathis Wackernagel. “ Early on we were accused of exaggerating ( we weren't ), so we \textbf{designed} the method to be conservative. ” The \textbf{effects} of overshoot – depleted forests, acidic oceans, drained aquifers, atmospheric carbon, depleted \textbf{soils}, and so forth – have weakened Earth ’s natural systemic ability to respond to \textbf{heating}.
A single species overshoot at this \textbf{scale} has never before plagued Earth during its entire history, much less in the Holocene.
At current \textbf{emission} \textbf{rates}, our \textbf{atmosphere} \textbf{will} reach a doubling of atmospheric CO 2 above pre-industrial concentrations by 2070.
In 1896, Swedish chemist Svante Arrhenius calculated this would be enough to trigger a 4-5°C temperature \textbf{increase}.
In 2013, 33 international scholars with the Palaeosens Project estimated in Nature that the doubling would cause a 2.2 – 4.8°C \textbf{increase}.
A 2009 MIT \textbf{study} estimated a 90 % chance of a 5.2°C \textbf{increase} by 2100.
Serious scientists know that we are attempting to predict the response of a complex \textbf{system}, and recognize the uncertainty in such \textbf{systems}.
Denialists, on the other hand, \textbf{use} normal scientific uncertainty to sow \textbf{doubt} about the climate \textbf{trends}, their causes, or about the seriousness of human carbon \textbf{emissions}. \textbf{None} of this is particularly difficult for serious meteorologists.
The fundamentals have been known for two centuries, and 65 years ago, even oil \textbf{companies} knew and published the \textbf{fact} that carbon \textbf{emissions} would heat the Earth.
The alleged “ climate debate ” is an entirely manufactured \textbf{public} relations campaign by vested interests and their operatives, not science.
On May 15, this year, both the Scripps and NOAA labs at Mauna Loa Observatory in Hawaii \textbf{reported} concentrations over 415.6 parts per million ( ppm ) of carbon-dioxide, the seasonal peak and new modern record.
Most of Earth ’s landmass and land-plants are in the northern hemisphere, so northern summer plant \textbf{growth} absorbs a lot of carbon-dioxide ( CO 2 ), reducing the \textbf{amount} in Earth ’s \textbf{atmosphere}.
In the fall, as photosynthesis declines, CO 2 \textbf{increases} in the \textbf{atmosphere}.
In the spring, when northern snowpacks melt, microbes release CO 2 contributing to the annual peak carbon-dioxide count each May.
This \textbf{cycle} explains the up and down fluctuation in Earth ’s steadily rising CO 2 concentration.
Daily CO 2 readings : CO2 Earth “ Carbon \textbf{Dioxide} at Mauna Loa Observatory reaches new milestone : \textbf{Tops} 400 ppm, ” Scripps, NOAA measurements cross threshold in same 24-hour period, ” NOAA 2013.
Temperature Record Chart, 1000 to 2019, Temperaturerecord.org.
Modern CO 2 surge : “ Greenhouse Gas Bulletin, ” 30 October, 2017, World Meteorological Organization, Atmospheric Environment Research Division ; World Data Centre for Greenhouse \textbf{Gases} Japan Meteorological Agency, Tokyo, Japan ; “ \textbf{Trends} in atmospheric carbon \textbf{dioxide}, ” NOAA, The National Oceanic and Atmospheric Administration Earth System Research Laboratory, 2016 ; and Butler, J.H. and S.A.
Montzka, “ The NOAA annual \textbf{greenhouse} \textbf{gas} index ( AGGI ), NOAA, 2016. “ How the World Passed a Carbon Threshold and Why It Matters, Nicola Jones, Yale Environment 360 January 26, 2017. “ Arctic Is Thawing So Fast Scientists Are Losing Their \textbf{Measuring} Tools, ” Dahr Jamail, Truthout, June 3, 2019.
Climate Sensitivity to doubling CO 2 = 2.2 – 4.8°C:”Making sense of palaeoclimate sensitivity, ” E.
J.
Rohling, A.
Sluijs, H.
A.
Dijkstra, et al., Nature, v. 491, 2012, Palaeosens Project CO 2 reaching 550ppm, and temperatures at + 5.2°C by 2100 : “ Climate change odds much worse than thought, ” David Chandler, MIT, 2009, \textbf{review} of \textbf{study} by Ronald Prinn, et al.
Modern Earth temperature \textbf{data} : NASA, Goddard Institute for Space Studies, Earth Sciences Division, Surface Temperature \textbf{Analysis}.
Earth ’s CO 2 concentration last ranged above 400 ppm in the mid-Pliocene, 3 million years ago, when temperatures reached 3°C above pre-industrial temperatures, 10°C higher in the Arctic, and when the sea \textbf{level} stood about 20 meters higher.
At that \textbf{time}, our primate ancestors developed a promising new \textbf{technology} : chipped stone cutting tools.
The last \textbf{time} CO 2 in Earth ’s \textbf{atmosphere} remained consistently above 400 ppm, some 16 million years ago in the warm mid-Miocene, our ancestors were venturing from trees onto the emerging African savanna.
Global \textbf{heating} forcings : “ Improved Attribution of Climate Forcing to \textbf{Emissions}, ” Drew T.
Shindell, Greg Faluvegi, Dorothy M.
Koch, et. al., Science, v.326, 30 Oct 2009 ; and “ Forcings in GISS Climate \textbf{Models}, ” updated 2011, NASA, Goddard Institute for Space Studies, Earth Sciences Division.
Global \textbf{heating} physics : “ A Tutorial on the Basic Physics of Climate Change, ” American Physical Society, APS, 2008 ; and a simpler summary : The Physics of Global Warming, ScienceBlogs, 2010. “ \textbf{Trends} in Atmospheric Carbon \textbf{Dioxide}, ” NOAA, Earth System Research Laboratory, Global Monitoring Division ; a \textbf{good} source of \textbf{data} and graphs. “ History of Earth ’s Climate 7, Cenozoic, Holocene, ” a denialism ’s attempt to dismiss anthropogenic global \textbf{heating} from CO2, a revealing tour of logical fallacies and pseudo-science, dandebat. “ Carbon \textbf{Dioxide} Emission-Intensity in Climate Projections : Comparing the Observational Record to Socio-Economic \textbf{Scenarios}. ” Felix Pretis, Max Roser, Oxford University Dept. of Economics. “ No way out?
The double-bind in seeking global prosperity alongside mitigated climate change, ” T.
J.
Garrett, Univ. of Utah : Earth Systems Dynamics. “ Why we ’ re losing the battle to keep global warming below 2C, ” The Guardian “ Target atmospheric CO2 : Where should humanity aim?
J.
Hansen, M.
Sato, P.
Kharecha, et. al. ( NASA, Columbia Univ., Univ.
Sheffield, Yale Univ., LSCE/IPSL, Boston Univ., Wesleyan Univ., UC Santa Cruz ) : Cornell University Library. “ A reconstruction of regional and global temperature for the past 11,300 years, ” Marcott, S.A., Shakun, J.D., Clark, P.U., Mix, A.C., Science, 339(6124), 1198-120, 2013. “ What ’s the hottest Earth has been lately? ”, Michon Scott, NOAA, 2014.
Reading a typical climate denial website \textbf{will} provide a plethora of doctored charts, logical fallacies, and mis-interpreted \textbf{data}.
One prominent denier of human-driven global \textbf{heating} recently asked me to name one single difference between our current warming and the normal fluctuations during the Holocene.
He claimed, “ not even the IPPC can name one. ” I gave him nine differences.
Here they are : “ Approaching a state \textbf{shift} in Earth ’s biosphere, ” Anthony D.
Barnosky, Elizabeth A.
Hadly, et al., Nature, 486, 2012.
Overshoot : “ Tracking the ecological overshoot of the human \textbf{economy}, ” Mathis Wackernagel, Niels B.
Schulz, Diana Deumling, et al., Proceedings of the National Academy of Sciences ( US ), PNAS, 2002 ; graphic representation at Earth Overshoot Day ; and William Rees in personal correspondence.
At the \textbf{end} of the last glacial period, 12,000 years ago, carbon-dioxide concentrations stood at about 240 ppm.
As Earth ’s glaciers melted, concentrations rose to 270 ppm and fluctuated up to around 280 ppm for 12 millennia, until the age of industrial hydrocarbon burning.
During the last two centuries, atmospheric CO 2 concentration has risen from about 280 ppm to the modern record of 415 ppm, a 48 % \textbf{increase}.
This equates to an extra quadrillion molecules of CO 2 per cubic centimetre of \textbf{atmosphere}.
Every one of those molecules serves as a tiny heat engine.
This \textbf{scale} of carbon-dioxide heat absorption in the \textbf{atmosphere} is unprecedented in the Holocene.
Throughout the Holocene, atmospheric carbon-dioxide \textbf{increased} at about 0.003 ppm / year ( 40 ppm over 12,000 years ).
The peak \textbf{rate} in the early Holocene reached about 0.024 ppm / year ( 24 ppm over 1,000 years ).
After indsutrialisation, the \textbf{rate} grew, and by 2017, the \textbf{rate} exceeded 2 ppm/year and the World Meteorological Society \textbf{reported}, “ The \textbf{rate} of \textbf{increase} of atmospheric carbon \textbf{dioxide} over the past 70 years is nearly 100 \textbf{times} larger than that at the \textbf{end} of the last ice age. … such abrupt changes in the atmospheric \textbf{levels} of CO2 have never before been seen. ” Today, human activity is now adding almost 3 ppm of CO2 to the \textbf{atmosphere} annually, over 100-times faster than the steepest Holocene \textbf{rate} and 1000-times faster than the \textbf{average} Holocene \textbf{rate}.
Annual and decade \textbf{average} \textbf{rate} of atmospheric CO2 \textbf{increase}, via Visual Carbon.
Due to the speed of human carbon-dioxide \textbf{increase}, global temperature rise lags behind the carbon-dioxide concentrations.
In meteorological science, this is called heat \textbf{energy} inertia.
Global temperatures have risen by over 1°C since the industrial \textbf{revolution}, but we ’ve added enough carbon already to take us to 2°C, and possibly higher with methane releases and other feedbacks.
This is the heat \textbf{energy} inertia already in the global climate \textbf{system}.
This too is unprecedented in the Holocene, and it is a big \textbf{deal}.

% matched lemmas: amount, analysis, atmosphere, average, break, company, cycle, datum, deal, design, dioxide, doubt, economy, effect, emission, end, energy, fact, footprint, gas, good, greenhouse, growth, heating, increase, investment, level, market, may, measure, model, none, public, quality, rate, report, review, revolution, scale, scenario, shift, soil, study, system, technology, time, top, trend, use, waste, will
\end{textsample}
