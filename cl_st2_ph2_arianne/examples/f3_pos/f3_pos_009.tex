\begin{textsample}{POS Dim 3 – human – Score 47.00 – t158\_human.txt}  \label{ex:f3_pos_009}
Airlines around the globe have been vying with each other on who has the greenest and shiniest \textbf{announcements} recently.
British Airways made headlines with its plan to \textbf{use} sustainable aviation fuel on a commercial \textbf{scale}, Air France claimed to aim for a 12 % cut in \textbf{emissions} by 2030, and Ryanair has called itself “ Europe ’s lowest \textbf{emissions} airline ” … What about renewable e-kerosene, synthetic kerosene made from \textbf{electricity} and a carbon source?
It ’s one of the few \textbf{alternative} fuels that can be produced in a relatively eco-friendly way if the \textbf{electricity} comes from 100 % renewable sources.
But this is a long way from being a done \textbf{deal} and would \textbf{need} significant \textbf{investment}, research and development.
The aviation \textbf{industry} and political leaders are relying on excessive optimism for false technological \textbf{solutions} – and it comes with a high \textbf{price} : \textbf{researchers} have warned that “ \textbf{technology} myths ” are stalling the necessary progress in climate policy for aviation.
More fuel-efficient planes are not a false \textbf{solution} as such, but they \textbf{will} not be sufficient to achieve decarbonisation in \textbf{time} to limit global \textbf{heating} below 1.5°C.
Under an optimistic \textbf{scenario}, Greenpeace expects an \textbf{efficiency} improvement ( \textbf{energy} \textbf{consumption} per passenger-kilometre ) of only 30 % by 2050 – not enough on its own to limit global \textbf{heating} below 1.5°C.
As the world becomes increasingly invested in tackling the climate emergency, the airline \textbf{industry} certainly wants us to think that they are part of the \textbf{solution}, not the \textbf{problem}.
We see greenwashing everywhere in the \textbf{sector} : from misleading communication and sponsorship of climate friendly initiatives to the promotion of \textbf{solutions} to tackle the environmental and social shortcomings of the \textbf{industry} that are either wrong, insufficient, or both.
There is a big discrepancy between the actual \textbf{emissions} \textbf{reduction} plans of airlines and the promotion of “ green ” PR by the airlines.
Advertisements by airlines pretend that there is no climate emergency and no reason to reduce the \textbf{number} of flights.
Airline advertising overwhelmingly focuses on low-cost flights, \textbf{deals} and promotions while evoking access to nature through flying, as found in a new Greenpeace Netherlands \textbf{report}.
The true \textbf{cost} of flying – the millions of \textbf{tonnes} of GHG \textbf{emissions} it causes – is not included in a low-cost fare.
And let ’s be honest : an individual airline ticket might be cheap, but this is only because airlines already benefit from taxpayers \textbf{money} through major \textbf{tax} cuts and \textbf{public} subsidies.
Unfortunately for the planet and us people, airlines are currently getting away with their tricks and false \textbf{solutions}.
Without political action to counter its \textbf{growth} prospects, the aviation \textbf{industry} could become one of the biggest emitting \textbf{sectors} globally.
At the same \textbf{time}, no other means of \textbf{transport} in Europe has been as heavily subsidised with \textbf{public} \textbf{money} through VAT and \textbf{tax} exemptions, state aid, bailouts and loans, as well as research and development support.
This has distorted \textbf{markets} for decades to the benefit of aviation above green mobility.
For example, airlines are exempt from kerosene \textbf{taxes} and VAT on international tickets, while railway \textbf{companies} pay high \textbf{energy} \textbf{taxes} and rail tolls.
On \textbf{top} of this, European airlines still receive a large proportion of their \textbf{emissions} allowances – permits to pollute under the EU ’s \textbf{Emissions} Trading \textbf{System} – for free.
There is a serious lack of strict laws to mitigate airlines ’ GHG \textbf{emissions} – and that ’s a major \textbf{problem}!
There is no way around the \textbf{need} to reduce flights to achieve real-zero \textbf{emissions} by 2040.
Greenpeace has calculated that in order to keep global \textbf{heating} below 1.5°C, European airlines \textbf{will} have to reduce their flights by 2 % annually.
Airlines must drop the illusion of “ carbon neutrality ”, dispel the myth of “ green flying ” and stop promoting false \textbf{solutions} that lull everyone into a false sense of security that airlines already have their climate damage under \textbf{control}.
Together, we can demand that the EU and European governments put a stop to greenwashing in the aviation \textbf{sector} : Sign this European Citizens ’ Initiative ( ECI ) launched by Greenpeace, together with the New Weather Institute and 30 other partners, to ban fossil fuel ads.
We must stop giving away free permits to pollute and taxpayers ’ \textbf{money} to the \textbf{sector} and make it pay for its pollution.
It ’s about \textbf{time} that the EU phased out all fossil fuel subsidies for the aviation \textbf{sector} ( including for airports ) and ensured that \textbf{tax} on kerosene is enhanced and swiftly implemented on all flights.
Boost rail and \textbf{public} \textbf{transport}!
We have to \textbf{start} building a mobility \textbf{system} that is \textbf{good} for the planet and the people by phasing out fossil fuel powered \textbf{transport}.
In doing so, we also have to ensure a just transition for workers in the aviation \textbf{sector} and an \textbf{end} to the \textbf{increasing} instability of working conditions – and not leave anyone behind.
And yet, in the face of the climate crisis, something in this gleaming rhetoric leaves a bitter taste.
Could an \textbf{industry} whose global \textbf{greenhouse} \textbf{gas} \textbf{emissions} have been growing by 3.4 % annually over the previous decade be fooling us about its efforts to become a responsible beacon of climate protection?
Herwig Schuster is a \textbf{transport} expert for the European Mobility For All campaign with Greenpeace Central and Eastern Europe office, based in Austria.
Bearing in mind that climate scientists have warned that the climate limit of 1.5°C is close to being \textbf{broken}, let ’s take a closer look at the green promises and climate actions of airlines – and dissect their carbon jargon!
A new \textbf{report} by Greenpeace Central Eastern Europe finds that there is little to no \textbf{substance} to the climate claims made by some of the biggest European airline groups that they \textbf{will} cut \textbf{greenhouse} \textbf{gas} \textbf{emissions} in the future.
Why?
Firstly because these \textbf{companies} mainly rely on false \textbf{solutions} and tricks that create a myth that aviation is green, despite the \textbf{fact} that flying remains the most climate-damaging \textbf{means} of transportation per passenger and per kilometre.
These are the \textbf{industry} ’s five most widely \textbf{used} tricks and false \textbf{solutions} : Globally, airlines have pledged to become carbon-neutral ( or “ net \textbf{zero} ” ) by 2050.
Doesn't sound too bad?
Think again!
The \textbf{term} “ carbon neutral ” does not actually \textbf{mean} cutting \textbf{greenhouse} \textbf{gas} \textbf{emissions} at the source.
Instead, it is based on the illusion that someone can release \textbf{greenhouse} \textbf{gases}, and balance them out by capturing these \textbf{emissions} somewhere else in the future, e.g. through carbon offsetting \textbf{schemes}.
Neither planting trees nor avoiding deforestation \textbf{will} make a flight “ carbon neutral ”.
Research has actually shown that only 2 % of offset projects have a high probability of resulting in additional \textbf{emissions} \textbf{reduction}.
Nevertheless, airlines continue to push the illusion that we can fly carbon neutral or net-zero.
Climate scientists have warned that the concept of “ net \textbf{zero} ” and “ carbon neutrality ” is a “ dangerous trap ” that has “ hastened the destruction of the natural world by \textbf{increasing} deforestation today, and greatly \textbf{increases} the risk of further devastation in the future ”.
The airline \textbf{industry} loves bringing up the magic word “ Sustainable Aviation Fuel ” ( SAF ) which refers to a variety of relatively new types of jet fuel based on e.g. biomass or \textbf{waste} intended to replace fossil fuel-based kerosene.
But the \textbf{problem} with so-called sustainable aviation fuel is : it is not sufficiently available and/or mostly not really sustainable.
While airlines present SAF as a key lever to decarbonise aviation, it only represents 0.05 % of the annual jet fuel \textbf{needed} in the EU.
In 2019, SAF accounted for at most 0.1 % of the \textbf{total} annual jet fuel \textbf{consumption} of airlines analysed in a new Greenpeace CEE \textbf{report}.
It ’s estimated that SAF \textbf{will} make up only 19 % of airline fuels by 2040 – \textbf{meaning} that 81 % \textbf{will} still be fossil-fuel based kerosene.
Not to mention that crop-based biofuel or so-called agrofuel which is made from food and feed crops like palm oil are often associated with deforestation and biodiversity loss.

% matched lemmas: alternative, announcement, break, company, consumption, control, cost, deal, efficiency, electricity, emission, end, energy, fact, gas, good, greenhouse, growth, heating, increase, industry, investment, market, mean, money, need, number, price, problem, public, reduction, report, researcher, scale, scenario, scheme, sector, solution, start, substance, system, tax, technology, term, time, tonne, top, total, transport, use, waste, will, zero
\end{textsample}
