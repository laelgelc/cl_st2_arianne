\begin{textsample}{POS Dim 3 – human – Score 42.00 – t192\_human.txt}  \label{ex:f3_pos_019}
Nuclear power is often hailed as a magic bullet \textbf{solution} for the rapid and large-scale decarbonisation of our societies which we all know \textbf{needs} to happen if we have any hope of mitigating the worst \textbf{effects} of the unfolding climate emergency.
Among politicians and \textbf{industry} groups, it is consistently favoured over meaningful \textbf{investment} in renewable \textbf{energy} \textbf{systems}, bolstered with misleading claims of its safety, \textbf{efficiency}, stability, and speed of deployment.
This can't be guaranteed in a \textbf{time} of climate crisis and extreme weather events either.
Nuclear power is a water-hungry \textbf{technology}.
Nuclear power plants consume a lot of water for cooling.
They are vulnerable to water stress, the warming of rivers, and rising temperatures, which can weaken the cooling of power plants and equipment.
Nuclear reactors in the United States and France are often shut down during heatwaves, or see their activity drastically slowed.
To protect the climate, we must abate the most carbon at the least \textbf{cost} and in the least \textbf{time}.
The \textbf{cost} of generating solar power ranges from \$36 to \$44 per megawatt-hour ( MWh ), the World Nuclear Industry Status Report said, while onshore wind power comes in at \$29–\$56 per MWh.
Nuclear \textbf{energy} \textbf{costs} between \$112 and \$189 per MWh.
Over the past decade, the World Nuclear Industry Status Report estimates levelised \textbf{costs} – which compare the \textbf{total} lifetime \textbf{cost} of building and running a plant to lifetime output – for utility-scale solar have dropped by 88 % and for wind by 69 %.
According to the same \textbf{report}, these \textbf{costs} have \textbf{increased} by 23 % for nuclear.* According to a November 2021 \textbf{study} released by Greenpeace France and the Rousseau Institute, power from the under-construction European Pressurised Reactor ( EPR ) at Flamanville in France would be 3 \textbf{times} as expensive as the country ’s most competitive renewable sources.
Stabilising the climate is an emergency.
Nuclear power is slow.
The 2021 World Nuclear Industry Status Report estimates that since 2009 the \textbf{average} construction \textbf{time} for reactors worldwide was just under 10 years, well above the estimate given by the World Nuclear Association ( WNA ) \textbf{industry} body of between 5 and 8.5 years.
The extra \textbf{time} that nuclear plants take to build has major implications for climate goals, as existing fossil-fueled plants continue to emit CO2 while awaiting substitution.
The construction of a nuclear plant is a long and complex process that obviously releases CO2, as does the demolition of decommissioned nuclear sites.
Uranium extraction, \textbf{transport} and processing is obviously not free of \textbf{greenhouse} \textbf{gas} \textbf{emissions} either.
All in all, nuclear power stations score comparable with wind and solar \textbf{energy}.
But this latter can be implemented much faster and on a much bigger \textbf{scale}.
We cannot wait for another decade for \textbf{emissions} to go down.
They \textbf{need} to go down now.
With clean renewable sources and \textbf{energy} \textbf{efficiency}, we can do that.
The multiple stages of the nuclear fuel \textbf{cycle} produce large \textbf{volumes} of radioactive \textbf{waste}.
No government has yet resolved how to safely manage this \textbf{waste}.
With the \textbf{costs} and \textbf{efficiency} of renewable \textbf{energy} \textbf{solutions} improving year on year, and the \textbf{effects} of our rapidly changing climate accelerating across the globe, we \textbf{need} to take an honest look at some of the myths being perpetuated by the nuclear \textbf{industry} and its supporters.
Here are six reasons why nuclear power is not the way to a green and peaceful \textbf{zero} carbon future.
Some of this nuclear \textbf{waste} is highly radioactive and \textbf{will} remain so for several thousand years.
Nuclear \textbf{waste} is a real scourge for our environment and for future generations, who \textbf{will} still have the responsibility of managing it in several centuries.
Countries like France are pushing hard for nuclear power at the EU \textbf{level}, hoping that when it comes to \textbf{waste}, out of sight is out of mind.
But nuclear \textbf{waste} \textbf{will} never go away, and \textbf{will} never be sustainable.
This is one of the obvious reasons why nuclear power shouldn't be eligible for green funding nor \textbf{marketed} as ‘ sustainable ’, as pointed out recently by countries like Austria, Denmark, Germany, Luxembourg, and Spain, who spoke against the inclusion of nuclear power in the EU ’s green finance taxonomy.
This is also one of the reasons why, on 9 March 2020, the EU Commission ’s Technical Expert Group on Sustainable Finance ( TEF ) rejected nuclear \textbf{energy} because it did not meet the EU ’s ‘ Do No Significant Harm ’ principle and recommended excluding nuclear power from the green taxonomy.
Nuclear \textbf{waste} management is \textbf{costing} taxpayers absurd \textbf{amounts} of \textbf{money}, \textbf{costs} for storage projects reaching into the \textbf{billions}.
This is true both for Europe and North America.
In 2019, a US Energy Department \textbf{report} showed the projected \textbf{cost} for long-term nuclear \textbf{waste} cleanup jumped more than \$100 \textbf{billion} in just one year.
The EPR nuclear reactor \textbf{technology} has been showcased by the French government and French nuclear operator EDF as a revolutionary \textbf{technology} announcing the dawn of a nuclear renaissance.
The reality is that this \textbf{technology} isn't any kind of technological leap.
More importantly, the French EPR reactor located in Flamanville is more than 10 years overdue and nearly four \textbf{times} over \textbf{budget}.
This so-called “ next-generation nuclear reactor ”, has also sustained multiple \textbf{problems}, delays and \textbf{cost} overruns in France, the United Kingdom, Finland and China.
Hypothetical new nuclear power \textbf{technologies} have been promised to be the next big \textbf{thing} for the last forty years, but in spite of massive \textbf{public} subsidies, that prospect has never panned out.
That is also true for Small Modular Reactors ( SMRs ).
And for nuclear fusion, an idea that is as old as the nuclear \textbf{industry}, which somehow always seems to be fifty years away.
The \textbf{cost} and uncertainty of fusion \textbf{mean} investing in thermonuclear reactors at the expense of other available clean \textbf{energy} \textbf{options}.
This \textbf{technology} won't arrive in \textbf{time}, if ever, and the \textbf{money} would be better invested elsewhere.
Let ’s exert the utmost caution when presented with pro-nuclear opinions coming from experts and organisations regularly working with stakeholders from the nuclear \textbf{sector} and potentially tainted by vested interests.
Nuclear \textbf{energy} has no place in a safe, clean, sustainable future.
It is more important than ever that we steer away from false \textbf{solutions} and leave nuclear power in the past.
Mehdi Leman is a content editor for Greenpeace International based in France In order to tackle climate change, we \textbf{need} to reduce fossil fuels in the \textbf{total} \textbf{energy} mix well before 2050 to 0 %.
According to \textbf{scenarios} from the World Nuclear Association and the OECD Nuclear Energy Agency ( both nuclear lobby organisations ), doubling the capacity of nuclear power worldwide in 2050 would only decrease \textbf{greenhouse} \textbf{gas} \textbf{emissions} by around 4 %.
But in order to do that, the world would \textbf{need} to bring 37 new large nuclear reactors to the grid every year from now, year on year, until 2050.
The last decade only showed a few to 10 new grid connections per year.
Ramping that up to 37 is physically impossible – there is not sufficient capacity to make large forgings like reactor vessels.
There are currently only 57 new reactors under construction or planned for the coming one-and-a-half decade.
Doubling nuclear capacity – different from the explosive \textbf{growth} of clean renewable \textbf{energy} sources like solar and wind – is therefore unrealistic.
And that for only 4 % when we already \textbf{need} to reduce 100 %.
Nuclear \textbf{factories} and plants are easy targets for malevolent acts : terrorist threats, the risk of unintentional or voluntary airliner crashes, cyberattacks or acts of war.
The enclosures of plants and certain ancillary buildings containing radioactive \textbf{materials} are not \textbf{designed} to withstand this type of attack or shock.
Nuclear power plants present unique hazards in \textbf{terms} of the potential consequences resulting from a severe accident.
Nuclear reactors and their associated high \textbf{level} spent fuel \textbf{stores} are vulnerable to natural disasters, as Fukushima Daiichi showed, but they are also vulnerable in \textbf{times} of military conflict.
For the first \textbf{time} in history, a major war is being waged in a country with multiple nuclear reactors and thousands of \textbf{tons} of highly radioactive spent fuel.
The war in southern Ukraine around Zaporizhzhia puts them all at heightened risk of a severe accident.
Nuclear power plants are some of the most complex and sensitive industrial installations, which require a very complex set of resources in ready state at all \textbf{times} to keep them operational.
This cannot be guaranteed in a war.

% matched lemmas: amount, average, billion, budget, cost, cycle, design, effect, efficiency, emission, energy, factory, gas, greenhouse, growth, increase, industry, investment, level, market, material, mean, money, need, option, problem, public, report, scale, scenario, sector, solution, store, study, system, technology, term, thing, time, ton, total, transport, volume, waste, will, zero
\end{textsample}
