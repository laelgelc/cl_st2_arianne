\begin{textsample}{POS Dim 3 – human – Score 42.00 – t767\_human.txt}  \label{ex:f3_pos_020}
For the world ’s rich, it \textbf{may} seem that life is getting better and that human expansion on Earth is not something to worry about.
But if we look a bit closer, the ecological data today shows that humanity and Earth ’s wildlife would all be better off had we heeded the warnings of Paul Ehrlich, 50 years ago.
At the dawn of the industrial age, when Earth was home to only about one \textbf{billion} people, the English economist, Thomas Malthus, warned that an exponentially growing population on a finite planet would reach ecological limits.
Likewise, in Principles of Political Economy from 1848, the economist John Stuart Mill lamented the sprawl of cities, farms, and \textbf{factories} across the landscape, and advocated a “ stationary state, ” a limit to economic and population \textbf{growth}. “ The \textbf{increase} of wealth is not boundless, ” Mill wrote, “ A stationary condition of capital and population implies no stationary state of human improvement.
There would be as much scope as ever for moral and social progress ; as much room for improving the art of living, and much more likelihood of it being improved. ” However, those who profited most from human \textbf{growth} mocked Malthus and Mill, and still do.
In our day, the profiteers also dismiss the 1972 Limits to \textbf{Growth} \textbf{study}, Ehrlich ’s population warnings, and environmentalism in general, since neoliberal economic theory denies the limits of a finite planet. \textbf{Technology} advocates repeat endlessly that the so-called “ Green Revolution ” in \textbf{agriculture} enabled humanity to feed \textbf{billions} more people.
But a closer look at \textbf{factory} farming reveals that the Green Revolution was actually a “ black ” \textbf{revolution}, fuelled by hydrocarbons and toxic \textbf{chemicals}.
Feeding \textbf{billions} led to ecological and \textbf{health} decline, the disruption of Earth ’s nitrogen \textbf{cycles}, global \textbf{heating}, and the relentless excavation of nutrients from Earth ’s fragile \textbf{soils}.
The European Union \textbf{reports} losing about a \textbf{billion} \textbf{tonnes} of topsoil annually, while over the last 25 years, Earth ’s \textbf{soil} productivity has declined by over 50 % in some regions. 75 \textbf{billion} \textbf{tons} of topsoil are lost annually worldwide from anthropogenic erosion.
Crop yields are now stagnating in many regions and even where they ’re rising, the \textbf{increase} lags behind population \textbf{growth} and rising demand, leading to higher food \textbf{prices}.
Fifteen years ago, David Pimentel warned that humanity and the ecosystems upon which it relies are “ threatened by overpopulation. ” Pimentel pointed out that in a world without fossil fuels, nations can support only about four people for every hectare of arable land.
That \textbf{means} that most countries won't be able to sustain even their current population, much less a growing one.
As ecologists from Malthus to Ehrlich have warned, humanity is now running into resource limits.
The world ’s wild fisheries catch stopped growing in the late 1980s, and has dropped by about 14 % since, in spite of advances in fishing \textbf{technology}.
The \textbf{amount} of fresh water per capita has declined by over 25 % since the late 1980s, and today, some 850 million people have no access to clean water.
Five million people, including \textbf{half} a million infants, die each year from waterborne diseases.
In 2016, the United Nations “ Global \textbf{Material} \textbf{Flows} ” \textbf{report} showed that global resource depletion diminishes human \textbf{health}, \textbf{quality} of life, and future development.
Although most nations still desire economic \textbf{growth}, the UN panel warned that “ rapid economic \textbf{growth} … \textbf{will} place much higher demands on \textbf{supply} \textbf{infrastructure} and the environment ’s ability to continue \textbf{supplying} \textbf{materials}. ” Numbers that matter Those who criticised Ehrlich for his warnings in The Population Bomb, often site the \textbf{fact} that, since the 1960s, the population \textbf{growth} \textbf{rate} has declined from about 2.2 % to 1.1 % annually.
This is true, but focusing on this percentage avoids the more relevant \textbf{fact} : the \textbf{number} of people on this planet has steadily \textbf{increased} every year.
Human population has been growing since time-immemorial, with two exceptions.
About 2000 years ago, urban crowding, plagues, and warfare caused the population to decline for the first \textbf{time} in history.
Population \textbf{growth} recovered for a while, then declined again in the 17th century due to genocide and disease brought on by the European colonisation of Africa, Asia, and the Western Hemisphere.
In 1968, Ehrlich published The Population Bomb, warning humanity that runaway human population would limit \textbf{quality} of life for humans, lead to \textbf{increased} starvation and malnutrition, and would contribute to ecological decline and biodiversity collapse.
At that \textbf{time}, the human population stood at about 3.5 \textbf{billion}.
Now, 50 years later, it has more than doubled to 7.6 \textbf{billion}, and we face the most severe biodiversity collapse the Earth has seen in 65 million years.
The \textbf{growth} \textbf{rate} recovered again by the \textbf{time} of Malthus, when a population of one \textbf{billion} people was growing by about 0.4 % annually, adding about 4 million more people each year.
Later, the oil boom in the 1940s and 50s accelerated population \textbf{growth}.
By the \textbf{time} Ehrlich wrote The Population Bomb in 1968, the \textbf{growth} \textbf{rate} had soared to 2.2 %, the population to 3.5 \textbf{billion}, and we were producing 73 million more humans every year.
Modern contraception has allowed the \textbf{growth} \textbf{rate} to drop ever since, but only where women have freedom of choice and access to it.
Today, although the \textbf{growth} \textbf{rate} has declined to 1.1 %, the population has reached over 7.6 \textbf{billion}, and we ’re growing more than ever before in history : 83 million more people each year.
Percentages can be misleading.
As the percentage of starving people allegedly declines, the net \textbf{number} of starving people \textbf{increases}.
About 1.5 \textbf{billion} people suffer from malnutrition, and about \textbf{half} of those ( 815 million people ), go to sleep hungry every night.
Nine million people starve to death every year.
That ’s one every 3.5 seconds.
There are more malnourished people today than the \textbf{number} of people alive at the \textbf{time} of Malthus.
The \textbf{problem} of \textbf{scale} In 1993, at a scientific World Summit, 58 international science academies warned, “ the magnitude of the [ environmental ] threat … is linked to human population size and resource \textbf{use} per person.
Resource \textbf{use}, \textbf{waste} \textbf{production}, and environmental degradation are accelerated by population \textbf{growth}. ” Humans and human \textbf{livestock} now comprise about 98.5 % of mammal biomass on Earth.
Ehrlich points out that, “ massive \textbf{numbers} of humans, their \textbf{livestock} and chickens ” displace wildlife from available habitats and lead to the “ disappearance of large, wild animals. ” He warns that, “ \textbf{increasing} the \textbf{scale} of human enterprise, both population \textbf{numbers} and per-capita \textbf{consumption}, are still the main \textbf{drivers} of the extinction crisis. ” The challenge we face, as \textbf{environmentalists}, or as concerned citizens, is that “ \textbf{scale} ” is almost a taboo subject in \textbf{public} discourse.
Since population and overconsumption remain two of the primary \textbf{drivers} of ecological destruction, perhaps we should take on the challenge of stabilising our population, along with managing over-consumption.
We cannot presume to engineer our way out of these ecological realities without attention to \textbf{scale}.
We must embrace the nagging question of human \textbf{scale}, and recognise the \textbf{need} to slow down and \textbf{control} human enterprise. “ If you don't have some sort of appreciation of the \textbf{economy} as being embedded in the natural \textbf{systems} of the planet, ” urges Peter Victor from York University in Canada, “ you ’re not going to get very far understanding why we ’ve got the \textbf{problems} we have with the environment, and how we ’re going to solve them. ” “ The Population Bomb has been both praised and vilified, ” Ehrlich wrote in 2009, “ but there has been no controversy over its significance in calling attention to the demographic \textbf{element} in the human predicament … its basic message is more important today than it was 40 years ago. ” References : Ehrlich, at 82, remains one of the most active, outspoken, and effective ecology activists on Earth.
He still lectures at Stanford University in the US, and last year, the Vatican ’s Pontifical Academy of Science invited him to Rome to speak about the causes of mass biological extinction.
Last July he wrote a piece entitled, “ You don't \textbf{need} a scientist to know what ’s causing the sixth mass extinction ” for the Guardian.
Ehrlich points to two key \textbf{drivers} of ecological destruction : population \textbf{growth} and burgeoning resource \textbf{consumption}, especially by the rich.
Paul Ehrlich, The Population Bomb, Sierra Club / Ballentine Books, 1968.
Paul and Anne Ehrlich : “ The Population Bomb Revisited, ” Electronic Journal of Sustainable Development ( 2009 ) ( PDF ).
Paul and Anne Ehrlich : Population, Resources, Environment, W.
H.
Freeman \& Co, 1970.
Paul Ehrlich : “ You don't \textbf{need} a scientist to know what ’s causing the sixth mass extinction ” The Guardian. “ Yield \textbf{Trends} Are Insufficient to Double Global Crop \textbf{Production} by 2050, ” Deepak K.
Ray, Nathaniel D.
Mueller, Paul C.
West, Jonathan A.
Foley, PLOS Journal, 2013. “ World Population, Food, Natural Resources, and Survival, ” David Pimentel and Marcia Pimentel, World Futures, 59 : 145-167, 2003. “ World population projected to reach 9.8 \textbf{billion} in 2050, and 11.2 \textbf{billion} in 2100 : UN, ” United nations Sustainable Development, World Population Project, 2017.
Summary : UN, Full \textbf{report} : UNPD/WPP. “ The Human Ecological Predicament : Wages of Self-Delusion ” William Rees, Professor emeritus, Human Ecology, University of British Columbia, Millennium Alliance for Humanity and Biosphere ( MAHB ), 2017. “ Harvesting the Biosphere : The Human Impact, ” Vaclav Smil, Population and Development Review 37(4), PDR 2011 ) ” Regarding \textbf{soil} degradation, which you highlight, you can add these, which reference the “ 50 % in some regions. ” Population and \textbf{consumption} “ Land Degradation : An overview, ” H.
Eswaran, R.
Lal, and P.F.
Reich, International Conference on Land Degradation and Desertification ; USDA and Oxford Press, India, 2001. “ \textbf{Soil} Degradation, Land Scarcity and Food Security, ” Tiziano Gomiero, MDPI, 2016 “ Comments on FAOs State of World Fisheries and Aquaculture, ” Daniel Pauly, Dirk Zeller, Science Direct Marine Policy, \textbf{Volume} 77, March 2017.
In 1972, when the Limits to \textbf{Growth} \textbf{study} appeared, Greenpeace cofounders Ben and Dorothy Metcalfe attended the world ’s first UN Conference on the Human Environment in Stockholm, with the intention of placing nuclear bomb testing on the conference agenda.
They met Ehrlich, who was there to add population \textbf{growth} to the agenda, as a driving force of ecological destruction.
The Metcalfes supported Ehrlich, but not all \textbf{environmentalists} – and \textbf{none} of the political delegates – agreed.
The ecologist, Barry Commoner, argued against Ehrlich, insisting that human population \textbf{growth} did not pose a critical environmental threat. \textbf{Technology}, he claimed, would allow us to feed \textbf{billions} more people.
Commoner thought the more pressing issue was wasteful \textbf{consumption} by the rich.
Ehrlich agreed with Commoner that excessive \textbf{consumption} was a root cause of our \textbf{problems}, but maintained that ignoring population was a mistake.
Regardless of technical progress, Ehrlich explained, population \textbf{growth} leads to habitat destruction, humanitarian catastrophes, and ecological decline.
Ehrlich proposed universal women ’s rights and a global contraception campaign to lower the human birth \textbf{rate}.
Ehrlich ’s proposals, however, collided with cultural, political, and religious resistance.
The Stockholm conference avoided discussing population, and the environmental movement since 1972 has largely ignored human population \textbf{growth}.
Nevertheless, the nagging issue remains, four \textbf{billion} people later.
The limits are real

% matched lemmas: agriculture, amount, billion, chemical, consumption, control, cycle, driver, economy, element, environmentalist, fact, factory, flow, growth, half, health, heating, increase, infrastructure, livestock, material, may, mean, need, none, number, price, problem, production, public, quality, rate, report, revolution, scale, soil, study, supply, system, technology, time, ton, tonne, trend, use, volume, waste, will
\end{textsample}
