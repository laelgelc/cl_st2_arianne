\begin{textsample}{POS Dim 3 – human – Score 48.00 – t168\_human.txt}  \label{ex:f3_pos_006}
Last year marked the 50th anniversary of Greenpeace, and this year the 50th anniversary of a book that set much of the ecological agenda for the five decades since : The Limits to \textbf{Growth}, commissioned by the Club of Rome, and compiled by four distinguished \textbf{systems} scientists at Massachusetts Institute of Technology ( MIT ) Donella Meadows, Dennis Meadows, Jørgen Randers, and William Behrens ( Signet, 1972 ).
In 1989, Ronald Bailey attacked Limits to \textbf{Growth} in Forbes magazine, calling the book “ wrong-headed. ” He continued his attack in Eco-Scam : The False Prophets of Ecological Apocalypse, a diatribe full of errors and misrepresentations.
Bailey claimed, “ In 1972, The Limits to Growth predicted that … the world would run out of gold by 1981, mercury by 1985, tin by 1987, zinc by 1990, petroleum by 1992, and copper, lead, and natural \textbf{gas} by 1993. ” The \textbf{problem} with these diatribes by Jaffee, Manning, Bailey, and others, is that the book makes no such predictions.
The critics simply invented false predictions and then attacked them.
In 2000, economist Matthew Simmons wrote in Energy Bulletin, “ After reading The Limits to \textbf{Growth}, I was amazed … There was not one sentence or a single word written about an oil shortage, or limit to any specific resource, by the year 2000. … The most amazing aspect of the book is how accurate many of the basic \textbf{trend} extrapolations still are. ” It appears that most of the book ’s critics never actually read it.
The outrage they express appears to be ideological : How dare these Earth scientists suggest that there are limits to human ingenuity!
Furthermore, these attacks presaged our own era of bad-faith misinformation attacks by climate change denialists and limitless-growth advocates.
The Limits to \textbf{Growth} did not actually make any predictions.
Rather the authors offered 12 \textbf{scenarios} that might unfold on Earth between 1972 and 2100, based on whether or not humanity recognized the ecological risks, and took appropriate action.
The first \textbf{scenario}, the “ \textbf{standard} run, ” was based on “ \textbf{business} as usual, ” with no intervention, and forecast serious ecological crises in the early 21st Century, as we now witness all around us.
The second \textbf{scenario} allowed for technological advances that might double human access to resources.
This \textbf{scenario} also led to crisis, delayed by a few decades.
The other 10 \textbf{scenarios} estimated the \textbf{effects} of interventions, individually and in combination, including : \textbf{Recycling}, pollution \textbf{controls}, \textbf{soil} restoration, stabilizing population, restricting economic \textbf{growth}, and extending the lifespan of industrial assets by eliminating planned obsolescence.
Some of these interventions have hardly been contemplated, \textbf{none} has been achieved on the global \textbf{scale} that the The Limits to \textbf{Growth} authors proposed, and as a result, we now find ourselves facing precisely the ecological crises forewarned in the “ \textbf{business} as usual ” \textbf{scenario} : Global \textbf{heating}, dying coral reefs, biodiversity collapse, dead lakes, drained aquifers, polluted \textbf{waterways}, endocrine disruptors and toxic \textbf{chemicals} in our bodies, \textbf{supply} \textbf{chain} disruptions from depleted resources, and a global pandemic, which is itself aggravated by human crowding and diminished wilderness habitats.
Lead author, Donella Meadows, passed away from cerebral meningitis in 2001, but spoke with Alice Friedemann at Energy Skeptic that year. “ We were at MIT, ” she said, “ we had been trained in science.
The way we thought about the future was utterly logical : if you tell people there ’s a disaster ahead, they \textbf{will} change course.
If you give them a choice between a \textbf{good} future and a bad one, they \textbf{will} pick the \textbf{good}.
They might even be grateful.
Naive, weren't we? ” In 1972, atmospheric carbon \textbf{dioxide} concentrations stood at 325 parts per million ( ppm ).
The Limits to \textbf{Growth} authors projected that, without intervention, concentrations could reach 380 ppm by 2000.
The actual concentration in 2000 was 370 ppm and is now over 420 ppm, the highest in some 4 million years, and now growing at the \textbf{rate} of about 3ppm per year.
The authors pointed out that there would be severe ecological \textbf{costs} to intensive industrial \textbf{agriculture} that required fossil fuels, fertilizers, \textbf{pesticides}, and massive irrigation.
Those \textbf{costs} now include global \textbf{heating}, \textbf{soil} degradation, habitat and biodiversity loss, toxic pollution, bee colony collapse, nutrient \textbf{cycle} disruption, fresh water depletion, and diminishing agricultural returns from additional fertilizer and \textbf{pesticide} \textbf{use}.
Since 1972, human population has doubled, \textbf{industry} has continued to profit from planned obsolescence, particularly in the \textbf{technology} \textbf{markets} for phones and computers, and our \textbf{consumption} of resources has more than tripled, from about 30-billion tonnes/year ( t/yr ) in 1972 to 100-billion t/yr now.
The Limits to \textbf{Growth} authors \textbf{used} a computer \textbf{model} ( “ World3 ” ) to track growing per-capita \textbf{consumption}, population, industrial \textbf{production} and agricultural \textbf{production}, plotted against resource depletion and \textbf{levels} of pollution, including CO 2.
They mapped out 12 future \textbf{scenarios}, depending on various social interventions that might mitigate a large-scale catastrophe.
With no intervention, their converging \textbf{trend} lines suggested serious ecological crises early in the 21st century, that is, roughly now.
Cassandra, the Trojan priestess of Greek mythology, earned the power to see the future, a gift from her admirer Apollo.
When she scorned him, he could not revoke his gift, but took revenge by cursing her that no one would ever believe her.
Before she died, Donella Meadows worked with Jorgen Randers and Dennis Meadows to complete Limits to \textbf{Growth} : The 30 Year Update, ( Chelsea Green ).
In a Summary of the 30-year research, she wrote : “ In 1998, more than 45 percent of the globe ’s people had to live on incomes \textbf{averaging} \$2 a day or less.
Meanwhile, the richest one-fifth of the world ’s population has 85 percent of the global GNP.
And the gap between rich and poor is widening. … Sea \textbf{level} has risen 10–20 cm since 1900 … 75 percent of the world ’s oceanic fisheries were fished at or beyond capacity.. 38 percent, or nearly 1.4 \textbf{billion} acres, of currently \textbf{used} agricultural land has been degraded … and nothing that has happened in the last 30 years has invalidated the book ’s warnings.
These are symptoms of a world in overshoot, where we are drawing on the world ’s resources faster than they can be restored, and we are releasing \textbf{wastes} and pollutants faster than the Earth can absorb them. ” Dennis Meadows recently told Richard Heinberg of the Post Carbon Institute, “ Whatever we recommended then certainly is not relevant now.
In 1972, the impact of humanity on the globe was probably below sustainable … the goal then was to slow \textbf{things} down before we hit the limit. “ Now it ’s clear that the \textbf{scale} of human activities is far above the limit, ” he said. “ And our goal is not to slow down, but to get back down : to find ways to maneuver the \textbf{system}, in a peaceful, equitable, hopefully fairly liberal way, and bring our demands back down to \textbf{levels} that can be borne by the planet. … One reason \textbf{technology} and \textbf{markets} are unlikely to prevent overshoot and collapse is that \textbf{technology} and \textbf{markets} are merely tools to serve goals of society as a whole.
If society ’s implicit goals are to exploit nature, enrich the elites, and ignore the long \textbf{term}, then society \textbf{will} develop \textbf{technologies} and \textbf{markets} that destroy the environment.. that hasten a collapse instead of preventing it. ” In 2014, Graham Turner at the University of Melbourne completed a 40year update : “ Is Global Collapse Imminent?
An Updated Comparison of The Limits to \textbf{Growth} with Historical \textbf{Data}. ” Turner gathered \textbf{data} from UNESCO, the UN Food and Agriculture Organization, US national oceanic and atmospheric administration ( NOAA ), the British Petroleum statistical \textbf{review}, and other sources, plotted alongside the Limits to \textbf{Growth} \textbf{scenarios}.
The Guardian published his charts showing that world food sources, industrial output, population, pollution, and resource decline all tracked closely to the 1972 “ \textbf{standard} run ” that led to crisis.
The Guardian concluded, as did Graham Turner, “ Limits to \textbf{Growth} was right. ” In 2016, the United Nations International Resource Panel ( IRP ), published “ Global \textbf{Material} \textbf{Flows} and Resource Productivity, ” a \textbf{report} that admits what Limits to \textbf{Growth} warned of decades earlier : Resources are limited, human \textbf{consumption} \textbf{trends} are unsustainable, and the resource depletion diminishes human \textbf{health}, \textbf{quality} of life, and future development.
Being an ecologist \textbf{means} rarely taking pleasure in being proven correct about the \textbf{scale} of our crisis.
Before she died in 2001, Donella Meadows said : “ I think we are headed for disaster.
That thought does not thrill me, and it does not panic me into trying to \textbf{fashion} a world so \textbf{controlled} that it is actually predictable.
Rather it energizes me to work toward a vision of a World-That-Works-For-Everyone, including all the nonhuman Everyones. ” Cassandra, remember, really did see the future.
The fools around her brought down Troy.
The Limits to Growth, Donella Meadows, Jorgen Randers, Dennis Meadows, William Behrens, Signet, 1972.
Limits to \textbf{Growth} : The 30 Year Update, 2004, Donella Meadows, Jorgen Randers, Dennis Meadows, Chelsea Green, and a “ Summary, ” donellameadows.org.
The 1970s witnessed a deluge of ecological awareness and literature, roused to a great extent by Rachel Carson ’s 1962 Silent Spring and by common experiences of oil spills, toxic pollution, and disappearing species. ( See a bibliography of 1970s ecology books below ). “ Limits to \textbf{Growth} was right.
New research shows we ’re nearing collapse, ” Graham Turner and Cathy Alexander, The Guardian, 2014.
Is Global Collapse Imminent?
An Updated Comparison of The Limits to \textbf{Growth} with Historical Data, Graham Turner, University of Melbourne, 2014.
Richard Heinberg : Interview with Dennis Meadows on the 50th anniversary of the publication of The Limits to Growth, Resilience, February 2022.
Limits and Beyond : 50 years on from The Limits to Growth, what did we learn and what ’s next?, Club of Rome, Carlos Alvarez Pereira, Ugo Bardi, Nora Bateson, Gianfranco Bologna, Yi-Heng Cheng, Wouter van Dieren, Sandrine Dixson-Declève, Sirkka Heinonen, Gaya Herrington, Julia Kim, Petra Künkel, Hunter Lovins, Dennis Meadows, Chandran Nair, Ndidi Nnoli-Edozien, Chuck Pezheski, Mamphela Ramphele, Jorgen Randers, Yury Sayamov, Ernst von Weizsäcker, Sviastolav Zabelin, Exapt Press, 2022. “ Revisiting The Limits to \textbf{Growth} : Could the Club of Rome have been correct, after all? ” Mathew R.
Simmons, Energy Bulletin, PDF, 2008. “ Global \textbf{Material} \textbf{Flows}, Resource Productivity, UNEP Report, 2016. “ The Shocks of a World of Cheap Oil, ” Amy Myers Jaffe and Robert A.
Manning, Foreign Affairs, Jan./Feb. 2000.
H.C.
Wallich quote, “ irresponsible nonsense, ” from “ To grow or not to grow, ” Newsweek ( 1972, 13 March ) p. 102–103.
Donella Meadows quotes from energyresources.com, Dec. 20, 2002 : private email from Alice Friedemann.
Barbara Ward and Rene Dubois, Only One Earth, Penguin, 1972.
In 1972, after the world ’s first global environmental conference in Stockholm, Barbara Ward and Rene Dubois published Only One Earth : The Care and Maintenance of a Small Planet ( Penguin, 1972 ).
Farley Mowat ’s 1972 A Whale for the Killing, ( McClelland and Stewart ) inspired the 1975 Greenpeace whale campaign, and Barry Commoner ’s 1971, The Closing Circle, influenced the Greenpeace “ Laws of Ecology. ” These and other ecology writings inspired me in my transition from a physics and engineering student to an ecologist and journalist.
Farley Mowat, A Whale for the Killing, ( McClelland and Stewart, 1972 ).
Barry Commoner, The Closing Circle, 1971.
Dolores LaChapelle, Earth Wisdom, ( Guild of Tutors, 1972 ).
Gregory Bateson, Steps to an Ecology of Mind, ( Jason Aronson, 1972 ) ; Gregory Bateson, Mind and Nature, Dutton, 1979.
Arne Naess, Ecology, Community and Lifestyle, Cambridge University, 1976.
Paul Shepard : The Tender Carnivore and the Sacred Game, Scribner, 1973.
Annie Dillard, Pilgrim at Tinker Creek, ( Harper \& Row, 1974 ).
Edward Abbey, The Monkey Wrench Gang, Lippincott, 1975.
Herman Daly, Steady-State Economics, University of Maryland ; World Bank, 1977.
The “ Three Laws of Ecology, ” published by Greenpeace in 1976, emphasized ecological interdependence, diversity, and finiteness.
The Limits to \textbf{Growth} authors rigorously examined this last characteristic of ecosystems, the \textbf{fact} that \textbf{energy} and \textbf{material} resources are finite and therefore impose limits on the expansion of all species.
The book sold over 12 million copies in 37 languages, the \textbf{best} selling ecology book of all \textbf{time}.
Howard Odum, Environment, Power, and Society, Wiley Interscience, 1971.
Frances Moore Lappe, Diet for a Small Planet, Ballantine Books, 1971.
Charles Reich, The Greening of America, Random House, 1970.
Garrett Hardin, Exploring New Ethics for Survival, Viking, 1972.
Joseph Meeker, The Comedy of Survival, Guild of Tutors, 1972.
Paul Singer, Animal Liberation, Harper, 1975.
Susan Griffin, Woman and Nature, Harper \& Row, 1978.
James Lovelock, Gaia.
A New Look at Life on the Earth, Oxford U., 1979.
William Catton, Overshoot, University of Illinois, 1980.
In 1972, the \textbf{fact} that ecosystems place limits on \textbf{growth} was not a radical or controversial idea among ecologists and biologists.
Nevertheless, the idea challenged some hallowed assumptions of mainstream industrial society : Human exceptionalism, unlimited \textbf{growth}, and the endless expansion of wealth.
Ecology, a recognition of natural law, transcended liberal and conservative ideologies about how the spoils of industrial \textbf{growth} should be shared.
Ecology questions industrial \textbf{growth} itself.
Establishment economists and mainstream media attacked The Limits to \textbf{Growth} viciously.
Within a week of its publication, in Newsweek magazine, Yale economist Henry Wallich, dismissed the book as “ a piece of irresponsible nonsense. ” Later, U.S. president Ronald Reagan declared “ There are no great limits to \textbf{growth}, when men and women are free to follow their dreams. ” This inspiring Reaganism expresses the essence of human-exceptionalism, the attitude that nothing can stop or limit humanity from achieving whatever people desire, at any \textbf{measure}, at any \textbf{scale}, in any \textbf{numbers}.
Since humanity has become so successful at occupying and consuming resources from every ecosystem on Earth, it is easy to see how these notions of exceptionalism arise.
Nevertheless, these are anti-ecological ideas, presuming that humans — through cleverness, imagination, and \textbf{technology} — can flourish outside the biological constraints of all living organisms.
The attacks continued over decades, and critics typically misrepresented the book.
In 2000, Amy Myers Jaffe and Robert Manning wrote in, Foreign Affairs, “ In its dramatic 1972 ‘ Limits to Growth ’ \textbf{report}, the group of prominent experts known as the Club of Rome wrote that only 550 \textbf{billion} barrels of oil remained and that they would run out by 1990. ”

% matched lemmas: agriculture, average, billion, business, chain, chemical, consumption, control, cost, cycle, datum, dioxide, effect, energy, fact, fashion, flow, gas, good, growth, health, heating, industry, level, market, material, mean, measure, model, none, number, pesticide, problem, production, quality, rate, recycling, report, review, scale, scenario, soil, standard, supply, system, technology, term, thing, time, trend, use, waste, waterway, will
\end{textsample}
