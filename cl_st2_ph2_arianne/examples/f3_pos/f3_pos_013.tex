\begin{textsample}{POS Dim 3 – human – Score 46.00 – t084\_human.txt}  \label{ex:f3_pos_013}
It ’s a volatile \textbf{time} for Europe.
War.
Historically high \textbf{gas}, oil, and \textbf{electricity} \textbf{prices}.
High inflation.
Looming recession.
Climate crisis.
Soon, winter \textbf{will} arrive, leaving people at risk of a choice between being able to afford to heat their homes or cook their meals.
Introduce occupancy detectors and \textbf{energy} saving LED lighting \textbf{Use} less hot water \textbf{Shift} to \textbf{energy} efficient equipment Manage demand – distribute \textbf{energy} \textbf{use} over the day and night Curtail operations of \textbf{top} industrial and non-essential \textbf{energy} users Everybody who can must contribute to reducing the \textbf{use} of \textbf{gas}, \textbf{electricity}, and oil but it is \textbf{industry} – the biggest users – who should be made to save the most.
This can be achieved through rationing and mandatory \textbf{measures} that prohibit \textbf{shifting} from \textbf{gas} to oil or coal.
We know it ’s possible.
For example, the Netherlands cut overall \textbf{gas} \textbf{consumption} by 25 % in the first \textbf{half} of 2022 compared to 2021, more than 30%in \textbf{energy} intensive \textbf{industries}.
The \textbf{energy} crisist requires governments to take extraordinary \textbf{measures} to ensure \textbf{energy} \textbf{reductions} across the board, especially from the biggest users, and \textbf{shift} the \textbf{burden} from individuals.
Introduce an affordable “ climate ticket ” for \textbf{public} \textbf{transport} ( like the German 9€‎ ticket ) Ban short haul flights where reasonable train \textbf{alternative} is available Ban private jets The first global \textbf{energy} crisis calls for an unprecedented \textbf{level} of intervention from national governments and the EU, the world ’s third biggest emitter of climate-wrecking pollution.
Europe must reduce its \textbf{consumption} of \textbf{gas}, oil, and \textbf{electricity} fast.
And it must be done in a way that ensures sustainability without exacerbating \textbf{energy} poverty. \textbf{Households} \textbf{need} fair \textbf{reduction} and fair redistribution to address the cost-of-living crisis. ( HINT : \textbf{Industry} is the biggest user and must be made to save the most! ).
Promote more efficient driving, lower speed limits New Greenpeace calculations show that short-term reforms would cut oil demand in the EU ´s \textbf{transport} \textbf{sector} by around 50 million \textbf{tonnes} of oil per year, and achieve annual \textbf{energy} savings of around 13 %.
The most effective \textbf{measures} to reduce \textbf{energy} \textbf{consumption} are affordable climate tickets for \textbf{public} \textbf{transport} across the EU, a \textbf{reduction} of flights, and efficient \textbf{car} usage.
Short-term reforms like more teleworking, affordable \textbf{public} \textbf{transport}, and lower speed limits could save EU \textbf{consumers} €63 \textbf{billion} on fuel.
These transport-related \textbf{energy} saving reforms would also lead to a \textbf{reduction} of \textbf{greenhouse} \textbf{gas} \textbf{emissions} by 180 million \textbf{tonnes} annually, \textbf{equivalent} to the \textbf{emissions} of 120 million fossil fuel-powered \textbf{cars} – almost \textbf{half} of the EU ’s \textbf{total} \textbf{car} fleet.
Ban the \textbf{sale} of new \textbf{gas} boilers Introduce support \textbf{schemes} for insulation and make home insulation mandatory Introduce support \textbf{schemes} for heat pumps, solar \textbf{heating} and photovoltaics ( PV ) Train more people to install insulation, heat pumps, solar \textbf{heating} and PV Boost/support \textbf{industries} producing insulation \textbf{material} People should not have to choose between heating or eating.
More heat pumps could make a big difference.
In the EU and UK, 2.2 million heat pumps were installed in 2021 which saved 1 \textbf{billion} cubic metres of \textbf{gas} \textbf{starting} from 2022.
Most of them replaced \textbf{gas} \textbf{heating}, some oil or electric \textbf{heating}.
The \textbf{growth} in 2021 was 34 percent.
This is not enough.
Doubling EU installation \textbf{rates} of heat pumps from the current trajectory would save 2 \textbf{billion} cubic metres of \textbf{gas} \textbf{use} within the first year.
Europe must import and manufacture more heat pumps so they can be readily available.
Governments should do everything they can to speed up the just transition – delivery \textbf{time} for rooftop PV to convert light into \textbf{electricity} could be 4-6 months.
To do this Europe \textbf{will} \textbf{need} more than 1 million solar workers in 2030 alone.
There should be an \textbf{end} to spending on dead \textbf{ends} like fossil \textbf{infrastructure} and nuclear power, and instead more \textbf{investment} in green jobs and a sustainable future for all.
And governments must stop the expansion of fossil fuel extraction and \textbf{infrastructure} to tackle the climate crisis that ’s affecting everyone and disproportionately harming the most vulnerable countries and communities worldwide.
Stabilising \textbf{energy} \textbf{systems} and safeguarding our environment are not mutually exclusive ; renewable \textbf{energy} is the \textbf{answer} to both the \textbf{energy} and climate crises.
Stop all subsidies that continue fossil fuels \textbf{consumption} Fossil fuel \textbf{companies} should be \textbf{taxed} 100 % of their windfall profits The proceeds from a defined windfall profits \textbf{tax} should be redistributed fairly at the national \textbf{level} or \textbf{used} to support communities in greater \textbf{need} of those resources.
It ’s \textbf{time} to make the polluters pay, and cut off the \textbf{money} for Putin ’s war. \textbf{Energy} \textbf{prices} globally have been rising fast since 2021 and it ’s gotten worse since the \textbf{start} of the war in Ukraine.
It doesn't look as if high \textbf{gas} and power \textbf{prices} \textbf{will} come to an \textbf{end} in the foreseeable future.
And it ’s \textbf{energy} producers, like Shell, who are making huge profits from this instability via windfall profits. \textbf{Companies} that produce or base their \textbf{business} on the \textbf{use} of fossil fuels ( i.e. oil and \textbf{gas} majors, refineries and fossil based utilities ) should have their windfall profits \textbf{taxed} to support \textbf{households} with lowest incomes and struggling small, and medium \textbf{companies} in exchange for \textbf{energy} \textbf{efficiency} \textbf{measures}.
Direct subsidies, fuel \textbf{tax} rebates, or lowering of \textbf{taxes} for fossil fuels must be avoided as they \textbf{will} not only serve those individuals, \textbf{companies}, and institutions who \textbf{use} the most \textbf{energy}, but also help to fund the war.
With a recession rolling across Europe, governments are introducing subsidies and \textbf{public} support \textbf{schemes} to counteract the \textbf{increased} \textbf{cost} of living. \textbf{Measures} include \textbf{increasing} minimum wage, sick-pay, unemployment benefits, rent subsidies, minimum pensions, supplementary benefits, and lower \textbf{taxes} for people with low incomes. \textbf{Public} \textbf{money} could dwindle fast so the \textbf{best} way to address inequality is to \textbf{tax} the massive profits of the fossil fuel \textbf{industry} to help cover the \textbf{public} spending \textbf{needed} globally to support the most vulnerable and transition to renewable \textbf{energy} for all.
People and the planet must come first, and it ’s the polluters that \textbf{need} to pay.
Check out the full policy brief on how to \textbf{deal} with Europe ’s \textbf{energy} crisis.
How decision makers handle this crisis \textbf{will} determine whether countries manage to meet commitments under the Paris climate agreement to keep global \textbf{heating} to 1.5°C, and therefore our future on this planet.
The International Energy Agency warned of “ the fragility and unsustainability of our current \textbf{energy} \textbf{system} ” and asked “ whether the crisis \textbf{will} be a setback for clean \textbf{energy} transitions or \textbf{will} catalyse faster action. ” Marie Bout is a communications \textbf{strategist} with the Energy Crisis in Europe team, based in France.
Europe ’s response must NOT be to promote new oil and \textbf{gas} extraction and export \textbf{infrastructure} at home or abroad.
That won't help anyone anywhere to tackle \textbf{energy} exclusion or \textbf{cost} of living crisis.
Across the African continent a colonial approach of extraction and exploitation continues to plague communities, paralyse \textbf{economies} and push ecosystems to the edge.
African activists are demanding \textbf{better} from their governments — to phase out fossil fuels and provide clean, safe decentralised renewable \textbf{energy} to the 600 million Africans \textbf{dealing} with \textbf{energy} poverty.
The long-term objective must be \textbf{energy} independence through 100 percent renewable \textbf{energy} for power, \textbf{heating}, \textbf{industry}, and \textbf{transport}.
While this can't be achieved quickly there are many \textbf{things} Europe ’s governments can do fast to help get us there.
Adjust indoor temperature – less \textbf{heating} in the winter, cooling in the summer Turn off ventilation and lights when not \textbf{needed}

% matched lemmas: alternative, answer, billion, burden, business, car, company, consumer, consumption, cost, deal, economy, efficiency, electricity, emission, end, energy, equivalent, gas, good, greenhouse, growth, half, heating, household, increase, industry, infrastructure, investment, level, material, measure, money, need, price, public, rate, reduction, sale, scheme, sector, shift, start, strategist, system, tax, thing, time, tonne, top, total, transport, use, will
\end{textsample}
