\begin{textsample}{POS Dim 3 – human – Score 48.00 – t513\_human.txt}  \label{ex:f3_pos_008}
I ’m in Vancouver, riding the skytrain, the metro-region ’s elevated and underground \textbf{public} \textbf{transport} \textbf{system}.
In a crowded cabin, I gaze above the seats and see this advertisement : Newsweek recently published an essay called “ This Controversial Way to Combat Climate Change Might Be the Most Effective ” by Michael Shank, who teaches sustainable development at NYU.
Shank encourages worldwide family planning to solve the ecological crisis.
Vaclav Smil, Professor Emeritus of Environment at the University of Manitoba, Canada, is considered among the world ’s experts in \textbf{energy} transitions.
His recent book, Growth : From Microorganisms to Megacities, describes an, “ irreconcilable conflict between the quest for continuous economic \textbf{growth} and the biosphere ’s limited capacity. ” All biological habitats are finite, all plants and animals reach individual maturity, and all natural communities reach collective maturity.
Predators grow until they deplete their \textbf{game}, then they die back.
Lake algae \textbf{will} grow until they deplete the available nutrients, then die back.
Ecosystems reach a “ climax status ” and endure in “ dynamic homeostasis ” until some catastrophe collapses the balance. \textbf{Growth} swamps \textbf{efficiency} Smil considers the idea that we can “ decouple ” economic \textbf{growth} from \textbf{material} throughput “ \textbf{total} nonsense. ” History shows that when human enterprise becomes more efficient with a resource, we \textbf{use} more of that resource, not less ; a phenomenon known as the “ rebound \textbf{effect} ” in \textbf{economics} or “ Jevon ’s paradox, ” in mechanical engineering.
In the 19th century, more efficient machines did not reduce coal \textbf{consumption}, but \textbf{increased} coal \textbf{consumption}.
In the 20th century, computers did not save resources, as some \textbf{technology} optimists predicted, but rather sped up the \textbf{economy}, leading to greater resource extraction.
As the rebound \textbf{effect} predicts, even with myriad modern \textbf{efficiencies}, we have burned through the highest \textbf{quality} \textbf{stores} of coal and oil, and are now digging into the dirty, low-grade shale oil and tar sands.
When Europeans first came to North America, they could pick up copper nuggets the size of watermelons from stream beds.
Now, to \textbf{supply} modern electronics, we have to dig giant pits – 4km wide, 1km deep – to scrape out low grade ore that contains 0.2 % of copper.
Limits don't necessarily \textbf{mean} that we “ run out ” of a resource, but that the \textbf{quality} declines as \textbf{costs} and ecological impact rise.
Smil emphasizes that we are now coming face-to-face with the real limits of human \textbf{scale}. “ That ’s our major \textbf{problem}, ” he writes, “ \textbf{scale}. ” Humans and their \textbf{livestock} now comprise 96 % of all mammal biomass on Earth.
That is a \textbf{problem} of \textbf{scale}.
Our climate crisis, our biodiversity crisis, our depleted \textbf{soils}, and humanitarian crises, are all symptoms of the unrealistic \textbf{scale} of human enterprise.
Smil argues that for the long-term survival of civilization, we have to accept the “ \textbf{end} of \textbf{growth}. ” We cannot engineer our way out of the contradiction or reconcile planetary constraints with unrestrained human aspirations.
We must abandon the idea that we can take baby steps.
We require a “ radical … bold vision … fundamental \textbf{shifts}, and unprecedented adjustments. ” Smil believes we could achieve this \textbf{shift} because of the tremendous “ slack in the \textbf{system}. ” The rich world \textbf{wastes} so much \textbf{energy}, \textbf{materials}, and food, that we could vastly reduce our \textbf{material} and \textbf{energy} \textbf{use} simply by avoiding \textbf{waste}.
The industrial \textbf{agriculture} \textbf{system}, for example, dumps vast quantities of toxic \textbf{pesticides} into our \textbf{soils}, burns millions of gallons of diesel fuel, disrupts Earth ’s nutrient \textbf{cycles} with ammonia and fertilizers, and then \textbf{wastes} 40 % of the harvest.
This is the “ slack ” in the \textbf{system} that could be \textbf{used} to help us reduce \textbf{consumption}.
There is no \textbf{option} C : “ Change our \textbf{system} to prevent it. ” The tiny home and shared office movements are a positive response to the massive \textbf{waste} in our private and \textbf{public} buildings.
Smil calculates that if all buildings were optimum size and well insulated, we could save about 20 % of our carbon \textbf{emissions}. “ But people aren't willing to do it, ” he warns. “ People want to have it all, giant houses with circular staircases, which are not properly insulated.
They want to have their SUVs and raspberries in January.
That ’s the \textbf{problem}. ” Furthermore, \textbf{household} \textbf{energy} \textbf{consumption} cannot be reduced simply with \textbf{efficiency}, because of the \textbf{increase} in home size.
According to William Rees at the University of British Columbia, since 1950, new US homes have grown from about 1000 ft 2 to over 2500 ft 2 while the \textbf{number} of people per home has dropped from 3.4 to 2.5 people.
In 70 years, the floor area per person has grown 240 %.
Meanwhile, since 1950, the US population has doubled.
World population has tripled.
Any marginal gains in \textbf{energy} \textbf{efficiency} have been swamped by \textbf{growth}.
Smil believes we could address our ecological crisis by taking advantage of this “ slack … the large margins for improvement … subsidized insulation retrofits.. down-sizing the \textbf{household}, duplex conversions, and upping \textbf{public} transit while limiting private \textbf{cars}. ” Population : the last taboo In the Newsweek essay, Michael Shank states that stabilizing and reducing human population is “ a necessary conversation that we can't keep avoiding. ” In “ The Last Taboo, ” written nine years ago in Mother Jones, Julia Whitty wonders : “ There are 7 \textbf{billion} humans on earth, so why can't we talk about population? ” She \textbf{reports} that scientists working on ways to address population often face harassment.
However, says Whitty, “ Voiced or not, the \textbf{problem} of overpopulation has not gone away. ” “ The only known \textbf{solution} to ecological overshoot, ” writes Whitty, “ is to decelerate our population \textbf{growth}.. and eventually reverse it [ and ] reverse the \textbf{rate} at which we consume the planet ’s resources.
Success in these twin endeavors \textbf{will} crack our most pressing global issues : climate change, food scarcity, water \textbf{supplies}, immigration, \textbf{health} care, biodiversity loss, even war. ” Fortunately, we do not \textbf{need} to resort to draconian laws to reverse population \textbf{growth}.
Scientists \textbf{studying} the \textbf{data} tell us that the most effective population action is to establish universal women ’s rights and universally available contraception.
Everywhere those goals are achieved, the birth \textbf{rate} plummets.
Slowing or reversing population \textbf{growth} would also help alleviate humanitarian challenges. “ The trial ahead ” Whitty writes, “ is to strike the delicate compromise between fewer people, and more people with fewer \textbf{needs}. ” In “ The Myth of Green Growth, ” Harry Haysom proposes that societies divert \textbf{money} from \textbf{consumption} to building green \textbf{infrastructure}, which he points out has been one of Greta Thunberg ’s arguments.
This little ad is a pitch by Metro Vancouver ’s transportation authority, TransLink.
They want the \textbf{public} to choose “ underwater transit \textbf{technology} ” – it \textbf{means} bigger \textbf{budgets} for them.
They dress up this choice with the hot \textbf{term} “ \textbf{technology} ” and invite us to “ embrace ” this.
They don't mention that dikes also require “ \textbf{technology}, ” and they certainly don't mention the taboo topic of \textbf{system} change.
There are plenty of \textbf{studies} demonstrating that after basic \textbf{needs} are met, more income and \textbf{consumption} do not necessarily create more happiness, and often create more stress and anxiety.
Arne Naess, who founded the deep ecology movement 50 years ago, put it this way : “ Richer lives, simpler \textbf{means}. ” Regardless of what else we propose to solve our ecological crisis, the \textbf{time} has come to reduce the \textbf{scale} of humanity ’s \textbf{footprint} on Earth.
It ’s \textbf{time} for Plan-C : Change our lifestyles and our economic \textbf{system}.
Vaclav Smil, “ \textbf{Growth} : From Microorganisms to Megacities, ” MIT Press, 2019.
Vaclav Smil, “ The Long-Term Survival of Our Civilization Cannot Be Assured, ” New York Magazine, Sept. 2019.
David Wallace-Wells, interview with Vaclav Smil, “ We Must Leave \textbf{Growth} Behind, New York Magazine, Sept. 24, 2019. “ The Myth of Green Growth, ” Harry Haysom, Financial Times, Oct 23, 2019.
Jørgen Stig Nørgård, John Peet, Kristín Vala Ragnarsdóttir, “ The History of The Limits to \textbf{Growth}, \textbf{Solutions} Journal, March 2010.
William Catton, “ Overshoot : The Ecological Basis of Revolutionary Change ”, University of Illinois Press, 1982.
Donella Meadows, et. al., “ Limits to \textbf{Growth} ” ( D.
H.
Meadows, D.
L.
Meadows, J.
Randers, W.
Behrens ), 1972 ; New American Library, 1977. “ Renewables 2019 : Status \textbf{Report}, ” Renewable Energy Policy Network for the 21st Century ( REN21 ).
Embracing the crisis with \textbf{technology} feels \textbf{good} ; it \textbf{means} growing our \textbf{economy}, advancing, having more, not giving up anything.
This blind spot remains our deep, unspoken \textbf{problem}.
We want to solve the ecological crisis and the humanitarian crisis with economic \textbf{growth} and “ \textbf{technology}, ” without changing anything.
We want it all.
Gunders, Dana. “ \textbf{Wasted} : How America is Losing Up to 40 Percent of Its Food from Farm to Fork to Landfill. ” Natural Resources Defense Council, 2017.
Chris Huber, “ World ’s food \textbf{waste} could feed 2 \textbf{billion} people, ” World Vision, 2017. “ Why Is Population \textbf{Control} Such a Radioactive Topic? ” Population Forum, Mother Jones, 2010.
Julia Whitty, “ The Last Taboo, ” Mother Jones, June 2010.
David Clingingsmith, “ Negative Emotions, Income, and Welfare, ” Department of Economics, Case Western Reserve University, September 2015 The \textbf{problem} is, we ’re avoiding the root \textbf{problem} and the genuine \textbf{solutions}.
The climate crisis, biodiversity crisis, and all other ecological challenges are symptoms of the larger, deeper \textbf{problem} : Ecological Overshoot.
No species can grow out of overshoot.
All genuine \textbf{solutions} to our ecological dilemma must include a contraction of human \textbf{scale}.
We must relinquish our expectation of endless economic \textbf{growth}.
However, this appears as the one \textbf{solution} ignored by most people, governments, corporations, and even many \textbf{environmentalists}.
Slack in the \textbf{system} Finding an effective way to discuss human \textbf{scale} as a fundamental \textbf{problem} is one of the biggest challenges I ’ve faced as an ecologist.
People find the topic uncomfortable.
The idea of living with less conflicts with our notions of unlimited human ingenuity, human destiny, human rights, and human specialness.
We find it difficult to admit that even with all our technological wonders, we are animals, subject to the biological and physical limits of our habitat.
Modern ecologists have known and described ecosystem limits for decades, highlighted in the 1972 Limits to \textbf{Growth} \textbf{study}.
Capitalist economists, however, mocked the idea that there exist any limits to human economic \textbf{growth}.
In the 1980s, U.S.
President Ronald Reagan claimed, “ There are no such \textbf{things} as limits to \textbf{growth}, because there are no limits to the human capacity for intelligence, imagination, and wonder. ” The denial feels appealing.
This long-standing taboo against talking about the limits of human \textbf{growth} is beginning to crumble, however, even within mainstream media.
Last month, the Financial Times, published an article by Harry Haysom, “ The Myth of Green Growth, ” in which he states that, “ green \textbf{growth} probably doesn't exist. ” As we build out our renewable \textbf{energy} \textbf{infrastructure}, he suggests, we \textbf{need} to simultaneously contract our \textbf{economies} and consume less \textbf{material} and \textbf{energy}.

% matched lemmas: agriculture, billion, budget, car, consumption, control, cost, cycle, datum, economics, economy, effect, efficiency, emission, end, energy, environmentalist, footprint, game, good, growth, health, household, increase, infrastructure, livestock, material, mean, money, need, number, option, pesticide, problem, public, quality, rate, report, scale, shift, soil, solution, store, study, supply, system, technology, term, thing, time, total, transport, use, waste, will
\end{textsample}
