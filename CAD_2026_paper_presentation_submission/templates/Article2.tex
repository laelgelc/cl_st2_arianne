%\selectlanguage{brazilian}

\title{AI Discourses on Climate Change: A Lexical Multidimensional Analysis of LLMs}
\author{Arianne Brogini} % Your name here
\date{November 2025}

\maketitle

\textcite{biberWhatRegisterAccounting2023}

\textcite{hughesIPCCPoliticsWriting2024}

Large Language Models (LLMs) have been producing and reproducing utterances that circulate widely across the social sphere and, as they spread, undergo a process of progressive naturalisation, gradually becoming incorporated into the human discursive repertoire. In this process, they also reach topics of high public relevance, such as issues related to climate change. Understanding discourse as ideologically driven representations of real-world phenomena that are socially shared, situated, produced within social practice and capable of generating meaning \parencite{berbersardinhaLexicalMultiDimensionalAnalysis2025}, the utterances generated by LLMs not only reflect or reiterate dominant voices \parencite{gillingsRiseLargeLanguage2025}, but also manifest, for instance, social identity biases \parencite{huGenerativeLanguageModels2024}. Their outputs may compete with, reinforce, or even replace discourses produced by human social groups, thus interfering with the collective construction of meaning on matters of public relevance.

Considering the scope and significance of climate change discourses, this study focuses on those produced and reproduced by LLMs and by human actors, with the aim of comparing their discursive and ideological patterns, as well as other lexically grounded constructs \parencite{berbersardinhaAIgeneratedHumanauthoredTexts2024}. To this end, the study employs Lexical Multidimensional Analysis \parencite{berbersardinhaLexicalMultiDimensionalAnalysis2025} to identify discursive profiles across key domains, within a robust corpus comparing utterances generated by LLMs (ChatGPT, Gemini, Grok), based on prompts, with utterances produced by different human actors on the internet, across a range of registers: press, governmental, non-governmental, and civil society.

The research questions are guided by the identification of: (i) which discourses are produced by LLMs on climate change; (ii) which discourses are produced by human actors on the same topic; and (iii) what the consonances -- understood as alignments or reinforcement of meanings -- and the tensions -- understood as divergences, shifts, or discursive resistances -- are between LLMs’ and humans’ discourses on climate change.

Lexical Multidimensional Analysis will allow the observation of statistically significant linguistic patterns, revealing the underlying discourses within the LLMs’ productions and reproductions on the phenomenon of climate change, as compared to those emerging from human social production

(<> characters)

\vspace{1em}
\textbf{Keywords}: <>, <>
