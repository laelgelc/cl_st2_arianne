%\selectlanguage{brazilian}

\title{8\textsuperscript{th} Corpora and Discourse International Conference (CAD 2026)}
\author{Rogério Yamada} % Your name here
\date{November 2025}

\maketitle

%\begin{abstract}
%    abstract
%
%    \vspace{1em}
%    \textbf{Keywords}: <>, <>, <>, <>, <>
%\end{abstract}

\section{\href{https://wp.lancs.ac.uk/cad-2026/}{Corpora and Discourse International Conference 2026}}

\begin{itemize}
    \item 23-25 June 2026, Lancaster University
    \item The conference dinner takes place on the evening of the 24\textsuperscript{th} June
\end{itemize}

The Corpora and Discourse International Conference brings together researchers from across the world who conduct research at the intersections of corpus linguistics and discourse analysis. Our Call for Papers can be found \href{https://wp.lancs.ac.uk/cad-2026/call-for-papers/}{here}.

Research presented at the conference might include (but is certainly not limited to) work that identifies as: corpus-assisted discourse studies, corpus-based (critical) discourse studies, corpus-based contrastive linguistics, corpus-based sociolinguistics, corpus-driven discourse studies, corpus-informed discourse studies, corpus stylistics, corpus pragmatics or corpus and discourse work that does not identify with a particular label.

The conference aims to showcase the diversity of research in this area and to give space for important conversations that help to move this varied field forward.

key dates:

\begin{itemize}
    \item 2 September 2025 -- \href{https://wp.lancs.ac.uk/cad-2026/call-for-papers/}{Call for Papers} opens
    \item 16 November 2025 -- Deadline for abstract submissions
    \item Mid-January 2026 -- Notifications of acceptance (or rejection)
    \item 2 February 2026 -- Early bird registration opens
    \item 6 April 2026 -- Late bird registration opens
    \item 1 June 2026 -- Registration closes
    \item 23-25 June 2026 -- Conference
\end{itemize}

\section{Submission Information}

As this is an in-person only conference, all presenting participants must attend the conference in person.

Oral presentations should present research that is either completed or that is ongoing with some substantial results. Research that is a work in progress and that is yet to yield substantial results is welcome in the form of a poster presentation.

Oral presentations will consist of a 20-minute talk followed by 5 minutes for questions and discussion.

For oral/poster presentations, please submit an unstructured abstract of up to 4000 characters (including references).

All abstracts should be submitted via this form: \href{https://forms.cloud.microsoft/e/qkm2sqPKA1}{https://forms.cloud.microsoft/e/qkm2sqPKA1}.

Please format your in-text references and bibliography in Chicago Citation Style.

\subsection{Title}

AI Discourses on Climate Change: A Lexical Multidimensional Analysis of LLMs

\subsection{Abstract}

Please enter an abstract of your paper (max 4000 characters, including references).


(<> characters)

\vspace{1em}
\textbf{Keywords}: <>, <>
